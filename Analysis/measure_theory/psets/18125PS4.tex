%%%This is a science homework template. Modify the preamble to suit your needs. 

\documentclass[12pt]{article}

\makeatother
%AMS-TeX packages
\usepackage{amsmath}
\usepackage{amssymb}
\usepackage{amsthm}
\usepackage{array}
\usepackage{amsfonts}
\usepackage{cancel}
\usepackage[all,cmtip]{xy}%Commutative Diagrams
\usepackage[pdftex]{graphicx}
\usepackage{float}
%geometry (sets margin) and other useful packages
\usepackage[margin=1in]{geometry}
\usepackage{sidecap}
\usepackage{wrapfig}
\usepackage{verbatim}
\usepackage{mathrsfs}
\usepackage{marvosym}
\usepackage{stmaryrd}
\usepackage{hyperref}
\usepackage{graphicx,ctable,booktabs}

\newtheoremstyle{norm}
{3pt}
{3pt}
{}
{}
{\bf}
{:}
{.5em}
{}

\theoremstyle{norm}
\newtheorem{thm}{Theorem}[section]
\newtheorem{lem}[thm]{Lemma}
\newtheorem{df}{Definition}
\newtheorem{rem}{Remark}
\newtheorem{st}{Step}
\newtheorem{pr}[thm]{Proposition}
\newtheorem{cor}[thm]{Corollary}
\newtheorem{clm}[thm]{Claim}

%Math blackboard, fraktur, and script commonly used letters
\newcommand{\A}[0]{\mathbb{A}}
\newcommand{\C}[0]{\mathbb{C}}
\newcommand{\sC}[0]{\mathcal{C}}
\newcommand{\cE}[0]{\mathscr{E}}
\newcommand{\F}[0]{\mathbb{F}}
\newcommand{\cF}[0]{\mathscr{F}}
\newcommand{\cG}[0]{\mathscr{G}}
\newcommand{\sH}[0]{\mathscr H}
\newcommand{\Hq}[0]{\mathbb{H}}
\newcommand{\cI}[0]{\mathscr{I}}%ideal sheaf
\newcommand{\N}[0]{\mathbb{N}}
\newcommand{\Pj}[0]{\mathbb{P}}
\newcommand{\bS}[0]{\mathbb{S}}
\newcommand{\sO}[0]{\mathcal{O}}
\newcommand{\cO}[0]{\mathscr{O}}
\newcommand{\Q}[0]{\mathbb{Q}}
\newcommand{\R}[0]{\mathbb{R}}
\newcommand{\Z}[0]{\mathbb{Z}}
%Lowercase
\newcommand{\ma}[0]{\mathfrak{a}}
\newcommand{\mb}[0]{\mathfrak{b}}
\newcommand{\fg}[0]{\mathfrak{g}}
\newcommand{\vi}[0]{\mathbf{i}}
\newcommand{\vj}[0]{\mathbf{j}}
\newcommand{\vk}[0]{\mathbf{k}}
\newcommand{\mm}[0]{\mathfrak{m}}
\newcommand{\mfp}[0]{\mathfrak{p}}
\newcommand{\mq}[0]{\mathfrak{q}}
\newcommand{\mr}[0]{\mathfrak{r}}
%Letter-related
\providecommand{\cal}[1]{\mathcal{#1}}
\renewcommand{\cal}[1]{\mathcal{#1}}
\newcommand{\bb}[1]{\mathbb{#1}}
%More sequences of letters
\newcommand{\chom}[0]{\mathscr{H}om}
\newcommand{\fq}[0]{\mathbb{F}_q}
\newcommand{\fqt}[0]{\mathbb{F}_q^{\times}}
\newcommand{\sll}[0]{\mathfrak{sl}}
%Shortcuts for symbols
\newcommand{\nin}[0]{\not\in}
\newcommand{\opl}[0]{\oplus}
\newcommand{\ot}[0]{\otimes}
\newcommand{\rc}[1]{\frac{1}{#1}}
\newcommand{\rra}[0]{\rightrightarrows}
\newcommand{\send}[0]{\mapsto}
\newcommand{\sub}[0]{\subset}
\newcommand{\subeq}[0]{\subseteq}
\newcommand{\supeq}[0]{\supseteq}
\newcommand{\nsubeq}[0]{\not\subseteq}
\newcommand{\nsupeq}[0]{\not\supseteq}
%Shortcuts for greek letters
\newcommand{\al}[0]{\alpha}
\newcommand{\be}[0]{\beta}
\newcommand{\ga}[0]{\gamma}
\newcommand{\Ga}[0]{\Gamma}
\newcommand{\de}[0]{\delta}
\newcommand{\De}[0]{\Delta}
\newcommand{\ep}[0]{\varepsilon}
\newcommand{\eph}[0]{\frac{\varepsilon}{2}}
\newcommand{\ept}[0]{\frac{\varepsilon}{3}}
\newcommand{\la}[0]{\lambda}
\newcommand{\La}[0]{\Lambda}
\newcommand{\ph}[0]{\varphi}
\newcommand{\rh}[0]{\rho}
\newcommand{\te}[0]{\theta}
\newcommand{\om}[0]{\omega}
\newcommand{\Om}[0]{\Omega}
\newcommand{\si}[0]{\sigma}
%Brackets
\newcommand{\ab}[1]{\left| {#1} \right|}
\newcommand{\ba}[1]{\left[ {#1} \right]}
\newcommand{\bc}[1]{\left\{ {#1} \right\}}
\newcommand{\pa}[1]{\left( {#1} \right)}
\newcommand{\an}[1]{\langle {#1}\rangle}
\newcommand{\fl}[1]{\left\lfloor {#1}\right\rfloor}
\newcommand{\ce}[1]{\left\lceil {#1}\right\rceil}
%Text
\newcommand{\btih}[1]{\text{ by the induction hypothesis{#1}}}
\newcommand{\bwoc}[0]{by way of contradiction}
\newcommand{\by}[1]{\text{by~(\ref{#1})}}
\newcommand{\ore}[0]{\text{ or }}
%Arrows
\newcommand{\hr}[0]{\hookrightarrow}
\newcommand{\xr}[1]{\xrightarrow{#1}}
%Formatting
\newcommand{\subprob}[1]{\noindent\textbf{#1}\\}
%Functions, etc.
\newcommand{\Ann}{\operatorname{Ann}}
\newcommand{\Arc}{\operatorname{Arc}}
\newcommand{\Ass}{\operatorname{Ass}}
\newcommand{\Aut}{\operatorname{Aut}}
\newcommand{\chr}{\operatorname{char}}
\newcommand{\cis}{\operatorname{cis}}
\newcommand{\Cl}{\operatorname{Cl}}
\newcommand{\Der}{\operatorname{Der}}
\newcommand{\End}{\operatorname{End}}
\newcommand{\Ext}{\operatorname{Ext}}
\newcommand{\Frac}{\operatorname{Frac}}
\newcommand{\FS}{\operatorname{FS}}
\newcommand{\GL}{\operatorname{GL}}
\newcommand{\Hom}{\operatorname{Hom}}
\newcommand{\Ind}[0]{\text{Ind}}
\newcommand{\im}[0]{\text{im}}
\newcommand{\Leb}[0]{\operatorname{Leb}}
\newcommand{\nil}[0]{\operatorname{nil}}
\newcommand{\ord}[0]{\operatorname{ord}}
\newcommand{\Proj}{\operatorname{Proj}}
\newcommand{\rad}{\operatorname{rad}}
\newcommand{\Rad}{\operatorname{Rad}}
\newcommand{\rank}{\operatorname{rank}}
\newcommand{\Res}[0]{\text{Res}}
\newcommand{\sign}{\operatorname{sign}}
\newcommand{\SL}{\operatorname{SL}}
\newcommand{\Spec}{\operatorname{Spec}}
\newcommand{\Specf}[2]{\Spec\pa{\frac{k[{#1}]}{#2}}}
\newcommand{\spp}{\operatorname{sp}}
\newcommand{\spn}{\operatorname{span}}
\newcommand{\Supp}{\operatorname{Supp}}
\newcommand{\Tor}{\operatorname{Tor}}
\newcommand{\tr}[0]{\text{trace}}
\newcommand{\Var}{\operatorname{Var}}
\newcommand{\vol}[0]{\operatorname{vol}}
%Commutative diagram shortcuts
\newcommand{\fiber}[3]{\xymatrix{#1\times_{#3} #2}\ar[r]\ar[d] #1\ar[d] \\ #2 \ar[r] & #3}
\newcommand{\commsq}[8]{\xymatrix{#1\ar[r]^{#6}\ar[d]^{#5} &#2\ar[d]^{#7} \\ #3 \ar[r]^{#8} & #4}}
%Makes a diagram like this
%1->2
%|    |
%3->4
%Arguments 5, 6, 7, 8 on arrows
%  6
%5  7
%  8
\newcommand{\pull}[9]{
#1\ar@/_/[ddr]_{#2} \ar@{.>}[rd]^{#3} \ar@/^/[rrd]^{#4} & &\\
& #5\ar[r]^{#6}\ar[d]^{#8} &#7\ar[d]^{#9} \\}
\newcommand{\back}[3]{& #1 \ar[r]^{#2} & #3}
%Syntax:\pull 123456789 \back ABC
%1=upper left-hand corner
%2,3,4=arrows from upper LH corner, going down, diagonal, right
%5,6,7=top row (6 on arrow)
%8,9=middle rows (on arrows)
%A,B,C=bottom row
%Other
%Other
\newcommand{\op}{^{\text{op}}}
\newcommand{\fp}[1]{^{\underline{#1}}}
\newcommand{\rp}[1]{^{\overline{#1}}}
\newcommand{\rd}[0]{_{\text{red}}}
\newcommand{\pre}[0]{^{\text{pre}}}
\newcommand{\pf}[2]{\pa{\frac{#1}{#2}}}
\newcommand{\pd}[2]{\frac{\partial #1}{\partial #2}}
\newcommand{\bs}[0]{\backslash}
\newcommand{\sia}[0]{ $\si$-algebra}
\newcommand{\ol}[1]{\overline{#1}}
\newcommand{\iy}[0]{\infty}
\newcommand{\nl}[1]{\left \Vert #1\right \Vert_{L^1}}
%Matrices
\newcommand{\coltwo}[2]{
\left[
\begin{matrix}
{#1}\\
{#2} 
\end{matrix}
\right]}
\newcommand{\matt}[4]{
\left[
\begin{matrix}
{#1}&{#2}\\
{#3}&{#4}
\end{matrix}
\right]}
\newcommand{\smatt}[4]{
\left[
\begin{smallmatrix}
{#1}&{#2}\\
{#3}&{#4}
\end{smallmatrix}
\right]}
\newcommand{\colthree}[3]{
\left[
\begin{matrix}
{#1}\\
{#2}\\
{#3}
\end{matrix}
\right]}
%
%Redefining sections as problems
%
\makeatletter
\newenvironment{problem}{\@startsection
       {section}
       {1}
       {-.2em}
       {-3.5ex plus -1ex minus -.2ex}
       {2.3ex plus .2ex}
       {\pagebreak[3]%forces pagebreak when space is small; use \eject for better results
       \large\bf\noindent{Problem }
       }
       }
       {%\vspace{1ex}\begin{center} \rule{0.3\linewidth}{.3pt}\end{center}}
       }
\makeatother


%
%Fancy-header package to modify header/page numbering 
%
\usepackage{fancyhdr}
\pagestyle{fancy}
%\addtolength{\headwidth}{\marginparsep} %these change header-rule width
%\addtolength{\headwidth}{\marginparwidth}
\lhead{Problem \thesection}
\chead{} 
\rhead{\thepage} 
\lfoot{\small\scshape 18.125 Real and Functional Analysis} 
\cfoot{} 
\rfoot{\scriptsize PS \# 4} % !! Remember to change the problem set number
\renewcommand{\headrulewidth}{.3pt} 
\renewcommand{\footrulewidth}{.3pt}
\setlength\voffset{-0.25in}
\setlength\textheight{648pt}



%%%%%%%%%%%%%%%%%%%%%%%%%%%%%%%%%%%%%%%%%%%%%%%
%
%Contents of problem set
%    
\begin{document}
\title{18.125 Real and Functional Analysis PSet \#4}% !! Remember to change the problem set number
\author{Holden Lee}
\date{3/21/11}% !! Remember to change the date
\maketitle
\thispagestyle{empty}
\begin{problem}{\it (3.3.17, Averaging near points of $\Leb(f)$)}
\subprob{(i)}
Since $x\in \Leb(f)$,
\[
\lim_{I\searrow \{x\}} \rc{\vol{I}}\int_I|f-f(x)|\,d\la_{\R}=0.
\]
But $-|f-f(x)|\le f-f(x)\le |f-f(x)|$ so
\[
\lim_{I\searrow \{x\}} \rc{\vol{I}}\int_I f-f(x)\,d\la_{\R}=0\implies \lim_{I\searrow \{x\}} \rc{\vol{I}}\int_I f\,d\la_{\R}=f(x).
\]
It suffices to note that the intervals $(x-\de,x]$ and $[x,x+\de)$ are shrinking to $\{x\}$ as $\de\searrow 0$, and that $\vol([x,x+\de))=\vol((x-\de,x])=\de$.\\

\subprob{(ii)}
Since $f$ is continuous with compact support by 3.2.18 we may treat the integrals as Riemann integrals. %(Implicitly we replace integrals over $\R$ with integrals over an interval containing the support of the integrand, use 3.2.18, and then write the integral over $\R$ again.)
\begin{align*}
f_{\ep}(x)&=\int_{\R} \rh_{\ep}(x-t) f(t)\,dt\\
&=\int_{\R}\ep^{-1}\rh(\ep^{-1}(x-t))f(t)\,dt\\
&=\int_{\R}\cancel{\ep^{-1}}\rho(-t')f(x+\ep t')\cancel{\ep} \,dt'&t\mapsfrom x+\ep t'\\
&=\int_{0}^{\iy}\rho(-t)f(x+\ep t)\,dt+
\int_{-\iy}^{0}\rho(-t)f(x+\ep t)\,dt\\
&=\int_{0}^{\iy}\rho(-t)f(x+\ep t)\,dt+
\int_{0}^{\iy}\rho(t)f(x-\ep t)\,dt
\end{align*}
(In the last step we made the substitution $t\mapsfrom -t$---this makes the limits of integration $\iy$ to 0, absorbing a negative sign from $-dt$ makes it $0$ to $\iy$.)

Using integration by parts, and noting $\rh(\iy)=0$,
\begin{align*}
\int_{[0,\iy)} \rh(-t)f(x+\ep t)\,dt
&=\left.\rh(-t)\int_0^t f(x+\ep u)\,du\right|^{\iy}_0-\int_0^{\iy} \pa{\int_0^t f(x+\ep u)\,du}(-\rh'(-t))\,dt\\
&=0+\int_0^{\iy}\pa{\rh'(-t)\int_x^{x+\ep t} \rc{\ep}f(s)\,ds}\,dt
\end{align*}
where we made the substitution $u\mapsfrom \frac{s-x}{\ep}$ in the last step. Similarly, with the substitution $u\mapsfrom \frac{x-s}{\ep}$,
\begin{align*}
\int_{[0,\iy)} \rh(t)f(x-\ep t)\,dt
&=\left.\rh(t)\int_0^t f(x-\ep u)\,du\right|^{\iy}_0-\int_0^{\iy} \pa{\int_0^t f(x-\ep u)\,du}\rh'(t)\,dt\\
&=0-\int_0^{\iy}\pa{\rh'(t)\int_{x-\ep t}^x \rc{\ep}f(s)\,ds}\,dt
\end{align*}

Now we prove it for general $f\in L^1$. By linearity of integrals, and by splitting $f$ into a positive and negative part, it suffices to consider nonnegative $f$. 
The set of continuous functions with compact support is dense in $L^1(\mu;\R)$ by 3.2.15. 
Thus given any $f$, there exist a sequence of such functions $f_n$ converging to $f$ in $L^1$. 
By replacing $f_n$ with $\max(\min(f,f_n),0)$ as necessary we may assume $0\le f_n\le f$. (This clearly preserves continuity, and $|f-f_n|$ does not increase after this replacement, so we still have $f_n\to f$.) 
By the above we have
\begin{equation}\label{p4-1-0}
\int_{[0,\iy)} \rh(-t)f_n(x+\ep t) \,\la_{\R}(dt) =\int_{(0,\iy)} \rh'(t)\pa{\rc{\ep}\int_{[x,x+\ep t)}f_n(s)\,\la_{\R}(ds)}\,\la_{\R}(dt).
\end{equation}
%Now $f_n(x\pm \ep t)$ converges to $f(x\pm \ep t)$ in $L^1$ as well (just change variables). Since $\rh$ is bounded, the LHS converges to the corresponding integral for $f$. 

Consider the LHS. Since $\rh$ is bounded,
\[
|\rh(-t)f_n(x+\ep t)|\le (\max |\rho|)f_n(x+\ep t).
\]
The right hand side is integrable over $t\in [0,\iy)$ (since $\max |\rho|$ is constant, $f_n$ is integrable, and $f_n(x+\ep t)$ is obtained from $f_n$ from substitution). Hence by Lebesgue dominated convergence it converges to the corresponding integral for $f$.

Consider the RHS. Now
\begin{align*}
\ab{\rh'(t)\pa{\rc{\ep}\int_{[x,x+\ep t)}f_n(s)\,ds}}
&\le |\rh'(t)|\pa{\rc{\ep}\int_{[x,x+\ep t)}f(s)\,ds}\\
&\le |\rh'(t)|\rc{\ep}\int_{[0,\iy)}f(s)\,ds
\end{align*}
which is integrable because $|\rh'(t)|$ is integrable. Hence by Lebesgue dominated convergence the RHS of~\ref{p4-1-0} converges to the corresponding integral for $f$, and equality in~\ref{p4-1-0} holds for $f_n$ replaced by $f$.\\

\subprob{(iii)}
\begin{align}
\nonumber
\lim_{\ep\searrow 0} f_{\ep}(x) &=\lim_{\ep\searrow 0}
\pa{
\int_{[0,\iy)} \rh(-t) f(x+\ep t)\,\la_{\R}(dt)+\int_{[0,\iy)} \rh(t)f(x-\ep t)\,\la_{\R}(dt)
}\\
\nonumber
&=\lim_{\ep\searrow 0}
\pa{
\int_0^{\iy}\rh'(-t)t\pa{\rc{\ep t} \int_{[x,x+\ep t)} f(s)\,\la_{\R}(ds)}\,\la_{\R}(dt)
-
\int_0^{\iy}\rh'(t)t\pa{
\rc{\ep t}\int_{(x-\ep t,x]} f(s)\,\la_{\R}(ds)
}\,\la_{\R}(dt)
}\\
\label{p4-1-1}
&=\int_0^{\iy} \rh'(-t)tf(x)\,\la_{\R}(dt)-\int_0^{\iy} t\rh'(t) f(x)\,\la_{\R}(dt)\\
\nonumber
&=-\int_{-\iy}^0 \rh'(t)t f(x)\,\la_{\R}(dt)-\int_0^{\iy} t\rh'(t) f(x)\,\la_{\R}(dt)\\
\nonumber
&=-f(x)\int_{\R} t\rh'(t)\,\la_{\R}(dt).
\end{align}
In~(\ref{p4-1-1}) we took the limit inside the integrand and used (i). We justify this as follows: %on the set $[0,T]$,
%\[
%\rc{\ep t} \int_{[t,x+\ep t)} f(s)\,ds\to f(x)
%\]
%uniformly as $\ep\to 0$, simply from (i).
Given $T,\ep'>0$ choose $\ep_0$ so that $\ab{\rc{\ep t} \int_{[x,x+\ep t)} f(s)\,ds- f(x)}\le \ep'$ for $0\le t\le T$, $0<\ep<\ep_0$. (By (i) the limit as $\ep t\to 0$ is $0$.) 
%On $[T,\iy)$, it is bounded by $\frac LT$ where $L=\rc{\ep}\int_{\R}|f|\,dx+f(x)$. Hence
Note furthermore, since $\rc{d} \int_{[x,x+d)} f(s)\,ds$ is continuous, has limit $f(x)$ as $d\to 0$ and as $d\to\iy$, it is bounded above, and therefore $\ab{\rc{\ep t} \int_{[x,x+\ep t)} f(s)\,ds- f(x)}$ is bounded for all $t$, say by $M$. Then
\begin{align*}
\int_0^{\iy} \ab{\rh'(-t)t\pa{\rc{\ep t}\int_{[x,x+\ep t)} f(s)\,ds-f(x)}}\,dt
&\le \int_0^T|t\rh'(-t)|\ep'\,dt+M\int_T^{\iy} |t\rh'(-t)|\,dt\\
&\le \ep'\int_0^{\iy}|t\rh'(-t)|\,dt+M\int_T^{\iy} |t\rh'(-t)|\,dt
\end{align*}
Choosing $T$ large and $\ep'\to 0$ this goes to 0, since $|t\rh'(t)|\in L^1$. Similarly for the other integral.

%\begin{lem}
%Let $f_n(x)$ be a sequence of functions converging to $f$ such that the convergence is uniform on each $[0,L]$. Then $\int_0^{\iy} f_n\,dx=\int_0^{\iy}$
%\end{lem}
%By integration by parts,
Now
\begin{align*}
1=\int \rh\,d\la_{\R}&=\int_{-\iy}^{\iy} \rh\,dx&\text{by 3.2.18}\\
&=t\rh(t)|^{\iy}_{-\iy}-\int_{-\iy}^{\iy}t\rh'(t)\,dt\\
&=-\int_{-\iy}^{\iy}t\rh'(t)\,dt.
\end{align*}
Hence $\lim_{\ep\searrow 0}f_{\ep}(x)=f(x)$.

\end{problem}
\begin{problem}{\it (3.3.18, Continuous and singular functions)}
\subprob{(i)}
By Exercise 1.2.25(iii),
\[
\Arc(F;(c,d])=\lim_{\Vert C\Vert \searrow 0} \sum_{I\in \cal C} \sqrt{\vol(I)^2+(\De_IF)^2}.
\]
The covers
\[
\cal C_n=\bc{\ba{0,\rc{2^n}},\ba{\rc{2^n},\frac{2}{2^n}},\ldots,
\ba{\frac{2^n-1}{2^n},1}
}
\]
have mesh tending to 0, so
\[
\Arc(F;(c,d])=\lim_{n\to \iy}\sum_{I\in \cal C_n} \sqrt{\vol(I)^2+(\De_IF)^2}
=\lim_{n\to\iy}L_n.
\]

\subprob{(ii)}
\begin{align*}
2^{-n}\sum_{k=0}^{2^n-1} \ba{
(1+f_n(k2^{-n}))-\sqrt{1+f_n(k2^{-n})^2}
}
&=\sum_{k=0}^{2^n-1}[2^{-n} +(F((k+1)2^{-n})-F(k2^{-n}))]\\
&\quad -\sum_{k=0}^{2^n-1} \sqrt{4^{-n}+(F((k+1)2^{-n})-F(k2^{-n}))^2}\\
&=1+F(1)-F(0)-L_n=2-L_n.
\end{align*}

\subprob{(iii)}
For $a\ge 0$,
\begin{align*}
&\frac{a}{1+a}\le (1+a)-\sqrt{1+a^2}\le \frac{2a}{1+a}\\
\iff &(1+a)-\frac{2a}{1+a}\le \sqrt{1+a^2}\le (1+a)-\frac{a}{1+a}\\
\iff &\frac{a^2+1}{a+1}\le \sqrt{1+a^2}\le \frac{a^2+a+1}{a+1}\\
\iff &(a^2+1)^2\le (a+1)^2(a^2+1)\le (a^2+a+1)^2\\
\iff &a^4+2a^2+1\le a^4+2a^3+2a^2+2a+1\le a^4+2a^3+3a^2+2a+1,
\end{align*}
the last statement clearly true.
Hence
\[
\int_{[0,1)} \frac{1+f_n(x)-\sqrt{1-f_n(x)^2}}{2}\la_{\R}(dx)
\le \int_{[0,1)} \frac{f_n(x)}{1+f_n(x)} \la_{\R}(dx) \le \int_{[0,1)} 1+f_n(x)-\sqrt{1-f_n(x)^2}\,\la_{\R}(dx).
\]
However, since $f_n(x)$ is constant on the intervals $[k2^{-n},(k+1)2^{-n})$, the integral on the RHS equals the sum in (ii), which equals $2-L_n$. Then the LHS equals $\frac{2-L_n}{2}$, as needed.\\

\subprob{(iv)}
By (i), $\lim_{n\to \iy} L_n=\Arc(F;[0,1])$.
Note
\[
F'(x)=\lim_{I\searrow \{x\}} \frac{F(I^+)-F(I^-)}{I^+-I^-}
\]
since the expression is between $\frac{F(I^+)-F(x)}{I^+-x}$ and $\frac{F(I^-)-F(x)}{I^--x}$ which both approach 0 as $I^+,I^-\to x$. But $f_n(x)$ is just this difference quotient for $I$ an interval in the form $[k2^{-n},(k+1)2^{-n}]$ containing $x$; the lengths of these intervals go to 0 as $n\to \iy$. Hence $\lim_{n\to \iy} f_n(x)=F'(x)$ provided $F'(x)$ exists. 

Take the limit of the expressions in (iii) as $n\to \iy$ to get
\begin{equation}\label{p4-2-1}
\frac{2-\Arc(F;[0,1])}{2}\le \int_{[0,1]}\frac{F'(x)}{1+F'(x)}\la_{\R}(dx)\le 2-\Arc(F;[0,1]).
\end{equation}
Note by Lebesgue dominated convergence we can take the limit inside the integrand in the middle term because $f_n(x)\in [0,1]$ for any $x,n$ and therefore $\frac{f_n(x)}{1+f_n(x)}\in [0,\rc 2]$; the $f_n$ are bounded by an integrable function, namely $\rc2$ on $[0,1]$. Moreover the integral is well-defined since $F'(x)$ exists almost everywhere by Lebesgue's differentiation theorem.

By~(\ref{p4-2-1}), $\Arc(F;[0,1])=2$ iff $\int_{[0,1]}\frac{F'(x)}{1+F'(x)}\la_{\R}(dx)=0$, iff $\frac{F'(x)}{1+F'(x)}=0$ almost everywhere on $[0,1]$, iff $F'(x)=0$ almost everywhere, iff $\int_{[0,1]}F'(x)\,\la_{\R}(dx)=0$, iff $F$ is singular (the last result being part of Lebesgue's differentiation theorem, 3.3.15).
\end{problem}
\begin{problem}{\it (4.1.9, Slicing a $\mu_1\times \mu_2$-measurable set)}
By symmetry it suffices to show each assertion for once choice of $(i,j)$.
\begin{enumerate}
\item $\Ga_{(1)}(x_2)\in \cal B_1$: Since $1_{\Ga}(x_1,x_2)$ is a measurable function on $\cal B_1\times \cal B_2$, by Lemma 4.1.2, $f(x)=1_{\Ga}(x,x_2)$ is measurable. Hence $f^{-1}(\{1\})=\Ga_{(1)}(x_2)\in \cal B_1$.
\item $g(x_1):=\mu_2(\Ga_{(2)}(x_1))$ measurable: The measure of a set is the integral of its characteristic function, so
\[
g(x)=\int_{E_2} 1_{\Ga_{(2)}(x)}(x_2)\,\mu_2(dx_2)=\int_{E_2}1_{\Ga}(x,x_2)\,dx.\]
This is measurable by Lemma 4.1.2.
\item $(\mu_1\times \mu_2)(\Ga)=0\iff \mu_2(\Ga_{(2)}(x_1))=0$ for $\mu_1$-almost every $x_1\in E_1$: By Lemma 4.1.3,
\begin{align*}
(\mu_1\times\mu_2)(\Ga)&=
\int_{E_1\times E_2} 1_{\Ga}(x_1,x_2)\,(\mu_1\times \mu_2)(dx_1\times dx_2)\\
&=\int_{E_1}\pa{\int_{E_2} 1_{\Ga}(x_1,x_2)\,\mu_2(dx_2)}\,\mu_1(dx_1)\\
&=\int_{E_1}\mu_2(\Ga_{(2)}(x_1))\,\mu_1(dx_1).
\end{align*}
The latter integral is 0 iff the integrand is 0 for $\mu_1$-almost every $x_1\in E_1$, as needed.

By symmetry, $(\mu_1\times \mu_2)(\Ga)=0\iff \mu_1(\Ga_{(1)}(x_2))=0$ for $\mu_2$-almost every $x_2\in E_2$. Hence $\mu_2(\Ga_{(2)}(x_1))=0$ for $\mu_1$-almost every $x_1\in E_1$, iff $\mu_1(\Ga_{(1)}(x_2))=0$ for $\mu_2$-almost every $x_2\in E_2$.
\end{enumerate}
\end{problem}
\begin{problem} {\it (4.1.10, Failure of product measure)}
$\mu_1(\Ga)=|\Ga|$ is a measure because $|\phi|=0$ and for disjoint sets,
\[
\ab{\bigcup_{n=1}^{\iy} \Ga_n}=\sum_{n=1}^{\iy} |\Ga_n|.
\]

Let $\Ga=\{(x,x):x\in (0,1)\}$. Note that $\Ga\in \cal B_1\times \cal B_2$ because 
\[
\Ga=\bigcap_{n=1}^{\iy} S_n
\]
where
\[
\left(0,\rc n\right]^2\cup \ba{\rc n,\frac 2n}^2\cup \cdots \cup \left[\frac{n-1}{n},1\right)^2.
\]
(Indeed, $S_n$ contains $\Ga$ and only contains elements with horizontal and vertical distance at most $\rc n$ from $\Ga$.) Given $x_1$ or $x_2$, there is exactly one element $x_2$, $x_1$, respectively, such that $(x_1,x_2)\in \Ga$. In $E_2$, a 1-point set has measure 0, while in $E_1$ a 1-point set has measure 1. Hence
\begin{align*}
\int_{E_2} 1_{\Ga}(x_1,x_2)\,\mu_2(dx_2)&=0&\text{for every }x_1\in E_1,\\
\int_{E_1} 1_{\Ga}(x_1,x_2)\,\mu_1(dx_1)&=1&\text{for every }x_2\in E_2.
\end{align*}
\end{problem}
\begin{problem}{\it (4.2.8, Hyperplane has Hausdorff measure 0)}
\begin{lem}
Let $T$ be an isometry. Then $H^N(\Ga)=H^N(T\Ga)$.
\end{lem}
\begin{proof}
Any cover $\cal C$ of $\Ga$ gives a covering for $T\Ga$, namely $T\cal C=\{TS:S\in \cal C\}$. Since isometries preserve distance, the radii of the sets are preserved. Hence $\sum_{C\in \cal C}\Om_N \rad(C)^N=\sum_{C\in T\cal C} \Om_N\rad(C)^N$. Conversely, by working with $T^{-1}$ any cover $\cal C$ of $T\Ga$ gives a covering of $\Ga$ with same value of the sum above.
\end{proof}
Given a hyperplane, there is isometry sending it to $Q=\{(x_1,\ldots, x_{n-1},0)\}$. Thus it suffices to show $H^N(Q)=0$.

As shown in the text, $H^N$ is countably subadditive. Thus
\[
H^N(Q)\leq \sum_{a_1,\ldots, a_{n-1}\in \Z} H^N([a_1,a_1+1]\times \cdots \times [a_{n-1},a_{n-1}+1]\times \{0\}).
\]
Since each of the sets $[a_1,a_1+1]\times \cdots \times [a_{n-1},a_{n-1}+1]\times \{0\}$ can be sent to $S=[0,1]^{n-1}\times\{0\}$ by a translation it suffices to show that $S$ has Hausdorff measure 0. Let
\[
\cal C_d=\bc{\ba{\frac{k_1}{d},\frac{k_1+1}{d}}\times \cdots \times \ba{\frac{k_{n-1}}{d},\frac{k_{n-1}+1}{d}}\times\{0\}:\text{integers }k_1,\ldots, k_{n-1}\in [0,d-1]}
\]
Now the radius of each set in $C_d$ is $\frac{\sqrt{n-1}}{d}$, so
\[
\sum_{C\in \cal C} \Om_N\rad(C)^N=d^{N-1}\cdot \Om_N\cdot \pf{\sqrt{n-1}}{d}^N=\Om_N\frac{\sqrt{n-1}^N}{d}
\]
and $H^{N,\rc d}(\Ga)\leq \Om_N\frac{\sqrt{n-1}^N}{d}\to 0$ as $d\to \iy$.  Hence $H^N(\Ga)=0$.
\end{problem}
\begin{problem}{\it (5.1.11, Integrability of $|f|^p$)}
By breaking $(0,\infty)$ into the intervals $[\rc{n+1},\rc{n}]$ and $[n,n+1]$ for $n\in \N$ and considering the maximum and minimum of $t^{p-1}\mu(|f|>t)$ on each interval, we get the following bound for $\int_{(0,\iy)}t^{p-1} \mu(|f|>t)\,\la_{\R}(dt)$:
\begin{align*}
& \sum_{n=1}^{\iy}\pa{\rc n-\rc{n+1}}\min\pa{\rc{n^{p-1}},\rc{(n+1)^{p-1}}}\mu(|f|>\rc{n})
+\sum_{n=1}^{\iy} \min(n^{p-1},(n+1)^{p-1})\mu(|f|>n+1)\\
&\leq \sum_{n=1}^{\iy}\int_{[\rc{n+1},\rc{n}]}t^{p-1} \mu(|f|>t)\,\la_{\R}(dt)+\sum_{n=1}^{\iy}\int_{[n,n+1]}t^{p-1} \mu(|f|>t)\,\la_{\R}(dt)\\
&\le \sum_{n=1}^{\iy}\pa{\rc n-\rc{n+1}}\max\pa{\rc{n^{p-1}},\rc{(n+1)^{p-1}}}\mu(|f|>\rc{n+1})
+\sum_{n=1}^{\iy} \max(n^{p-1},(n+1)^{p-1})\mu(|f|>n).
\end{align*}
The sums on the LHS and RHS can be written as
\begin{gather}
\sum_{n=1}^{\iy}\pa{\rc{n(n+1)}}\min\pa{\rc{n^{p-1}},\rc{(n+1)^{p-1}}}\mu(|f|>\rc{n})
+\sum_{n=2}^{\iy} \min((n-1)^{p-1},n^{p-1})\mu(|f|>n)
\label{p4-6-1}\\
\sum_{n=2}^{\iy}\pa{\rc{n(n-1)}}\min\pa{\rc{(n-1)^{p-1}},\rc{n^{p-1}}}\mu(|f|>\rc{n})
+\sum_{n=1}^{\iy} \min(n^{p-1},(n+1)^{p-1})\mu(|f|>n)
\label{p4-6-2}
\end{gather}
By the limit comparison test~(\ref{p4-6-1}) converges iff~(\ref{p4-6-2}) converges iff
\begin{equation}\label{p4-6-3}
\sum_{n=1}^{\iy} \rc{n^{p+1}} \mu\pa{|f|>\rc{n}} +\sum_{n=1}^{\iy} n^{p-1}\mu(|f|>n)
\end{equation}
converges. Hence the integral is finite iff~(\ref{p4-6-3}) converges.
\end{problem}
\begin{problem}{\it (5.1.13)}
\subprob{(i)}
\begin{align}
\nonumber
\pa{\int_{\R} e^{-\frac{|x|^2}{2}}\,dx}^2
&=\pa{\int_{\R} e^{-\frac{|x|^2}{2}}\,dx}
\pa{\int_{\R} e^{-\frac{|y|^2}{2}}\,dy}\\
\label{p4-7-1}
&=\int_{\R}\int_{\R}e^{-\frac{|x|^2+|y|^2}{2}}\,dxdy\\
\label{p4-7-2}
&=\int_{\R^2} e^{-\frac{|x|^2}{2}}\,dx\\
\label{p4-7-3}
&=\int_{\bS^1}\int_{(0,\iy)} re^{-\frac{|r\om|^2}{2}}\,dr\,\la_{\bS}(d\om)\\
\nonumber
&=\int_{\bS^1}  \int_{(0,\iy)} re^{-\frac{r^2}{2}}\,dr \,\la_{\bS}(d\om)\\
\label{p4-7-4}
&=2\pi\int_0^{\iy} re^{-\frac{r^2}{2}}\,dr\\
\nonumber
&=2\pi[e^{-\frac{r^2}{2}}]^{\iy}_0=2\pi.
\end{align}
Note~(\ref{p4-7-1}) equals~(\ref{p4-7-2}) by Fubini's Theorem (and one is finite iff the other is). The next step~(\ref{p4-7-3}) comes from  Theorem 5.1.8. In~(\ref{p4-7-4}) we use $\la_{\bS^1}(\bS^1)=2\la_{\R^2}(B_{\R^2}(0,1)\bs \{0\})=2\pi$.

Since $A$ is symmetric, we can write $A=PDP^{-1}$, where $P$ is orthogonal and $D$ is diagonal with entries $\la_1,\ldots, \la_n>0$. Then by Change of Variables (5.1.1) and (2.2.15) with $\det(P)=1$,
\begin{align*}
\int_{\R^N}e^{-\rc{2}\an{x,A^{-1}x}}\,dx
&=\int_{\R^N}e^{-\rc{2}\an{P^{-1}x,D^{-1}P^{-1}x}}\,dx\\
&=\int_{\R^N}e^{-\rc{2}\an{x,D^{-1}x}}\,[(P^{-1})_*\la_{\R}](dx)\\
&=\int_{\R^N}e^{-\rc{2}\an{x,D^{-1}x}}\,\la_{\R}(dx)\\
&=\int_{\R^N}e^{-\rc{2}(\la_1^{-1}x_1^2+\cdots+\la_n^{-1}x_n)}\,dx\\
&=\prod_{i=1}^N\int_{\R} e^{-\rc2\la_i^{-1}x_i^2}\,dx_i&\text{by Fubini}\\
&=\prod_{i=1}^N\int_{\R}e^{-\rc2x_i^2} \sqrt{\la_i}\,dx_i&\text{change of variables (5.1.1), }x_i\mapsfrom x_i\sqrt{\la_i}\\
&=\prod_{i=1}^N \sqrt{\la_i}\sqrt{2\pi}=\det(A)^{\rc2}(2\pi)^{\frac N2}.
\end{align*}

\subprob{(ii)}
Using integration by parts,
\begin{align*}
\int_{(0,\iy)} t^{\ga-1}e^{-t}\,dt
&=\left.e^{-t}\frac{t^{\ga}}{\ga}\right|^{\infty}_0-\int_0^{\iy} \frac{t^{\ga}}{\ga}(-e^{-t})\,dt\\
&=\rc{\ga}\Ga(\ga+1)
\end{align*}
so $\Ga(\ga+1)=\ga\Ga(\ga)$. Using $\Ga(1)=\int_0^{\iy} e^{-t}\,dt=1$, by induction, $\Ga(n+1)=n!$.

Using the substitution $t=\frac{u^2}{2}$,
\begin{align*}
\Ga\pa{\rc 2}&=\int_0^{\iy} t^{-\rc 2}e^{-t}\,dt\\
&=\int_0^{\iy}\sqrt 2 u^{-1} e^{-\frac{u^2}{2}}u\,du\\
&=\frac{\sqrt{2}}{2}\int_{\R} e^{-\frac{u^2}{2}}\,du\\
&=\frac{\sqrt{2}}{2}\sqrt{2\pi}=\sqrt{\pi}.
\end{align*}

\subprob{(iii)}
We induct on $N$. For $N=1$ we have $\Om_1=\la_{\R}([-1,1])=2=\frac{\pi^{\rc2}}{\Ga(3/2)}$; for $N=2$, $\Om_2=\pi=\frac{\pi}{\Ga(2)}$. For the induction step $N\to N+2$:
\begin{align*}
\Om_{N+2}&=\la_{\R^{N+2}}(B_{\R^{N+2}}(0,1)\bs \{0\})\\
&=\int_{B_{\R^{N+2}}(0,1)}\,d\la_{\R^{N+2}}\\
&=\int_{x_{N+1}^2+x_{N+2}^2\le 1}\int_{x_1^2+\cdots +x_N^2\leq 1-x_{N+1}^2-x_{N+2}^2} \,d\la_{\R^{N}}\,d\la_{\R^2}\\
&=\int_{x_{N+1}^2+x_{N+2}^2\leq 1} \la_{\R^N}\pa{B_N\pa{0,\sqrt{1-x_{N+1}^2-x_{N+2}^2}}}\,d\la_{\R^2}\\
&=\int_{x_1^2+x_2^2\le 1}\Om_N \sqrt{1-x_1^2-x_2^2}^{N}\,d\la_{\R^2}&\text{by 5.1.1, 2.2.15}\\
&=\Om_N\int_{\bS_1}\int_0^1 r(1-r^2)^{\frac N2}\,dr\la_{\bS^1}(d\om)\\
&=2\pi \Om_N\ba{
\frac{-(1-r^2)^{\frac N2+1}}{N+2}
}^1_0\\
&=\Om_N\cdot \frac{\pi}{(N+2)/2}\\
&=\frac{\pi^{\frac N2}}{\Ga\pa{\frac N2+1}}\cdot \frac{\pi}{\frac N2+1}\\
&=\frac{\pi^{\frac {N+2}2}}{\Ga\pa{\frac {N+2}2+1}}
\end{align*}
%where in the last step we use $\frac{N+1}{2}\Ga\pf{N+1}{2}=\Ga\pf{N+3}{2}$. 
Then
\[
\om_{N-1}=\la_{\bS^{N-1}}(\bS^{N-1})=N\la_{\R^N}(B_N(0,1)\bs\{0\})=N\Om_N=2\frac{\pi^{\frac{N}2}}{\frac{\Ga(N/2+1)}{N/2}}=\frac{2\pi^{\frac N2}}{\Ga\pf{N}{2}}.
\]

\subprob{(iv)}
Let 
\begin{align*}
A&=\int_{(0,\iy)} t^{-\frac 12} e^{-\al^2 t-\frac{\be^2}{t}}\,dt\\
B&=\int_{(0,\iy)} t^{-\frac 32} e^{-\al^2 t-\frac{\be^2}{t}}\,dt.
\end{align*}

%By 5.1.10 (change of variables), with $F(t)=\al t^{-\rc 2} +\be t^{-\frac 32}$, 
Making the substitution $s=\al t^{\rc 2} -\be t^{-\rc2}$ and 
noting that as $t$ ranges from 0 to $\iy$, $s$ ranges from $-\iy$ to $\iy$,
\begin{align}
\nonumber
\frac{e^{-2\al\be}}{\al}\int_{-\iy}^{\iy} e^{-s^2}\,ds
&=\frac{e^{-2\al\be}}{\al}\int_{0}^{\iy}e^{-\al^2 t+2\al\be -\be^2t^{-1}}\pa{\rc2\al t^{-\rc 2}+\rc 2\be t^{-\frac 32}}\,dt\\
\nonumber
&=\rc{\al}\int_0^{\iy} t^{-\rc 2} e^{-\al^2 t-\frac{\be^2}{t}}\pa{\rc 2 \al+\rc 2\be t^{-1}
}\,dt\\
\label{p4-7-5}
\implies \frac{e^{-2\al\be}}{\al}(2\pi)^{\rc2}&=\rc{2}A+\frac{\be}{2\al}B
\end{align}
Now making the substitution $t\mapsfrom \frac{\be^2}{\al^2u}$ gives
\begin{align*}
A&=\int_{\iy}^0 \frac{\al}{\be}u^{\rc2} e^{-\al^2u-\frac{\be^2}u}\frac{\be^2}{\al^2}\cdot -\rc{u^2}du\\
&=\frac{\be}{\al}\int_{(0,\iy)} u^{-\frac 32}e^{-\al^2u-\frac{\be^2}u}\,du\\
&=\frac{\be}{\al}B.
\end{align*}
Substituting this into~(\ref{p4-7-5}) gives $A=\frac{(2\pi)^{\rc2}e^{-2\al\be}}{\al}$. Then $B=\frac{\al}{\be}A=\frac{(2\pi)^{\rc2}e^{-2\al\be}}{\be}$.

\end{problem}
\begin{problem}{\it (5.1.14)}
\subprob{(i)}
\begin{enumerate}
\item
If $\Ga$ is nonempty open in $\bS^{N-1}$, %then $\Ga=\Ga'\cap \bS^{N-1}$ for some nonempty open $\Ga'\subeq \R^N$. 
then considering $\Phi$ as a function $B^N(0,1)\bs\{0\}\to \bS^{N-1}$, $\Phi^{-1}(\Ga)$ must be a nonempty open set in $B^N(0,1)\bs\{0\}$ and hence in $\R$ (since the former is open). Hence
\[
\la_{\bS^{N-1}}(\Ga)=N\la_{B^N(0,1)\bs\{0\}}(\Phi^{-1}(\Ga))=N\la_{\R^N}(\Phi^{-1}(\Ga))>0
\]
since nonempty open sets in $\R^N$ has positive measure (every open set in $\R^N$ contains a rectangle).
\item
Note 
\[
\Phi(T_{\sO}(x))=\frac{T_{\sO}x}{|T_{\sO}x|}=\frac{T_{\sO}x}{|x|}=T_{\sO}\pf{x}{|x|}=T_{\sO}(\Phi(x)).
\]
Now
\begin{align*}
[(T_{\sO})_*\la_{\bS^{N-1}}](\Ga)
&=\la_{\bS^{N-1}}(T_{\sO}^{-1}\Ga)\\
&=N\la_{B^N(0,1)\bs\{0\}} (\Phi^{-1}T_{\sO}^{-1}\Ga)\\
&=N\la_{B^N(0,1)\bs\{0\}} (T_{\sO}^{-1}\Phi^{-1}\Ga)&\text{ since }T_{\sO},\Phi\text{ commute}\\
&=N\la_{B^N(0,1)\bs\{0\}}(\Phi^{-1}\Ga)&\la_{\R^N} \text{ invariant under }T_{\sO}\\
&=\la_{\bS^{N-1}}(\Ga)
\end{align*}
so $\la_{\bS^{N-1}}$ is orthogonal invariant.
\item Let $T$ be an orthogonal linear transformation taking $\xi$ to $-\xi$ and acting as the identity on $\xi^{\perp}$ (i.e. a reflection).
\begin{align*}
\int_{\bS^{N-1}} \an{\xi,\om}\la_{\bS^{N-1}}(d\om)
&=\int_{\bS^{N-1}} \an{\xi,\om} [T_* \la_{\bS^{N-1}}](d\om)
&\text{orthogonal invariance}\\
&=\int_{\bS^{N-1}}\an{\xi, T\om} \,\la_{\bS^{N-1}}(d\om)
&\text{by 5.1.1 (COV)}\\
&=\int_{\bS^{N-1}}\an{T^{-1}\xi,\om}\,\la_{\bS^{N-1}}(d\om)\\
&=\int_{\bS^{N-1}} \an{-\xi,\om}\,\la_{\bS^{N-1}}(d\om).\\
&=-\int_{\bS^{N-1}} \an{\xi,\om}\,\la_{\bS^{N-1}}(d\om).
\end{align*}
Thus the integral equals 0.
\item By linearity of the integral, $f(\xi,\eta):=\int_{\bS^{N-1}} \an{\xi,\om}\an{\eta,\om}\,\la_{\bS^{N-1}}(d\om)$ is a bilinear form in $\xi,\eta$. Since $\Om_N\an{\xi,\eta}$ is bilinear form, to show these are equal it suffices to show they agree on an orthonormal basis. Thus it suffices to show that $f(\xi,\eta)=0$ if $\xi\perp \eta$ and
$f(\xi,\eta)=\Om_N$ if $\xi=\eta$. For the first part, let $T$ be the same transformation as above.
\begin{align*}
\int_{\bS^{N-1}} \an{\xi, \om}\an{\eta, \om} \,\la_{\bS^{N-1}}(d\om)
&=\int_{\bS^{N-1}} \an{\xi, \om}\an{\eta, \om} [T_* \la_{\bS^{N-1}}](d\om)\\
&=\int_{\bS^{N-1}}\an{\xi, T\om}\an{\eta, T\om} \,\la_{\bS^{N-1}}(d\om)\\
&=\int_{\bS^{N-1}}\an{T^{-1}\xi, \om}\an{T^{-1}\eta, \om} \,\la_{\bS^{N-1}}(d\om)\\
&=\int_{\bS^{N-1}}\an{-\xi, \om}\an{\eta, \om} \,\la_{\bS^{N-1}}(d\om)\\
&=-\int_{\bS^{N-1}} \an{\xi, \om}\an{\eta, \om} \,\la_{\bS^{N-1}}(d\om).
\end{align*}
Thus the integral equals 0. 
For the second part, first note $\la_{\bS^{n-1}}$ is orthogonally invariant, so we can replace $\om$ with $T_{\sO}\om$. Choosing $\sO$ sending $(1,0,\ldots, 0)$ to $\xi$, and using $\an{\xi, T_{\sO}\om}=\an{T_{\sO}^{-1}\xi,\om}$, we may assume $\xi=(1,0,\ldots, 0)$.  Then
\begin{align*}
\int_{\bS^{N-1}} \an{\xi,\om}^2 \,\la_{\bS^{N-1}}(d\om)
&=\int_{\bS^{N-1}} \an{\xi,\om}^2 \,(\Phi_* N\la_{B(0,1)\bs\{0\}})(d\om)\\
&=N\int_{B(0,1)\bs \{0\}} \an{\xi, \Phi(x)}^2 \,\la_{\R^N}(dx)\\
&=N\int_{B(0,1)\bs \{0\}} \rc{|x|}\an{\xi, x}^2 \,\la_{\R^N}(dx)\\
&=N\int_{B(0,1)\bs\{0\}}\frac{x_1^2}{x_1^2+\cdots +x_n^2}\,\la_{\R^N}(dx).
\end{align*}
But by symmetry, $\int_{B(0,1)\bs\{0\}}\frac{x_i^2}{x_1^2+\cdots +x_n^2}\,\la_{\R^N}(dx)$ are equal for all $i$. Summing over $1\le i\le n$ gives $\int_{B(0,1)\bs\{0\}}\,\la_{\R^N}(dx)=\Om_N$. Hence each is equal to $\rc{N}\Om_N$ and the integral above equals $\Om_N$, as needed.
\end{enumerate}
\subprob{(ii)}
Define $\sO_{\theta}=\smatt{\cos\theta}{-\sin\theta}{\sin\theta}{\cos\theta}$. 
%Now
%\begin{align*} 
%\int_{\bS^1} f\,d\nu&=\rc{2\pi} \int_{[0,2\pi]} \int_{\bS^1} f(\om)\,\nu(d\om)\,d\te\\
%&=\rc{2\pi} \int_{[0,2\pi]} \int_{\bS^1} f(\om)\,\nu(d\om)\,d\te\\
%&=\rc{2\pi}\int_{[0,2\pi]} \int_{\bS^1} f(\om)\,T_{\sO_{\te}}^*\nu(d\om)\,d\te&\text{orthogonal invariance}\\
%&=\rc{2\pi} \int_{[0,2\pi]} \pa{
%\int_{\bS^1} f\circ T_{\sO_{\te}}(\om)\,\nu(d\om)}\,d\te&\text{ by 5.1.1.}
%\end{align*}
%Applying this with $f=1_{\Ga}$ gives
%\begin{align*}
%\nu(\Ga)&=\rc{2\pi} \int_{[0,2\pi]} \int_{\bS^1} 1_{\Ga} (T_{\sO_{\te}}(\om))\,\nu(d\om)\,d\te\\
%&=\rc{2\pi}  \int_{\bS^1} \int_{[0,2\pi]} 1_{\Ga} (T_{\sO_{\te}}(\om))\,d\te\,\nu(d\om).
%%&=\rc{2\pi}\int_{[0,2\pi]} \int_{\bS^1}1_{T_{\sO_{\te}}^{-1}}(\om)\,\nu(d\om)\,d\te\\
%%&=\rc{2\pi}\nu(T_{\sO_{\te}}^{-1}(\Ga))\,d\te)\\
%%&=\rc{2\pi}\int_{[0,2\pi)} \nu(\Ga)\,d\te&(\nu \text{ rotationally invariant})\\
%%&=\rc{2\pi}\nu(\Ga).
%%&=\rc{2\pi}\int_{[0,2\pi]} \int_{\bS^1} 1_{\Ga} (T_{\sO_{\te}}(\om))\,\nu(d\om)\,d\te\\
%\end{align*}

We have
\begin{align}
\nonumber
\nu(\Ga)&=\rc{2\pi}\int_{[0,2\pi)} \nu(\Ga)\,d\te\\
\nonumber
&=\rc{2\pi}\int_{[0,2\pi)} \nu(T_{\sO_{\te}}^{-1}(\Ga))\,d\te&(\nu \text{ rotationally invariant})\\
\nonumber
&=
\rc{2\pi}\int_{[0,2\pi]} \int_{\bS^1}1_{T_{\sO_{\te}}(\Ga)^{-1}}(\om)\,\nu(d\om)\,d\te\\
%\nonumber
%&=\rc{2\pi}  \int_{\bS^1} \int_{[0,2\pi]} 1_{\Ga} (T_{\sO_{\te}}(\om))\,d\te\,\nu(d\om)\\
%\nonumber
%&=
%\rc{2\pi} \int_{[0,2\pi]} \int_{\bS^1} 1_{\Ga} (T_{\sO_{\te}}(\om))\,\nu(d\om)\,d\te\\
\label{p4-9-1}
&=\rc{2\pi}  \int_{\bS^1} \int_{[0,2\pi]} 1_{\Ga} (T_{\sO_{\te}}(\om))\,d\te\,\nu(d\om).
%&=\rc{2\pi}\int_{[0,2\pi]} \int_{\bS^1}1_{T_{\sO_{\te}}^{-1}}(\om)\,\nu(d\om)\,d\te\\
%&=\rc{2\pi}\nu(T_{\sO_{\te}}^{-1}(\Ga))\,d\te)\\
%&=\rc{2\pi}\int_{[0,2\pi)} \nu(\Ga)\,d\te&(\nu \text{ rotationally invariant})\\
%&=\rc{2\pi}\nu(\Ga).
\end{align}

Note the inner integral is constant. Indeed, suppose $\om=(\cos\phi,\sin\phi)=\Psi(\phi)$. Since $T_{\sO_{\te}}$ is rotation by $\te$,
\begin{align*}
\int_{[0,2\pi]} 1_{\Ga} (T_{\sO_{\te}}(\om))\,d\te
&=\int_{[0,2\pi]} 1_{\Ga}(\Psi(\te+\phi))\,d\te\\
&=\int_{[0,2\pi]} 1_{\Ga}(\Psi(\te))\,d\te
\end{align*}
since as $\te$ ranges over $[0,2\pi]$, $\te+\phi$ also ranges over $[0,2\pi]$ modulo $2\pi$ (in the positive direction). Putting this in~(\ref{p4-9-1}) gives gives
\begin{align*}
%\rc{2\pi}\int_{\bS^1}\int_{[0,2\pi]} 1_{\Ga}(\Psi(\te))\,d\te\,\nu(d\om)
\rc{2\pi}\int_{\bS^1}\int_{[0,2\pi]} 1_{\Ga}(\Psi(\te))\,\la_{[0,2\pi]}(d\te)\,\nu(d\om)
&=\rc{2\pi}\int_{\bS^1}\int_{[0,2\pi)} %1_{\Psi_*\la_{[0,2\pi]}}(\te)
1_{\Psi^{-1}\Ga}(\te) \,\la_{[0,2\pi]}(d\te)\,\nu(d\om)\\
&=\rc{2\pi}\int_{\bS^1} \la_{[0,2\pi]}(\Psi^{-1}(\Ga))\,\nu(d\om)\\
&=\frac{\nu(\bS^1)}{2\pi} (\Psi_*\la_{[0,2\pi]})(\Ga).
\end{align*}

In particular, since $\la_{\bS^1}$ is rotationally invariant by the previous problem, $\la_{\bS^1}=\frac{\la_{\bS^1}(\bS^1)}{2\pi}\mu=\mu$.


\end{problem}
\end{document}
