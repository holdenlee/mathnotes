\lecture{Fri. 4/1/11}

\subsection{Polar coordinates}
Let $\Phi:\R^N\bs \{0\}\to \bS^{N-1}$ be defined by $\Phi(x)=\frac{x}{|x|}$. Now
\[
\la_{\bS^{N-1}}(\Ga)=N\la_{\R^N}(\{x:\Phi(x)\in \Ga\}).
\]
\begin{thm}
\[
\int \ph \,d\la_{\R} = \int_{(0,\iy)} \rh^{N-1} \pa{\int_{\bS^{N-1}}\ph(\rh \om)\,\la_{\bS^{N-1}}(d\om)}\,d\rh.
\]
\end{thm}
\begin{proof}
Let $f(r)=\int_{B(0,r)} \ph \,dx$. Then
\begin{align*}
f(r+h)-f(r)&=\int_{B(0,r+h)\bs B(0,r)} \ph\, dx\\
&=\int_{B(0,r+h)\bs B(0,r)}\ph(r\Phi(x))\,\la(dx)
+%\underbrace{
\int_{B(0,r+h)\bs B(0,r)}(\ph(x)-\ph(r\Phi(x)))\,\la(dx)%}_{o(h)}.
\\
%not trying to, succeeding.
&=\int_{B(0,r+h)}\ph(r\Ph(x))\,dx-\int_{B(0,r)}\ph(r\Ph(x))\,dx+o(h)\\
&=(r+h)^N\int_{B(0,1)} \ph(r\Ph(y))\,dy-r^N\int_{B(0,1)}\ph(r\Ph(y))\,dy+o(h).
\end{align*}
Therefore 
\[
f'(r)=\lim_{h\searrow 0}\frac{f(r+h)-f(r)}{h}=Nr^{N-1}\int_{\bS^{N-1}}\ph(r\om) \,\la_{\bS^{N-1}}(d\om).
\]
Now integrate from 0 to $\iy$.
\end{proof}

The polar coordinate formula can be viewed as a special case of surface measure.
\subsection{Surface measure}
\begin{df}
A \textbf{hypersurface} is a set $M\subeq \R^N$ such that for every $p\in M$ there exists $r>0$, $F\in C^3(B(p,r); \R)$, so that $|\nabla F|>0$ and 
\[M\cap B(p,r)=\{x\in B(p,r):F(x)=0\}.\]
The tangent space $T_p(M)$  is the set of $v\in \R^N$ such that there exists $\ep>0$ so there exists $\ge\in C^1((-\ep,\ep),M
)$ such that so that $\ga(0)=p$, $\dot{\ga}(0)=v$. 
\end{df}
In the language of manifolds, $M$ is a submanifold of $\R^n$, locally characterized by the fact that the 0 is a regular point of $F$.

\begin{lem}
\begin{enumerate}
\item
$M$ is a countable union of compact sets.
\item $T_p(M)$ is a $N-1$ dimensional subspace of $\R^N$ and $T_p(M)=\{v\in \R^N:v\perp \nabla F(p)\}$.
\end{enumerate}
\end{lem}
\begin{proof}
For the second part, noting the directional derivative is given by dot product with the gradient,
\[
0=\frac{d}{dt}F(\ga(t))|_{t=0}=\an{\nabla F(p),v}.
\]

Conversly, suppose $\an{\nabla F(p),v}=0$. By ODE theory (existence of solutions), we can find $\ga$ locally so that $\dot{\ga}(0)=v$ and
%\[
%\ga'(t)-\frac{\pa{(\nabla F(\ga(t)),y]}}{|\nabla F(\ga(t))|^2}\De F(\ga+x).
%\]
\[
\dot{\ga}(x)=v-\frac{\an{v,\nabla F(x)}}{|\nabla F(\ga(t))|^2}\nabla F(x).
\]
%All we need is a Lipschitz consant.
Now $\frac{d}{dt}F(\ga(t))=\an{\dot{\ga}(t),\nabla F(\ga(t))}=0$.

%%Let $\Ga=\n 

%Let 
%\[
%\Ga(\rh)=\{x:\exists p\Ga \,x-p\in \in T_p(M)\text{ and }|x-\rh|<\rh\}.
%\]
%
%%For example, $\Ga(\rh)
%%Let $\Ga=\{
%$\la_M(\Ga)=\lim_{\rh\searrow 0}\rc{2\rh} \la_{\R^N}\la_{\R^N}(\Ga(\rh))$.
%
%$M=R\times\{0\}$ in $\R^2$; $\Ga=A\times \{0\}$.
%\begin{align*}
%\Ga(\rho)&=A\times (-\rh,\rh)\\
%\frac{\la_{\R^2}^*\Ga(\rh)}{2\rh}&=\la_{\R^1}(
%A).
%\end{align*}
%
%$\Ga\in \cal B_{\bS^{N-1}}$; then $\Ga(p)=\{r\om: 1-\rh<r<1+\rh,\om\in \Ga\}$.
%\[
%\la_{\R^N}(\Ga(\rh))=\int_{1-\rh}^{1+\rh} r^{N-1} \la_{\bS^{N-1}}(\Ga)
%\]
\end{proof}