\lecture{Wed. 4/6/11}

\subsection{Divergence}
\begin{df}
The \textbf{divergence} of $V:\R^N\to \R^N$ is
\[
\text{div}(V)=\sum_{i=1}^N\partial_{x_i}V_i.
\]
\end{df}
\begin{thm}[Divergence theorem]
Let $G$ be a smooth bounded domain in $\R^N$, i.e. it is an open set and its boundary is a hypersurface. Let $V:\R^N\to \R^N$ be a function with two bounded continuous derivatives. Then
\[
\int_{G}\text{div}(V)\,dx=\int_{\partial G} \an{V,n}\,d\la_{M}
\]
where $n$ is the outward pointing normal.
\end{thm}
\begin{proof}
Let $\Phi$ be the flow of $V$, i.e. $\dot{\Phi}(t,x)=V(\Phi(t,x))$, $\Phi(0,x)=x$. The flow property states that $\Phi(s+t,x)=\Phi(t,\Phi(s,x))$. Thus $I=\Phi(t,\Phi(-t,x))$; the $\Phi(t,\cdot)$'s are bijective, in fact diffeomorphisms. Differentiating $\dot{\Phi}(t,x)=V(\Phi(t,x))$ with respect to $x$,
\begin{equation}\label{diffxv}
\partial_t\pd{\Phi}{x}(t,x)=\pd{V}y\Phi(t,x)\pd{\Phi}{x}(t,x).
\end{equation}
We are interested in $\partial_t\det\pa{\Phi}$.

For a matrix $A=((a_{ij}))$, $\partial_{a_{ij}}\det(A)=A^{(ij)}$ where $A^{(ij)}$ is the $(i,j)$th cofactor of $A$. From~(\ref{diffxv}), and $\sum_{j=1}^N a_{kj} A^{(ij)}=\det(A)$, 
%Let $\La((a_{ij}))$; then $\la_{a_{ij}}\det A$ is a cofactor. %Take $i=j$ to get 
%Then
Using the chain rule,
\begin{align*}
\frac{d}{dt}\det\pa{\pd{\Phi(t,x)}{x}}
&=\sum_{1\le i,j,k\le N}\pa{\pd{\Phi(t,x)}{x}}^{(ij)}
\pa{\pd{V}{y}(\Phi(t,x))}_{ik}\pa{\pd{\Phi(t,x)}{x}}_{kj}\\
&=\text{div}(V)(\Phi(t,x))\det\pa{\pd{\Phi(t,x)}{x}}.
%\det(\pd{\Phi}{x})&=\sum_{i,k} \pd{\Phi}x^{(i,k)} \partial_t\pf{\Ph}{x}_{ik}\\
%&=\sum_{i,k} \pa{\pd{\Phi}{x}}\pa{\pd{V}{y}}_{kj} \pa{\pf{\Phi}{x}}_{jk}\\
%&=\det\pa{\pd{\Phi}{}+}\sum_i\pa{%\fd 
%Vy}_{i,j}.
\end{align*}
Solving this differential equation, and using $\rc{J\Phi(t,x)}=J\Phi(-t,\Phi(t,x))$ gives
\begin{align*}
%\text{div} V(\Phi(t,x))\det(\pd{\Phi}x)(t,x)\
\det\pa{\pd{\Phi}{x}}&=e^{\int_0^t \text{div}(V)(\Phi(t,x))\,dt}\\
\int\ph\circ \Phi(t,x)%\rc{J\Phi(t,x)}J\Ph(t,x)
\,dx
&=\int \ph(y)J\Phi(-t,y)\,dy\\
&=\int\ph(y)e^{-\int_0^{t}\text{div}(V)(\Phi(-t,y))}\,dt\,dy.
\end{align*}
%Now look at difference between stuff already in $G$ and ?
Now consider
\begin{align*}
\la_{\R}(G)-\la_{\R^N}(\Phi(t, G))=\int_G (1-e^{\int_0^{-t}V(\Phi(t,x))\,dt})dx.
\end{align*}
So
\[
\lim_{t\to 0} \frac{\la_{\R}(G)-\la_{\R^N}(\Phi(H,G))}{t}=?%=\dov FIX!
\]
Now we calculate another way.
\begin{align*}
\la_{\R}(G)-\la_{\R^N} (\Phi(t,G))
&=\int 1_{G(x)}-1_{G}(\Phi(t,x))\,dx\\
&=\int_G 1_{G^c}(\Phi(t,x))\,dx-\int_{G^c} 1_G(\Phi(t,x))\,dx.
%bdary wo walk around
\end{align*}
%%In what rate is %tuffilosing 
%\[
%\int_G \text{div}(V)\,d\la_{\R^N}
%\]

%Now look at $\int_G1_G(\Phi(t,x))\,dx$ Yadayada.
%Get move at flow so no longer in $G$.
%$x\in G$ band $x\nin \Phi(t,x)\nin G^c$
%
%Something something %wtf am I so tired
%$x\in(\partial G)^{\Vert V\Vert} r)$
%\[
%\an{n(\Phi(t;x)),\Phi(t,x)-p(\Phi(t,x)}\ge 0.
%\]
%%The latter can't be any larger than the maximum of $x$ times $t$.
%%x lies prop to g on bdary
%Now
Do something. Then
\begin{align*}
\an{n\Phi(t,x),\Phi(t,x)-p\Phi(t,x)}
&=\an{n(x),\Phi(t,x)-p\Phi(t,x)}+O(t^2)\\
&=\an{n(x),\Phi(t,x)-x}+\an{n(x), x-p(x)}\\
&\quad+\an{n(x),p(x)-p(\Phi(t,x))}+O(t^2)\\
&=t\an{n(x), V(x)}+\an{n(x), x-p(x)}\\
&\quad+\an{n(x),p(x)-p(\Phi(t,x))}+O(t^2)\\
&=t\an{n(x), V(x)}+\an{n(x),x-p(x)}+E(t,x).
\end{align*}
Note we used $\an{n(x),p(\Phi(t,x))-p(x)}$ is 0 when $t=0$ and the derivative with respect to $t$ is $0$ since the second argument is on the boundary, so its derivative is $t=0$ in the tangent space, perpendicular to $n(x)$, so is $O(t^2)$.
\end{proof}
