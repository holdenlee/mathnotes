\lecture{Mon. 3/7/11}

\subsection{Lebesgue dominated convergence}
Suppose we have a measure space $(E,\cal B, \mu)$. We say that a property holds almost everywhere with to $\mu$ if it holds except on a set of measure 0. %A set that mu cannot see.
For example, $f=g$ (a.e., $\mu$) means that $\mu(f\neq g)=0$.
%\lim_{n\to \infty} f_n =f a.e., \mu

\begin{thm}[Lebesgue Dominated Convergence Theorem]
%A sequence of functions 
Suppose $\{f_n:n\ge 1\}\cup \{f\}$ is measurable, and there exists $g\in L^1(\mu)$ such that 
\begin{enumerate}
\item
for all $n$, $|f_n|\le g$ almost everywhere w.r.t. $\mu$, and %a.e.,\mu
\item
$f_n\to f$ almost everywhere,
\end{enumerate}
Then $\ab{\int f_n-\int f}\le \int |f_n-f|\to 0$.
\end{thm}
\begin{proof}
Note
\[
\{x:\exists n,\, |f_n(x)|>g(x)\}\cup \{x:f_n(x)\not\to f(x)\}
\]
has measure 0 because it is the countable union of measure 0 sets.
Let $\hat E$ be the set of $x$ such that $f_n(x)\to f(x)$ and $|f_n(x)|\le g$. Then $\mu(E\bs \hat E)=0$, as mentioned above.
Thus it suffices to show the theorem, without the ``almost everywhere" business.

Now $|f|\le g$ gives $|f_n-f|\le 2g$. By the Fatou Lemma, (using $\ab{\int h}\leq \int | h|$)
\[
\varlimsup_{n\to \infty} \int |f_n-f|\le \int \varlimsup_{n\to \iy} |f_n-f|=0.
\]
\end{proof}
Note this is not true without the $|f_n(x)|\le g$ hypothesis. (Consider a ``moving spike.")

\begin{thm}\label{lieb}
Suppose $\{f_n:n\ge 1\}\cup \{f\}\subeq L^1(\mu;\R)$, and that $f_n\to f$ (a.e., $\mu$). Then
\[
\lim_{n\to \iy} \ab{
\nl{f_n}-\nl{f}-\nl{f_n-f}
}
=
\lim_{n\to \iy}\int \ab{
|f_n|-|f|-|f_n-f|
}\,d\mu=0.
\]
If $\nl{f_n}\to \nl{f}<\infty$, then $\nl{f-f_n}\to 0$.
\end{thm}
\begin{proof}
Note $\ab{|f_n|-|f|-|f_n-f|}\to 0$ \am{}, and by the Triangle Inequality,
$\ab{|f_n|-|f|-|f_n-f|}\leq \ab{|f_n|-|f_n-f|}+|f|\le 2|f|$. Thus the right-hand equality holds by Lebesgue's Dominated Convergence Theorem.

By the Triangle inequality,
\[
\ab{
\nl{f_n}-\nl{f}-\nl{f_n-f}
}
\leq 
\int \ab{
|f_n|-|f|-|f_n-f|
}\,d\,mu.
\]
Taking $n\to \infty$ gives the left-hand equality.
\end{proof}
\subsection{Convergence in $\mu$-measure}
%Suppose $\{f(n):n\ge 1\}\cup \{f\}\subeq L^1(\mu)$ and $f_n\to f$ almost everywhere. Then by Fatou's Lemma, $\int |f|\leq \lim_{n\to \iy} \int |f_n|$. (Set of measure 0 doesn't matter.)
%%What's a set of measure 0 among friends?
%
%We also know $\varliminf_{n\to \infty} |f_n-f|\ge 0$.
%%Lebesgue dominant.
%
%Then
%\[
%\lim_{x\to \iy}\ab{
%\int |f_n-f|-\pa{\int |f_n|-int |f|}}
%\leq
%\lim_{n\to \infty} \int |f_n-f|+|f|+|f_n|-0.
%\]
%\begin{proof}
%Note $|f_n-f|-|f_n|+\hat f\le 2 |f|.$ Use the triangle inequatlity.
%%integrand to 0, integral to 0.
%\end{proof}

\begin{df}
A sequence of measurable functions $f_n$ converges to $f$ {\bf in $\mu$-measure} if for every $\ep>0$, 
\[\lim_{n\to \infty}\mu(|f_n-f|\ge \ep)=0.
\]
\end{df}
By Markov's inequality, if $\nl{f_n-f}\to 0$, then $f_n\to f$ in $\mu$-measure.
Using the triangle inequality, $f_n$ can only converge to one function $f$.

\textbf{Failure of Fatou}: Just because $\nl{f_n}=\int |f_n|\,d\mu \to 0$ does not mean $\lim_{n\to \infty} f_n=0$. 
Defining $f_{2^m+l}=1_{[2^{-m}l,2^{-m}(l+1)]}$ (that is, characteristic functions covering smaller and smaller intervals in $[0,1]$), we see $\lim_{n\to \infty} f_n(x)=1$ for every $x$ (because there's infinitely many intervals in the sequence covering it), but $\lim_{n\to \infty}\int_{[0,1]}f_n\,\la_{\R}=0$.
%Let
%\[
%1_{[im, i_m]}\le 1_{m2^-n}ma
%\]
%for $0\le m<2^n$. Then $\int f_k\le 2^{-m}$ if 
%%notion of convergence of tunid[s
%%eac
%%funcs conrted by integrals.
%\[
%\mu(|f|\ge \la)\le \rc ? 
%\]
%Measure set etting smaller and smaller. Plafece where smal 
%Now $f_n\to f$ in $\mu$-measure means that fo every $\ep>0$
%\[
%\lim{n\to \iy}\mu(|f_n\ge q0|\geq c0.
%\]
%conclude f to g almost everywhere.

In Fatou's Lemma, Lebesgue's Dominated Convergence Theorem, and Theorem~\ref{lieb}, $f_n\to f$ almost everywhere can be replaced with $f_n\to f$ in $\mu$-measure.

Note when $|E|=n$ with the discrete topology, $L^1(\mu;E)$ is homeomorphic to $\R^n$ via the natural identification.

%Notes here incomprehensible:(. Read the text.

%%Consider the set $[
%\[\mu(|g-h|\ge h)\le \mu(???\ge |\rc{2k}|)\le \mu(|h-f_n|\ge \rc{2k})\to 9\]
%%$tria anled r
%%Colnvlusion (p
%Now
%\[
%\mu(|g-h|\ge h)\nearrow \mu(|f-g|>0).
%\]
%Can replace conditions in theorems with blah.
%
%$L^1(\mu;E)$. Suppose $1,2, lots $.
%\[
%||f||_{L^1(\mu_)}
%\]
%Consider $E=\{1,\ldots, n\}$, $m(\{m\}=1$, $1\le m\le n$. Now $L^1(\mu)=\R^n$, $||f||_{m=1}^{\mu}=\sum_{n\-1}^{\infty}|f_m(m)|$
%Now
%%ocmparable to Euclidasn notion of lenght
%\[
%\pa{\sum_{m=1}^{\infty} a_m^2}^{\rc w}\le \sum_{m=1}^{\infty} \leq n^{\rc 2}\le n^{\rc 2}\sum_{n=1}^{\iy} a_m^2.
%\]
%Then follws from Cauchcy-Schwarz.
%
%Topology, same as on $\R^n$
%
%Norm 
%%sepble (ctable denseO , complete
%\begin{enumerate}
%\item $||f||_{L^1(\mu)}=\int |f|$
%\item $||\al f||_C=|\al| ||f||_{L^1(\mu)}$
%\item $\nl{f}=0$ iff $f=0$.
%\item $\nl{f+g}\le \nl{f}+\nl{g}$.
%\end{enumerate}•
%finite dim v space false in this setting. % bounded closed not compact