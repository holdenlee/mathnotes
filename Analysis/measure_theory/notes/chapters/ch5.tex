\lecture{Fri. 2/11/2011}

\subsection{Measure}

%\begin{df}
%A \textbf{measure} $\mu$ is a function which assigns nonnegative numbers (in $[0,\infty]$) to subsets $\Ga$ of a set $E$, satisfying \textbf{countable additivity}: 
%If $\{\Ga_n:n\geq 1\}$ is a family of pairwise disjoint subsets of $E$, then
%\[
%\mu\pa{\bigcup_{n=1}^{\infty}}=\sum_{m=1}^{\infty} \mu(\Ga_n),
%\]
%i.e. the volume of the whole is the sum of the volume of the parts.
%\end{df}
For a set $E\neq \phi$ define the power set
\[\mathcal P(E)=2^E=\{\Ga:\Ga\subeq E\}.\]
\begin{df}\label{salgdf}
A subset $\mathcal B\subeq \mathcal P(E)$ is a $\sigma$\textbf{-algebra} it satisfies the following properties:
\begin{enumerate}
\item $E\in \mathcal B$.
\item $\mathcal B$ is closed under complementation: $\Ga\in \mathcal B$ implies $\Ga^c=E\backslash \Ga\in \cal B$.
\item $\{\Ga_n:n\geq 1\}\subeq \cal B$ implies $\bigcup_{n=1}^{\infty} \Ga_n\in \cal B$.
\end{enumerate}
(If item 2 is satisfied just for finite instead of countable unions then we call $\mathcal B$ an algebra.)
\end{df}
Note that items 2 and 3 imply that a countable intersection of elements in $\mathcal B$ is in $\mathcal B$, and a difference of sets in $\mathcal B$ is in $\mathcal B$.
\begin{df}
We call 
$(E, \cal B)$ is a measurable space. A measure on $(E,\cal B)$ is a map $\mu:\cal B\to [0,\infty]$ such that 
\begin{enumerate}
\item $\mu(\phi)=0$.
\item (Countable additivity) If $\{\Ga_n:n\geq 1\}$ is a family of pairwise disjoint subsets of $E$, then
\[
\mu\pa{\bigcup_{n=1}^{\infty}\Ga_n}=\sum_{n=1}^{\infty} \mu(\Ga_n),
\]
i.e. the volume of the whole is the sum of the volume of the parts.
\end{enumerate}
\end{df}
Compare this to the definition of a  topological space---measurable spaces have measureable sets while topologies have open sets.

\begin{ex}
Define a measure $\mu$ on the integers $\Z$ by associating some $\mu_i\geq 0$ for each integer $i$, and setting
\[
\mu(\Ga)=\sum_{i\in \Ga}\mu_i.
\]
\end{ex}
%Lebesgue provided a measure for the reals---a more complicated set---that agrees with our previous definition of volume for intervals.

Our strategy is to start with some class of nice, well-defined subsets, and generate more.
\begin{df}
For a family of subsets $\cal C\subeq \cal P(E)$, define the
$\sigma$-algebra generated by $\cal C$, denoted by $\sigma(\cal C)$, to be the smallest $\sigma$-algebra containing $\cal C$. In other words it is the intersection of all $\sigma$-algebras containing $\cal C$. (This is well-defined since the power set is a $\sigma$-algebra containing $\cal C$.)

If $E$ is a topological space and $\cal C=\{\Ga\subeq E:\Ga\text{ open}\}$ then $\sigma(\cal C)=\cal B_E$ is called the \textbf{Borel }$\sigma$\textbf{-algebra}.
\end{df}
Lebesgue showed that there exists a unique Borel $\sigma$\textbf{-algebra} on $\cal B_{\R_N}$ such that $\mu_{\R^N}(I)=\vol(I)$. 

\subsection{Basic results}
\begin{pr}\label{measure-basic}
\begin{enumerate}
\item If $A\subeq B$ are sets in $\cal B$ then $\mu(A)\leq \mu(B)$.
\item (Countable subadditivity) Let $\{\Ga_n:n\geq 1\}\subeq \cal B$. Then 
\[
\mu\pa{
\bigcup_{n=1}^{\infty} \Ga_n}\leq \sum_{n=1}^{\infty} \mu(\Ga_n).
\]
(The sets are not necessarily disjoint, so the RHS counts ``overlap.")

\item A countable union of subsets of measure zero has measure 0.
\item We write $B_n \nearrow B$ if $B_1\subeq B_2\subeq \cdots $ and $\bigcup_{n=1}^{\infty} B_n=B$.

If $B_n\nearrow B$ then $\mu(B_n)\nearrow \mu(B)$ (i.e. $\mu(B_n)\to \mu(B)$ from below).
\item We write $B_n \searrow B$ if $B_1\supeq B_2\supeq \cdots $ and $\bigcap_{n=1}^{\infty} B_n=B$.

If $B_n\searrow B$ \textit{and} $\mu(B_1)<\infty$ then $\mu(B_n)\searrow \mu(B)$ (i.e. $\mu(B_n)\to \mu(B)$ from above).
\end{enumerate}
\end{pr}
\begin{proof}
\begin{enumerate}
\item Note $B\backslash A\in \cal B$. Hence
\[
\mu(B)=\mu(A)+\mu(B\backslash A)\geq \mu(A).
\]
\item 
Let
\[
B_n=\Ga_n\backslash \bigcup_{m=1}^{n-1}\Ga_m.
\]
Then by countable additivity,
\[
\mu\pa{
\bigcup_{n=1}^{\infty} \Ga_n}
=
\mu\pa{
\bigcup_{n=1}^{\infty} B_n}
\leq \sum_{n=1}^{\infty} \mu(B_n)
\leq \sum_{n=1}^{\infty} \mu(\Ga_n).
\]
In the last step we used $B_n\subeq \Ga_n$ and part 1.
\item Follows directly from part 2.
\item Like in part 2, take $A_n=B_n\backslash B_{n-1}$. Then
\[
\mu(B_n)=\sum_{m=1}^n\mu(A_m)\nearrow \sum_{m=1}^{\infty} \mu(A_m)=\mu(B).
\]
\item By the previous part, $B_1\backslash B_n\nearrow B_1\backslash B$ giving $\mu(B_1\backslash B_n)\nearrow\mu(B_1\backslash B)$.

Now use %$\mu(B)=\mu(A)+\mu(B\backslash A)$ is always true but 
$\mu(B\backslash A)=\mu(B)-\mu(A)$, which holds because $\mu(B)<\infty$. %may not be true because of infinities. 

\end{enumerate}
\end{proof}
Note item 5 is false without the assumption that $\mu(B_1)<\infty$.
%We get that if $B_n\searrow B$ and $\mu(B_1)<\infty$ then $\mu(B_n)\searrow \mu(B)$. 
For example, consider the measure on $\Z$ with $\mu(\Ga)=|\Ga|$, and take $B_n=\{i:i\geq n\}$. 

Note from item 3, the existence of Lebesgue measure implies $\R$, or any interval of $\R$, is uncountable, since all countable subsets have measure 0 and any interval does not.
%Favorite choices of generating sets
%Closed under intersection nice
\begin{df}
We say $\cal C$ is a $\Pi$-system if $\cal C$ is closed under intersection, i.e. if $A\in \cal C$ and $B\in \cal C$ then $A\cap B\in \cal C$.
\end{df}
%In the next lecture we give  additional conditions a  pi system have to have to be $\sigma$-alg

%compute measure of pi system of subset 

%if agree on pi system then on sigma alg generated by pi system - like two continuous functions equal on dense set.