\lecture{Mon. 4/11/11}

\subsection{Lebesgue spaces}

For $p\in [1,\iy)$, let
$L^p(\mu, \R)$ be the set of measurable functions $\xi: E\to \R$ such that
\[
\ve{\xi}=\pa{
\int |f|^p
}^{\rc p}<\iy.
\]
Note that
\begin{align*}
\ve{\xi+\psi}_p&\le \ve{\xi}_p+\ve{\psi}_p\\
\ab{\ve{\xi}_p-\ve{\psi}_p}&\le\ve{\xi-\psi}_p\\
\ve{\al \xi}_p&=|\al|\ve{\xi}_p
\end{align*}
Again we form equivalence classes, considering two functions the same if they differ on a set of measure 0. This turns $L^p(\mu, \R)$ into a metric space; it satisfies the triangle inequality by Minkowski's inequality~(??).

Let $L^{\iy}(\mu, \R)$ be the set of measurable functions that are bounded Off a set of measure 0. Define the norm to be 
\[
\ve{f}_{\iy} =\inf\{M:|f|\le M (\text{ a.e., }\mu)\}.
\]

%Why study these spaces?
Some intuition.
\begin{enumerate}
\item As $p$ changes, we emphasize different features of the function. As $p$ gets large, large values $f$ takes (``spikes") are emphasized. Small $p$ is forgiving on spikes.
\item 
If the space is infinite, a function in $L^p$ for small $p$ must have rapid ``decay". 
If $p$ is large and $|f|<1$, $|f|^p$ is small, so $f$ need not decay as rapidly. For example $\rc{1+|x|}$.
\item 
Often you can't get pointwise estimates of functions (ex. solutions to differential equations). However, using various trick, you can get estimates of integrals of the functions. We want the flexibility of being able to choose $p$. 
%(Sobolev imbedding theorem)
The Sobolev imbedding theorem gives an inequality for $\ve{f}_p$ given that
%says for 
$f\in L^p(\R^N, \R)$, $|\nabla f|\in L^p$, and $p>N$.
%, then 
%\[
%\ve{f}_p\le C
%\]
\item
Some spaces are geometrically nicer than others. For example, for $L^1$, for a space on two points, a circle is a diamond and has nasty corners. For $1<p<\iy$ it is smooth and convex. For $p=\iy$ we again get nastiness (a square).
\end{enumerate}
\begin{thm}\label{lpbasic}
\begin{enumerate}
\item
If $f_n\to f$ in $L^p$, then $f_n\to f$ in $\mu$-measure.
\item %no (Converse, analogue of Lebesgue dominated convergence theorem)
If $f_n\to f$ in $\mu$-measure of $\mu$-almost everywhere. Then 
\[
\ve{f}_p\le \varliminf_{n\to \iy} \ve{f_n}_p.
\]
\item $\sum_{n>m} \ve{f_n-f_m}_p\to 0$ as $m\to \iy$, then $f_n$ converges to a function in $L^p$. Hence $L^p$ is a complete metric space.
\item If $\mu$ is finite, $\cal B=\si(\cal C)$ is a $\Pi$-system, and the $\al_m$ are dense in $\R$, then 
\[
S=\bc{\sum_{m=1}^n \al_m1_{\Ga_m}}
\]
is dense in $L^p$ for $p<\iy$. If the measure space is $\si$-finite and the \sia{} can be generated by a countable set, then $L^p$ is separable for $p<\iy$.
%separability is obviously false wrt 
%p=\iy: functions of [0,1] Can find uncountably many f lies in? no
%integration statements 
\item Suppose $(E,\rh)$ is a metric space and $\{E_n:n\ge 1\}$ is a countable set such that $E_n\nearrow E$, with $\mu(E_n)<\iy$, $p<\iy$. Then the set of continuous functions which vanish off $E_n$ for some $n$ is densein $L^p$.
\item (Lebesgue dominated convergence) Suppose $p<\iy$. If $f_n\to f$ in $\mu$-measure or $\mu$-almost everywhere and there exists $g\in L^p$ such that $|f_n|\ge g$ almost everywhere for every $n$. Then $f_n\to f$ in $L^p$.
\item (Lieb) if $f_n\to f$ in $\mu$-measure or $\mu$-almost everywhere, and $\sup_n \ve{f_n}_1<\iy$, then 
\[
\lim_{n\to \iy} \int \ab{|f_n|^p-|f|^p-|f-f_n|^p}\,d\mu=0
\]
Thus if $\ve{f_n}_{p}\to \ve{f}_{p}$, then $\ve{f_n-f}_p\to 0$. 
\end{enumerate}
\end{thm}
\begin{proof}
%Cauchy criterion in sense of $\mu$-measure implies convergence to something in $\mu$-measure. 
The arguments are the same as for $L^1$. 

For Lieb,  note there is $K_p$ such that \[\ab{|b|^p-|a|^p-|b-a|^p}\le K_p(|a||b-a|^{p-1}+|a|^{p-1}|b-a|)\] for all $a,b$. (This follows from homogeneity and $\ab{|c|^p-1-|c-1|^p}\le K_p(|c-1|^{p-1}+|c-1|)$. Use calculus!) Put $a=f_n(x)$ and $b=f(x)$. Now integrate and note that 
%blah
%By homogeneity it suffices to prove $
%[
%\mu(|\xi|\ge R)=\mu(|f|
%\]
if $\xi_n\in L^p$, $\xi_n\to 0$ in $\mu$-measure or a.e., then 
\[
 \lim_{n\to \iy}\int |\xi_{n}|^{p-1} |\psi|\,d\mu=0=\lim_{n\to \iy } \int |\xi_n||\psi|^{p-1}\,d\mu.
\]
To prove this, assume they're positive. Now divide into 2 parts. For $\de>0$,
\begin{align*}
\int \xi_n^{p-1}\psi &=\int_{\xi_n\le \de\psi} \xi_n^{p-1}\psi\,d\mu+\int_{\xi_n>\de\psi} \xi_n^{p-1} \psi\,d\mu\\
&\le \de^{p-1} \ve{\psi}_p^p
+\int_{\ph_m\ge\de^2} \xi_n^{p-1}\psi\,d\mu+\int_{\psi\le \de}\xi_n^{p-1} \psi\,d\mu
\end{align*}
Take $p'=\frac{p}{p-1}$ and use H\"older's inequality:
\[
\int_{\ph_m\ge\de^2} \xi_n^{p-1}\psi\,d\mu+\int_{\psi\le \de}\xi_n^{p-1} \psi\,d\mu
\le \pa{\int \xi_n^p}^{\frac{p-1}{p}}\ba{\pa{\int_{\xi_n\ge \de^2}|\psi|^p}^{\rc p}+\pa{\int_{\psi\le \de}\psi^p\,d\mu}^{\rc p}}
\]
Now let $n\to\iy$, $\de\to 0$, and use Lebesgue's Dominated Convergence Theorem. The other inequality follows similarly.
\end{proof}