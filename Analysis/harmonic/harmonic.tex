\def\filepath{C:/Users/Owner/Dropbox/Math/templates}

\input{\filepath/packages_article.tex}
\input{\filepath/theorems_with_boxes.tex}
\input{\filepath/macros.tex}
\input{\filepath/formatting.tex}
%\input{\filepath/other.tex}

%\def\name{NAME}

%\input{\filepath/titlepage.tex}

\pagestyle{fancy}
%\addtolength{\headwidth}{\marginparsep} %these change header-rule width
%\addtolength{\headwidth}{\marginparwidth}
\lhead{Harmonic analysis}
\chead{} 
\rhead{} 
\lfoot{} 
\cfoot{\thepage} 
\rfoot{} 
\renewcommand{\headrulewidth}{.3pt} 
%\renewcommand{\footrulewidth}{.3pt}
\setlength\voffset{0in}
\setlength\textheight{648pt}

\begin{document}
\section{Hardy-Littlewood Maximal Function}
%Real-variable theory
Throughout we work in $\R^n$.
\begin{df}
Given a function $f$, define the \vocab{maximal function} as the maximal average of absolute value over balls centered at $f$.
\[
Uf(x)=\sup_{r>0} \rc{m(B(x,r))} \int_{B(x,r)} |f(y)|\,dy.
\]
\end{df}
\begin{thm}[Maximal inequality]\label{thm:max-ineq}
\begin{enumerate}
\item (Weak $L^1$ bound)
For all $\al>0$, 
\[
m(\set{x}{Mf(x)>\al})\precsim \rc{\al}\int |f|\,dx.
\]
This essentially says that if $f\in L^1$, then $Mf\in L^{1,\iy}$. 
\item If $f\in L^p$ and $1<p\le \iy$, then $Mf\in L^p$, and 
\[
\ve{Mf}_{L^p}\precsim \ve{f}_{L^p}.
\]
\end{enumerate}
\end{thm}
To prove this we need the covering lemma. 
\begin{lem}[Vitali covering lemma]\label{lem:covering}
Let $E$ be a nonempty measurable set, covered by balls $B_1,\ldots, B_M$. There exists a subcollection $\{B_1,\ldots, B_N\}$ that is mutually disjoint and
\[
\sum_{k=1}^N m(B_k) \ge 3^n m(E).
\]
\end{lem}
%We can choose a disjoint subcollection that gives us some control over the original collection.

Suppose we have some weird set $E$. If we can cover it by a finite collection of balls, we can find a disjoint subcollection that still tells us something about the measure of $E$.

\begin{proof}
Use the greedy method (``analysis is being greedy").
Let $B_1\in \{B_\al\}$ have the largest radius. Let $B_2$ be the ball disjoint from $B_1$ which has largest radius, and so forth.

They are mutually disjoint by construction. Suppose $B\in \{B_\al\}$ which is not one of $B_1,\ldots, B_N$. Then for some $l$, $B\cap B_\ell\ne \phi$. Suppose $B_l$ is the largest (first) ball that it intersects. Then $r(B_l)>r(B)$ because otherwise $B$ would have been chosen instead of $B_l$ at that stage. Then the dilation of $B_l$ by a factor of 3 covers $B$: $B\subeq 3B_l$. 

Then $3B_1,\ldots, 3B_N$ cover $E$, so
\[
m(E)\le \sum_{k=1}^N m(3B_k)=3^n \sum_{k=1}^N m(B_k).
\]
\end{proof}
\begin{proof}[Proof of maximal inequality~\ref{thm:max-ineq}]
Define the \vocab{uncentered maximal function} by 
\[
\wt M f(x)=\sup_{B\ni x} \rc{m(B)} \int_B|f(y)|\,dy.
\]
It's clear that $Mf(x)\le \wt Mf(x)$.

It suffices to prove the theorem for $\wt Mf$.

Set
\[
E_\al = \set{x}{\wt Mf(x)>\al}.
\]
For all $x\in E_\al$, there exists $B_x\ni x$ such that 
\[
\rc{m(B_x)}\int_{B_x} |f(y)|\,dy>\al,
\]
i.e. $m(B_x)\le \rc{\al}\int_{B_x}|f(y)|\,dy$. Let measure of $E$ can be made arbitrarily close to that of $E_\al$ because of inner regularity.)
Let  $\{B_x\}_{x\in E_\al}$ be a collection of balls 
 that cover $E_\al$. %(It suffices to look at compact subsets because $E_\al$ is measurable.) 
By compactness we can choose a finite number $B_1,\ldots, B_M$ that cover $E$. By the Covering Lemma~\ref{lem:covering} we can choose $B_1,\ldots, B_N$ mutually disjoint and $\sum_{k=1}^N m(B_k)\ge 3^{-n} m(E)$. We have an upper bound on this already:
\bal
m(E)&\le 3^n \sum_{k=1}^N m(B_k)\\
&\le \fc{3^n}{\al} \sum_{k=1}^N\int_{B_f}|f(y)|\,dy\\
&\le \fc{3^n}{\al}\int |f|\,dx\\
\implies m(E_\al)&\precsim \rc{\al} \int|f|\,dx.
\end{align*}
%Inner regular (compact)
\end{proof}
The norm is independent of the dimension.

\begin{df}
Let $f$ be measurable. The \vocab{distribution function} of $f$ is a function $\la_f(\al):[0,\iy)\to [0,\iy)$ given by
\[
\la_f(\al)=m(\{|f(x)|>\al\}).
\]
\end{df}
This gives us useful information. For example, 
\[
\int|f|^p\,dx=p\int_0^{\iy} \al^{p-1}\la_f(\al)\,dx.
\]
``Cut the cake" horizontally.
\begin{proof}[Proof of part 2 of Theorem~\ref{thm:max-ineq}]
Let $f\in L^p$, $1<p<\iy$. Then
\[
\int|\wt Mf(x)|^p\,dx=
p\int_0^{\iy} \al^{p-1}m\pa{\set{x}{\wt Mf(x)>\al}}\,dx
\le 
p\int_0^{\iy} \al^{p-1}m\pa{\set{x}{\wt Mg(x)>\fc{\al}2}}\,dx.
\]
where
\[
g(x):=\begin{cases}
|f(x)|,&|f(x)|>\fc{\al}{2}\\
0,&\text{otherwise}.
\end{cases}
\]
We show the last inequality.
The new function $g$ contains all the information we need.
We know
\bal
|f(x)|&\le \max\{g(x),\fc{\al}{2}\}\\
&\le |g(x)|+\fc{\al}2.
\end{align*}
This translates to the maximal function.
\bal
\wt Mf(x)&\le \wt Mg(x)+\fc{\al}{2}\\
\wt Mf(x)>\al&\implies \wt Mg(x)>\fc{\al}2\\
\{\wt Mf(x)>\al\}&\subeq \{\wt M g(x)>\fc{\al}2\}.
\end{align*}
%only supported where $f$ is large, so we can use the first inequality. $g\in L^1$. The only it can fail is by blowing up somewhere. But $p>1$.
We have
\bal
\int|\wt Mf(x)|^p\,dx&\precsim \int_0^{\iy} \al^{p-1}\rc{\al}\pa{\int_{\R^n}|g(x)|\,dx}\,d\al\\
&=\int_0^{\iy} \al^{p-2} \pa{\int_{\set{x}{|f(x)|>\fc{\al}2}}|f(x)|\,dx}\,d\al\\
&=\int_{\R^n} |f(x)|\pa{\int_0^{2|f(x)|}\al^{p-1}\,d\al}\,dx \precsim \int|f|^p\,dx.
\end{align*}
\end{proof}
Let's build up what we need for the Calderon-Zygmund decomposition. First we need another (more annoying) covering lemma.

We can cover the complement of a closed set with cubes, where the diameter of the cube is proportion to the distance from the cube to the set. 
\begin{lem}[Whitney decomposition]\label{lem:whitney}
Let $F$ be a nonempty closed set. There exists a sequence of almost disjoint cubes (intersecting only on a set of measure 0, i.e., their boundary) $\{Q_k\}$ such that 
\[
F^c=\bigcup Q_k,
\]
and there exists $C>0$ such that 
\[
\diam(Q_k)\le d(Q_k,F)\le C\diam(Q_k).
\]
\end{lem}
\begin{proof}
%Consider cubes 
Let $M_0$ be cubes of unit side length with vertices in $\Z^n$.

Define $M_{k}$ by bisecting each cube in $M_{k-1}$, so that $M_k$ consists of cubes with side length $2^{-k}$, vertices $2^{-k}\Z^n$. (This is the ``dyadic decomposition.")

Let $\Om=F^c$; set
%bands around the original set
\[
\Om_k = \set{x\in \R^n}{C2^{-k}\le d(x,F)\le 2C2^{-k}},
\] 
where $C$ is a constant to be chosen. Think of them as bands around the original set.

For $Q\in M_k$, include $Q\in \cal F$ if $Q\cap \Om_k\ne \phi$.
We need a lower bound of the distance of the cube to the original set $F$. Let $x\in Q$. Then 
\bal
d(x,F)&\ge C2^{-k}-\ub{\diam(Q)}{\sqrt n2^{-k}}\\
&=(C-\sqrt n) 2^{-k} \ge \sqrt n 2^{-k} =\diam(Q).
\end{align*}
where we take $C=2\sqrt n$. This gives one direction.

For the other, 
\bal
d(Q,F)&\le 2C2^{-k}&\text{since it intersects the band}\\
&=2\cdot 2\sqrt{n}2^{-k}\\
&=4d(Q).
\end{align*}
For all $Q\in \cal F$, we have
\[
\diam (Q)\le d(Q,F)\le 4\diam(Q).
\]
We also have to remove redundant cubes. Proof omitted. %any cube in collection that contains it can only be so large. Get small cube by bisecting fixed number of times; get rid of smaller cubes.

\end{proof}

\begin{thm}[Calderon-Zygmund decomposition]
\llabel{thm:c-z}
Let $f\in L^1$. Fix $\al>0$. Then there exists a decomposition $f=g+\sum_{k}b_k$ and a sequence of almost disjoint cubes $\{Q_k\}$ such that 
\begin{enumerate}
\item
(the good part is bounded) $|g(x)|\le \al$,
\item $\Supp(b_k)\subeq Q_k$, $\int_{Q_k} b_k=0$, and $\int|b_k(x)|\precsim \al\cdot m(Q_k)$,
\item $\sum m(Q_k)\precsim \rc{\al}\int |f|\,dx$.
\end{enumerate}
\end{thm}
Bounds ($\precsim$) depend only on the dimension.

This is very useful. To construct $g,b_k$, it's tempting to just cut off $f$ at $\al$. But the correct thing to do is to cut off the maximal function at $\al$. %gives us more control.

Idea: The bad set will be covered by a bunch of cubes; if you dilate the cubes by a fixed factor it intersects the good set, so you have a bound on the maximal function.
\begin{proof}
%apply whitney decomp to compl?
We cutoff where $\wt Mf(x)>\al$. Let $E_{\al}=\set{x}{\wt Mf(x)>\al}$. $E_{\al}^c$ is closed (WLOG nonempty and not all of $\R^n$). Apply Whitney decomposition~\ref{lem:whitney} to cover $E_\al$ by $\{Q_k\}$. Set
\[
g(x)=\begin{cases}
|f(x)|,&x\in E_\al^c\\
\rc{m(Q_k)}\int_{Q_k} f(y)\,dy,&x\in Q_k\\
0,&\text{elsewhere.}%edges where intersect, for consistency.
\end{cases}
\]
%The remaining piece will have 0 average.
On $E_\al^c$ we have $|f(x)|\precsim |\wt Mf(x)|\le \al$. %Lebesgue differentiation theorem, have it almost everywhere.
\begin{clm*}
We have $|g(x)|\precsim \al$ almost everywhere.
\end{clm*}
If $x\in Q_k$ (This is a set completely contained in the bad region, but we can blow it up so that it overlaps the good region.),
\bal
|g(x)|&\le \rc{m(Q_k)}\int_{Q_k}|f(y)|\,dy \\
&\le \fc{4^n}{m(Q_k^*)}\int_{Q_k^*} |f(y)|\,dy&Q_k^*:=4Q_k\\
&\precsim\al.
\end{align*}
Let 
\[
b_k(x)=\chi_{Q_k}(x)\pa{
f(x)-\rc{m(Q_k)}\int_{Q_k} f(y)\,dy
};
\]
note that $\int b_k(x)=0$; we have $f=g+\sum b_k$.
%Notice that if you integrate over $Q_k$, this will give you 0.

Now we need to estimate the $L^1$ norm of each. We find (the ``$\le \al$" comes from taking $x\in Q_k^*\cap E_\al$ and noting $\rc{m(Q_k^*)}\int_{Q_k^*}|f(y)|\,dy\le \wt Mf(x)$  by definition of $\wt Mf$)
\bal
\int |b_k(x)|\,dx&\le 2\int_{Q_k} |f(y)|\,dy\\
& \le \int_{Q_k^*}|f(y)|\,dy \\
&=m(Q_k^*)\cdot \ub{\rc{m(Q_k^*)} \int_{Q_k^*} |f(y)|\,dy
}{\le \al}\\
&\precsim \al m(Q_k^*)\precsim \al m(Q_k)\\
\sum m(Q_k)
%cover bad set 
&=m(\set{x}{\wt Mf(x)>\al})\\
&\precsim \rc{\al}\int|f|\,dx.
\end{align*}

\end{proof}
What can you do with this decomposition? You can estimate in singular integrals. This comes up in $L^p$ elliptic regularity results for Laplace's equation. There's an integral kernel that can be estimated using Calderon-Zygmund.
%get all exponents in between for interpolation. Adjoint: get between 1 and $\iy$ for free.

\blu{2-17-15}

\section{Singular integrals}

\subsection{Approximation technique}
We first discuss the Lebesgue differentiation theorem. A closely related topic is approximation to the identity. They use an approximation theorem that is very important and will occur many times in proofs about singular integrals.

\begin{thm}[Lebesgue differentiation theorem]
Assume $f\in L^1_{\text{loc}}$ (locally integrable function, i.e., integrable when restricted to any ball). 
\begin{enumerate}
\item
Then
\[
\lim_{r\to 0}\rc{m(B(r))}\int_{B(X,r)} |f(y)-f(x)|\,dy=0
\]
for almost every $x$.
\item 
\[
\lim_{r\to 0}\rc{m(B(r))}\int_{B(X,r)} f(y)\,dy\to f(x)
\]
for almost every $x$.
\end{enumerate}•
\end{thm}
\begin{proof}
\begin{enumerate}
\item
Assume $f\in C_c$ (continuous and compactly supported). Then $f$ is uniformly continuous: 
\[\forall\ep>0\qquad\exists\de,\qquad\forall |x-y|<\de,\qquad|f(x)-f(y)|<\ep.\]
Then for $r<\de$, the integral is $<\ep$.
\item Assume $f\in L^1$. We approximate it with a continuous function
\[
\forall \ep>0,\qquad \exists g\in C_c \text{ such that }\ve{f-g}_{1}<\ep.
\]
%We want to bound $|f(y)-f(x)|$ by $|g(y)-g(x)|$ but there is an error:
By the triangle inequality,
\[
|f(y)-f(x)|\le |g(y)-g(x)|+|f(y)-g(y)|+|f(x)-g(x)|.
\]
Averaging over a ball, taking the limsup, and using (1),
\bal
\limsup \rc{m(B(r))} \int_{B(x,r)} |f(y)-f(x)| 
&\le \sup_{r\to 0}
\rc{m(B(r))}\int |f(y)-g(y)|\,dy + |f(x)-g(x)|\\
&\le M(f-g)(x)+|f-g|(x)\\
m\pa{\limsup_{r\to 0} \rc{m(B(r))}\int |f(y)-f(x)|\,dy \ge \al} &\le m(M(f-g)(x)>\fc{\al}2)+m(|f-g|(x)>\fc{\al}2)\\
&\precsim \rc{\al}\ve{f-g}_1 < \fc{\ep}{\al}.
\end{align*}
where we used the weak $L^1$ bound for the maximal function, Theorem~\ref{thm:max-ineq}(1).
%by chebyshev
Now take $\ep\to 0$, so we get the LHS is 0. 
Now take $\al=\rc{n},n\to \iy$.
%countable union of sets whose measure is 0.
\end{enumerate}
\end{proof}

\begin{thm}[Approximation to identity]
Let $|K(x)|<(1+|x|)^{-n-\ep}$. Let
%R(x)$ where $R$ is radial decreasing, $R\in L^1$ (for example $R=(1+|x|)^{-n-\ep}$). Let 
\[
K_t(x)=\rc{t^n} K\pf xt 
\]
(note this is normalized so $\int K_t=\int K$). Then the following hold.
\begin{enumerate}
\item
$\sup_{t>0}|f*K_t(x)|\precsim Mf(x)$
\item 
As $t\to 0$, 
$f*K_t(x)\to f(x)\int K$ for almost all $x$.
\end{enumerate}
\end{thm}
We use the same approximation argument.
\begin{proof}
WLOG assume $f>0,K>0$.
\begin{enumerate}
\item
First we do the easier case where $K=\sum_{j=1}^N c_j 1_{B_j(0,r_j)}$. (For example, something that looks like stacked cylinders.) 
By linearity you can consider $K$ to the characteristic function of a single ball, $K=1_{B(1)}, K_t=\rc{t^n} 1_{B(t)}$. Then $f*K_t$ is the maximal function, so this follows from the maximal inequality, Theorem~\ref{thm:max-ineq}.
Then
\bal
f*K_t(x)&=\int f(x-y)K_t(y)\,dy \\
&=\rc{t^n} \int_{B(x,t)} f(y)\,dy\\
&\precsim c_n Mf(x)\ve{K}_{\iy}.
\end{align*}
Now take $\sup_{t>0}$.
\item %We can assume 
For general $k$, choose $K_j>0$, simple $K^{(j)}\nearrow K$. By the Monotone Convergence Theorem, as $j\to \iy$,
\[
c*K_t^{(j)}(x) \to  f*K_t(x)
\]
%radial and decreasing.
(We check that $c*K_t^{(j)}(x) \le c_n Mf(x)\ve{K^{(j)}}_{\iy}\le c_n Mf(x)\ve{K}_{\iy}$, $c*K_t^{(j)}(x) \le c_n Mf(x)\ve{K}_{\iy}$.) 
\item For continuous functions we have that as $t\to 0, f*K_t(x)\to f(x)\int  K$ for almost all $x$. Now approximate $L^1$ functions by continuous functions.
\end{enumerate}
\end{proof}

\subsection{Singular integrals}

\begin{df}
\vocab{Singular integrals} are integrals of the form
\[
Tf(x)=\int K(x,y) f(y)\,dy
\]
satisfying the following.
\begin{enumerate}
\item
$T$ is bounded on $L^q$ for some $q>1$.
\item
(Regularity assumption $K$)
For some $c>1$, $\int_{|x-y|>c|y-y'|} |K(x,y)-K(x,y')|\,dx\le c$.\footnote{Commonly $K(x,y)\sim \rc{|x-y|^n}$; then $DK(x)\sim \rc{|x|^{n+1}}$. Then this condition is $\int_{C|y-y'|}^{\iy} \fc{|y-y'|}{r^{n+1}} r^{n-1}\,dr\le C$.
It's like an $L^1$ bound on the gradient of $K$.
``Away from the diagonal, you have good bounds." Ex. This shows up as the Newtonian potential when solving Laplace's equation.}
\item If $f\in L^q$ is compactly supported, then $Tf(x)=\int K(x,y)f(y)\,dy$ is absolutely convergent.
\end{enumerate}
%the usual kernel, actually 
\end{df}

\begin{pr}
The following are true. 
\begin{enumerate}
\item
$T$ is weak $(1,1)$:
\[
\forall f\in L^1,\al>0, \qquad \inf (Tf>\al) \precsim \rc{\al}\ve{f}_1.
\]
\item
For all $1<p<q$, $T$ is strong $(p,p)$ (i.e., $\ve{Tf}_p\precsim_{p,n}\ve{f}_p$).
\end{enumerate}•
\end{pr}
We prove the first claim using Calderon-Zygmund, and the second claim using the first and Marcikiewicz interpolation.
\begin{proof}
Use the Calderon-Zygmund decomposition~\ref{thm:c-z} to get
\bal
f&=g+\ub{\sum_k b_k}b\\
g&\precsim \al\\
\Supp b_k &\subeq Q_k,\qquad \int_{Q_k}|b_k|\precsim \al m(Q_k)\\
\int_{Q_k} b_k&=0\\
\sum_k m(Q_k)&\precsim \rc{\al}\ve{f}_1.
\end{align*}
Let $F=\R^n\bs \bigcup_k Q_k$. 
We bound
\[
m(Tf>\al) \le m\pa{Tg>\fc{\al}2}+m\pa{Tb>\fc{\al}2}.
\]
\begin{enumerate}
\item
For $m(Tg>\fc{\al}2)$ we use the $L^q$ bound for $T$. First we obtain a $L^q$ bound for $g$. To do this we use our $L^1$ bound for $g$.
\bal
\int|g|&=\int_F|g| + \sum_k \int_{Q_k}|g|\\
&\le \int_F |f| + \al \sum_k m(Q_k)\\
&\precsim \ve{f}_1\\
\int |g|^q&\precsim \al^{q-1}\ve{f}_1\\
\implies \int |Tg|^q &\precsim \al^{q-1} \ve{f}_{1}\\
\implies m\pa{Tg>\fc{\al}2}&\precsim \fc{\al^{q-1}\ve{f}_1}{\al^q}.
\end{align*}
the last step by Chebyshev.
%weak L^1 bounded by L^1
\item 
For the bad part, we have to get our hands dirty. 
Let the cube $cQ_k$ have the same center as $Q_k$ but be dilated by $c$. Let 
\[
F'=\R^n \bs \bigcup_k cQ_k.
\]
First look at the part bounded away from the cubes.
\begin{enumerate}
\item
Claim: 
\[
m(\{Tb > \fc{\al}2\}\cap F')\precsim \rc{\al}\ve{f}_1.
\]
Proof: 
\bal
|Tb(x)|&=|\sum_k Tb_k (x)|\\
&=\ab{\sum_k \int_{Q_k} K(x,y) b_k(y)\,dy}\\
%in hope that difference is small
\text{(fix $y_k\in Q_k$)}\quad & \le \sum_k \int_{Q_k} |K(x,y)-K(x,y_k)| |b_k(y)|\,dy.\\
&\quad\text{(interchange integrals)}\\
\int_{F'}|Tb(x)|&\le \sum_k \int_{Q_k} |b_k(y)|\,dy \int_{F'} |K(x,y)-K(x,y_k)|\,dx\\
&\quad \text{(used zero-mean condition, take absolute value last)}\\
&\quad x\nin c_n (1+c)Q_k, \quad y\in Q_k\\
\implies |x-y|& \succsim \diam(Q_k) \succsim |y-y_k|\\
%apply bund on regularity of $k$
|x-y| &\precsim \sum_k \int_{Q_k} |b_k(y)|\,dy \\
& \precsim \al \sum_k m(Q_k)\precsim \ve{f}_1.\\
%Bound L^1 of Tb by L^1 of f, implies weak.
m(\{Tb>\fc{\al}2\}\cap \bigcup_k  cQ_k) &\precsim_k m(Q_k)\precsim \rc{\al}\ve{f}_1.
\end{align*}
Put all these together. 
%integrate over $F'$.
%interchange the integral
\item We have to interpolate between a weak $(1,1)$ and strong $(p,p)$ inequality.
We use the identity (distribution formula for $L_p$ norm)
\beq{eq:cutcake}
\int |f|^p=p\int _0^{\iy} \al^{p-1} m(|Tf|>\al)\,d\al.
\eeq
We decompose $f$ into 1 pieces,
\[
f=f_\al+f^\al,\qquad f_\al=1_{|f|\le \al}f,\qquad f^\al = 1_{|f|>\al}f
\]
Use the $L^p$ bound to the first and the weak $L^1$ bound to the second. We have the inequalities
\bal
m(|g|>\al)&\precsim \fc{\ve{g}_1}{\al}\\
m(|g|>\al)& \precsim \fc{\ve{g}_1^p p}{\al^p}
\end{align*}
useful when $\al\precsim |g|$, $\al \succsim |g|$, respectively.
%useful when $\al\precsim |f|$.
%close to 1 cannot get large domain for integration.
%huge increase in small part.
Applying these 2 inequalities to the 2 parts,
\bal
m(|Tf|>\al) & \le m(|Tf_{\al}|>\fc{\al}2) + m(|Tf^{\al}|>\fc{\al}2)\\
&\precsim \rc{\al^q} \ve{Tf_\al}_q^q +\rc{\al}\ve{f^{\al}}_1\\%know L^q, want L^p.
&\precsim \rc{\al^q}\int_{|f|\le \al}|f|^q +\rc{\al}\int_{|f|>\al}|f|.
\end{align*}
Finally, plugging into~\eqref{eq:cutcake}
\bal
\int |Tf|^p & \precsim_p \int_0^{\iy} \pa{
\fc{\al^{p-1}}{\al^q}\int_{|f|\le \al} |f|^q
+ \fc{\al^{p-1}}{\al} \int_{|f|>\al} |f|
}\,d\al\\
& = \int\int_{|f|}^{\iy} \al^{p-q-1} |f|^q \,d\al\,dx + 
\int\int_{0}^{|f|} \al^{p-2} |f|\,d\al\\
&\precsim \int |f|^{p-q}|f|^q+\int |f|^{p-1}|f|\,dx\\
&\precsim \ve{f}_p^p.
\end{align*}
Bootstrap by getting $L^q$ norm. As long as you have 1 $L^q$ norm you can get all these results. How to get the $L^q$ norm of $T$ in the first case? Use Fourier analytic methods.
%translation invariant.

%1-2, then 2-\iy by duality.
Assume $K(x,y)=K(x-y)$. We have $\wh{f*K}=\wh f*\wh K$ so 
\[
\ve{f*K}_2=\ve{\wh f\wh K}_2\le \ve{f}_2\ve{\wh K}_{\iy}.
\]
If $\wh K$ is bounded then $Tf = f*K$ is bounded on $L^2$.
Given the inequality for $1<p<2$, we get the inequality for $2<p<\iy$ by a duality argument. 
Let $2<p<\iy$ and $p^*$ be such that $\rc p +\rc{p^*}=1$. Then
\bal
\ve{Tf}_{p}&=\sup_{\ve{g}_{p^*}=1} \int Tf(x)g(x)\,dx \\
&=\sup \iint f(y)K(x,y) g(x)\,dy\,dx\\
&=\sup \int f(y) T^* g(y)\\
&\le \sup_{\ve{g}_{p^*}=1} \ve{f}_p\ve{g}_{p^*}.
\end{align*}
%usually forced at 1 $\iy$.
\end{enumerate}

\end{enumerate}

\end{proof}
\section{Ch 2 continued}

\fixme{Everything after this is messy}
We'll discuss:
\begin{enumerate}
\item
Atomic decomposition
\item Averge of general collections of balls.
\item Singular approximations to the identity.
\end{enumerate}
\begin{df}
Define $F:\R_+^{n+1}\to \R$, 
\[
F^*(x) = \sup_{|x-y|<t}|F(y,t)|.
\]
\end{df}

Let $\ph=|B(0,r)|^{-1}$. Define $F(x,t)=f*\ph_t$, $\ph_t(x)=t^{-n}\ph\pf nt$. Then $F^*$ is the uncentered maximal function $\cal F f$.
Define $F^*(x)=\sup_{|y-x|<t}|F(y,t)|$. $F\in N$, tent space, if $F^*\in L^1(\R^n)$. $\ve{F}_N:=\ve{F^*}_{L^1}$.

split tent function into atoms which are easy to understand individually.

The closer a point is to the boundary, the smaller.

$T(0)=\bigcup_{x\in O}T(B(x,d(x,O^c)))$.

\subsection{Atomic decomposition}
\begin{df}
An \textbf{atom} is a function supported on $T(B)$ for some $B\subeq \R^n$ such that $|a|\le |B|^{-1}$.
%a^* supported on the ball, so the tent norm of $a$ satisfies $\ve{a^*}_N\le 1$.
\end{df}
In particular, $\ve{a^*}_N\le 1$.
\begin{thm}
If $F\in N$, then there exist atoms $\{a_k\}$ and $\{\la_k\}$ (scalars) such that 
\[
F=\sum_k \la_ka_k\in N
\]
and $\sum_k \la_k\le C\ve{F}_N$.
\end{thm}

Let $\cal B$ be a collection of balls in $\R^n$. Define the maximal function 
\[
(M_{\cal B}f)(x) = \sup_{B\in \cal B} \rc{|B|} \int_B |f(x-y)|\,dy.
\]
Does
\[
\rc{|B|}\int_B f(x-y)\,dy \to f(x)\qquad \text{a.e.}?
\]
%If I scale each ball by $B$, it will contain the origin.
E.g. if $|C(B)|\le k\diam(B)$, then $|M_{\cal B}(f)|\le k^nM(f)$.

The boundedness of the ordinary maximal function implies the boundedness of this maximal functions.

Consider the larger collection of balls 
\[
\ol{\cal B} = \set{B'}{B'\supeq B\text{ for some }B\in \cal B}.
\]
This is a  kind of completion process. If you can bound the maximal operator for the original collection, then you can bound it for the completed collection. Let
\[
\cal B(r)=\bigcup_{B\in \ol{\cal B},\text{radius}(B)=r} B.
\]

\begin{thm}
If $|\cal B(r)|\le cr^n$, then $M_{\cal B}$ is bounded on $L^p$ for all $1<p\le \iy$ and also on $L^{1,\iy}$.

Conversely, if $M_{\cal B}$ is bounded on some $L^{p_0}$, then $|\cal B(r)|\le cr^n$.
\end{thm}
%Union of $n$ balls that covers $r$, then this is true. Conversely, if you have this inequality, then you have a fixed $n$ so that you have balls of radius $O(r)$, such that it has a cover of that many balls of that radius. 

\begin{proof}
$\ve{M\ol{\cal B}}_{p_0}\le A\ve{M\cal B}_{p_0}$.
Define $m_B=|B|^{-1} \chi_B$. Let $B_{2r}$ be the bll of $2\text{radius}(\ol B)$ centered at 0. Claim: $2^{-n}m_{\ol B}\le m_{B_{2r}}*m_B$.

Then $M_{\ol{\cal B}}f\le M(M_{\cal B}f)$.
For $x\in \ol B$, $B(x,2r)\supeq \ol{B}\supeq B$.
\[
(m_{B_{2r}}* m_B) (x)=\rc{|B_{2r}|}=2^{-n} \rc{|B_r|} = 2^{-n}\rc{|\ol B|}.
\]
Using the boundedness of the ordinary maximal operator $M$,
\[
\ve{M_{\ol{\cal B}}f}_{p_0}\le A\ve{M_{\cal B}f}_B.
\]
Test function $\chi_{B_{2r}(0)}$. Claim: $\cal B(r)\subeq \set{x}{(M_{\cal B}f)(x)\ge 1}$. %If know have bound on $L^p$ norm 
This implies that $\vol(\ol{\cal B}(r))\le A\vol(B_{2r})$.

To show the converse, study the tent space. %If we want maximal functions with respect to general collection of balls
Given $F:\R_+^{n+1}\to \R$, define $F_{\ol{\cal B}}^*(x)=\sup_{B(y,t)\in \ol{\cal B}}|F(x-y,t)|$.

Claim: $\int_{\R^n}F_{\ol{\cal B}^*}(x)\,dx\le C\int_{\R^n} F^*(x) \,dx$.

Use atomic decomposition: Let $a$ be an atom supported on $B(0,r)$, assume $a_{\cal B}^*(x)\ne 0$. Then for some $(y,t)$, $B(y,t)\in \ol{\cal B}$, then $|x-y|\le r-t$. This implies $x\in B(y,r)\in \ol{\cal B}$. Thus $x\in \ol{B(r)}$.
%true for general F. To weak apply to char func of distribution function.

\[
\ve{a_{\cal B}}_{L^1} \le |B(0,r)|^{-1}\int_{\cal B(r)}\,dx\le C.
\]
We obtain that $M_{\ol{\cal B}}$ is bounded on $L^p$ and $L^{1,\iy}$ for $1< p\le \iy$. 
$F(x,t)=f*\ph_t$, $\ph_t$ distribution function.
\end{proof}

Recall the setup. Suppose $\ph\ge 0$ is bounded and integrable $\int \ph=1$. If $\ph$ has sufficiently uniform decay at $\iy$, then 
\[
(f*\ph_t)(x)=\int_{\R^n}f(x-ty)\ph(y)\,dy\to f(x)\text{ a.e.}
\]
where $\ph_t(x)=t^{-n}\ph\pf xt$. This is true if $|\ph(x)|\le (1+|x|)^{-n-\ep}$, $\ep>0$.
%closely connected to the study of the maximal operator.

We weaken the conditions; then the same techniques as before cannot be applied. 
Claim: If $\ph(rx)$ is decreasing in $r$, for all $0\ne x\in \R^n$ ($\ph$ bounded, $\ge 0$, integrable), then 
\[
(M\ph)(x)=\sup_{t>0}|(f*\ph_t)(x)|
\]
is bounded on $L^p$ for $1<p<\iy$. 
%a.e., need weak $1,1$ bound. It's open whether there is a weak $1,1$ bound.

The proof uses the method of rotations.
\begin{proof}
We want to bound (uniformly in $t>0$)
\[
\int_{|\xi|=1}\int_0^{\iy} f(x-rs) \ph_t(rs)r^{n-1}\,dr\,d\xi
\]
We claim that this is $\le \int_{|\xi|=1} (M^{\xi}f)(x)\int_0^{\iy} \ph_t(-s) r^{n-1}\,dr\,d\xi$.
Here $(M^{(\xi)})=\sup_{r>0}\rc{2r}\int_{-r}^r |f(x-t\xi)|\,dt$ is the maximal function in 1 direction.
Claim: $\ve{M^{(\xi)}f}_p\le A_p \ve{f}_p$
precompose with element of orthogonal group, can assume in $x_1$ direction, integrate with respet to $x_1$, then other coordinates.

Firstly, 
\[
\int_{|\xi|=1}\int_0^{\iy} f(x-r\xi)\ph(r\xi)|r|^{n-1} \,dr\,d\xi.
\]
We want to apply the result, majorizing $\ph(r\xi)|r|^{n-1}$ by a radially decreasing function
\[
\ph(r\xi) |r|^{n-1}\le \int_{|r|}^{\iy}s^{n-1} \,d\ph_S(rs)=\psi(r)
\]
by integration by parts.
We calculate
\[
\iiiy -\psi(r)\,dr=\int_0^{\iy} d_r\psi(r)\le A\int \ph.
\]
Then 
\[
\int_{|\xi|=1}\int_0^{\iy} f(x-r\xi)\psi(r)|r|^{n-1}\,dr\,d\xi\le \int_{|\xi|=1}(M^{(\xi)}f)(x)\iiy \ph(r\xi)r^{n-1}\,dr\,d\xi.
\]
Use the Minkowski inequality and the claim bounding $\ve{M^{(\xi)}(f)}_p$. Then $\ve{M\ph f}_p\le A\ve{f}_p$ for $1<p\le \iy$.
\end{proof}

Let $\eta$ be a Dini modulus of continuity. This means that
\begin{enumerate}
\item
\[
\int_0^1 \fc{\eta(s)}{s}\,ds<\iy,
\]
\item $\eta(0)=0$,
\item $\eta:[0,1]\to \R$,
\item $\eta$ is non-decreasing.
\end{enumerate}
If $\eta(s)=O(s^{\ep})$ for $\ep>0$ then $\eta$ is a Dini modulus of continuity.
(The inverse function of $e^{-\rc{x^2}}$ is not a Dini modulus of continuity.)
Without this condition, the following question is open. 
\begin{thm}
If $\ph:\R^n\to \R$ and if 
\[
\int_{\R^n}|\ph(x-y)-\ph(x)|\,dy\le \eta(|y|)
\]
and if 
\[
\int_{|x|\ge R} |\ph(x)|\le \eta(R),
\]
 then $M\ph$ is bounded $L^1\to L^{1,\iy}$.
\end{thm}
If $\ph$ is compactly supported, this is satisfied.
\begin{proof}
If $\ph(rx)$ is decreasing in $r$ for $x\ne 0$, then 
\[
M_\ph f=\sup_{t>0}|f*\ph_t| \le 2 \sup |f*\ph_{2j}|
\]
%radially decreasing
%2^j\ge t.
Claim: there exists $A$ such that
\bal
\int_{|x|\ge 2|y|} \sup|\ph_{2j}(x-y)-\ph_{2j}(x)|&\le A\\
&\le \ub{\sum_{2^j<|y|} \int_{|x|\ge 2|y|} |\ph_{2^j}(x-y)-\ph_{2^j}(x)|\,dx}{(1)} + \ub{\sum_{|y|\le 2j}\int_{|x|\ge 2|y|}|\ph_{2^j}(x-y)-\ph_{2^j}(x)|\,dx}{(2)}\\
(1)&\le 2\sum_{2^j<|y|}\int_{|x|\ge 2|y|}|\ph(x-2^{-j}y)-\ph(x)|\,dx\\
&\le 2\sum_{2^j<|y|}\eta(2^{-j}|y|)\\
&\le 4\sum_{2^j<|y|}\fc{\eta(2^{-j}|y|)}{2^{-j}|y|}2^{-j-1}|y|\\
&\le 4\int_0^1 \fc{\eta(s)}{s}\,ds<\iy\\
(2)&\le 4 \sum_{2^j<|y|}\int_{|x|\ge 2|y|}|\ph(2^{-j}|y|)|\,dx\\
&\le 4 \sum_{2^j<|y|} \eta(2^{-j}|y|)\\
&\le 4 \int_0^1 \fc{\eta(s)}{s}\,ds.
\end{align*}
%(1) using a change of variables, Riemann sum.

If $K=\{\ph_{2^j}\}_{j\in\Z}$, we want a bound $f\to f*K$. Using the Banach space version of boundedness of singular integrals, the result follows.
\end{proof}


%\chapter{Motivation}

%We'll see lots of analysis in action.
\section{Dirac delta function}

What is the derivative of the Dirac delta?

You may have seen the mysterious ``\textbf{Dirac delta function}" defined by 
\beq{eq:dist0-1}
\int_{-\iy}^{\iy} \de(x-x_0)f(x)\,dx=f(x_0), \qquad \de(x)=0,\,x\ne 0.
\eeq
This emerged from Fourier's classical treatise on heat. It was there implicitly in his work. Cauchy and Dirac noticed it. It is used in math, applied math, physics, engineering.
%does the job we're asking it

But there is no function in any sense of the word that does this job! It makes no mathematical sense!

%Let's look at~\eqref{eq:dist0-1} in an abstract sense. 
However, looking at~\eqref{eq:dist0-1} in an abstract sense, the ``process" which takes $\de(x-x_0)$ and the ``nice" function $f(x)$, and spits out $f(x_0)$ is well-defined.
%There's some $\de(x-x_0)$, given $f(x)$, some process happens and spits out $f(x_0)$. 

However, people don't just talk about the delta function, they also talk about its derivative! Trying to differentiate something that doesn't exist...? %We'll put on our applied maths hat on and try to define the derivative.

How can we define the derivative $\de'(x-x_0)$? A first attempt might be (assuming $f$ is nice)
\begin{align}
\nonumber
\int \de'(x-x_0)f(x)\,dx&=\lim_{h\to 0} \int 
\pf{\de(x-x_0+h)-\de(x-x_0)}{h}f(x)\,dx\\
\nonumber
&=\lim_{h\to 0} \fc{f(x_0-h)-f(x_0)}{h}\\
\nonumber
&=-f'(x_0)\\
\llabel{eq:dist0-2}
&=-\int \de(x-x_0)f'(x)\,dx.
\end{align}
The equality~\eqref{eq:dist0-2} suggests that we could have simply integrated by parts
\[
\int \de'(x-x_0)f(x)\,dx=-\int \de(x-x_0)f'(x)\,dx+\ub{\text{boundary term}}{0}.
\]
This function that doesn't exist seems to follow the usual rules of calculus! 
This suggests there is a way of interpreting all the integrals in a consistent way
We can make all this rigorous using the theory of distributions. %looks like the usual rules of integral calculus can be applied to $\de(x-x_0)$.

\section{Fourier transforms of polynomials}

The \textbf{Fourier transform} is defined by (abbreviate $\int:=\iiy$)
\[
\cal F:f\mapsto \wh f(x)=\int e^{-i\la x}f(x)\,dx.
\]
This makes sense if $f$ is absolutely integrable:
\[
\ab{
\int e^{-i\la x}f(x)
}\le \int |e^{-i\la x}f(x)|\,dx=\int |f(x)|\,dx<\iy.
\]
What if $f\nin L^1$, in particular, what if $|f|\not\to 0$ as $|x|\to \iy$? You may have seen
\beq{eq:dist0-3}
\de(\la)=\rc{2\pi}\int e^{-i\la x}\,dx.
\eeq
There is a way of interpreting this so that it makes sense. This seems to suggest that the Fourier transform of $\rc{2\pi}$ is equal to $\de(\la)$. What if $f(x)=x^n$? Can we take the Fourier Transform?
\bal
\int e^{-i\la x}x^n\,dx&=\int \pa{i\pd{}{\la}}^n e^{-i\la x}\,dx\\
&=\pa{i\pd{}{\la}}^n \int e^{-i \la x}\,dx\\
&=2\pi e^{-i \pi n/2}\de^{(n)}(\la).
\end{align*}
If we can make sense on the derivatives of $\de$, then we can define the Fourier transform of polynomials.

An alternative way of defining Fourier transform of $f(x)=x^n$ would be to use Parseval's Theorem, which states
\beq{eq:dist0-4}
\int f(x)\wh g(x)\,dx =\int \wh f(\la)g(\la)\,d\la
\eeq
for all ``nice" functions $f$ and $g$. We could define $\hat f(\la)$ where $f(x)=x^n$, as the function for which
\beq{eq:dist0-5}
\int x^n\hat g(x)\,dx=\int \hat f(\la)g(\la)\,d\la\text{ for all nice functions } g
\eeq
we could say that $\hat f(\la)$ is the Fourier transform of $f(x)=x^n$. Note $\hat g(x)$ decays quickly, so this makes sense. This can be done rigorously using the theory of distributions. %If we can find $\hat f(x)$

``Everything's easy when you know the answer." It's only perfectly natural when you've been shown its perfectly natural. To prove consistency is not quite so easy.

\section{Discontinuities to Differential Equations}
%we want to see things like that happen.

There are some important genuinely important physical scenarios in which we would like a solution to a PDE to have discontinuities. For example, in acoustics we want the pressure $p(x,t)$ to solve the wave equation
\beq{eq:dist0-6}
p_{xx}-p_{tt}=0,
\eeq
but for $p$ to jump either side of a shock wave.
%propagate out?
%pavilion g

Is there any meaningful way to say that the function 
\[
u(x,y)=\al(x-y)+\be(x+y)
\]
is a solution to the wave equation $u_{xx}-u_{yy}=0$ if $\al,\be \nin C^2$? Assume $\al,\be\in C^2$ and $u_{xx}-u_{yy}=0$ so that for any nice function $f(x,y)$ (say $f=0$ when $x^2+y^2$ is sufficiently large),
\bal
0&=\int\int f(x,y)\pa{\pd{{}^2u}{x^2}-\pd{{}^2u}{y^2}}\,dx\,dy\\
&=\int\int u(x,y) \pa{\pd{{}^2}{x^2}-\pd{{}^2f}{y^2}}\,dx\,dy
\text{integration by parts twice}.
\end{align*}
If we can find $u(x,y)$ such that
\beq{eq:dist0-7}
\int\int u(x,y) \pa{\pd{{}^2}{x^2}-\pd{{}^2f}{y^2}}\,dx\,dy=0
\text{ for all nice functions }f,
\eeq
we say that $u=u(x,y)$ is a \textbf{weak solution} to the PDE $u_{xx}-u_{yy}=0$. We can use distribution theory to study weak solutions to PDE's.
%3 reasons want formalize

\section{Summary}

In each motivating example we introduced a family of ``nice" functions that allowed us to extend classical definitions to a wider class of problems. This is the underlying idea in distribution theory. Given a vector space $V$ of ``nice" functions we define the distributions on $V$ to be the topological dual $V^*$, which consists of all the continuous linear forms $V\to \C$. 

For example, if $V=C(\R)$ then we can define Dirac delta by the linear form\footnote{You may be more familiar with the notation $\an{x,y}=x\cdot y$ for finite-dimensional vector spaces.}
\beq{eq:dist0-8}
\an{\de_{x_0},f}=f(x_0).
\eeq
%linear map.
In general, any $u\in V^*$ is linear, so $\an{u, \al f+\be g}=\al\an{u,f}+\be \an{u,g}$ for arbitrary constants $\al,\be$ and $f,g\in V$. We need functional analysis because we require continuity. (The algebraic dual is too big to be interesting. Hence we supplement $V$ with a topology, i.e. a notion of convergence $f_n\to V$ in $V$. This is the motion of $w^*$-convergence, $\an{u,f_n}\to \an{u,f}$.

\blu{Lecture 2}
\chapter{Distributions}

Recap:
\begin{itemize}
\item
Delta function doesn't make sense.
\item
Way to define distributions is to first define a ``nice" space of functions $V$ (having all the properties we want it to have) and define distributions as continuous linear maps from $V$ to $\C$. 
\end{itemize}
%V has all properties we want it to have.

We'll always work with continuous functions, so we can define continuity of functions $V\to \C$ very explicitly.

\section{Notation and preliminaries}

An element of $\R^n$ will be written $x,y,z,\ldots$ so that 
\[
x=(x_1,x_2,\ldots, x_n)
\]
and we will use $dx=dx_1\,dx_2\,\cdots \,dx_n$ to denote the volume element in $\R^n$. Capitals $X,Y,Z$ will always denote open subsets of $\R^n$ and $K$ will always denote a compact (closed and bounded) subset of $\R^n$. Integrals over all $\R^n$ or over $X\subeq \R^n$ will be denoted by $\int [\cdot]\,dx$ and $\int_X[\cdot]\,dx$, respectively.
We will use multi-index notation $\al,\be,\ga$ (Greek letters) will denote multi-indices $\al=(\al_1,\ldots, \al_n)\in \Z_+^n=\{0,1,2,3,\ldots\}^{\times n}$. Multi-index notation reads as follows.
\bal
\pl^{\al} &\equiv\pa{\pdd{x_1}}^{\al_1}\cdots \pa{\pdd{x_n}}^{\al_n},&x^{\al}&\equiv x_1^{\al_1}\cdots x_n^{\al_n}\\
\al!&\equiv \al_1!\cdots \al_n!&|\al|&\equiv \al_1+\cdots +\al_n.
\end{align*}
%functions for which it's switched on.
We will often write $\pl^{\al}_x$ to make it clear what we're differentiating with respect to. We will also use $D=-i\pl$ when we do Fourier analysis. Define the \textbf{support} of a function $f$ by 
\[
\Supp(f)=\ol{\set{x\in \R^n}{f(x)\ne 0}}.
\]
We will often want to take limits inside integrals. To do this we refer to the dominated convergence theorem: (See for instance~\url{https://dl.dropboxusercontent.com/u/27883775/math\%20notes/18.125.pdf}, Theorem 15.1.)
\begin{thm}[Dominated convergence theorem]\llabel{thm:dct}
Given a sequence of absolutely integrable functions $\{f_m\}_{m\ge 1}$ such that $f_m(x)\to f(x)$ for each $x$ and $|f_m(x)|\le g(x)$, where $g$ is absolutely integrable, then 
\[
\lim_{m\to \iy}\int_Xf_m(x)=\int_X\ba{\lim_{m\to \iy} f_m(x)}\,dx=\int_X f(x)\,dx.
\]
\end{thm}
If $f$ is absolutely integrable on $X$, i.e.,
\[
\int_X |f|\,dx<\iy,
\]
we say that $f\in L^1(X)$.

\section{Test functions and distributions}
We need to define our first vector space of test functions.
\begin{df}
The space $D(X)$ consists of all the smooth functions from $X$ to $\C$ that have compact support. We say that a sequence $\{\ph_m\}_{m\ge1}$ tends to zero in $D(X)$ if there exists some compact set $K\subeq X$ such that $\Supp(\ph_m)\subeq K$ and $\pl^{\al}\ph_m\to 0$ uniformly for each multi-index $\al$.

The space $D(X)$ is often written $C^{\iy}_c(X)$.
\end{df}
(For convergence, the function is not allowed to have its mass moving away to infinity.)
%integrate by parts, evaluated on the boundary, derivatives shift to the other side.

Since the functions in $D(X)$ vanish at the boundary of $X$, we have
\[
\int_X\ph \pl^{\al}\psi\,dx=(-1)^{\al}\int_X\psi\pl^{\al}\ph\,dx,
\qquad \ph,\psi\in D(X)
\]
by integration by parts $|\al|$ times. We have Taylor's Theorem to any order
\beq{eq:taylor}
\ph(x+h)=\sum_{|\al|\subeq N}\fc{h^{\al}}{\al!} \pl^{\al}\ph(x)+R_N(x,h)
\eeq
where the remainder is $o(|h|^N)$ uniformly in $x$. %treat it as  ncice polynomial plus remainder.

%we have space of tests. distributions will be defined as linear maps. We want them to be continuous. $u(\ph)\equiv \an{u,\ph}\in \C$. If $\ph_m\to 0$ in $D(X)$, $\an{u,\ph_m}\to 0$ in $\C$.
\begin{df}\llabel{df:distribution}
A linear map $u:D(X)\to \C$ is a \textbf{distribution} (on $X$) if
for each compact $K\subeq X$, there exist $C,N$ such that 
\beq{eq:dist1-1}
|\an{u,\ph}|\le C\sum_{|\al|\le N} \sup|\pl^{\al}\ph|
\eeq
%vs sequential continuity. 
for each $\ph\in D(X)$ with $\Supp(\ph)$ with $\Supp(\ph)\subeq K$. The space of all such maps is denoted $D'(X)$. If we can use the same $N$ for all $\ph\in D(X)$, we call the least such $N$ the order of $u$, denoted $\ord(u)$.
\end{df}
%in the back of mind have this example, go back to familiar example.
\begin{rem}
Thinking of $D(X)$ as a locally convex space (in fact, Fr\'echet space) with seminorms defined by $\sup|\pl^{\al}\ph|$, we have $D'(X)=D(X)^*$. See Example 2.2.4 in Functional Analysis\footnote{\url{https://dl.dropboxusercontent.com/u/27883775/math\%20notes/part_iii_functional.pdf}}.
The example there is actually a larger space ($\cal E(X)$ of the next chapter), but it contains our $D(X)$.
\end{rem}

\begin{ex}
\begin{enumerate}
\item
We check that the Dirac delta $\de_x$ is a distribution. $\de_x$ is defined by
\[
\an{\de_x,\ph}=\ph(x)\qquad\ba{\int \de(x-y)\ph(y)\,dy=\ph(x)}.
\]
We want to check if~\eqref{eq:dist1-1} holds:
\[
\ab{\an{\de_x,\ph}}=|\ph(x)|\le \sup|\ph|
\] 
so~\eqref{eq:dist1-1} holds with $C=1$, $N=0$,  no matter what $\ph$ we choose. So $\de_x$ is a distribution of order 0.
\item
Here is a more useful example. Consider the linear map $T$ on $D(X)$ defined by 
\[
\an{T,\ph}:=\sum_{|\al|\le M} \int_X f_{\al}\pl^{\al} \ph\,dx,
\]
$f_{\al}\in C(X)$. Now for $\ph\in D(X)$, with $\Supp(\ph)\subeq K$. By definition of $T$, 
\bal
|\an{T,\ph}|&=\ab{
\sum_{|\al|\le M}\int_K f_{\al} \pl^{\al}\ph\,dx
}\\
&\le \sum_{|\al|\le M} \int_K |f_{\al}||\pl^{\al} \ph|\,dx\\
&\le \sum_{|\al|\le M} \sup_{|\al|\le M}|\pl^{\al}\ph| \int_K |f_{\al}|\,dx\\
&\le \pa{\max_{\al}\int_K |f_{\al}|\,dx}\sum_{|\al|\le M} \sup |\pl^{\al} \ph|
\end{align*}
Note that the test functions have compact support, so it doesn't matter if $f_{\al}$ blows up at the boundary. So we have estimate~\eqref{eq:dist1-1} with 
\[
C=\max_{|\al|\le M}\int_K |f_{\al}|\,dx,\qquad N=M.
\]
Note that $C=C_K$. We could have done this only assuming that $\{f_{\al}\}$ were locally integrable on $X$ (integrable on every compact subset of $X$), written $f_{\al}\in L_{\text{loc}}^1(X)$.

Note here the constant $C$ depends on the support test function. $N$ can also depend on it.
%$N$ can also depend on the support of the test function.
\end{enumerate}
\end{ex}
\begin{lem}\llabel{lem:dist1-1}
A linear map $u:D(X)\to \C$ belongs to $D'(X)$ if $\an{u,\ph_m}\to 0$ for every sequence $\{\ph_m\}_{m\ge 1}$ in $D(X)$ that tends to 0.
\end{lem}
\blu{Lecture 3 (27 Jan)} 
\begin{rem}
That Definition~\ref{df:distribution} and Lemma~\ref{lem:dist1-1} are equivalent conditions for continuity is a special case of Lemma 2.2.5 in the functional analysis notes.
\end{rem}
\begin{proof}
\begin{itemize}
\item
$\implies$: If $u\in D(X)$ and $\ph_m\to 0$ in $D(X)$ then
\[
|\an{u,\ph_m}|\le \sum_{|\al|\le m} \sup|\pl^{\al} \ph_m|\to 0.
\]
\item
$\Leftarrow$: Assume not, so there exists a compact set $K\subeq X$ such that estimate~\eqref{eq:dist1-1} does not hold for any $C,N$. In particular, it doesn't hold for $C=N=m$. So there exist $\phi_m\in D(X)$ such that $|\an{u,\phi_m}|>m\sum_{|\al|\le m}\sup |\pl^{\al} \phi_m|$. WLOG, we can assume that $\an{u,\phi_m}=1$, by setting $\wt{\phi_m}=\fc{\phi_m}{\an{u,\phi_m}}$. This implies
\bal
\implies \sum_{|\al|\le m} \sup|\pl^{\al}\phi_m|&<\rc m\\
\implies \sup|\pl^{\al}\phi_m|&<\rc m,&|\al|\le m\\
\implies \phi_m\to 0& \text{ in }D(X).
\end{align*}
But this is a contradiction.
\end{itemize}
\end{proof}
\section{Limits in $D'(X)$}
Often we have  sequence of distributions $\{u_m\}_{m\ge 1}$. If there exist some $u\in D'(X)$ such that $\an{u_m,\ph}\to \an{u,\ph}$ for all $\ph\in D(X)$, then we say that $u_m\to u$ in $D'(X)$.

Limits in $D'(X)$ often look strange. 
\begin{ex}
For instance, define the distribution $u_m\in D'(\R)$ by the locally integrable function
\[
u_m(X):=\sin (mx).
\]
Then $u_m\to 0$ in $D'(\R)$. 

%compact support, ibp no boundary terms.
Proof: We have using \blu{integration by parts}
\bal
\an{u_m,\ph}&=\int \sin(mx)\ph(x)\,dx\\
&=\rc{m} \int \cos(mx)\ph'(x)\,dx
\end{align*}
Hence $u_m\to 0$ in $D'(\R)$. 
\end{ex}
%think about Fourier expansion of $\ph$
\begin{thm}[Closure under pointwise convergence]\llabel{thm:dist1-1}
If $u_m\in D'(X)$ is such that $\lim_{m\to \iy}\an{u_m,\ph}$ exists for every $\ph\in D(X)$, then the linear map
\[
\an{u,\ph}:=\lim_{m\to \iy}\an{u_m,\ph}
\]
is an element of $D'(X)$. 
\end{thm}
It's obvious that the LHS will satisfy the estimate or the definition. Use the principle of uniform boundedness (See the Banach-Steinhaus Theorem, 4.2.7 in FA notes).

\section{Basic operations}

\subsection{Differentiation and multiplication by smooth functions} 
If $u\in C^{\iy}(X)$, then $\pl^{\al}u$ defines an element of $D'(X)$ for every multi-index $\al$ by
\bal
\an{\pl^{\al}u,\ph}&=\int_{X}\pl^{\al}u\ph\,dx\\
&=(-1)^{|\al|} \int_X u\pl^{\al} \ph\,dx&\text{integration by parts}\\
&=\an{u,(-1)^{|\al|}\pl^{\al}\ph}.
\end{align*}
%add in f
%Makes sense for any distribution, so it's a good definition.
The RHS is well-defined for any $u\in D'(X)$. We arrive at the following.
\begin{df}\llabel{df:dist1-3}
For $f\in C^{\iy}(X)$, $u\in D'(X)$ and any multi-index $\al$ we define $\pl^{\al}(fu)$ by 
\[
\an{\pl^{\al}(fu),\ph}:=\an{u,(-1)^{|\al|} f\pl^{\al}\ph}.
\]
We call $\pl^{\al}u$ the \textbf{distributional derivatives} of $u$.
\end{df}
\begin{ex}
Take the Dirac delta $\de_x$. Then $\pl^{\al}\de_x$ is defined by
\[
\an{\pl^{\al}\de_x,\ph}:=\an{\de_x,(-1)^{|\al|} \pl^{\al}\ph} = (-1)^{|\al|} \pl^{\al}\ph(x).
\]

Consider the Heaviside function
\[
H(x)=\begin{cases}
1,&x>0\\
0,&x\le 0.
\end{cases}
\]
Then this defines an element of $D'(\R)$. We compute $H'$:
\[
\an{H',\ph}:=\an{H, -\ph'}=\int_0^{\iy} -\ph'(x)\,dx = -\ph|^{\iy}_0=\ph(0)=\an{\de_0,\ph}.
\]
Hence $H'=\de_0$. In general if $\an{u,\ph}=\an{v,\ph}$ for all $\ph\in D(X)$ then we say $u=v$ in $D'(X)$. 
%gradient infinite.
\end{ex}
Now we ask: how does the calculus for normal functions carries over to calculus for distributions?
\begin{lem}
If $u'=0$ in $D'(\R)$ then $u$ is a constant.
\end{lem}
\begin{proof}
Note that $u'=0$ means 
\[0=\an{u',\psi}=-\an{u,\psi'}.\]
%every function that is a total derive.
%{\it Note that }

{\it \blu{Idea: we'd like to say that given $\ph$, we can write $\ph=\ddd x\int_{-\iy}^x \ph(y)\,dy=\psi'$, and use the above to conclude $0=\an{u,\ph}$, so $u$ is constant. The problem is that when we integrate a test function, we don't necessarily get a test function. We need to adjust our function so that the integral is 0 for large $x$.}}

Fix $\te\in D(\R)$ with $\an{1,\te}=\int \te\,dx=1$. For arbitrary $\ph\in D(\R)$ write 
\bal
\ph&=(\ph-\an{1,\ph}\te)+\an{1,\ph}\te\\
&\equiv \ph_A+\ph_B
\end{align*}
Then 
\[
\an{1,\ph_A} =\an{1,\ph}-\an{1,\ph}\an{1,\te}=0.
\]
This is helpful because 
\[
\psi_A(x)=\int_{-\iy}^x \ph_A(y)\,dy\in D(\R).
\]
%
We have $\ph_A=\psi_A'$. So 
\bal
\an{u,\ph}&=\an{u,\ph_A}+\an{u,\ph_B}\\
&=\an{u,\psi_A'}+\an{1,\ph}\ub{\an{u,\te}}b=0+c\an{1,\ph} =\an{c,\ph}.
\end{align*}
This implies that $u$ is a constant.
\end{proof}
\subsection{Reflection and translation}
\begin{df}
For $\ph\in D(\R^n)$ then we can define its \textbf{translation} by $h\in \R^n$ by 
\[
(\tau_h\ph)(x):=\ph(x-h)
\]
and \textbf{reflection}
\[
\check \ph(x)=\ph(-x).
\]
\end{df}
(Motivation: $\an{u,\ph}=\int u\ph\,dx$ gives
$\an{\tau_h u,\ph}=\int u(x-h)\ph(x)\,dx=\int u(x)\ph(x+h)\,dx=\an{u,\tau_{-h}\ph}$.) By duality, the definitions of these operations on $u\in D'(\R^n)$,
\bal
\an{\tau_hu,\ph}&:= \an{u,\tau_{-h}\ph}\\
\an{\check u,\ph}&:=\an{u,\check{\ph}}.
\end{align*} 
\begin{lem}\llabel{lem:dist1-3}
For $u\in D'(\R^n)$ define
\[
v_h=\fc{\tau_{-h} u-u}{|h|}.
\]
Then $v_h\to n\cdot \pl u$, where $\lim_{h\to 0} \fc{h}{|h|}=n\in S^{n-1}$. 
\end{lem}
\begin{proof}
We have
\bal
\an{v_h,\ph}&=\an{\fc{\tau_{-h}u-u}{|h|},\ph}\\
&=\an{u,\fc{\tau_h\ph-\ph}{|h|}}.
\end{align*}
By Taylor's Theorem,
\bal
\tau_h \ph(x)-\ph(x)&=-h\pl \ph(x)+\ub{R(x,h)}{o(|h|)\text{ in }D(\R^n)}\\
\an{u,\fc{\tau_h\ph-\ph}{|h|}}&=\an{u,-\fc{h}{|h|}\pl \ph}+\ub{\an{u,\fc{(R(\cdot ,h))}{h}}}{\to 0\text{ as $|h|\to 0$}}\\
&=n\cdot \an{\pl u, \ph}.
\qedhere
\end{align*}
\end{proof}
This shows the distributional derivative coincides with the normal notion of derivative as difference quotient.

\blu{Lecture 4 (29 Jan)}

\subsection{Convolution between $D(\R^n)$ and $D'(\R^n)$}
If we combine the operations of reflection and translation, we get 
\[
(\tau_x\check{\ph})(y)=\check{\ph} (y-x)=\ph(x-y).
\]
If $u\in D(\R^n)$, we define the \textbf{convolution} of $u$ and $\ph\in D(\R^n)$ with
\[
(u*\ph)(x):=\int u(x-y)\ph(y)\,dy=\int \ph(x-y)u(y)\,dy=(\ph*u)(x).
\]
Using $\tau_h$ and $\check{\bullet}$, we can write 
\[
u*\ph(x)=\an{u,\tau_x \check{\ph}}.
\]
This is well defined for all $u\in D'(\R^n)$.
\begin{df}\llabel{df:dist1-5}
For $u\in D'(\R^n)$ and $\ph\in D(\R^n)$, define their convolution by 
\[u*\ph(x)=\an{u,\tau_x\check{\ph}}.\]
\end{df}
It is clear that $u*\ph(x)$ is just some function of $x\in \R^n$. It is actually smooth.
\begin{lem}\llabel{lem:dist1-4}
Define $\Phi_x(y)=\phi(x,y)$ where $\phi\in C^{\iy}(\R^n\times \R^n)$ and $\phi(\cdot, y)=0$ for $y$ outside some compact $K\subeq \R^n$. Then for $u\in D'(\R^n)$, 
\[
\pl_x^{\al} \an{u,\Phi_x}=\an{u, \pl^{\al}_x \Phi_x}.
\]
\end{lem}
We can take the derivatives inside the bracket.
\begin{proof}
By Taylor's theorem,  
%left with is linear map on $H$.
%use defn of $D'$
\[
\Phi_{x+h}(y)-\Phi_x(y)=\sum_i h_i \pd{\phi}{x_i} (x,y)+R_x(y,h)
\]
It is not difficult to show that $R_x(y,h)=o(|h|)$ in $D(\R^n)$ for each $x\in \R^n$. Hence
\[
\an{u,\Phi_{x+h}}-\an{u,\Phi_x}=\sum_i h_i \an{u,\pl{\Phi_x}{x_i}}+\an{u,R_x(\cdot, h)}.
\]
%sequence of test functions tend to 0, then continuity means $u$ acting on that sequence tends to 0.
Since $R_x(\cdot, h)=o(|h|)$ in $D(\R^n)$, dividing by $|h|$ and taking $h\to 0$ gives ($u=\fc{h}{|h|}$)
\[
n\cdot \an{u, \Phi_x}=n\cdot \an{u,\pl_x\Phi}.
\]
So the result follows.
\end{proof}
\begin{cor}\llabel{cor:dist1-1}
If $u\in D'(\R^n)$ and $\ph\in D(\R^n)$ then $u*\ph$ is smooth and $\pl^{\al}(u*\ph)=u*\pl^{\al}\ph$. 
\end{cor}
\begin{proof}
By Lemma~\ref{lem:dist1-4}, $\pl^{\al}(u*\ph)=\pl^{\al}_x\an{u,\tau_x \check{\ph}}=u*\pl^{\al}\ph$.
\end{proof}
%dist are normal functions, just take a derivative at the end.
\subsection{Density of $D(\R^n)$ in $D'(\R^n)$}
We have just seen the following.\\

\cpbox{
If $u\in D'(\R^n)$ and $\ph\in D(\R^n)$ then $u*\ph$ is smooth.}
\vskip0.15in
This is extremely useful. No matter how wild $u$ is, $u*\ph$ is nice. For this reason, $u*\ph$ is often called a  \textbf{regularisation} of the distribution $u$.
%wild creatures. give me any wild distribution, convolute with anything, get a smooth function. This is what we'll use to prove the density result.
We will use this fact to prove $D(\R^n)$ is dense in $D'(\R^n)$, i.e., for each $u\in D^1(\R^n)$ there exists a sequence of test functions $\{\ph_m\}_{m\ge 1}$ in $D(\R^n)$ such that $\ph_m\to u$ in $D'(\R^n)$, i.e., $\an{\ph_m,\te}\to \an{u,\te}$ for all $\te\in D(\R^n)$.

\blu{How to apply this: Suppose we have a problem about a distribution we don't know anything about. We know for some sequence of $\phi_m$, $\ph_m=u*\phi_m\to u$. We replace $u$ with $\ph_m$, do manipulations there, run the argument and take a limit.}


We need a technical lemma.
\begin{lem}\llabel{lem:dist1-5}
For $\ph,\psi\in D(\R^n)$ and $u\in D'(\R^n)$ we have
\[
(u*\ph)*\psi=u*(\ph*\psi).
\]
%insert dummy var
\end{lem}
\begin{proof}
The LHS is
\bal
(u*\ph)*\psi(x)&=\int (u*\ph)(x-y)\psi(y)\,dy\\
&=\int \an{u,\tau_{x-y} \wh \ph}\psi(y)\,dy\\
&=\int\an{u(z),\ph(x-y-z)\psi(y)}\,dy&\text{$z$ is ``dummy" variable}\\
&=\lim_{h\to 0} \sum_{m\in \Z^n}\an{u(z), \ph(x-z-hm)\psi(hm)h^n}&\text{Riemann sum}\\
&=\lim_{h\to 0}\an{u(z), \sum_{m\in\Z^n}\ph(x-z-hm)\ph(hm)h^n}
\end{align*}
(We want to take the $\int$ inside the $\an{\cdot}$, and we do this by turning it into a Riemann sum. Note the sum only has finitely many terms for each $m$, so this is legal.)
%sum eventually dead, only finitely many terms.
It is not hard to show that 
\[
\sum_{m\in \Z^n}\ph(x-hm)\psi(m)h^n\to \ph*\psi(x)\text{ in }D(\R^n)\text{ as }h\to 0.
\]
Continuing the above calculation,
\bal
&=\an{u(z),\ph*\psi(x-z)}\\
&=\an{u,\tau_x (\ph*\psi)\check{\,}}\\
&=u*(\ph*\psi)(x).
\end{align*}
%compact support, LHS has compact support, by fixed compact set, can differentiate
\end{proof}
\begin{thm}\llabel{thm:dist1-2}
$D(\R^n)$ is dense in $D'(\R^n)$. 
\end{thm}
We would like to say
\[
u(x)=\int\de(x-y)u(y)\,dy\stackrel?= \lim_{m\to \iy} \int \de_m(x-y)u(y)\,dy
\]
The $\de_m(x)$ are such that $\int \de_m(x)\,dx=1$ and get squashed closer and closer to $\de$. %(``a family of good kernels").
\begin{proof}
Fix $\psi\in D(\R^n)$ with $\int \psi\,dx=1$ and set
\[
\phi_m(x)=m^n \psi(mx)\qquad \pa{\int \phi_m\,dx=\int \psi\,dx=1}.
\]
%convolution of look-like de with original
Also introduce the bump function $\chi\in D(\R^n)$ with $\chi=1$ on $|x|<1$ and $\chi=0$ on $|x|>1$. Now set $\chi_m\pf xm$ and
\[
\ph_m=\pur{\chi_m(x)}(u*\phi_m)(x).
\]
%why can't $\ph$ of this guy?
(The purpose of $\pur{\chi_m(x)}$ is to make $\ph_m$ have compact support.)
Choose $\an{\ph_m,\te}$ for $\te\in D(\R^n)$ arbitrary, giving
\bal
\an{\ph_m,\te}&=\an{u*\phi_m, \chi_m\te}\\
&=(u*\phi_m)*(\chi_m\te)\check{\,}(0)\\
%&=u*\check{\ph}(0)\\
&=u*(\phi_m*(\chi_m \te)\check{\,})(0)\\
\phi_m*(\chi_m\te)\check{\,}(x)&=\int\phi_m (x-y)\chi_n(-y)\te(-y)\,dy\\
&=\int m^n \phi(m(x-y))\chi\pf{-y}{m}\te(-y)\,dy& y'=m(x-y)\implies y=x-\fc{y'}m\\
%Make the substitution 
&=\int\phi(y)\chi\pa{\fc y{m^2}-\fc xm}\te\pa{\fc ym -x}\,dy\\
&=\te(-x)+\ub{\int \phi(y)\chi\pa{\fc y{m^2}-\fc xm}\ba{
\te\pa{\fc ym-x}-\te(-x)
}\,dy}{=:R_m(-x)}\\
&=(\check{\te}+\check{R_m})(x).
\end{align*}
We can show $R_m\to 0$ in $D(\R^n)$. So \[\an{\ph_m,\te}=\an{u,\te}+\ub{\an{u,R_m}}{\to 0\text{ as }m\to \iy}.\] Hence $\an{\ph_m,\te}\to \an{u,\te},\te\in D(\R^n)$, giving $\ph_m\to u$ in $D'(\R^n)$ i.e. $D(\R^n)$ dense in $D'(\R^n)$. 
\end{proof}

 
%\bibliographystyle{plain}
%\bibliography{refs}
\end{document}