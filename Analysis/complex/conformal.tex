\def\filepath{C:/Users/Owner/Dropbox/Math/templates}

\input{\filepath/packages_article.tex}
\input{\filepath/theorems_with_boxes.tex}
\input{\filepath/macros.tex}
\input{\filepath/formatting.tex}
%\input{\filepath/other.tex}

%\def\name{NAME}

%\input{\filepath/titlepage.tex}

\pagestyle{fancy}
%\addtolength{\headwidth}{\marginparsep} %these change header-rule width
%\addtolength{\headwidth}{\marginparwidth}
\lhead{Conformal Maps}
\chead{} 
\rhead{} 
\lfoot{} 
\cfoot{\thepage} 
\rfoot{} 
\renewcommand{\headrulewidth}{.3pt} 
%\renewcommand{\footrulewidth}{.3pt}
\setlength\voffset{0in}
\setlength\textheight{648pt}

\begin{document}

All sets will be connected unless said otherwise.
\section{Schwarz Reflection}
Schwarz reflection is a way to analytically continue a function.

\begin{thm}[Schwarz reflection]\llabel{thm:schwarz-reflection}
Suppose $\Om$ is an open set symmetric about the real axis. Let $\Om^+=\Om\cap \set{z}{\Im z>0}$ and similarly for $\Om^0,\Om^-$. Suppose $f^+,f^-$ are holomorphic on $\Om^+$ and $\Om^-$, can be {\it continuously} extended to $\Om^0$, and are equal there. 

Then the pasting $f$ of $f^+,f^-$ is holomorphic.
\end{thm}
\begin{proof}
We use Morera's Theorem. By splitting, we just have to consider curves (or triangles) one of whose edges are on the real axis. Use continuity to move the path upwards/downwards slightly.
\end{proof}

The following special case is useful.

\begin{cor}\llabel{cor:schwarz-reflection}
Let $\Om, \Om^{\pm}$ be as above. 
Suppose $f$ is holomorphic on $\Om^+$ and real on $\Om^0$. Then $f$ can be analytically continued to $\Om$.
\end{cor}
\begin{proof}
Let $f^-=\ol{f(\ol{z})}$ in Schwarz reflection~\ref{thm:schwarz-reflection}. (We have $\lim_{\ep\to 0}\fc{\ol{f(\ol{z+\ep})}-\ol{f(\ol{z})}}{\ep}=\ol{f'(\ol z)}$.)
\end{proof}

\section{Schwarz's lemma}
Let $D$ be the open disc.
\begin{lem}[Schwarz]\llabel{lem:schwarz}
Let $f$ be a map $D\to D$ with $f(0)=0$.

If either $f'(0)=1$ or $\sup_{z\in D} |f(z)|=1$, then $f(z)=cz$ for some $|c|=1$.
\end{lem}
We can use this to find $\Aut(D)$.
\begin{lem}\llabel{lem:autD}
The automorphism of $D$ are in the form
\[
f(z)=e^{i\te} \ub{\fc{\al - z}{1-\ol{\al}z}}{:=\psi_\al}
\]
for $\al\in D$. Here, $ \fc{\al - z}{1-\ol{\al}z}$ is an involution switching 0, $\al$, and $z\mapsto e^{i\te}z$ is a rotation by $\te$.

The automorphisms of $\cal H$ are in the form $z\mapsto \fc{az+b}{cz+d}$ with $\smatt abcd\in \SL_2(\R)$. We have $\Aut(D)\cong \Aut(\cal H)\cong \SL_2(\R)$.
\end{lem}
\begin{proof}[Proof of Lemma~\ref{lem:schwarz}]
By the maximal modulus principle, $\af{f(z)}{z}\le 1$ inside the disc. (Consider the circle $C_r$, and let $r\to 1^-$. This is necessary because $f$ many not be defined on $\pl D$.)

If equality is attained anywhere, then $\fc{f(x)}{x}$ is constant. 
\end{proof}

\begin{proof}[Proof of Lemma~\ref{lem:autD}]
Suppose $f\in \Aut(D)$ sends $\al$ to 0.
Then $f$ is the composition of $\fc{\al-z}{1-\ol{\al}z}$ and an automorphism fixing 0.

To get the automorphisms of $\cal H$, conjugate by the isomorphism between $D$ and $\cal H$.
\[
\xymatrix{
D\ar@/^1pc/[r]^{i\fc{1-z}{1+z}} & \cal H \ar@/^1pc/[l]^{\fc{i-z}{i+z}}
}
\]
The rest is calculation.
\end{proof}

Tip: How to remember/think about the maps? 
\begin{enumerate}
\item
On the boundary, the map $\cal H\to D$ is given by $\tan \te\mapsto \cos 2\te+i\sin 2\te$. By trig identities, this is $\fc{1-z^2}{1+z^2}+\fc{2zi}{1+z^2}=\fc{i-z}{i+z}$. 
\end{enumerate}
\section{On complex analytic maps}
\begin{thm}[Open mapping theorem]\llabel{thm:omt}
If $f$ is holomorpic on $\Om$ and $\forall z\in \Om, f'(z)\ne 0$, then $f$ is open.

Moreover, $f'(z_0)\ne0$ iff $f$ is injective in some open set around $z_0$.
\end{thm}
\begin{proof}
Use Rouch\'e's Theorem on $f(z)\approx f'(z_0)(z-z_0)+f(z_0)$ to show there is exactly 1 solution to $f(z)=y$ for $y$ close to $f(z_0)$. Thus the image of an open is open.

For the second part, use Rouch\'e on $f\approx f(z_0)+a_k(z-z_0)^k$.
\end{proof}

\begin{lem}
If $f:U\to V$ satisfies $\forall z\in U,f'(z)\ne 0$ and is bijective, then the inverse map is holomorphic.
\end{lem}
\begin{proof}
We can certainly define $f^{-1}$. Theorem~\ref{thm:omt} shows it's continuous. The standard calculation shows that $(f^{-1})'(f(z))=\rc{f'(z)}$, which is defined.

\end{proof}
Define a biholomorphic map to be $f:U\to V$ that is bijective holomorphic with $\forall z\in U,f'(z)\ne0$.
\section{Normal families}
Say a property holds compact-locally if it holds for every compact subset.
\begin{thm}[Arzela-Ascoli]\llabel{thm:arzela-ascoli}
\begin{enumerate}
\item
Suppose $X\subeq \R^n$ is compact. A closed and bounded equicontinuous family of functions in $C(X)$ is compact. 
In other words, if $\cal F$ is an infinite family of pointwise bounded equicontinuous functions in $C(X)$, then any sequence in $\cal F$ has a uniformly convergent subsequence. 
\item
The same is true of any open set $U$, if we ask these conditions hold compact-locally. (Define \textbf{normal} to mean every sequence has a subsequence converging compact-locally.)
\end{enumerate}
\end{thm}

\begin{proof}
See Real Analysis notes. For (2) diagonalize over an exhaustion by compact subsets (e.g., in $\R^n$).
\end{proof}
%Let $\cal F$ be an infinite family of functions. If $\cal F$ is compact-locally equicontinuous then there exists a subsequence that is compact-locally uniformly convergent.

In the $\C$ world, much looser criteria force $\cal F$ to be an  family of pointwise bounded equicontinuous functions.
\begin{thm}[Montel]\llabel{thm:montel}
If $\cal F$ is compact-locally uniformly bounded, then $\cal F$ is compact-locally equicontinuous and hence normal. 
\end{thm}
Counterexample for $\R$: $\sin(nx)$.
\begin{proof}
One way to show good continuity/limit/boundedness properties in complex analysis is to use the integral representation. Given $K$, for points $|x-y|<\ep_1$, write $f(x)=\rc{2\pi}\int_{C_{\ep_2}}\fc{f(z)}{z-x}\,dz$ over a circle containing $x,y$. Now bound the difference $f(x)-f(y)$ uniformly $\to 0$ in terms of $\sup_{z\in K, f\in \cal F} f$ as $\ep_1\to 0$.

Now use Arzela-Ascoli~\ref{thm:arzela-ascoli}.
\end{proof}

We'll take $\cal F$ to be an injective family of functions below, so we'll want this to be preserved under limit.
\begin{lem}\llabel{lem:inj-conv} (Pr. 8.3.5)
If $\Om$ is a connected open subset, $f_n$ are injective, and $f_n\to f$ compact-locally uniformly, then $f$ is injective or constant.
\end{lem}
\begin{proof}
Idea: we can count the number of solutions of $f_n(z)-w$ using a holomorphic function. By convergence, the number of solutions must stay constant.

Suppose $f(0)=0$ is a problem point, WLOG $f_n(0)=0$. Just use $\rc{2\pi i}\int_\ga \fc{f_n'}{f_n}\,dz$.
\end{proof}

\section{Riemann Mapping Theorem}
\begin{thm}[Riemann mapping theorem]\llabel{thm:rmt}
All proper simply connected open set in $\C$ are isomorphic (complex analytically).
\end{thm}

It suffices to prove that if $\Om$ is a simply connected open set then there is a biholomorphism $f:\Om \to D$.

\begin{proof}
\begin{enumerate}
\item
Construct an injective map $f:\Om \to D$ with $f'\ne 0$ and with 0 in the image.
\begin{enumerate}
\item
Find $f_1:\Om\to \Om_1$ where $\Om_1$ avoids an open set around a point. 
Let $f_1=\ln (x-a)$ where $a\nin \Om$; this is well-defined because $\Om$ is simply connected. 
Note $w\in f(\Om)$ implies $w+ 2\pi i\nin \ol{f(\Om)}$. (To see it's not in the closure, note otherwise (becuase $f$ is open) there is a point close by with $w',w'+2\pi i\in f(\Om)$, contradiction. Alternatively, replace 2 by $2+\ep$ so this argument is unnecessary.)
\item
Now $\Om_1$ avoids an open set around $\be$. Let $f_2=\fc{1}{z-\be}$; $f_2(\Om_1)$ is bounded in norm.
\item
Translate and dilate so it fits in the unit disc and includes 0.
\end{enumerate}
\item 
We can now assume $0\in \Om\subeq D$. 
Recast the existence problem as a maximization problem. Consider $\cal F:=\set{f:\Om \to D}{f\text{ injective, }f'\ne 0}$, and $s=\sup_{f\in \cal F}|f'(0)|$. Claim 1: The maximum is attained. Claim 2: $f$ attaining the maximum is an isomorphism $\Om \to D$.

Proof of Claim 1: $\cal F$ is uniformly bounded so normal by Theorem~\ref{thm:montel}. Thus there exists a subsequence $f_n\to f$ with $f_n'(0)\to s$. In the real world, this is not enough to say $f'(0)=s$. In the complex world it is, because we can express $f_n'$ again {\it in terms of $f_n$} using the integral representation, and just use uniform convergence of the $f_n$. We have $f\in \cal F$ by Lemma~\ref{lem:inj-conv} ($f$ can't be constant because $f'(0)\ne 0$.)
\item
Proof of Claim 2: Otherwise (if $f$ is not surjective) we exhibit $g$ with larger $|g'(0)|$. Idea: If $f$ is not surjective, write
\[
f=\Phi\circ F
\]
where $|\Phi'(0)|<1$ (proved using Schwarz's lemma~\ref{lem:schwarz}) forcing $|F'(0)|>|f'(0)|$. (Use chain rule.)

If $\Phi$ is a map $D\to D$ that is not injective, it can't be in the form $cz,|c|=1$, so by Schwarz must have $|\Phi'(0)|<1$. Thus we let $F$ be a function whose inverse is multivalued on $D$ but that is injective on $\Om$. Use the square root!

BWOC, let $\al\nin f(\Om)$. $F=\psi_{\sqrt{\al}}\circ \sqrt{}\circ \psi_\al\circ f$ and let $\Phi$ be the inverse $\psi_{\sqrt{\al}}\circ \sqrt{}\circ \psi_\al$. This gives $F\in \cal F$ with $|F'(0)|$ larger, contradiction.
\end{enumerate}
\end{proof}

The Riemann mapping theorem does not tell us what happens on the boundary of the open sets (if the map can be extended, etc.). We show the map behaves nicely in the case of a polygon.

\begin{thm}[Extending to the boundary]\llabel{thm:extend-boundary}
Let $D,P$ be open sets with ``regular" boundary, i.e., for every point $p\in \pl S$, there is a small open set $U\ni p$ with $U\cap \pl S$ topologically equivalent to a line segment. (Ex. $D$ is a disc, $P$ is a polygon.)
%Let $P$ be a polygon; 
Suppose $f:D\to P$  is an isomorphism. Then $f$ can be extended to a continuous bijection function $\ol{D}\to \ol{P}$.
\end{thm}
The idea is that if there is $>1$ possible value for $\lim_{n\to \iy} f(z_n)$ given $z_n\to z\in \pl D$, then area would not be preserved. There would be $z_n,z_n'$ such that $|z_n-z_n'|\to 0$ but $f(z_n)-f(z_n')$ is large.

\begin{lem}\llabel{lem:area}
Let $f:U\to V$ be an isomorphism. Then $\Area V = \int |f'(z)|^2\,dx\,dy$.
\end{lem}
\begin{proof}
Think of $f$ as a map $\R^2\to \R^2$. The factor is the Jacobian. Calculate it using the Cauchy-Riemann equations.
\end{proof}

\begin{proof}[Proof of Theorem~\ref{thm:extend-boundary}]
\begin{enumerate}
\item (4.5) First, to prove continuity at $z\in \pl D$, using the triangle inequality, it suffices to consider nearby points $z\in \pl D$ and $z'\in D$.
\item
If $z_n\to z$ then by (sequential) compactness of $\ol{P}$, $f(z_n)$ has a limit point. If for each such sequence $f(z_n)$ has only 1 limit point it must be the actual limit. Thus if $f$ does not extend continuously to $z$, then there are sequences $z_n,z_n'$ such that $f(z_n),f(z_n')$ converge to different values $w,w'$. This suggests $f(x)-f(y)$ can stay away from 0 even though $x,y\to z$. 

To formalize this, suppose $f(z_n),f(z_n')$ converge to different values. We find continuous curves $\ga(r),\ga'(r)$ containing $z_n,z_n'$ ($r$ is distance to $z$) such that $|\ga(r)-\ga'(r)|>\de$ for $r,\de$ small enough. To do this, choose disjoint open balls around $w,w'$ and take inverse images of curves containing $f(z_n),f(z_n')$ there.

(4.3) Let $\ga(r)= z+re^{i\te_1(r)}$ and similarly for $\ga'$. 
Let $U$ be the region between $\ga,\ga'$ from radius 0 to $s$.
We have
\bal
\int_{\te_1(r)}^{\te_2(r)}r|f'(z)|\,dz& = \int_{\ga(r)}^{\ga'(r)} |f'(z)|\,dz\ge \de\\
\iy > \Area(f(U))&=\iint_{U}|f'(z)|^2\,dz\\
&= \int_0^s\int_{\te_1(r)}^{\te_2(r)} r|f'(z)|^2 \,d\te\,dr&(z=x+re^{i\te}))\\
&\ge \int_0^s\fc{(\int_{\te_1}^{\te_2} r|f'(z)|\,d\te)^2}{\int_{\te_1}^{\te_2} r\,d\te}\,dr&\text{C-S}\\
&\ge \int_0^s\fc{c}{r}\,dr= \iy.
\end{align*}
\item Now apply to $P\to D$ as well to get $g=f^{-1}$.
\end{enumerate}
\end{proof}

\section{Schwarz-Christoffel formula}

The Riemann Mapping Theorem does not give an explicit formula. We give an explicit formula for $\cal H\to P$, $P$ a polygon.

How might we create one? The key is knowing how to the straight line $\R$ to a single sharp corner. Then multiplying functions together we can create a series of corners.

To get one corner, consider the function $f_\be(z)= z^{\be}$ which maps $\R$ to the rays of an angle $\be\pi$. 
Now we would like a function on $\R$ such that $\arg F$ is a piecewise step function, so that if we integrate it we will move along the sides of a polygon. Just multiply translates of functions $f_\be$ and integrate. This motivates the formula below.
% with internal angle $\al\pi$. We can have corners at $A_1,A_2$ by considering $f_{\al_1}(z-A_1)$, $f_{\al_2}(z-A_2)$.
%$\int_{0}^z \ze^{-\be}\,d\ze$.
\begin{thm}[Schwarz-Christoffel]\llabel{thm:schwarz-christoffel}
\begin{enumerate}
\item
Suppose $\be_1,\ldots, \be_k>0$ with $1<\sum_k \be_k\le 2$ and $A_1<\cdots <A_n$ be real. 
The following is an analytic map $\cal H\to P$ for some polygon $P$.
\[
S(z)=\int_{-\iy}^z\prod_k (\ze-A_k)^{-\be_k}\,d\ze.
\]
$P$ starts at $(0,0)$ in the $+\Re$ directions and its edges going around have exterior angles $\be_k\pi$ and $\be_0\pi:= (2-\sum_k \be_k)\pi$. $A_kA_{k+1}$ gets mapped to the edge forming exterior angle $\be_k\pi$ (absolute angle $\sum_k \be_k\pi$), where $A_{n+1}=A_0:=\iy$.
\item For any such polygon, there is a isomorphism $\cal H\to P$ that is given by $c_1S(z)+c_2$ for some $c_1,c_2$.
\end{enumerate}
\end{thm}
Let $a_k$ be the image of $A_k$. 
Note that if $\sum_k \be_k=2$, then the last exterior angle is 0, i.e. $a_{\iy}$ is on the segment $a_na_1$.
\begin{proof}
We prove the case where $\sum_k \be_k=2$ first.
\begin{enumerate}
\item
To see how it maps $\R$, note that 
\[
\arg \ddd z = \arg \prod_k (z-A_k)^{-\be_k} = \sum_{A_k>z} -\be_k\pi.
\]
(The $-$ is so that the angle increases by $\be_k\pi$ after passing $A_k$, and to make things convergent.)

Note $S$ is well-defined because $\be_k<1$, so the integral converges around $A_k$; $\sum b_k>1$, so $\lim_{|z|\to \iy}S(z)$ exists (because integrating around a large arc gives something on the order of $r^{-\sum b_k +1}$). 
\item
Let $f$ be the isomorphism given by Riemann mapping~\ref{thm:rmt}, extend it to the boundary by~\ref{thm:extend-boundary}, and let $A_k$ be the preimages of the vertices (maybe one of them is $\iy$, in which we will be trying to match $f$ up with $S$ in the case $\sum \be_k<2$). 

We will use a Liouville-type argument to show $f$ is given by the (a rotation+translation of the) Schwarz function. We can't hope to apply such an argument to $f$ because $f$ can't extend. We instead apply it to $\fc{f'}{f}$. 

We want a function we can extend by Schwarz reflection. Thus we get rid of the corner. Let $h_k = (f(z) - a_k)^{\rc{\al_k}}$. Then $h_k$ sends $A_{k-1}A_{k+1}$ to a straight line. By Schwarz reflection~\ref{cor:schwarz-reflection} we can extend $h_k$ to the strip between $A_{k-1}$ and $A_{k+1}$. (We are in the case of~\ref{cor:schwarz-reflection} once we map the straight line into the real axis.) (Check $h_k'\ne 0$ by checking it is injective on small balls around real points (This suffices by Theorem~\ref{thm:omt}.). Check the latter by examining the Schwarz construction.)

Computing the derivatives gives an expression for $\fc{f''}{f'}$ in terms of $h_k$ (without all those exponents). (We expect $\fc{f''}{f'}=(\ln f)'$ to be nice because if $f=S(z)$ this would be nice.) Its only pole in the strip is $A_k$. Since the strips cover $\C$, this analytically continues $\fc{f''}{f'}$.

In the case of $[A_n,\iy), (-\iy,A_1]$, we can in fact analytically continue $f$ across these, and find its continuation there to be bounded (because $P$ is bounded, and the construction is just reflection), so $f$ is holomorphic at $\iy$. Expanding in power series gives: if $f$ is holomorphic at $\iy$ then $\lim_{|z|\to \iy}\fc{f''}{f'}\to 0$.

Using Liouville, $\fc{f''}{f'}=-\sum \fc{\be_k}{z-A_k}$. Integrating gives $F'=cS'$.
\item In the case where one of the vertices is mapped from $\iy$, use a M\"obius map to reduce to the above case.
\end{enumerate}
\end{proof}

\section{Examples, Elliptic integrals}
\begin{thm}
The function
\[
\int_0^z \rc{\sqrt{(1-\ze^2)(1-k^2\ze^2)}}\,d\ze
\]
is a function from $\cal H$ to a rectangle of side lengths 
\bal
A&=2\int_0^1\rc{\sqrt{(1-\ze^2)(1-k^2\ze^2)}}\,d\ze\\
B&=\int_1^{\rc k} \rc{\sqrt{-(1-\ze^2)(1-k^2\ze^2)}}\,d\ze.
\end{align*}
It maps 
\bal
\iy&\mapsto -Bi\\
-\rc k &\mapsto \fc A2-Bi\\
-1&\mapsto \fc A2\\
0&\mapsto 0\\
1&\mapsto -\fc A2\\
\rc k & \mapsto -\fc A2 + Bi.
\end{align*}
The inverse function can be meromorphically extended to the whole plane and is elliptic with lattice $\an{2A,2Bi}$. In fact it equals $\fc{c}{\wp(z)-\wp(Bi)}$ for some $c$. \fixme{Find it!}
\end{thm}
Note we can also get a version for $\ze(\ze-1)(\ze-\la)$. (These 2 forms of an elliptic curve have always puzzled me, but now it's clear: the corresponding functions on $\cal H$ are related by  a $\fc{az+b}{cz+d}$ transformation that takes $-\rc k, -1, 1, \rc k$ to $1,\la,\iy,0$.)
\begin{proof}
This function does what we claim by Schwarz-Christoffel~\ref{thm:schwarz-christoffel}. To extend the inverse function, use Schwarz reflection. These rectangles tile the plane. After reflecting 2 times we get back to the original function, so it is periodic with respect to the lattice $\an{2A,2Bi}$. Let $f$ be the inverse function.

Consider the function $(\wp, \wp'):\C\to E:=\{y^2=(x^2-1)(k^2x^2-1)\}$ (include the point at infinity). We have maps
\[
\xymatrix{
E\ar[d]_{(x,y)\mapsto x} & \\
\cal H & \C\ar[l]_{f} \ar[lu]_{(\wp,\wp')}
}
\]
We have $(\wp,\wp')^*\fc{dx}y = dz$; restricting to the original domain of $f$, we find this means $\wp^*\fc{d\ze}{\sqrt{\cdots}} = dz$. By FTC, $(\int_0^z\rc{\sqrt{\cdots}} \,d\ze=\cdots(z))^* dz = \fc{d\ze}{\sqrt{\cdots}}$. The differentials agree on an open set so must agree everywhere. This tells us that $\wp = f$ up to translation by some constant. \fixme{But what about the fact the pole changed?}
\end{proof}

%\chapter{Motivation}

%We'll see lots of analysis in action.
\section{Dirac delta function}

What is the derivative of the Dirac delta?

You may have seen the mysterious ``\textbf{Dirac delta function}" defined by 
\beq{eq:dist0-1}
\int_{-\iy}^{\iy} \de(x-x_0)f(x)\,dx=f(x_0), \qquad \de(x)=0,\,x\ne 0.
\eeq
This emerged from Fourier's classical treatise on heat. It was there implicitly in his work. Cauchy and Dirac noticed it. It is used in math, applied math, physics, engineering.
%does the job we're asking it

But there is no function in any sense of the word that does this job! It makes no mathematical sense!

%Let's look at~\eqref{eq:dist0-1} in an abstract sense. 
However, looking at~\eqref{eq:dist0-1} in an abstract sense, the ``process" which takes $\de(x-x_0)$ and the ``nice" function $f(x)$, and spits out $f(x_0)$ is well-defined.
%There's some $\de(x-x_0)$, given $f(x)$, some process happens and spits out $f(x_0)$. 

However, people don't just talk about the delta function, they also talk about its derivative! Trying to differentiate something that doesn't exist...? %We'll put on our applied maths hat on and try to define the derivative.

How can we define the derivative $\de'(x-x_0)$? A first attempt might be (assuming $f$ is nice)
\begin{align}
\nonumber
\int \de'(x-x_0)f(x)\,dx&=\lim_{h\to 0} \int 
\pf{\de(x-x_0+h)-\de(x-x_0)}{h}f(x)\,dx\\
\nonumber
&=\lim_{h\to 0} \fc{f(x_0-h)-f(x_0)}{h}\\
\nonumber
&=-f'(x_0)\\
\llabel{eq:dist0-2}
&=-\int \de(x-x_0)f'(x)\,dx.
\end{align}
The equality~\eqref{eq:dist0-2} suggests that we could have simply integrated by parts
\[
\int \de'(x-x_0)f(x)\,dx=-\int \de(x-x_0)f'(x)\,dx+\ub{\text{boundary term}}{0}.
\]
This function that doesn't exist seems to follow the usual rules of calculus! 
This suggests there is a way of interpreting all the integrals in a consistent way
We can make all this rigorous using the theory of distributions. %looks like the usual rules of integral calculus can be applied to $\de(x-x_0)$.

\section{Fourier transforms of polynomials}

The \textbf{Fourier transform} is defined by (abbreviate $\int:=\iiy$)
\[
\cal F:f\mapsto \wh f(x)=\int e^{-i\la x}f(x)\,dx.
\]
This makes sense if $f$ is absolutely integrable:
\[
\ab{
\int e^{-i\la x}f(x)
}\le \int |e^{-i\la x}f(x)|\,dx=\int |f(x)|\,dx<\iy.
\]
What if $f\nin L^1$, in particular, what if $|f|\not\to 0$ as $|x|\to \iy$? You may have seen
\beq{eq:dist0-3}
\de(\la)=\rc{2\pi}\int e^{-i\la x}\,dx.
\eeq
There is a way of interpreting this so that it makes sense. This seems to suggest that the Fourier transform of $\rc{2\pi}$ is equal to $\de(\la)$. What if $f(x)=x^n$? Can we take the Fourier Transform?
\bal
\int e^{-i\la x}x^n\,dx&=\int \pa{i\pd{}{\la}}^n e^{-i\la x}\,dx\\
&=\pa{i\pd{}{\la}}^n \int e^{-i \la x}\,dx\\
&=2\pi e^{-i \pi n/2}\de^{(n)}(\la).
\end{align*}
If we can make sense on the derivatives of $\de$, then we can define the Fourier transform of polynomials.

An alternative way of defining Fourier transform of $f(x)=x^n$ would be to use Parseval's Theorem, which states
\beq{eq:dist0-4}
\int f(x)\wh g(x)\,dx =\int \wh f(\la)g(\la)\,d\la
\eeq
for all ``nice" functions $f$ and $g$. We could define $\hat f(\la)$ where $f(x)=x^n$, as the function for which
\beq{eq:dist0-5}
\int x^n\hat g(x)\,dx=\int \hat f(\la)g(\la)\,d\la\text{ for all nice functions } g
\eeq
we could say that $\hat f(\la)$ is the Fourier transform of $f(x)=x^n$. Note $\hat g(x)$ decays quickly, so this makes sense. This can be done rigorously using the theory of distributions. %If we can find $\hat f(x)$

``Everything's easy when you know the answer." It's only perfectly natural when you've been shown its perfectly natural. To prove consistency is not quite so easy.

\section{Discontinuities to Differential Equations}
%we want to see things like that happen.

There are some important genuinely important physical scenarios in which we would like a solution to a PDE to have discontinuities. For example, in acoustics we want the pressure $p(x,t)$ to solve the wave equation
\beq{eq:dist0-6}
p_{xx}-p_{tt}=0,
\eeq
but for $p$ to jump either side of a shock wave.
%propagate out?
%pavilion g

Is there any meaningful way to say that the function 
\[
u(x,y)=\al(x-y)+\be(x+y)
\]
is a solution to the wave equation $u_{xx}-u_{yy}=0$ if $\al,\be \nin C^2$? Assume $\al,\be\in C^2$ and $u_{xx}-u_{yy}=0$ so that for any nice function $f(x,y)$ (say $f=0$ when $x^2+y^2$ is sufficiently large),
\bal
0&=\int\int f(x,y)\pa{\pd{{}^2u}{x^2}-\pd{{}^2u}{y^2}}\,dx\,dy\\
&=\int\int u(x,y) \pa{\pd{{}^2}{x^2}-\pd{{}^2f}{y^2}}\,dx\,dy
\text{integration by parts twice}.
\end{align*}
If we can find $u(x,y)$ such that
\beq{eq:dist0-7}
\int\int u(x,y) \pa{\pd{{}^2}{x^2}-\pd{{}^2f}{y^2}}\,dx\,dy=0
\text{ for all nice functions }f,
\eeq
we say that $u=u(x,y)$ is a \textbf{weak solution} to the PDE $u_{xx}-u_{yy}=0$. We can use distribution theory to study weak solutions to PDE's.
%3 reasons want formalize

\section{Summary}

In each motivating example we introduced a family of ``nice" functions that allowed us to extend classical definitions to a wider class of problems. This is the underlying idea in distribution theory. Given a vector space $V$ of ``nice" functions we define the distributions on $V$ to be the topological dual $V^*$, which consists of all the continuous linear forms $V\to \C$. 

For example, if $V=C(\R)$ then we can define Dirac delta by the linear form\footnote{You may be more familiar with the notation $\an{x,y}=x\cdot y$ for finite-dimensional vector spaces.}
\beq{eq:dist0-8}
\an{\de_{x_0},f}=f(x_0).
\eeq
%linear map.
In general, any $u\in V^*$ is linear, so $\an{u, \al f+\be g}=\al\an{u,f}+\be \an{u,g}$ for arbitrary constants $\al,\be$ and $f,g\in V$. We need functional analysis because we require continuity. (The algebraic dual is too big to be interesting. Hence we supplement $V$ with a topology, i.e. a notion of convergence $f_n\to V$ in $V$. This is the motion of $w^*$-convergence, $\an{u,f_n}\to \an{u,f}$.

\blu{Lecture 2}
\chapter{Distributions}

Recap:
\begin{itemize}
\item
Delta function doesn't make sense.
\item
Way to define distributions is to first define a ``nice" space of functions $V$ (having all the properties we want it to have) and define distributions as continuous linear maps from $V$ to $\C$. 
\end{itemize}
%V has all properties we want it to have.

We'll always work with continuous functions, so we can define continuity of functions $V\to \C$ very explicitly.

\section{Notation and preliminaries}

An element of $\R^n$ will be written $x,y,z,\ldots$ so that 
\[
x=(x_1,x_2,\ldots, x_n)
\]
and we will use $dx=dx_1\,dx_2\,\cdots \,dx_n$ to denote the volume element in $\R^n$. Capitals $X,Y,Z$ will always denote open subsets of $\R^n$ and $K$ will always denote a compact (closed and bounded) subset of $\R^n$. Integrals over all $\R^n$ or over $X\subeq \R^n$ will be denoted by $\int [\cdot]\,dx$ and $\int_X[\cdot]\,dx$, respectively.
We will use multi-index notation $\al,\be,\ga$ (Greek letters) will denote multi-indices $\al=(\al_1,\ldots, \al_n)\in \Z_+^n=\{0,1,2,3,\ldots\}^{\times n}$. Multi-index notation reads as follows.
\bal
\pl^{\al} &\equiv\pa{\pdd{x_1}}^{\al_1}\cdots \pa{\pdd{x_n}}^{\al_n},&x^{\al}&\equiv x_1^{\al_1}\cdots x_n^{\al_n}\\
\al!&\equiv \al_1!\cdots \al_n!&|\al|&\equiv \al_1+\cdots +\al_n.
\end{align*}
%functions for which it's switched on.
We will often write $\pl^{\al}_x$ to make it clear what we're differentiating with respect to. We will also use $D=-i\pl$ when we do Fourier analysis. Define the \textbf{support} of a function $f$ by 
\[
\Supp(f)=\ol{\set{x\in \R^n}{f(x)\ne 0}}.
\]
We will often want to take limits inside integrals. To do this we refer to the dominated convergence theorem: (See for instance~\url{https://dl.dropboxusercontent.com/u/27883775/math\%20notes/18.125.pdf}, Theorem 15.1.)
\begin{thm}[Dominated convergence theorem]\llabel{thm:dct}
Given a sequence of absolutely integrable functions $\{f_m\}_{m\ge 1}$ such that $f_m(x)\to f(x)$ for each $x$ and $|f_m(x)|\le g(x)$, where $g$ is absolutely integrable, then 
\[
\lim_{m\to \iy}\int_Xf_m(x)=\int_X\ba{\lim_{m\to \iy} f_m(x)}\,dx=\int_X f(x)\,dx.
\]
\end{thm}
If $f$ is absolutely integrable on $X$, i.e.,
\[
\int_X |f|\,dx<\iy,
\]
we say that $f\in L^1(X)$.

\section{Test functions and distributions}
We need to define our first vector space of test functions.
\begin{df}
The space $D(X)$ consists of all the smooth functions from $X$ to $\C$ that have compact support. We say that a sequence $\{\ph_m\}_{m\ge1}$ tends to zero in $D(X)$ if there exists some compact set $K\subeq X$ such that $\Supp(\ph_m)\subeq K$ and $\pl^{\al}\ph_m\to 0$ uniformly for each multi-index $\al$.

The space $D(X)$ is often written $C^{\iy}_c(X)$.
\end{df}
(For convergence, the function is not allowed to have its mass moving away to infinity.)
%integrate by parts, evaluated on the boundary, derivatives shift to the other side.

Since the functions in $D(X)$ vanish at the boundary of $X$, we have
\[
\int_X\ph \pl^{\al}\psi\,dx=(-1)^{\al}\int_X\psi\pl^{\al}\ph\,dx,
\qquad \ph,\psi\in D(X)
\]
by integration by parts $|\al|$ times. We have Taylor's Theorem to any order
\beq{eq:taylor}
\ph(x+h)=\sum_{|\al|\subeq N}\fc{h^{\al}}{\al!} \pl^{\al}\ph(x)+R_N(x,h)
\eeq
where the remainder is $o(|h|^N)$ uniformly in $x$. %treat it as  ncice polynomial plus remainder.

%we have space of tests. distributions will be defined as linear maps. We want them to be continuous. $u(\ph)\equiv \an{u,\ph}\in \C$. If $\ph_m\to 0$ in $D(X)$, $\an{u,\ph_m}\to 0$ in $\C$.
\begin{df}\llabel{df:distribution}
A linear map $u:D(X)\to \C$ is a \textbf{distribution} (on $X$) if
for each compact $K\subeq X$, there exist $C,N$ such that 
\beq{eq:dist1-1}
|\an{u,\ph}|\le C\sum_{|\al|\le N} \sup|\pl^{\al}\ph|
\eeq
%vs sequential continuity. 
for each $\ph\in D(X)$ with $\Supp(\ph)$ with $\Supp(\ph)\subeq K$. The space of all such maps is denoted $D'(X)$. If we can use the same $N$ for all $\ph\in D(X)$, we call the least such $N$ the order of $u$, denoted $\ord(u)$.
\end{df}
%in the back of mind have this example, go back to familiar example.
\begin{rem}
Thinking of $D(X)$ as a locally convex space (in fact, Fr\'echet space) with seminorms defined by $\sup|\pl^{\al}\ph|$, we have $D'(X)=D(X)^*$. See Example 2.2.4 in Functional Analysis\footnote{\url{https://dl.dropboxusercontent.com/u/27883775/math\%20notes/part_iii_functional.pdf}}.
The example there is actually a larger space ($\cal E(X)$ of the next chapter), but it contains our $D(X)$.
\end{rem}

\begin{ex}
\begin{enumerate}
\item
We check that the Dirac delta $\de_x$ is a distribution. $\de_x$ is defined by
\[
\an{\de_x,\ph}=\ph(x)\qquad\ba{\int \de(x-y)\ph(y)\,dy=\ph(x)}.
\]
We want to check if~\eqref{eq:dist1-1} holds:
\[
\ab{\an{\de_x,\ph}}=|\ph(x)|\le \sup|\ph|
\] 
so~\eqref{eq:dist1-1} holds with $C=1$, $N=0$,  no matter what $\ph$ we choose. So $\de_x$ is a distribution of order 0.
\item
Here is a more useful example. Consider the linear map $T$ on $D(X)$ defined by 
\[
\an{T,\ph}:=\sum_{|\al|\le M} \int_X f_{\al}\pl^{\al} \ph\,dx,
\]
$f_{\al}\in C(X)$. Now for $\ph\in D(X)$, with $\Supp(\ph)\subeq K$. By definition of $T$, 
\bal
|\an{T,\ph}|&=\ab{
\sum_{|\al|\le M}\int_K f_{\al} \pl^{\al}\ph\,dx
}\\
&\le \sum_{|\al|\le M} \int_K |f_{\al}||\pl^{\al} \ph|\,dx\\
&\le \sum_{|\al|\le M} \sup_{|\al|\le M}|\pl^{\al}\ph| \int_K |f_{\al}|\,dx\\
&\le \pa{\max_{\al}\int_K |f_{\al}|\,dx}\sum_{|\al|\le M} \sup |\pl^{\al} \ph|
\end{align*}
Note that the test functions have compact support, so it doesn't matter if $f_{\al}$ blows up at the boundary. So we have estimate~\eqref{eq:dist1-1} with 
\[
C=\max_{|\al|\le M}\int_K |f_{\al}|\,dx,\qquad N=M.
\]
Note that $C=C_K$. We could have done this only assuming that $\{f_{\al}\}$ were locally integrable on $X$ (integrable on every compact subset of $X$), written $f_{\al}\in L_{\text{loc}}^1(X)$.

Note here the constant $C$ depends on the support test function. $N$ can also depend on it.
%$N$ can also depend on the support of the test function.
\end{enumerate}
\end{ex}
\begin{lem}\llabel{lem:dist1-1}
A linear map $u:D(X)\to \C$ belongs to $D'(X)$ if $\an{u,\ph_m}\to 0$ for every sequence $\{\ph_m\}_{m\ge 1}$ in $D(X)$ that tends to 0.
\end{lem}
\blu{Lecture 3 (27 Jan)} 
\begin{rem}
That Definition~\ref{df:distribution} and Lemma~\ref{lem:dist1-1} are equivalent conditions for continuity is a special case of Lemma 2.2.5 in the functional analysis notes.
\end{rem}
\begin{proof}
\begin{itemize}
\item
$\implies$: If $u\in D(X)$ and $\ph_m\to 0$ in $D(X)$ then
\[
|\an{u,\ph_m}|\le \sum_{|\al|\le m} \sup|\pl^{\al} \ph_m|\to 0.
\]
\item
$\Leftarrow$: Assume not, so there exists a compact set $K\subeq X$ such that estimate~\eqref{eq:dist1-1} does not hold for any $C,N$. In particular, it doesn't hold for $C=N=m$. So there exist $\phi_m\in D(X)$ such that $|\an{u,\phi_m}|>m\sum_{|\al|\le m}\sup |\pl^{\al} \phi_m|$. WLOG, we can assume that $\an{u,\phi_m}=1$, by setting $\wt{\phi_m}=\fc{\phi_m}{\an{u,\phi_m}}$. This implies
\bal
\implies \sum_{|\al|\le m} \sup|\pl^{\al}\phi_m|&<\rc m\\
\implies \sup|\pl^{\al}\phi_m|&<\rc m,&|\al|\le m\\
\implies \phi_m\to 0& \text{ in }D(X).
\end{align*}
But this is a contradiction.
\end{itemize}
\end{proof}
\section{Limits in $D'(X)$}
Often we have  sequence of distributions $\{u_m\}_{m\ge 1}$. If there exist some $u\in D'(X)$ such that $\an{u_m,\ph}\to \an{u,\ph}$ for all $\ph\in D(X)$, then we say that $u_m\to u$ in $D'(X)$.

Limits in $D'(X)$ often look strange. 
\begin{ex}
For instance, define the distribution $u_m\in D'(\R)$ by the locally integrable function
\[
u_m(X):=\sin (mx).
\]
Then $u_m\to 0$ in $D'(\R)$. 

%compact support, ibp no boundary terms.
Proof: We have using \blu{integration by parts}
\bal
\an{u_m,\ph}&=\int \sin(mx)\ph(x)\,dx\\
&=\rc{m} \int \cos(mx)\ph'(x)\,dx
\end{align*}
Hence $u_m\to 0$ in $D'(\R)$. 
\end{ex}
%think about Fourier expansion of $\ph$
\begin{thm}[Closure under pointwise convergence]\llabel{thm:dist1-1}
If $u_m\in D'(X)$ is such that $\lim_{m\to \iy}\an{u_m,\ph}$ exists for every $\ph\in D(X)$, then the linear map
\[
\an{u,\ph}:=\lim_{m\to \iy}\an{u_m,\ph}
\]
is an element of $D'(X)$. 
\end{thm}
It's obvious that the LHS will satisfy the estimate or the definition. Use the principle of uniform boundedness (See the Banach-Steinhaus Theorem, 4.2.7 in FA notes).

\section{Basic operations}

\subsection{Differentiation and multiplication by smooth functions} 
If $u\in C^{\iy}(X)$, then $\pl^{\al}u$ defines an element of $D'(X)$ for every multi-index $\al$ by
\bal
\an{\pl^{\al}u,\ph}&=\int_{X}\pl^{\al}u\ph\,dx\\
&=(-1)^{|\al|} \int_X u\pl^{\al} \ph\,dx&\text{integration by parts}\\
&=\an{u,(-1)^{|\al|}\pl^{\al}\ph}.
\end{align*}
%add in f
%Makes sense for any distribution, so it's a good definition.
The RHS is well-defined for any $u\in D'(X)$. We arrive at the following.
\begin{df}\llabel{df:dist1-3}
For $f\in C^{\iy}(X)$, $u\in D'(X)$ and any multi-index $\al$ we define $\pl^{\al}(fu)$ by 
\[
\an{\pl^{\al}(fu),\ph}:=\an{u,(-1)^{|\al|} f\pl^{\al}\ph}.
\]
We call $\pl^{\al}u$ the \textbf{distributional derivatives} of $u$.
\end{df}
\begin{ex}
Take the Dirac delta $\de_x$. Then $\pl^{\al}\de_x$ is defined by
\[
\an{\pl^{\al}\de_x,\ph}:=\an{\de_x,(-1)^{|\al|} \pl^{\al}\ph} = (-1)^{|\al|} \pl^{\al}\ph(x).
\]

Consider the Heaviside function
\[
H(x)=\begin{cases}
1,&x>0\\
0,&x\le 0.
\end{cases}
\]
Then this defines an element of $D'(\R)$. We compute $H'$:
\[
\an{H',\ph}:=\an{H, -\ph'}=\int_0^{\iy} -\ph'(x)\,dx = -\ph|^{\iy}_0=\ph(0)=\an{\de_0,\ph}.
\]
Hence $H'=\de_0$. In general if $\an{u,\ph}=\an{v,\ph}$ for all $\ph\in D(X)$ then we say $u=v$ in $D'(X)$. 
%gradient infinite.
\end{ex}
Now we ask: how does the calculus for normal functions carries over to calculus for distributions?
\begin{lem}
If $u'=0$ in $D'(\R)$ then $u$ is a constant.
\end{lem}
\begin{proof}
Note that $u'=0$ means 
\[0=\an{u',\psi}=-\an{u,\psi'}.\]
%every function that is a total derive.
%{\it Note that }

{\it \blu{Idea: we'd like to say that given $\ph$, we can write $\ph=\ddd x\int_{-\iy}^x \ph(y)\,dy=\psi'$, and use the above to conclude $0=\an{u,\ph}$, so $u$ is constant. The problem is that when we integrate a test function, we don't necessarily get a test function. We need to adjust our function so that the integral is 0 for large $x$.}}

Fix $\te\in D(\R)$ with $\an{1,\te}=\int \te\,dx=1$. For arbitrary $\ph\in D(\R)$ write 
\bal
\ph&=(\ph-\an{1,\ph}\te)+\an{1,\ph}\te\\
&\equiv \ph_A+\ph_B
\end{align*}
Then 
\[
\an{1,\ph_A} =\an{1,\ph}-\an{1,\ph}\an{1,\te}=0.
\]
This is helpful because 
\[
\psi_A(x)=\int_{-\iy}^x \ph_A(y)\,dy\in D(\R).
\]
%
We have $\ph_A=\psi_A'$. So 
\bal
\an{u,\ph}&=\an{u,\ph_A}+\an{u,\ph_B}\\
&=\an{u,\psi_A'}+\an{1,\ph}\ub{\an{u,\te}}b=0+c\an{1,\ph} =\an{c,\ph}.
\end{align*}
This implies that $u$ is a constant.
\end{proof}
\subsection{Reflection and translation}
\begin{df}
For $\ph\in D(\R^n)$ then we can define its \textbf{translation} by $h\in \R^n$ by 
\[
(\tau_h\ph)(x):=\ph(x-h)
\]
and \textbf{reflection}
\[
\check \ph(x)=\ph(-x).
\]
\end{df}
(Motivation: $\an{u,\ph}=\int u\ph\,dx$ gives
$\an{\tau_h u,\ph}=\int u(x-h)\ph(x)\,dx=\int u(x)\ph(x+h)\,dx=\an{u,\tau_{-h}\ph}$.) By duality, the definitions of these operations on $u\in D'(\R^n)$,
\bal
\an{\tau_hu,\ph}&:= \an{u,\tau_{-h}\ph}\\
\an{\check u,\ph}&:=\an{u,\check{\ph}}.
\end{align*} 
\begin{lem}\llabel{lem:dist1-3}
For $u\in D'(\R^n)$ define
\[
v_h=\fc{\tau_{-h} u-u}{|h|}.
\]
Then $v_h\to n\cdot \pl u$, where $\lim_{h\to 0} \fc{h}{|h|}=n\in S^{n-1}$. 
\end{lem}
\begin{proof}
We have
\bal
\an{v_h,\ph}&=\an{\fc{\tau_{-h}u-u}{|h|},\ph}\\
&=\an{u,\fc{\tau_h\ph-\ph}{|h|}}.
\end{align*}
By Taylor's Theorem,
\bal
\tau_h \ph(x)-\ph(x)&=-h\pl \ph(x)+\ub{R(x,h)}{o(|h|)\text{ in }D(\R^n)}\\
\an{u,\fc{\tau_h\ph-\ph}{|h|}}&=\an{u,-\fc{h}{|h|}\pl \ph}+\ub{\an{u,\fc{(R(\cdot ,h))}{h}}}{\to 0\text{ as $|h|\to 0$}}\\
&=n\cdot \an{\pl u, \ph}.
\qedhere
\end{align*}
\end{proof}
This shows the distributional derivative coincides with the normal notion of derivative as difference quotient.

\blu{Lecture 4 (29 Jan)}

\subsection{Convolution between $D(\R^n)$ and $D'(\R^n)$}
If we combine the operations of reflection and translation, we get 
\[
(\tau_x\check{\ph})(y)=\check{\ph} (y-x)=\ph(x-y).
\]
If $u\in D(\R^n)$, we define the \textbf{convolution} of $u$ and $\ph\in D(\R^n)$ with
\[
(u*\ph)(x):=\int u(x-y)\ph(y)\,dy=\int \ph(x-y)u(y)\,dy=(\ph*u)(x).
\]
Using $\tau_h$ and $\check{\bullet}$, we can write 
\[
u*\ph(x)=\an{u,\tau_x \check{\ph}}.
\]
This is well defined for all $u\in D'(\R^n)$.
\begin{df}\llabel{df:dist1-5}
For $u\in D'(\R^n)$ and $\ph\in D(\R^n)$, define their convolution by 
\[u*\ph(x)=\an{u,\tau_x\check{\ph}}.\]
\end{df}
It is clear that $u*\ph(x)$ is just some function of $x\in \R^n$. It is actually smooth.
\begin{lem}\llabel{lem:dist1-4}
Define $\Phi_x(y)=\phi(x,y)$ where $\phi\in C^{\iy}(\R^n\times \R^n)$ and $\phi(\cdot, y)=0$ for $y$ outside some compact $K\subeq \R^n$. Then for $u\in D'(\R^n)$, 
\[
\pl_x^{\al} \an{u,\Phi_x}=\an{u, \pl^{\al}_x \Phi_x}.
\]
\end{lem}
We can take the derivatives inside the bracket.
\begin{proof}
By Taylor's theorem,  
%left with is linear map on $H$.
%use defn of $D'$
\[
\Phi_{x+h}(y)-\Phi_x(y)=\sum_i h_i \pd{\phi}{x_i} (x,y)+R_x(y,h)
\]
It is not difficult to show that $R_x(y,h)=o(|h|)$ in $D(\R^n)$ for each $x\in \R^n$. Hence
\[
\an{u,\Phi_{x+h}}-\an{u,\Phi_x}=\sum_i h_i \an{u,\pl{\Phi_x}{x_i}}+\an{u,R_x(\cdot, h)}.
\]
%sequence of test functions tend to 0, then continuity means $u$ acting on that sequence tends to 0.
Since $R_x(\cdot, h)=o(|h|)$ in $D(\R^n)$, dividing by $|h|$ and taking $h\to 0$ gives ($u=\fc{h}{|h|}$)
\[
n\cdot \an{u, \Phi_x}=n\cdot \an{u,\pl_x\Phi}.
\]
So the result follows.
\end{proof}
\begin{cor}\llabel{cor:dist1-1}
If $u\in D'(\R^n)$ and $\ph\in D(\R^n)$ then $u*\ph$ is smooth and $\pl^{\al}(u*\ph)=u*\pl^{\al}\ph$. 
\end{cor}
\begin{proof}
By Lemma~\ref{lem:dist1-4}, $\pl^{\al}(u*\ph)=\pl^{\al}_x\an{u,\tau_x \check{\ph}}=u*\pl^{\al}\ph$.
\end{proof}
%dist are normal functions, just take a derivative at the end.
\subsection{Density of $D(\R^n)$ in $D'(\R^n)$}
We have just seen the following.\\

\cpbox{
If $u\in D'(\R^n)$ and $\ph\in D(\R^n)$ then $u*\ph$ is smooth.}
\vskip0.15in
This is extremely useful. No matter how wild $u$ is, $u*\ph$ is nice. For this reason, $u*\ph$ is often called a  \textbf{regularisation} of the distribution $u$.
%wild creatures. give me any wild distribution, convolute with anything, get a smooth function. This is what we'll use to prove the density result.
We will use this fact to prove $D(\R^n)$ is dense in $D'(\R^n)$, i.e., for each $u\in D^1(\R^n)$ there exists a sequence of test functions $\{\ph_m\}_{m\ge 1}$ in $D(\R^n)$ such that $\ph_m\to u$ in $D'(\R^n)$, i.e., $\an{\ph_m,\te}\to \an{u,\te}$ for all $\te\in D(\R^n)$.

\blu{How to apply this: Suppose we have a problem about a distribution we don't know anything about. We know for some sequence of $\phi_m$, $\ph_m=u*\phi_m\to u$. We replace $u$ with $\ph_m$, do manipulations there, run the argument and take a limit.}


We need a technical lemma.
\begin{lem}\llabel{lem:dist1-5}
For $\ph,\psi\in D(\R^n)$ and $u\in D'(\R^n)$ we have
\[
(u*\ph)*\psi=u*(\ph*\psi).
\]
%insert dummy var
\end{lem}
\begin{proof}
The LHS is
\bal
(u*\ph)*\psi(x)&=\int (u*\ph)(x-y)\psi(y)\,dy\\
&=\int \an{u,\tau_{x-y} \wh \ph}\psi(y)\,dy\\
&=\int\an{u(z),\ph(x-y-z)\psi(y)}\,dy&\text{$z$ is ``dummy" variable}\\
&=\lim_{h\to 0} \sum_{m\in \Z^n}\an{u(z), \ph(x-z-hm)\psi(hm)h^n}&\text{Riemann sum}\\
&=\lim_{h\to 0}\an{u(z), \sum_{m\in\Z^n}\ph(x-z-hm)\ph(hm)h^n}
\end{align*}
(We want to take the $\int$ inside the $\an{\cdot}$, and we do this by turning it into a Riemann sum. Note the sum only has finitely many terms for each $m$, so this is legal.)
%sum eventually dead, only finitely many terms.
It is not hard to show that 
\[
\sum_{m\in \Z^n}\ph(x-hm)\psi(m)h^n\to \ph*\psi(x)\text{ in }D(\R^n)\text{ as }h\to 0.
\]
Continuing the above calculation,
\bal
&=\an{u(z),\ph*\psi(x-z)}\\
&=\an{u,\tau_x (\ph*\psi)\check{\,}}\\
&=u*(\ph*\psi)(x).
\end{align*}
%compact support, LHS has compact support, by fixed compact set, can differentiate
\end{proof}
\begin{thm}\llabel{thm:dist1-2}
$D(\R^n)$ is dense in $D'(\R^n)$. 
\end{thm}
We would like to say
\[
u(x)=\int\de(x-y)u(y)\,dy\stackrel?= \lim_{m\to \iy} \int \de_m(x-y)u(y)\,dy
\]
The $\de_m(x)$ are such that $\int \de_m(x)\,dx=1$ and get squashed closer and closer to $\de$. %(``a family of good kernels").
\begin{proof}
Fix $\psi\in D(\R^n)$ with $\int \psi\,dx=1$ and set
\[
\phi_m(x)=m^n \psi(mx)\qquad \pa{\int \phi_m\,dx=\int \psi\,dx=1}.
\]
%convolution of look-like de with original
Also introduce the bump function $\chi\in D(\R^n)$ with $\chi=1$ on $|x|<1$ and $\chi=0$ on $|x|>1$. Now set $\chi_m\pf xm$ and
\[
\ph_m=\pur{\chi_m(x)}(u*\phi_m)(x).
\]
%why can't $\ph$ of this guy?
(The purpose of $\pur{\chi_m(x)}$ is to make $\ph_m$ have compact support.)
Choose $\an{\ph_m,\te}$ for $\te\in D(\R^n)$ arbitrary, giving
\bal
\an{\ph_m,\te}&=\an{u*\phi_m, \chi_m\te}\\
&=(u*\phi_m)*(\chi_m\te)\check{\,}(0)\\
%&=u*\check{\ph}(0)\\
&=u*(\phi_m*(\chi_m \te)\check{\,})(0)\\
\phi_m*(\chi_m\te)\check{\,}(x)&=\int\phi_m (x-y)\chi_n(-y)\te(-y)\,dy\\
&=\int m^n \phi(m(x-y))\chi\pf{-y}{m}\te(-y)\,dy& y'=m(x-y)\implies y=x-\fc{y'}m\\
%Make the substitution 
&=\int\phi(y)\chi\pa{\fc y{m^2}-\fc xm}\te\pa{\fc ym -x}\,dy\\
&=\te(-x)+\ub{\int \phi(y)\chi\pa{\fc y{m^2}-\fc xm}\ba{
\te\pa{\fc ym-x}-\te(-x)
}\,dy}{=:R_m(-x)}\\
&=(\check{\te}+\check{R_m})(x).
\end{align*}
We can show $R_m\to 0$ in $D(\R^n)$. So \[\an{\ph_m,\te}=\an{u,\te}+\ub{\an{u,R_m}}{\to 0\text{ as }m\to \iy}.\] Hence $\an{\ph_m,\te}\to \an{u,\te},\te\in D(\R^n)$, giving $\ph_m\to u$ in $D'(\R^n)$ i.e. $D(\R^n)$ dense in $D'(\R^n)$. 
\end{proof}

 
%\bibliographystyle{plain}
%\bibliography{refs}
\end{document}