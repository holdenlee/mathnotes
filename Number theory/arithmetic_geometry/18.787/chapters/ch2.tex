\lecture{Tue. 9/11/12}

Last time we introduced group schemes. Let $S$ be a scheme; then $\gps\subeq \schs$ (but it is not a full subscheme). For $\ph:H\to G$ in $\gps$ we defined $\ker\ph$ by 
\[
(\ker\ph)(T)%=\ker(\ph(T))
=\ker(\ph(T):H(T)\to G(T)).
\]
and saw it was representable. 
What is the cokernel $\coker\ph$ (when the image of $\ph$ is a normal group)? The naive idea is to find $Q\in \schs$ such that 
\[Q(T)=\coker(\ph(T):H(T)\to G(T)).\]
However this doesn't quite work. For example, consider the multiplication-by-$n$ map
%surj on Qbar
\bal
[n]:\G_m&\to \G_m\\
g&\mapsto g^n
\end{align*}
Think of this as ($\G_m$ appears $n$ times)
\[
\xymatrix{
\G_m \ar[r]^-{\text{diagonal}} \ar[rd]_{[n]} & \G_m\times \cdots \times \G_m\ar[d]\\
& \G_m.
}
\]
%Let $S=\Spec \Q$. 
We write $Q(K)$ as short for $Q(\Spec K)$. 
Now $Q(\Spec(\Q))=\Q^{\times}/(\Q^{\times})^n$, and this is not contained in $Q(\Spec(\ol{\Q}))=\{1\}$. However, if $Q$ were a scheme, when we pass to a larger field, we should get more points! (Since $\Q\hra \ol{\Q}$, we have $\Spec(\ol{\Q})\hra\Spec(\Q)$, and hence $Q(\Q)\hra Q(\ol{\Q})$.) 

There doesn't exist such a scheme $Q$, in other words, the functor $Q(\bullet)$  
%enlarge to fppf sheaves, try to solve problem there.
%quotient of sheave is subtle, have to sheafify, same here. 
is not representable.

Today we'll talk about quotients, and see how to remedy this problem. There are three ways.
\begin{enumerate}
\item Take {\it categorical quotients}.
\item Take {\it geometric quotients} (geometric invariant theory).
\item Take the {\it fppf quotient}. We enlarge the category of schemes to fppf sheaves, and try to solve the problem there. Sheafify in the fppf topology and hope it's represented by a scheme.
\end{enumerate}
Or we could enlarge the category to stacks...

Recall that taking the quotient of sheaves was subtle, too; we had have to sheafify. Something similar is at play here. 


We follow~\cite[\S4]{GGBM}.
%sheaf, when rep by scheme.

\subsection{Group scheme actions}
It is actually no harder to study a group scheme action
\[
G\cir X
\]
where $G$ is a group scheme over $S$ and $X$ is a scheme over $S$. If you're Grothendieck, then you can study something even more general, the quotient by any kind of equivalence relation (we won't cover this). Specializing to the left or right action of a subgroup scheme on a group scheme,
\[
H\cir G
\]
where $H\subeq G$ in $\gps$, we get quotient group schemes.


The idea for 1 (categorical quotients) and 2 (geometric quotients) is as follows. Consider the analogous case where a group $G$ acts on a topological space $X$:
\[
G\cir X.
\]
\begin{enumerate}
\item
The quotient space is the space $G\bs X$ such that for any space $Y$ and $G$-invariant map $X\to Y$ there exists a unique continuous map making the diagram commute.
\[
\xymatrix{
X\ar[rr]^{G\text{-invariant}}\ar[rd] && Y\\
& G\bs X\ard{ur}_{\exists!}&
}
\]
\item
Define $G\bs X$ as a topological space to be $X/\sim$ where $x\sim gx$ for every $g\in G$, with the quotient topology. 
The topology is such that the continuous functions on $C(G\bs X)$ are exactly those that come from $G$-invariant continuous functions on $X$.
\[
C(G\bs X)=C(X)^G
\]
%Lifted functions to $X$ are $G$-invariant; descent to quotient space.
%coequalize in schemes, vs. ringed spaces
\end{enumerate}
Taking categorical quotients is akin to 1; taking geometric quotients is akin to 2.
\subsubsection{Group action on schemes}
\begin{df}
Let $G\in \gps$ and $X\in \schs$. A \text{(left) action} of $X$ is one of the two equivalent things:
\begin{enumerate}
\item
For every $T\in \schs$, a map
\[
G(T)\times X(T)\to X(T)
\]
that is a group action, and that is functorial in $T$.
\item
a morphism $\mu:G\times_SX\to X$ such that  the following composition is the identity
\[
\xymatrix{
X\aq{r}\ar@/_1pc/[rrr]_{\id_X} &  S\times_S X\ar[r]^{(e,\id_X)}& G\times_S X\ar[r]^{\mu}& X}
\]
and
\[
\commsq{G\times_S G\times_S X}{G\times_S X}{G\times_SX}{X}{(\id_G,\mu)}{(m,\id_X)}{\mu}{\mu}
\]
(This says $g_1(g_2x)=(g_1g_2)x$.)
\end{enumerate}
\end{df}
Again, the equivalence holds by the Yoneda Embedding. In the first action, when we say $G(T)\times X(T)\to X(T)$ is a group action, we mean that an analogous diagram to the diagram in the 2nd definition holds (with $G(T)$ and $X(T)$ instead of $G$ and $T$).

\begin{df}
Given a group action on a scheme $G\cir X$, we say that a map $X\to Y$ in $\schs$ is \textbf{$G$-invariant} if one of the following equivalent conditions hold.
\begin{enumerate}
\item $G(T)\cir X(T)\to Y(T)$ is $G(T)$-invariant for all $T$ (i.e. $f(x)=f(gx)$).
\item The following diagram commutes. ($p_i$ is the projection to the $i$th component.)
\[
\xymatrix{
& X \ar[rd]^f &\\
G\times_S X \ar[ru]^{p_2}\ar[rd]_{\mu} & & Y\\
& X\ar[ru]_f&
}
\]
i.e. $f\circ p_2=f\circ \mu$.
\end{enumerate}
\end{df}
%concoct diagrams

\subsection{Categorical quotient}
We can now define a categorical quotient of $X$ by $G$.
\begin{df}
A \textbf{categorical quotient} of $X$ by $G$ is $(Y,X\xra{\pi} Y)$ such that for any $G$-invariant map $X\to Y'$, there exists a unique dotted morphism making the following commute.
\[
\xymatrix{
X\ar[rr]^{G\text{-invariant}}\ar[rd]_{\pi} & & Y'\\
& Y\ard{ru}_{\exists !} &
}
\]
%coequalizer action
\end{df}
In other words,
\bal
\Hom_{\schs} (Y,Y')&=\ker(\xymatrix{\Hom(X,Y')\ar@/^/[r]^-{p_2^*}\ar@/_/[r]_-{\mu^*} & \Hom(G,Y')\times \Hom(X,Y')})\\
f&\mapsto f\circ \pi.
\end{align*}
By $\ker$ we actually mean the universal equalizer, so the RHS is the set of functions $g$ satisfying $p_2\circ g=\mu\circ g$. We have $p_2\circ (f\circ \pi)=\mu\circ (f\circ \pi)$. The uniqueness and existence of $f$ making the diagram commute says that the map is a bijection.

This is a flexible definition, because we can do the same in other categories. We work within any given category without enlarging it. Because of this, this approach also has its limitations.

We choose the best object to be the quotient, in the category of schemes. This quotient may or may not have the nice properties we may want.

We revisit our example.
\begin{ex}
We have
\bal
\mu:\G_m\times_S\G_m& \to \G_m\\
(g,x)&\mapsto g^nx
\end{align*}
For simplicity we take $S=\Spec k$ where $k$ is a field. 
We can show that $(\Spec k,\G_m\xra{\text{trivial}} \Spec k)$ is the categorical quotient of $\G_m$ by $\G_m$ under this action. Then for any $T$,
\[
[\coker(\G_m\xra{[n]} \G_m)](T)=\{1\}.
\]
We expect this---as $\G_m\xra{[n]}\G_m$ is surjective over an algebraically closed field (captured in our fake definition where the quotient had a single point over an algebraically closed field), we suspect that the quotient should just be a point over {\it any} field and actually any scheme.
%our fake definition of the quotient had a single point over algebraically closed field, we suspect that the quotient should just be a point, over {\it any} field.

Observe that given a $G$-invariant map $X\to Y'$, the unique map $Y\to Y'$ making the diagram commute is the composition of $e:Y\to X$ with $X\to Y'$:
\[
\xymatrix{
X=\G_m\ar[rr]^{G\text{-invariant}}\ar[rd] && Y'\\
& Y=\Spec k\ard{ur}_{\exists!}\ar@/^1pc/[ul]^{e}&
}
\]
%map to single point. unique image. show that induces imomorphism onto point which is image
\prbbox{
Check these assertions.
}
\end{ex}
\subsection{Geometric quotient}
The geometric quotient is nicer, in the sense that we can construct, not just characterize it.

As a warm-up we consider the following simple case.
\begin{lem}
Let $\Ga$ be a finite group, and $X=\Spec A$. %want Ga-invariant functions on quotient space.
Consider the group action 
\[
\Ga\cir X=\Spec A\quad (S=\Spec k).
\]
Let $Y=\Spec A^{\Ga}\xleftarrow{\pi} X$ be induced by the inclusion $A^{\Ga}\subeq A$.
\begin{enumerate}
\item
$A$ is integral over $A^{\Ga}$. 
\item The map $Y=\Spec A^{\Ga}\xleftarrow{\pi} X$ gives
\begin{enumerate}
\item
the map on topological spaces 
\[
\xymatrix{
|X|\ar[rr]^{\pi^{\text{top}}}\ar[rd]_{\text{can.}} &&|Y|\\
&|X|/\sim\ar[ru]_{\cong}&}
\]
and
\item
the map on sheaves 
\[
\xymatrix{
\cO_Y\ar[rr]^{\pi^{\sharp}}\ar[rd]_{\cong} &&\pi_*\cO_X\\
&(\pi_*\cO_X)^{\Ga}\ha{ru}&}
\]
\end{enumerate}
\end{enumerate}
\end{lem}
Note also that $\pi$ is a closed map.

Think of $Y$ as the scheme whose topological space has the quotient topology, and whose sections are the $\Ga$-invariant functions.
%$\Ga$-action

Let's prove it in this special case.

%like doing Galois theory.
\begin{proof}
\begin{enumerate}
\item
Every $a\in A$ is a root of $\prod_{\ga\in \Ga} (X-\ga(a))\in A^{\Ga}[X]$, so $A$ is integral over $A^{\Ga}$.
\item We need to show
\begin{enumerate}
\item
\ul{$\Ga\bs |X|\to |Y|$ is well defined:} Given $\mfp \in \Spec(A)$, its image in $A^{\Ga}$ is $\mfp\cap A^{\Ga}$. We need $\mfp \cap A^{\Ga}=\ga(\mfp)\cap A^{\Ga}$; this holds because $\Ga$ doesn't do anything on the $\Ga$-invariant subset of $\mfp$. \\

\ul{Surjectivity} follows from the fact that $A$ is integral over $A^{\Ga}$ and the going-up theorem, which says that every prime ideal of $A^{\Ga}$ comes from a prime ideal of $A$.

\begin{thm}[Going up]
Let $R\subeq R'$ be an integral extension of rings, and $\mfp$ a prime of $R$. Then there exists a prime $\mq$ of $R'$ such that $\mq\cap R=\mfp$.
\end{thm}
%Integral extension, every prime ideal here comes from prime ideal of $A$. (Here we use (1).) 


\ul{Injectivity:} Assume $\mfp\cap A^{\Ga}=\mfp'\cap A^{\Ga}$. Then for every $x\in \mfp'$, we have $x\mid \prod_{\ga\in \Ga}\ga(x)\in \mfp'\cap A^{\Ga}=\mfp \cap A^{\Ga}$, so $\ga(x)\in \mfp$ for some $\ga$, i.e. 
\[
\mfp'\subeq \bigcup_{\ga\in \Ga}\ga(\mfp).
\]
We use the following theorem from commutative algebra.
\begin{thm}[Prime avoidance]
Suppose $\ma$ is a subset of a ring $R$ stable under addition and multiplication (e.g. an ideal). Suppose $\mfp_j$ are prime ideals.  %such that $\mfp_3,\ldots, \mfp_n$ are prime (e.g. they're all prime).
If $\ma\subeq \bigcup_{i=1}^n \mfp_i$, then $\ma\subeq \mfp_i$ for some $i$.
\end{thm}
This tells us that actually $\mfp'\subeq \ga(\mfp)$ for some $\ga$. By symmetry, $\mfp \subeq \de(\mfp')$ for some $\de$. %, and we must have $\de=\ga^{-1}$. 
This shows $\ga(\mfp)=\mfp'$, i.e. $\mfp$ and $\mfp'$ lie in the same $\Ga$-orbit, i.e. are the same in $\Ga\bs |X|$.
\item \ul{The map on sheaves:} Because the distinguished open sets $D_f:=\Spec(A^{\Ga})_f$ for $f\in A^{\Ga}$ form a basis for the topology on the affine scheme $Y=\Spec A^{\Ga}$, it suffices to check the isomorphism on the $D_f$. On $D_f$ we have the map 
\[
(A^{\Ga})_f=\cO_Y(D_f)\to \pi_*\cO_X(D_f).
\]
The image is $(A_{\Ga})_f=(A_f)^{\Ga}=(\pi_*\cO_X)^{\Ga}(D_f)$, as needed.
\end{enumerate}
\end{enumerate}
\end{proof}
Because we actually constructed $Y$, the geometric quotient is a more hands-on approach.

We considered the affine case; to discuss the more general case where $G$ is group scheme and $X$ not necessarily affine, we need to enlarge our playground to {\it ringed spaces}.

We constuct a ringed space that looks like a candidate for the quotient, and show it is represented by scheme when possible.

We'll still assume $G\cir X$ where $G$ is a finite flat group scheme over $S$ and $X$ is a scheme over $S$. %mumford: spec of alg closed field.
This suffices for many purposes.

We have 
\[\schs\sub (\text{locally rings spaces}/S)\sub (\text{ringed spaces}/S)\]
We write $(LRS/S)$ and $(RS/S)$ in shorthand. The left inclusion is a full subcategory while the right one is not. 

We'll construct a candidate for the quotient in the category $(RS/S)$ and try to show it's actually in $\schs$.

We assume the following hypothesis on orbits (essential in this approach):
\begin{itemize}
\item
For any closed point $x\in X$, $G\cdot x\subeq |X|$ is contained in an open affine subscheme of $X$.
\end{itemize}
\begin{df}
Let $G\cir X$ where $G$ is a group scheme over $S$ and $X$ is a scheme over $S$. Define the \textbf{geometric quotient} of $X$ by $G$ as the ringed space
\[
(G\bs X)_{rs} = (|X|/\sim, (\pi_{rs*}^{\text{top}}\cO_X)^G)
\]
where $\sim$ is given by the $G$-action, $\pi^{\text{top}}_{rs}:|X|\rra |X|/\sim$ is the canonical quotient map and for every open $V\subeq |X|/\sim$,
%we have to say what the second thing is
%open? (nonaffine ok?)
\[
(\pi_{rs*}^{\text{top}}\cO_X)^G(V)=\set{a\in \pi_*^{\text{top}}\cO_X(V)=\cO_X(\pi^{\text{top}-1}(V))}
{p_2^{\sharp}(a)=\mu^{\sharp}(a)}\subeq (\pi^{\text{top}}_{rs*}\cO_X)^G(V).
\]
\end{df}
Here the condition $p_2\sh(a)=\mu\sh(a)$ says that $a$ is a $G$-invariant function. Recall that $p_2$ and $\mu$ were maps
%The condition is $G$-invariance. Function same on these two pullbacks, regard as $G$-invariant function.
\[
\xymatrix{
G\times X\ar@/^/[r]^{\mu}\ar@/_/[r]_{p_2} & X.
}
\]

Next time, we'll show that under the hypothesis, this ringed space is actually a scheme.

%We have to somehow prove this is a scheme.
%
%\begin{thm}
%Under the hypothesis,
%\begin{enumerate}
%\item
%There exists $Y\in \schs$ and a quotient morphism $\pi:X\to Y$ such that
%\[
%(Y,\pi)\cong ((G\bs X)_{rs},\pi_{rs}).
%\]
%\item $(Y,\pi)$ is a categorical quotient
%\end{enumerate}
%\end{thm}
%%Questions: how related to categ, what properties (when $X$ and $G$ have good, do inherit?)
%\begin{rem}
Why do we assume such a hypothesis? The hypothesis says that the orbits of a closed point is contained in an affine neighborhood, so it allows us to reduce to the affine case. 

If $\pi^{\text{top}}:|X|\to |X|/\sim=|Y|$ is affine (i.e. any inverse image of affine scheme is affine), %scheme is such that each point has affine nbhd
for $x\in |X|$ mapping to $y\in |Y|$, there exists $V\ni y$ affine such that $\pi^{\text{top}-1}(V)\subeq X$ is affine. But $\pi^{\text{top}-1}(V)$ contains the orbit of $x$, i.e. some affine neighborhood contains the orbit of $x$. Because the quotient morphism will often be affine, it is helpful to assume such a hypothesis.
%\end{rem}
%\begin{proof}[Proof sketch]
%\begin{enumerate}
%\item
%Reduce to the case where $S=\Spec Q$ and $X=\Spec A$. To reduce to the case where $X$ is affine, use the hypothesis to show $X$ can be covered by $G$-stable open affine subschemes. Work on each and try to glue together.
%
%Note $G$ is finite over $S$, so $G=\Spec R$ for some finite-dimensional algebra $R$ over $A$.
%
%We have
%\[
%\xymatrix{
%G\times_S X\ar[r]^{\mu}\ar[r]_{p_2} & X
%}
%\to 
%\xymatrix{
%R\ot_Q A & \ar[l]\ar[l] A.
%}
%\]
%%spec of g-inv funcs on $A$, can characterize by sharp equal.
%Define
%\[
%B=\set{a\in A}{p_2^{\sharp}(a)=\mu^{\sharp}(a)}.
%\]
%Show ($Y=\Spec B$, $X\xra{\pi} Y$) given by $B\hra A$. One main step is to show $A$ is integral over $B$.
%\end{enumerate}%can glue locally ringed spaces
%\end{proof}
%We skip 2, see reference on website.
%
%Where use flat in proof? Get nice properties want to have for quotient.
%
%Few more words about quotient next time.