\lecture{Tue. 10/30/12}

\subsection{Structure of $\Hom$ and $\End$}

Recall that $\abk$ denotes the category of abelian varieties over $k$, while $\abk^{0}$ denotes the category of abelian varieties, but with morphisms $\Hom^0(A,B):=\Hom_{(\text{Gp}/k)}(A,B)\ot_{\Z}\Q$. 
An isogeny becomes a morphism in the new category.

Now $\Hom^0(A,B)$ is a $\Q$-vector space and $\End^0(A)$ is a $\Q$-algebra (with multiplication as composition), but we don't know that they are finite-dimensional yet. We'll show this today. 
\begin{lem}
\begin{enumerate}
\item
$\Hom(A,B)$ is a torsion-free $\Z$-module.
\item
$\Hom(A,B)\subeq \Hom^0(A,B)$.
\end{enumerate}
\end{lem}
\begin{proof}
We want to show that if $[n]\circ f=0$ then $f=0$.
\[
\cmp{A}{B}{B}{f}{}{[n]}{}{}{0}
\]
Suppose $\dim A=g$. Now $\im f\subeq B[n]$ has dimension 0, so $\ker f$ has dimension $g$. This means $\ker f=A$, because the only subscheme of an irreduced irreducible scheme with same dimension is the scheme itself.

(1) implies (2) directly.
\end{proof}
Recall that an abelian variety $A$ is $k$-simple %depends on choice of basefield
if there is no abelian subvariety $B\sub A$ with $B\ne 0$. 
\begin{lem}
If $A$ is simple then $\End^0(A)$ is a division ring. 
\end{lem}
\begin{proof}
We need to show every nonzero morphism is invertible: if $0\ne f\in \End^0(A)$, is $f\in \End^0(A)^{\times}$? 

There exists $n\in \N$ such that $[n]\circ f\in \End(A)$; we have $[n]\circ f\ne 0$ because $\End(A)$ is torsion-free. We have $[n]\in \End^0(A)^{\times}$ because its inverse is $\rc n$. The fact that $A$ is simple implies that every nonzero endomorphism is invertible, so $[n]f\in \End^0(A)^{\times}$ (being an isogeny). Hence $f\in \End^0(A)^{\times}$.
\end{proof}
This tells us something about the structure of the endomorphism algebra. Recall that Poincar\'e reducibility tells us that $A$ is isogenous to some product of simple varieties:
\[
A\sim A_1^{n_1}\times \cdots \times A_r^{n_r}
\]
with $A_i\nsim A_j$ for $i\ne j$ and with $n_i\ge 1$. This tells us that the endomorphism ring decomposes
\[
\End^0(A)\cong \prod_{i=1}^r \cal M_{n_i}(D_i),\qquad \text{where }D_i=\End^0(A_i)
\]
(Because the $A_i$ are simple, there is no map between different $A_i$'s except zero.)
The $D_i$ are division $\Q$-algebras. We've reduced the study of the $\End(A)$ to the case when $A$ is simple. 
The basic questions are the following.
\begin{enumerate}
\item
Is $\dim_{\Q} \End^0(A)<\iy$?
\item
Is $\Hom(A,B)$ a finitely generated $\Z$-module?
%abelian group in fin gen vspace may not be fin gen
\end{enumerate}
If item 2 is true, then because we know $\Hom(A,B)$ is torsion-free, $\Hom(A,B)$ is actually a finite free $\Z$-module.

%We first establish finite-dimensionality, then classify based on division algebras.
We now answer these questions.
%We'll state a result, then explain the proof.
\begin{thm}[Mumford~\cite{Mu70}, Theorem 19.3]\llabel{thm:787-HomAB}
\begin{enumerate}
\item
$\Hom(A,B)$ is a finitely generated $\Z$-module.
\item
Assume that $(\ell,\chr k)=1$. Then 
\[
\Z_{\ell}\ot_{\Z}\Hom(A,B)\hra\Hom_{\Z_{\ell}}(T_{\ell}A,T_{\ell}B)
\]
is injective. 
\end{enumerate}
\end{thm}
We'll see why we need $(\ell,\chr k)=1$ in the proof. %a priori far from clear that injective.
We don't expect the map $\Z_{\ell}\ot_{\Z}\Hom(A,B)\to\Hom_{\Z_{\ell}}(T_{\ell}A,T_{\ell}B)$ to be a bijection. Consider when $A=B$ are elliptic curves over $\C$. Then $\End(A)=\Z$ or $\Z\opl \Z$, a rank 2 module in an imaginary quadratic field. The RHS has rank 4 over $\Z_{\ell}$, while the LHS has rank 1 or 2. Galois invariance prevents this from being bijective.

When the field is finitely generated over $\Q$, conjecturally, the image is the commutant of Galois.

\begin{cor}
\begin{enumerate}
\item
$\rank_{\Z}(\Hom(A,B))=\dim_{\Q}\Hom^0 (A,B)\le 4\dim A 4\dim B$.
\item
$\End^0(A)$ is a finite-dimensional semisimple algebra.
\end{enumerate}•
\end{cor}
\begin{proof}
%This follows from earlier results. (ref.)
We have $T_{\ell}A\cong \Z_{\ell}^{2\dim A}$ and $T_{\ell}B\cong \Z_{\ell}^{2\dim B}$. Just multiply the dimensions.
%finite division algebra, finite product. 
\end{proof}
We'll prove the theorem starting in 2 steps. As preparation, we'll prove the following.
\begin{thm}[Mumford~\cite{Mu70}, Theorem 19.2]\llabel{thm:deg-poly}
The map
\begin{align*}
\End(A)&\to \Z\\
\phi&\mapsto \deg \phi
\end{align*}
(where $\deg\phi:=0$ if $\phi$ is not an isogeny) extends to a homogeneous polynomial $f$ of degree $2g$ on $\End^0(A)$, i.e., for any $\phi,\psi\in \End^0(A)$, the following function
\begin{align*}
\Q^2&\to \Q\\
(m,n)&\mapsto f(m\phi+n\psi)
\end{align*}
is a homogeneous polynomial of degree $2g$. %(Recall that $\End^0(A)$ is a $\Q$-vector space (actually algebra).) ($f(m\phi)=m^{2g}f(\phi)$.)
%degree same as what does to fundamental class in homology.
%highest etale cohomology
\end{thm}
\begin{proof}
We recall some facts.
\begin{enumerate}
\item[(i)]
$\deg[n]=n^{2g}$. 
\item[(ii)]
$\chi_A(f^*L)=(\deg f)\chi_A(L)$. (Both sides are 0 if $f$ is not an isogeny.)
%if f is not isogeny still true, both sides are 0.
\item[(iii)]
If $L$ and $M$ are ample line bundles then the function
\[
n\mapsto \chi(L^{\ot n}\ot M)
\]
is a polynomial in $n$ of degree at most $g$.
\end{enumerate}
In fact there is a precise Riemann-Roch type formula for $\chi_A(L)$ in Mumford~\cite{Mu70}, Section 16.

\begin{enumerate}
\item
We have  by (i) that
\[
\deg(n\phi)=\deg n\deg \phi=n^{2g}\deg \phi.
\]
This shows homogeneity.
\item
We will show $n\mapsto \deg (n\phi+m\psi)$ is a polynomial in $n$.
%$deg \phi/n = \deg\phi/n^{2g}$.  Extend to whole, well-defined. Poly should be homogeneous with degree 2g.
\end{enumerate}•
Together 1 and 2 give us what we want, namely, $\det(n\phi+m\psi)$ is a polynomial in $m$ and $n$, homogeneous of degree $2g$>

Let's show item 2. We will use the Euler characteristic of the line bundle. Choose an ample line bundle $L$. Then $\chi(L)\ne 0$. This comes from $H^i(A,L)=0$ for $i>0$. (See Mumford~\cite{Mu70}, Section 16.) 

We know by (ii) that
\[
\deg(n\phi+\psi)=\fc{\chi((n\phi+\psi)^* L)}{\chi(L)}.
\] 
It suffices to show that 
\[
n\mapsto \chi(\underbrace{(n\phi+\psi)^*L}_{=:L(n)})
\]
is a polynomial.
The idea is to get an expression in terms of a tensor power of $L$, using Theorem of the Cube~\ref{thm:cube}. Apply the Theorem of the Cube to $f_1=(n-1)\phi+\psi$ and $f_2=f_3=\phi$. We get a recursive formula which we solve to get
\[
L(n):=L_1^{\fc{n(n-1)}{2}}\ot L_2^n\ot L_3.
\]
Now 2 applications of 
Then we can appeal to (iii) to conclude that $\chi(L(n))$ is a polynomial in $n$: (iii) implies
\[
(m,n)\mapsto \chi(L_1^m\ot L_2^n \ot L_3)
\]
is a polynomial in $m$ and $n$, so $n\mapsto \chi(L(n))$ is a polynomial in $n$, of degree at most $2g$. 
%n^2 multiples by g to get degree 2g. 
%In a sense, Euler char func on endo alg is quadratic in nature. 
\end{proof}
Let's get back to the proof of Theorem~\ref{thm:787-HomAB}.
\begin{proof}[Proof of Theorem~\ref{thm:787-HomAB}]
We prove this in two steps.\\

\step{1} Recall that $\Hom(A,B)\sub \Hom^0(A,B)$. For every finitely generated $\Z$-submodule $M$ of $\Hom(A,B)$, $\Q M\cap \Hom(A,B)$ is a finitely generated $\Z$-module.\\

\step{2} Prove the injectivity of (ii). \\

Once we have steps 1 and 2, noting tensoring is an exact functor (?), tensoring with $\ql$ gives
\[
\Q_{\ell}\ot_{\Z} \Hom(A,B)=\Q_{\ell} \ot_{\Q} \Hom^0(A,B) \hra \Q_{\ell}\ot_{\Z_{\ell}} \Hom(T_{\ell}A, T_{\ell}B).
\]
The RHS is finite dimensional, so $\Hom^0(A,B)$ is finite dimensional. Taking any $M$ containing a $\Q$-basis of $\Hom^0(A,B)$, step 1 implies (1) and we'll be done.

We have to prove Steps 1 and 2. For step 1, we'll make use of the fact that the degree function is polynomial.\\

\step1 (Proof) We use the following.
\begin{lem}
Let $A'\to A$ and $B\to B'$ be isogenies. Then we have an injection $\Hom(A,B)\hra \Hom(A',B')$.
\end{lem}
\begin{proof}
%kind of covering, majorize by isogeny.
Given $A'\to A$, we can find $m$ such that $[m]:A\to A$ factors through $A'\to A$. Given $B\to B'$ we can find $n$ such that $[n]:B\to B$ factors through $B\to B'$.
\[
\xymatrix{
A\ard{d}_{\exists} \ar[rd]^{[m]} & & & B\\
A'\ar[r] & A\ar[r] & B \ar[r] \ar[ru]^{[n]} & B'.\ard{u}_{\exists}
}
\]
Then we find that $[n]f[m]=0$. By torsion-freeness we have $f=0$.
\end{proof}

We make some reduction steps. First we may assume $A,B$ are simple. Poincar\'e reducibility tells us that for some finite product of simple abelian varieties we have $\prod A_i^{m_i}\to A$ and $B\to \prod_j B_j^{n_i}$. The lemma says that 
\[
\Hom(A,B)\hra \prod_{i,j}\Hom(A_i^{m_i},B_j^{n_j})=\prod_{i,j}\cal M_{m_i\times n_j}\Hom(A_i,B_j).
\]
%\fixme{powers of $A_i$?}
If the conclusion true for each $\Hom(A_i,B_j)$, then it is true for $\Hom(A,B)$.

We can assume $A=B$: If there is no isogeny $B\to A$, then $\Hom(A,B)=0$. If there is an isogeny $B\to A$, then $\Hom(A,B)\hra \Hom(A,A)$. If conclusion of step 1 true for $\Hom(A,A)$, then true for $\Hom(A,B)$.

Now we use use Theorem~\ref{thm:deg-poly}. Let $M$ be a finitely generated $\Z$-submodule of $\End(A)$.  We have to show $\Ga:=\Q M\cap \End (A)\sub \Q\cdot M\cong \Q^r$ is finitely generated. Now we claim that 
\[
\Ga\cap \set{\al\in \End^0(A)}{|\deg(\al)|<1}=\{0\}.
\]
Indeed, the degree is a polynomial on $\End^0(A)$. The reason is that on $\End(A)$, the degree function assumes integer values; %$\al\in \End A$, $\deg \al\in \Z$,
we have $\deg \al=0$ iff $\al=0$. This shows $|\deg(\al)|<1$ contains an open neighborhood of 0. Hence $\Ga$ is a disrete subgroup of $\Q M$. A standard fact shows that a discrete subgroup is finitely generated. This marks the end of step 1.\\
%algebraic number theory: finiteness for class group, unit group

\step 2 Suppose by way of contradiction that the map is not injective. Then some element is sent to 0:
\[
T_{\ell}:0\ne \sum_{i=1}^r \al_i\phi_i \mapsto 0,\qquad \al_i\in \Z_{\ell}, \,\phi_i\in\Hom(A,B).
\]
Let $M:=\Z\an{\phi_1,\ldots, \phi_r}$. We may enlarge $M$ such that 
\[
M=\Q M\cap \Hom(A,B). 
\]
%RHS finite generated
Choose free generators $\phi_1,\ldots, \phi_r$ of $M$. However, it's difficult to work with homomorphisms with $\Z_{\ell}$ coefficients, so we approximate $\Z_{\ell}$ coefficients with $\Z$ coefficients. Choose $\be_i\in \Z$ such that $\be_i-\al_i\in \ell\Z_{\ell}$ (i.e., the $\be_i$ are a first-order $\ell$-adic approximation to $\al_i$). Then $\phi=\sum_{i=1}^r \al_i\phi_i+(\be_i-\al_i)\phi_i$ maps to $T_{\ell}(\phi)=0+\ell(\cdots)$. Now we have a commutative diagram
\[
\commsq{T_{\ell}A}{T_{\ell}B}{A[\ell](\ol k)}{B[\ell](\ol k).}{\bmod{\ell}}{T_{\ell}(\phi)}{\bmod{\ell}}{\phi}
\]
Going clockwise we have 0. However, the vertical maps are surjections, so we must have the lower map $\phi=0$.

Now $\phi|_{A[\ell](\ol k)} =0$. Using $(\ell,\chr k)=1$, we have
\[
\phi|_{A[\ell]\times_k\ol k}=0.
\]
%where $A[\ell]\times_k\ol k$ is the constant group scheme $(Z/\ell)^{2g}$. 
This gives us $\phi|_{A[\ell]}=0$. The rank of the kernel doesn't change from $k$ to $\ol k$. If the kernel has full rank in either field, it has full rank in both fields.

We now have a map
\[
\xymatrix{
A\ar[r]^{\phi} \ar[d] & B\\
A/A[\ell] \ard{ru} \ar[d]^{\sim}_{[\ell]}& \\
A\ard{ruu}_{\phi'=\fc{\phi}{\ell}}&
}
\] 
in $\Hom(A,B)\cap \Q M=M$. Now $\phi'=\sum_{i=1}^r \al_i'\phi_i$. Thus $\sum \ell \al_i' \phi_i=\sum_{i=1}^r \be_i \phi_i$. This implies that $\al_i$ should be infinitely divisible by $\ell$: we may assume there exists $i$, $\ell\nmid \al_i$, using $\ell$-torsion-freeness of the RHS.  
%\phi=\sum_{i=1}\be_i\phi_i.
We we a contradiction because by iniital choice of $\be_i$, there exists $i$ so that $\ell\nmid \be_i$. 
\end{proof}
Next time we'll discuss the characteristic polynomial, Section 19 of Mumford~\cite{Mu70}.
