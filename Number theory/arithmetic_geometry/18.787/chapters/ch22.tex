\lecture{Thu. 11/29/12}

Today we'll talk about finite commutative group schemes over a field. Next time we'll talk about Dieudonn\'e theory, and our final topic will be Honda-Tate theory. %(Next Thursday 12/6: any volunteers to speak?)

\subsection{Frobenius and Verschiebung}
\subsubsection{Frobenius}
The key to studying finite group schemes over a field (or an arbitrary base scheme) is the Frobenius map. Recall that for a $\Fp$-scheme $X$, we defined the absolute Frobenius map $F_X:X\to X$ by acting on topological spaces as the identity $X^{\text{top}}\xra{\id}X^{\text{top}}$ and acting on the underlying ring as the $p$th power map, $a^p\mapsfrom a$. 

For an scheme $X\xra fS$ in $(\text{Sch}/\Fp)$, we define the relative Frobenius map $F_{X/S}$ using the fiber diagram
\[
\xymatrix{
\pull{X}{f}{\exists ! F_{X/S}}{}{X^{(p)}}{}{X}{}{f}\back{S}{F_S}{S}
}
\]
%We obtain $F_{X/S}:X\to X^{(p)}$, the relative Frobenius.

We're interested in group schemes. 
%
\subsubsection{Verschiebung}

Let $S$ be a $\Fp$-scheme, and let $G\in  \pat{Gp/$S$}$. We have a map $F_{G/S}:G\to G^{(p)}$ a morphism of $S$ group schemes. There is a useful map going the opposite direction. %(The map is functorial enough to be true, we won't check the details.)

\begin{fct}
Suppose $G$ is commutative. Then there exists a canonical map 
\[
V_G:G^{(p)}\to G
\]
in $\gps$ called the \textbf{Verschiebung} such that
\[
F_{G/S}\circ V_G=[p],\qquad V_G\circ F_{G/S}=[p].
\]
\end{fct}

We only care about the case where $k$ is a field. Suppose $G\in \pat{FCGp/$k$}$ where FCGp stands for ``finite locally free commutative group scheme." (We could work over a general scheme $S$, but we won't need this generality.)

It turns out we can express $V_G$ in terms of the Frobenius map.

\begin{pr}
We have
\[
V_G=(F_{G^{\vee}})^{\vee}
\]
where ${}^{\vee}$ denotes Cartier dual.
\end{pr}

By this we mean the following. Dualizing 
the map
\[
G^{\vee}\xra{F_{G^{\vee}}} G^{\vee(p)}
\]
gives 
\[
\xymatrix{
G^{\vee \vee} \aq{d} & G^{\vee(p)\vee}\ar[l]_{F_{G^{\vee}}^{\vee}} \aq{d}\\
G&G^{(p)}\ar[l]^{V_G}
}
\]
%For psychological comfort, 
One can check that $G^{(p)}$, $F_G$, $V_G$ are all functorial in $G$.

\subsection{On $\pat{FCGp/$k$}$.}
Let $k$ be any field.
\subsubsection{Approach using fppf sheaves}
\begin{df}
A \textbf{fppf scheme} is a map 
\[
\schk\xra{\mathcal F} \pat{Grp}
\]
%grothendieck topology
satisfying the fppf sheaf axioms (finitely presented, faithfully flat).
\end{df}
We can embed the category of schemes into the category of fppf sheaves:
\[
\pat{FCGp/$k$}\hra \pat{fppf/$k$}.
\]
This is a fully faithful functor (over abstract groups).

The different notions of quotients in these categories are the same: an exact sequence in $\pat{FCGp/$k$}$ is an exact sequence as fppf sheaves. If we have $H\hra G$, we can define $G/H$ as a geometric quotient. Or we can look at them as fppf sheaves, take the presheaf quotient, sheafify, and then see it's represented by an object in $\pat{FCGp/$k$}$. We hence have two ways to view the quotient.

\subsubsection{Decomposition into $\ell$-primary parts}
Let $G\in\pat{FCGp/$k$}$. Then there is a (finite) product decomposition
\[
\prod_{\ell\text{ prime}} \ub{G[\ell^{\iy}]}{\varinjlim G[\ell^n]}.
\]
%actually the limit ends after finitely many steps.

The proof uses the fact that if $r=|G|$, then $[r]=0$ on $G$. (The analogue for groups is that raising an element to the order of a group gives the identity element. However the proof for schemes is nontrivial.)

Thus it's enough to study the $\ell$-primary part for each $\ell$.
\subsubsection{$\ell$-groups}

\begin{fct}
Suppose $\ell\ne \chr k$.
Let $G\in\pat{FCGp/$k$}$. Let $r=\rank G$. If $r^{-1}\in k$ (i.e., $\ell\nmid r$), the $G$ is \'etale over $\Spec k$.
\end{fct}
There are several definitions of ``\'etale." One of them is ``smooth of relative dimension 0."

It suffices to check $G$ is \'etale after base change. %(See references on website.)
\subsubsection{$\pat{\'Etale FCGp/$k$}$}
Let $G=\Spec R\to \Spec k$.

Write $R=\prod_{i=1}^r R_i$, where $R_i$ are finite local $k$-algebras.
Then $G$ is \'etale iff $R_i$ are finite separable extensions of $k_i$ for each $i$.

\begin{pr}
There is an equivalence
\[
\pat{\'EtFCGp/$k$}\xrc \bt{discrete finite}{$G(k\sep/k)$-modules}.
\]
(Here discrete means that the stabilizer of a point in the Galois group is open.)
The map sends
\[
\cal F\mapsto \cal F(k\sep).%\cil Gal
\]
\end{pr}
In other words, we can understand an \'etale finite commutative group scheme over $k$ just by understanding its $k\sep$ points and the Galois action on them.
%\'etale fund group here. stalk in \'etale topology.
We generally believe that the guy on the RHS is easier than the guy on the LHS. 

But there are still lots of non-\'etale schemes, and they remain mysterious.

\subsubsection{Conclusion}
We can focus on $p$-groups over $k$ where $\chr k=p$. 
%goal of d theory is to classify objects in a way no more difficult than prop above.
Dieudonn\'e theory classifies these objects.
\subsection{Types of group schemes}
We'll take a step back to give some strategies and overview. First we need some basic definitions.
\begin{df}
Let $k$ be any field. %can be char 0 but less interesting.
Let $G\in \pat{FCGp/$k$}$.
\begin{enumerate}
\item
$G$ is \textbf{infinitesimal} if $G$ is connected. (This is true iff $G_{\ol k}$ is connected, iff $G_{\text{top}}=\{\cdot\}$ (as $G$ has a finite number of points), iff $G_{\ol k,\text{ top}}=\{\cdot \}$.)
\item
$G$ is \textbf{unipotent} iff $G^{\vee}$ is infinitesimal. 

$G$ is \textbf{multiplicative} (diagonalizable) iff $G^{\vee}$ is \'etale.

$G$ is \textbf{bi-infinitesimal} iff $G$ and $G^{\vee}$ are infinitesimal.
\end{enumerate}•
\end{df}
%But if we can only account for a small number of group schemes, this is bad.
We can restrict to studying these types of schemes, because everything is an extension of an \'etale by an infinitesimal group scheme; they account for everything if you allow extensions.
%cannot be etale for both g and dual.
\begin{pr}
Let $\chr(k)=p>0$, and let $G$ be a $p$-group scheme. If $G$ is  \'etale, then $G^{\vee}$ is infinitesimal.
\end{pr}
\begin{proof}
First reduce to $k=\ol k$ by using the fact that \'etale group schemes are preserved by basechange.
%cartier dual works nicely

%bite
Now $\Ga:=G(k)$ is finite abstract $p$-group.
We have $G^{\vee}=\Spec k[\Ga]$. 
Let
\[
\mm=\set{
\sum_{\ga\in \Ga} a_{\ga}\ga
}{\sum_{\ga\in \Ga} a_{\ga}=0}.
\]
We can check
\begin{itemize}
\item
$\mm$ is nilpotent: For large $N$ we have
\[
\pa{\sum a_{\ga}\ga}^{p^N} = \pa{\sum a_{\ga}}^{p^N}\cdot 1=0.
\]
\item
$\mm$ is maximal:
We see that $k[\Ga]/\mm=k$ from the exact sequence %have the exact sequence
\[
0\to \mm \to k[\Ga]\xra{\tr} k\to 0
\]
where the trace sends $\sum a_{\ga}\ga\mapsto \sum a_{\ga}$.
\end{itemize}
Every prime ideal contain nilpotents, so it contains $\mm$ and  nothing else. Thus $G_{\text{top}}^{\vee}=\{\cdot\}$.
\end{proof}
\begin{lem}
If $G$ is \'etale and infinitesimal, then $G$ is trivial.
\end{lem}
\begin{proof}
Reduce to the affine case and write $G=\Spec\pa{\prod_{i=1}^r R_i}$. 
Because $G$ is infinitesimal, $r=1$. Because $G$ is \'etale, $R_1$ is finite separable over $k$. Because there exists an identity section, $R_1=k$.
\end{proof}
Say that $G$ is a finite commutative $p$-group. 
We have the following summary.\footnote{Note that unipotent group schemes contain bi-infinitesimal group schemes, and multiplicative group schemes contain bi-infinitesimal group schemes as well. For ease of classification we will sometimes write ``unipotent"/``multiplicative" to exclude bi-infinitesimal schemes; hopefully this is clear from context.}
%Right top \chr k=p.

\begin{center}
\begin{tabular}{|c|c|c|}
\hline 
$G\,\backslash\, G^{\vee}$  & \'etale & infinitesimal\tabularnewline
\hline 
\'etale & $\chr k\ne p$ & unipotent\tabularnewline
\hline 
infinitesimal & multiplicative & bi-infinitesimal\tabularnewline
\hline 
\end{tabular}
\end{center}

Mumford calls bi-infinitesimal group schemes ``local-local." The definitions coincide when $k$ is perfect.

\begin{ex}
We have the following examples.

\begin{center}
\begin{tabular}{|c|c|}
\cline{2-2} 
\multicolumn{1}{c|}{} & $\ul{\Z/p\Z}$\tabularnewline
\hline 
$\mu_{p}$ & $\al_{p}$\tabularnewline
\hline 
\end{tabular}
\end{center}

Here
\[
\mu_p=\ker([p]:\G_m\to \G_m)
\]
and %$\al_p=\Spec k[x]/(x^p)$ with the same comultiplication as $\G_a$, and
\[
\al_p:=\ker(F_{\G_a}:\G_a\to \G_a).
\]
(Recall addition on $\G_a$ was given by  the map $x\mapsto x\ot 1+1\ot x$.) We have $\al_p=\Spec k[x]/(x^p)$ with the same comultiplication as $\G_a$. %\fixme{(?)}. 
Note $\al_p^{\vee}\cong \al_p$ which can be seen by the existence of a perfect pairing 
%cannot make sense of 1/p! if p=0.
\[
\al_p\times \al_p\to \mu_p
\]
given by 
\[
(x,y)\mapsto \exp(xy)=1+xy+\cdots +\fc{(xy)^{p-1}}{(p-1)!}.
\]
\end{ex}
\begin{fct}
When $k=\ol k$, $\ul{\Z/p\Z}$, $\mu_p$, and $\al_p$ are the only groups of order $p$ in $\pat{FCGp/$k$}$.

All other $p$-groups are extensions of them, we can find a filtration such that each quotient is one of these groups:
\[
G=G_0\supset \cdots\supset G_r=\{0\}.
\]
\end{fct}
\begin{ex}%\fixme{look into this example more.} 
For instance, consider elliptic curves over $\fpb$. Ordinary curves are extensions of $\Z/p\Z$ by $\mu_p$. Supersingular curves are  extensions of $\al_p$ by another $\al_p$.
\end{ex}
\begin{pr}\llabel{pr:787-FV}
The following hold.
\begin{enumerate}
\item
$G$ is \'etale iff $F_{G/k}$ is an isomorphism.

$G$ is infinitesimal iff $F_{G/k}$ is nilpotent.
\item
$G$ is multiplicative iff $V_G$ is an isomorphism.

$G$ is unipotent iff $V_G$ is nilpotent.

$G$ is bi-invariant iff both $F_{G/k}$ and $V_G$ are nilpotent.
\end{enumerate}
\end{pr}
\begin{proof}
Going from (1) to (2), just use the fact that the dual of Frobenius is Verschiebung.

Think of $F_{G/k}$ as a twist. We have that $F_{G/k}$ is nilpotent iff for some $r>0$, the following composition is 0:
\[
\xymatrix{
G\ar[r]^{F_{G/k}}\ar@/_1pc/[rrr]_{F_{G/k}^{(r)}} & G^{(p)}\ar[r]^{F_{G^{(p)}/k}}& \cdots \ar[r] & G^{(p^r)}.
}
\]
Consider $G$ over $\Spec k$:
\[
\xymatrix{
G=\Spec R\ar[d]\\
\Spec k. \ar@/_1pc/[u]_e
}
\]
We can write where $R=k\opl J$ where $J$ is the augmentation ideal. Then we have the commutative diagram
\[
\xymatrix{
\Spec R=G\ar[r] & G^{(p^r)} \aq{r} & \Spec(R\ot_k k)\\
\ker F_{G/k}^{(r)}\ar[r]\ar[u] & \Spec k\ar[u] \aq{r} & \Spec R/J\ot_k k.
}
\]
%\fixme{???} 
Here the inclusion $k\to R\ot_k k$ is given by $F:k\to k$. Here $\ker F_{G/k}^{(r)}=\Spec R/J_r$, where $J_r$ is the ideal generated by $F^r(J)$. $F:R\to R$, $x\mapsto x^p$. We have $J_r\subeq J^{p^r}$ because $F^r(J)\subeq J^{p^r}$.

On the other hand, we have an inclusion in the other direction: $J$ has $p^N$ generators, for $N\gg 0$, and then $J^{p^{r+N}}\subeq J_r$ by purely algebra considerations (Pigeonhole Principle).
%one generator appear $p^r$ times, which appears in $F^r(J)$.

Once we have this, we can conclude that $F_G$ is nilpotent iff $J_r=0$ for $r\gg 0$. This is true iff $J^{p^s}=0$ for $s\gg 0$, iff $G$ is infinitesimal. ($J$ is maximal and nilpotent, so $G$ can only be infinitesimal.) 
\end{proof}
\subsection{Connected-\'etale sequence}
This is very important.

For simplicity, we assume $\chr k=p$ and let $G\in (p\text{-FCGp}/k)$.
\begin{pr}
There exists a canonical exact sequence
\[
0\to G^0 \to G\to G_{\text{\'et}}:=G/G^0\to 0
\]
%think of in fppf topology
where $G^0$ is the connected component of the identity element $e$, $G^0$ is infinitesimal, and $G_{\text{\'et}}$ is \'etale.
\end{pr}
%(We have to check $G_{\text{et}}$ is an \'etale group scheme.)
Note $G^0$ is infinitesimal because %we took 
the connected component is just a point. %, so there is just a point.
\begin{proof}
We may assume $k=\ol k$ because if the scheme is connected it is still connected over $\ol k$. Because the inclusion $G^{\circ}\hra G$ is open and closed, the inclusion 
\[
\Spec k\cong G^{\circ}/G^{\circ}\hra G/G^{\circ}
\]
is open. This says $\cO_{G_{\text{\'et}},e}\cong k$. %\fixme{Why?} 
By group translation, $\cO_{G/G_{\text{\'et}},p}$ is the same for every point $p$, and $G_{\text{et}}$ is \'etale. (A scheme is \'etale if it is \'etale at every point.) %can see at all physical points?
%dualize. etale, infin, or in between.
%refine?
\end{proof}
\subsubsection{Refining $G^{0}$}
Now we refine $G^{0}$. Apply the proposition to $G^{\vee}$ to obtain
\[
0\to (G^{\vee})^{0}\to G^{\vee}\to (G^{\vee})_{\text{\'et}}\to 0.
\]
By Cartier duality, \fixme{?}
\[
G_{\text{mult}}:=(G^{\vee}_{\text{\'et}})^{\vee}\to G^{\vee\vee}\cong G.
\]
Because $G_{\text{mult}}$ is multiplicative, i.e., its dual is \'etale, it must be infinitesimal. The map hence factors through $G^{\circ}$, the maximal infinitesimal subgroup scheme of $G$:
\[
\xymatrix{
G_{\text{mult}}:=(G_{\text{et}}^{\vee})^{\vee} \ar[r] \ha{rrd} & G^{\vee\vee} \aq{r} & G\\
&& G^{\circ}\ha{u}_{\text{max infinitesimal}}.
}
\]
We get an exact sequence
\[
0\to G_{\text{mult}} \to G^{0}\to G^{0}/G_{\text{mult}}\to 0.
\]
Here $G^{0}/G_{\text{mult}}=:G_{\text{bi}}$. We can check this is bi-infinitesimal.

\subsubsection{Splittings}
Assume $k$ is perfect and $\chr k=p$. We get $G\rd\subeq G$ is a subgroup scheme. (The condition that $k$ be perfect is essential. See~\cite[Exercise 3.2]{GGBM}.) We get
\[
\xymatrix{
G\rd \ar[r]\ar@/_1pc/[rr]_{\cong} & G \ar[r] & G_{\text{et}}.
}
\]
We can check that $G\rd(\ol k)\xrc G(\ol k)\xrc G_{\text{et}}(\ol k)$. %\fixme{So?} 
Hence we obtain a splitting
\[
\xymatrix{
0\ar[r] & G^{\circ} \ar[r] & G\ar[r] & G_{\text{et}} \ar[r] & 0\\
&&&G\rd\ha{lu}^{\exists\text{ section}} \aq{u}
}
\]
Hence $G\cong G^0\times G_{\text{et}}$ canonically. We can similarly split the sequence for $G^{\circ}$. The upshot is that we have the following.
\begin{thm}\llabel{thm:decomp-fcgp}
Suppose $G\in \fcg$ is a $p$-group, and $\chr k=0$. Then there is a unique decomposition
\[
G\cong G_{\text{mult}}\times G_{\text{bi}}\times G_{\text{\'et}}
\]
where $G_{\text{mult}}$ is multiplicative (with \'etale dual), $G_{\text{bi}}$ is bi-infinitesimal, and $G_{\text{\'et}}$ is \'etale.
\end{thm}