\lecture{Tue. 12/4/12}

Let $k$ be a field and let $G\in \pat{FCGp/$k$}$ be a $p$-group. Recall we had four possibilities. We are mainly interested in the characteristic $p$ case; then we have 3 possibilities.

\begin{center}
\begin{tabular}{|c|c|c|}
\hline 
$G\,\backslash\, G^{\vee}$  & \'etale & infinitesimal\tabularnewline
\hline 
\'etale & $\chr k\ne p$ & unipotent\tabularnewline
\hline 
infinitesimal & multiplicative & bi-infinitesimal\tabularnewline
\hline 
\end{tabular}
\end{center}

Recall that $G$ is \'etale iff $F_G:G\to G^{(p)}$ is an isomorphism, and $G$ is infinitesimal iff $F_G:G\to G^{(p)}$ is nilpotent. For $k$ perfect, we saw that we could decompose
\[
G\cong G_{\text{\'et}}\times G_{\text{mult}}\times G_{\text{bi}}.
\]
%\subsection{Dieudonn\'e Theory}
%linear alg category. easy to manip lin alge data
The goal of Dieudonn\'e Theory is to classify $p$-groups and $p$-divisible groups in terms of linear algebraic data. 

\subsection{Witt vectors}
\begin{df}
Let $n\ge 1$. Let $R$ be a ring and let $\mathbb W_n(R):=R^n$ equipped with the ring structure such that the map 
\[
\phi=(\phi_1,\ldots, \phi_n):\mathbb W_n(R)\xra R^n
\]
given by 
\[
\phi_i(X_1,\ldots, X_n)=X_1^{p^{i-1}}+pX_2^{p^{i-2}}+\cdots +p^{i-1}X_i
\]
is a homomorphism. %isomorphism. \fixme{is this an isomorphism?}
\end{df}
In other words, we define the ring structure on $\mathbb W_n(R)$ as the transport of the ring structure on $R^n$ via $\phi$.

We can define Witt vectors of infinite length by taking an inverse limit.
\begin{df}
Define
\[
\mathbb W(R):=\varprojlim \mathbb W_n(R)
\]
where the maps are the obvious projection maps 
\bal
\mathbb W_{n+1}(R)&\to \mathbb W_n(R)\\
(x_1,\ldots, x_{n+1})&\mapsto (x_1,\ldots, x_n).
\end{align*}
\end{df}
\begin{rem}
By taking $\Spec$, $\mathbb W_n$ and $\mathbb W$ can be though of as ring schemes over $\Spec \Z$.
\end{rem}
What's nice about this construction is that it comes with nice operators $F$, $T$, and $V$. %We introduce two operators: Verschiebung and translation. 
\begin{df}
Define the \textbf{Verschiebung}, \textbf{translation}, and \textbf{Frobenius maps}
%\[
%\xymatrix{
%\mathbb W_n \ar[r]^V\ar[rd]_T & \mathbb W_n\\
%& \mathbb W_{n+1}
%}
%\]
\bal
V:\mathbb W_n&\to \mathbb W_n\\
T:\mathbb W_n&\to \mathbb W_{n+1}\\
F:\mathbb W_n&\to \mathbb W_n
\end{align*}
by
\bal
V(x_1,\ldots, x_n)&=(0,x_1,\ldots, x_{n-1})\\
T(x_1,\ldots, x_n)&=(0,x_1,\ldots, x_{n})\\
F(x_1,\ldots, x_n)&=(x_1^p,\ldots, x_n^p).
\end{align*}
Taking the inverse limit, $F$ and $V$ operate on $\mathbb W$. For an extension $k/\Fp$, define $\mathbb W_k:=\mathbb W\times_{\Z} k$, and extend the action of $F$ and $V$ to $\mathbb W_k$.
\end{df}
Note that if $W$ is over $k$ and $k$ has characteristic $p$, then $F$ is a ring homomorphism and $V$ is an additive homomorphism.
\begin{lem}
$FV=VF=p$.
\end{lem}
Here $p$ is the map $x\mapsto \ub{x+\cdots +x}{p}$.
\begin{proof}
Computation.
\end{proof}

%\begin{df}
We are interested in the case where $k$ is a perfect field of characteristic $p$. Let
\[
W:=W_k=\mathbb W(k)=\varprojlim \mathbb W_n(k)
\]
In this case we also denote $F$ by $\si$; it's an automorphism.
%\end{df}
\begin{ex}
If $k=\Fp$, then $W=\zp$. \fixme{Ring integers in max unram extension of $\qp$. Integer ring of unram ext of $\qp$ of degree $r$.}
\end{ex}
We've completed the first step in introducing our linear algebraic category.
\subsection{Dieudonn\'e theory I}
\begin{df}
Fix a perfect field $k$ with characteristic $p$. 
Define the \textbf{Dieudonn\'e ring} $W[F,V]=W(k)[F,V]$ as the non-commutative ring generated by $F,V$ over $W$ with relations
\[
F\la =\si(\la) F ,\qquad  \la V =V\si(\la) ,\qquad FV=VF=p
\]
for all $\la\in W$.
\end{df}
The key definition is the following. This category classifies finite commutative group schemes over $k$ of $p$-power order.

\begin{df}
Define the category of \textbf{left Dieudonn\'e modules} by
\[
\Dmod:=\bt{left $W[F,V]$-modules}{of finite length as $W$-modules}
\]
\end{df}
Note that any object is killed by a power of $p$. This is true of a module $M$ of length 1 because in this case $pM$ is a submodule and hence must be 0; now use an induction argument. %bbecause torsion mod over ... killed power of p. induction argument

Rephrasing the definition, $M\in \Dmod$ is a Dieudonn\'e module iff $M$ is a $W$-module of finite length with maps
\begin{itemize}
\item
$F:M\to M$ that is $\si$-linear $F(\la m)=\si(\la) F(m)$,
\item
$V:M \to M$ that is $\si^{-1}$-linear, 
\end{itemize}•
such that $FV=VF=p$. 
Equivalently, $M$ has maps
\begin{itemize}
\item
$F_M:M^{(p)}\to M$ and
\item
$V_M:M\to M^{(p)}$
\end{itemize}•
where $M^{(p)}:=M\ot_W\to W$ with the map in the tensor product given by $\si:W\to W$. Note $M^{(p)}$ is also written $M^{\si}$ (Grothendieck).

We will see in a moment that these maps correspond to the Frobenius and Verschiebung maps on group schemes.
Our main theorem is the following.

\begin{thm}
Let $k$ be a perfect field of characteristic $p$.

There exists a canonical anti-equivalence of categories
\[
\pfcg \xra{\mathbb D} \Dmod
\]
with a canonical isomorphism $\mathbb D(G^{(p)})\cong \mathbb D (G)^{\si}$, such that the maps 
\begin{align*}
G&\xra{F_G} G^{(p)}\\
G & \lra{V_G} G^{(p)}
\end{align*}
corresponds under $\mathbb D$ to the maps 
\begin{align*}
M&\xla{F_M} M^{\si}\\
M& \xra{V_M} M^{\si}
\end{align*}
where $M=\mathbb D(G)$. In other words,
\begin{align*}
\mathbb D(F_G)&=F_{\mathbb D(G)}\\
\mathbb D(V_G)&=V_{\mathbb D(G)}.
\end{align*}
\end{thm}
One natural thing to check is how the Diedonn\'e functor is affect by base change. We have the following.
\begin{pr}
Let $K/k$ be perfect. Then
\[
\mathbb D(G\times_k K)\cong \mathbb D(G)\ot_{W_k}W_K
\]
functorially.
\end{pr}
%This is proved as a corollary of the main theorem.
This will fall out as a corollary of our main theorem.

Another natural question is how the order of the group is related to the length of the Dieudonn\'e module.
\begin{pr}
We have
\[
\log_p(|G|)=\text{length}_{W}\mathbb D(G)
\]
\end{pr}
%Z/pZ, mu_p, al_p
\begin{ex}
Let's revisit our three prototypical examples $\ul{\Z/p\Z},\mu_p,\al_p$. We can write down the corresponding Dieudonn\'e modules corresponding to them. The Diedonn\'e module is 
\[
\mathbb D(G)=W/pW=k
\]
with structure depending on $G$: (we'll justify these choices below.)
\begin{enumerate}
\item
If $G=\ul{\Z/p\Z}$, then $F=\si$ and $V=0$. ($F$ is a bijection.)
\item
If $G=\mu_p$, then $F=0$ and $V=\si^{-1}$. ($V$ is a bijection.) \fixme{$\si$ is really the identity?}
\item
If $G=\al_p$, then $F=V=0$.
\end{enumerate}•
\end{ex}
How do we actually match them up? %only stated abstract theorem. how do know correspond?
To check, we first reduce to the case where $k=\ol k$. 
Because $\mathbb D$ is an equivalence, we try to find simple objects in each category. If a Dieudonn\'e module has length 1, must be one of these three modules; these are the only $F$ and $V$ actions we can define.  %What $F$ and $V$ actions can we define? 
On the finite commutative group scheme side, $\ul{\Z/p\Z},\mu_p,\al_p$ are exactly all the simple objects, in the sense that cannot be decomposed further as smaller $p$-group schemes.

By considering simple objects on both sides, there are 6 possibilities for matching them! How should they be matched?

We use Proposition~\ref{pr:787-FV}. 
\begin{enumerate}
\item
Since $\ul{\Z/p\Z}$ is \'etale, the Frobenius should be bijective, so we must have $F=\si$.
\item
Since $\mu_p$ multiplicative, the Verschiebung should be bijective.
\item
Since $\al_p$ is bi-infinitesimal, $V,F$ are nilpotent.
\end{enumerate}
 
This gives a starting point for understanding $\mathbb D$. Using this example, we get a relationship between length of Dieudonn\'e modules and the order of the group schemes. To see this, we use the fact that given an arbitrary finite commutative $p$-group scheme ext of group scheme, we can find a filtration 
\[
0=G_0\sub\cdots \sub G.
\]
%Prove by induction argument. If we have arbitrary $G$
Each quotient will correspond to one of $\Z/p\Z$, $\mu_p$, and $\al_p$. 
%chain of short exact sequences.
%not only p-group schemes of order p. etale order scheme of order p. As many as Galois modules on Z/pZ
%Distinguished group schemes corresp to easy Dieu modules.

\subsection{Construction of $\mathbb D$}
\begin{df}
Consider an inductive system of group schemes
\[
\underrightarrow{\mathbb W}_k=\varinjlim \mathbb W_{n,k}
\]
via the translation maps $T: \mathbb W_{n,k}\to  \mathbb W_{n+1,k}$ defined by 
\[
T(x_1,\ldots, x_n)=(0,x_1,\ldots, x_n).
\]
%ind lim of ring schemes
We can define a $W[F,V]$-action on $ \underrightarrow{\mathbb W}_k$ because $F,V$ commute with $T$; this gives $F,V$ acting on $\underrightarrow{\mathbb W}_k$.
\end{df}
%There exists a $W$-action. 
%$R$-valued points give ring, comes with $W$-module structure. 
%See Grothendieck's exposition.

The idea of the construction is as follows. We carry it out in 3 steps.

Recall that for $p$-groups in characteristic $p$, group schemes are organized as follows: we have unipotent groups, multiplicative groups, and the intersection consists of the bi-infinitesimal group schemes.
%venn diagram
Because there is a product decomposition into these 3 parts, it is enough to define the Dieudonn\'e functor $\mathbb D$ on unipotent and on multiplicative groups, and then show we can put these definitions of $\mathbb D$ together.
%check the constructions agree on the intersection.
\begin{enumerate}
\item
$G$ is unipotent (i.e., $G^{\vee}$ is infinitesimal). Then define the functor
\[
\mathbb D(G):=\Hom_{k\text{-group}} (G,\underrightarrow{\mathbb W}_k).
\]
Note $\underrightarrow{\mathbb W}_k$ carries a $W[F,V]$-action. So this is actually a $W[F,V]$-module; we show it is of finite length. 
%because FV=VF=p, limit still have prop.
We still have $FV=VF=p$ because it is true at every finite level. 
\fixme{Pass from $G$ to $G^{\si}$, keep track of what Frobenius does on Dieudonn\'e module. Corresp to Frob on Dieudonn\'e module. (?)}

We have the following.
\begin{pr}\llabel{pr:787-23-unip}
There is an anti-equivalence of categories
\[
\bt{unipotent}{FCGp/$k$}\xra{\mathbb D}%\cong
\Mod_{W[F,V]}^{\text{fl, $V$-nilp}}.
\]
such that $\mathbb D(G^{(p)})\cong \mathbb D(G)^{\si}$, $\mathbb D(F_G)=F_{\mathbb D(G)}$, and $V_G=V_{\mathbb D(G)}$.

Decomposing into the \'etale and bi-infinitesimal part, this functor is the same as the product functor
\[
\bt{\'etale}{FCGp/$k$}\times \bt{bi-infinitesimal}{FCGp/$k$}\xra{\mathbb D} \Mod_{W[F,V]}^{\text{fl, $F$-bij}}\times \Mod_{W[F,V]}^{\text{fl, $F,V$-nilp}}.
\]
\end{pr}
\item
We do the same for multiplicative groups.
\begin{pr}\llabel{pr:787-23-mult}
There is an anti-equivalence of categories
\[
\bt{multiplicative}{FCGp/$k$}\xra{\mathbb D}  \Mod_{W[F,V]}^{\text{fl, $V$-bij}}.
\]
\end{pr}
\begin{proof}
The idea is to dualize twice and use the equivalence of categories from Proposition~\ref{pr:787-23-unip}. We have
\[
\bt{\'etale}{FCGp/$k$}\cong \Mod_{W[F,V]}^{\text{fl, $F$-bij}}.
\]
%Actually, dualize twice (if we only do it once we'd get a covariant functor). 
First take the Cartier dual. Then take the Pontryagin dual on category of finite abelian group schemes ($\Hom(\bullet, \qp/\zp)$). 
\end{proof}
\item
Decompose the linear category. Given an arbitrary Dieudonn\'e module, we claim there is a functorial decomposition into a $V$-nilpotent and $V$-bijective part. This is basically a linear algebra problem.
%group sch side, decomp in terms of category.
\begin{lem}\llabel{lem:787-23-decomp}
\[
\Mod_{W[F,V]}^{\text{fl}}
\cong
\Mod_{W[F,V]}^{\text{fl, $V$-nilp}}
\times
\Mod_{W[F,V]}^{\text{fl, $V$-bij}}
\]
\end{lem}
%This reduces the proof to $V$-nilpotent and $V$-bijective part.
\end{enumerate}
%jargon.
Propositions~\ref{pr:787-23-unip} and~\ref{pr:787-23-mult} and Lemma~\ref{lem:787-23-decomp} give the main theorem. A good exercise is to think about the Dieudonn\'e modules for $\Z/p^n\Z$ and $\mu_{p^n}$.

This is the Dieudonn\'e theory for finite commutative group schemes.

\subsection{$p$-divisible groups}
These are also called $p$-Barsotti-Tate groups. We define $p$-divisible groups and then explain what Dieudonn\'e theory tells us about $p$-divisible groups.
\begin{df}
A \textbf{$p$-divisible group} $\underrightarrow{G}/S$ of height $h\in \N_0$ is an inductive group schemes over $S$,
\[\underrightarrow{G}
=\varinjlim(G_n,i_n)
\]
such that $\rank G_n=p^{nh}$, the sequence
\[
0\to G_n \xhookrightarrow{i_n} G_{n+1} \xra{[p^n]} G_{n+1}
\]
is exact.

We have
\begin{align*}
\underrightarrow G:\schs &\to \gps\\
T&\mapsto \varinjlim G_n(T).
\end{align*}
This is represented by a fppf sheaf. The sheaf is $p$-divisible in that $[p]:\underrightarrow G\to \underrightarrow G$ is a fppf surjection.
\end{df}
The example to keep in mind (and the primary motivation for defining $p$-divisible groups) is the following.
\begin{ex}
Let $A\in \Abs$. Then $A[p^{\iy}]=\varinjlim A[p^n]$ is a $p$-divisible group.

For instance, $\ul{\qp/\zp}=\varinjlim \pa{\rc{p^n}\zp/\zp}$ and $\mu_{p^{\iy}}=\varinjlim \mu_{p^n}$ are both $p$-divisible groups. 
%frob twist for each G_n
\end{ex}
%suh that analogue true. Take proj lim of what done.
\begin{thm}
Let $k$ be perfect. There exists a canonical equivalence of categories
\[
\pat{$p$-divisible group/$k$}\xra{\mathbb D}\bt{free finite rank}{$W[F,V]$-modules}
\]
with a canonical isomorphism $\mathbb D(\underrightarrow{G}^{(p)})\cong \mathbb D(\underrightarrow{G})^{\si}$.
%isog not iso. Then lin alg cat involve simple classification. Rational number 0 to 1.
The equivalence is given by
\[
\mathbb D(\underrightarrow{G}):=\varprojlim \mathbb D(G_n).
\]
%$ $p$-divisible correspond to $W$-free. 
\end{thm}
