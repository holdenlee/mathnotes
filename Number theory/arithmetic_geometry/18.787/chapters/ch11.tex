\lecture{Thu. 10/18/12}

Today we'll continue talking about duality of abelian schemes (duality strikes back!). 

This section is a mess. See Mumford~\cite[\S13]{Mu70} or \cite[\S7.2]{GGBM}.

%Padma: Isogeny is an equivalence relation. Can we always find an isogeny back? Choose large $n$ such that kernel of isogeny is in $\ker[n]$. Then we get
%\[
%\xymatrix{
%A\ar[r] \ar[rd]_{[n]} & B\ard{d}\\
%& A/A[n].
%}
%\]
%%is there a section for multiplication by $n$? A/A[n] back to A??? \cong 
%(Work over a locally noetherian scheme.) %take rank of kernel
\subsection{Complements on $\Pic$}

Last time we worked modulo $p_2^*$ all the way. Today we'll give a slightly different but equivalent definition of $\Pic$. Let $A\in \Abs$. To avoid confusion, write
\begin{align*}
\Line(A)&=\{\text{line bundles over }A\}\\
\Pic(A)&=\Line(A)/\cong.
\end{align*}
\begin{df}
Define 
\[
\Pic_{e,A/S}:\schs\to \pat{Sets}
\]
by mapping
\[
T\mapsto \bt{$(L,\ep):\, L\in \Line(A\times_ST)$}{$\ep:\cO_T\xrc e_{A,T}^*L$}/\cong
\]
%no iso-> ignore such L
where $e_{A,T}$ is a section ($p_2\circ e_{A,T}=\id_T$), $e_A$ is a section ($f\circ e_A=\id_S$), and $p_1\circ e_{A,T}=e_A\circ g$:
\[
\xymatrix{A\ar[d]^f & A\times_S T\ar[l]_{p_1} \ar[d]_{p_2}\\
S\ar@/^/[u]^{e_A} & T\ar[l]_g\ar@/_/[u]_{e_{A,T}}.}
\]
\end{df}
Think of Definition~\ref{df:rel-pic} as the point variant of this definition. Here we put some condition. We  can show that
\begin{gather*}
\Pic_{e,A/S}\cong \Pic_{A/S}\\
(L,\ep)\mapsto L.
\end{gather*}
\fixme{
\[
T\mapsto \fc{\Pic(A\times_ST)}{p_2^*\Pic(T)}.
\]
arbitrary line bundle, no chance such iso in general, bc L lots line bundles in generl. But can mod by $p_2^*\Pic(T)$ and go back.Careful in choose what want.}
%L\to L', compat with $\ep$ with $\ep'$

Recall that we defined a subfunctor $\Pic_{A/S}^{\circ}$. Let the image of $\Pic_{A/S}^{\circ}$ under the above isomorphism be $\Pic_{e,A/S}^0$. We can take either definition as the dual abelian scheme $\av$ of $A$.

Our black-box theorem~\ref{thm:pic0-rep} tells us $\av \in \Abs$. The definition was fiberwise: $\Pic_{A/S}^{\circ}$ is basically the bundles in the connected components of every fiber. 

%Poincar\'e line bundle.
We have an isomorphism
\[
\av(\av)\cong \Pic_{e,A/S}^{\circ}(\av) =\set{(L,\ep)}{L\in \Line(A\times_S A^{\vee}), \ep:\cO_{\av}\xrc e_{A,\av}^*L}
\]
sending $\id_A$ to $(\cal P_A,\ep_A)$ where $\cal P_A$ is the Poincar\'e line bundle.  $\ep$ is called a \textbf{rigidification}. 
%unravel moduli-theoretic definition. 

\fixme{only way get line bundle on $A\times T$. $A\times T\to A\times \av$. Identity on $A$ and given map on $T$. }

Now what is $f\in \av(T)$ sent to under the isomorphism $\av(T)\cong \Pic_{e,A/S}^{\circ}(T)$? We saw above what $\id_A\in \av(\av)$ got sent to under $\av(\av)\cong \Pic_{e,A/S}^{\circ}(\av)$. Now the only way to get a line bundle on $A\times T\to A\times \av$ is to use the map $A\times T\to A\times \av$, which is the identity on $A$ and the given map on $T$. This motivates the following lemma.
\begin{lem}
We have that under the isomorphism $\av(T)\cong \Pic_{e,A/S}^{\circ}(T)$, $f$ is sent to $(1\times f)^* \cal P_A$:
\[
\xymatrix{
\av(T) \aq{r}\aq{d}& \Pic_{e,A/S}^{\circ}(T)\\
\Hom_S(T,\av) & {\rotatebox[origin=c]{90}{$\in$}}\\
f \ar@{}[u]|-{\rotatebox[origin=c]{90}{$\in$}}\ar@{|->}[r] & (1\times f)^*\cal P_A. 
}
\]
\end{lem}
Recall that given $f:A\to B$, we had an associated dual map $f^{\vee}:B^{\vee}\to \av$. We also constructed a canonical map $K_A:A\to (\av)^{\vee}$.  \\

\prbbox{Show that if we have maps of abelian varieties
$A\xra{f}B \xra{g}C$, then \[(g\circ f)^{\vee}=f^{\vee}\circ g^{\vee}.\]}

\subsection{Duality theorems}
Let $A,B\in \Abs$ be of relative dimension $g$. Let $f:A\to B$ be a morphism. 
\begin{pr}
\llabel{pr:787-11-1}
Let $A$ be an abelian variety of relative dimension $g$. Then
\[
\dim(A)=\dim(\av)=g.
\]
\end{pr}
\begin{pr}\llabel{pr:787-11-2}
\begin{enumerate}
\item
If $f$ is an isogeny, then $f^{\vee}$ is an isogeny. 
\item
$\deg f=\deg f^{\vee}$.
\item $K_A$ is an isomorphism.
\end{enumerate}•
\end{pr}

\fixme{We have 
\[
\xymatrix{
\Abs \ar@/^/[r]^{\vee} & \Abs\op\ar@/^/[l]^{\vee}
}
\]
giving that $\id\cong ((\bullet)^{\vee})^{\vee}$.}

\begin{proof}
Step 1 is to reduce to the case where $S=\Spec k$. Because $A$ and $\av$ are the same dimension by Proposition~\ref{pr:787-11-1}, it suffices to check $f^{\vee}$ is an isogeny fiberwise. 

\fixme{If know fiberwise isomorphism, then ?

For (3), check $K_A$ is an isogeny fiberwise. Also $\ker K_A$ has rank 1, $\deg K_A=1$. Why isogeny of degree 1 an isomorphism (exercise). 
This gives isomorphism}\\

\prbbox{
Exercise: An isogeny of degree 1 is an isomorphism. We have $\ker f\to S$ is locally free of rank 1. However we have a section:
\[
\xymatrix{
\ker f\ar[d]\\
S\ar@/^/[u]^{e_A}
}
\]
Hence $\ker f\cong S$.
\fixme{Using this, one can show Algebra, locally free of rank 1, but already have copy of $A$, show rest is 0.
\[
\xymatrix{
B=A\opl (\cdot)\\
A\ar[u]
}
\]
$(\cdot)=0$. }}

We can now work over fields. When $S=\Spec k$, Then $(L,\ep)$, and $(L,\ep')$ are always related by an isomorphism. Hence we can forget about $\ep$; only the existence of $\ep$ matters. \fixme{(Exercise. Mumford \S13) Isomorphism without specify.
Pullback along id section trivial.}
\end{proof}
Let's start the proof of Proposition~\ref{pr:787-11-1}. One proof uses cohomology of abelian varieties with line bundles as coefficients.
%keep making funny shape

\begin{proof}[Proof of Proposition~\ref{pr:787-11-1}]
We'll need 2 black boxes in our proof of Propositions~\ref{pr:787-11-1} and~\ref{pr:787-11-2}.
\begin{fct}[Black box 1,~{\cite[\S12.2]{Mu70}}]\llabel{fct:787-11-1}
%k alg closed, not nec.
Let $A\in \Abs$ be of relative dimension 0. Then
\[
\dim_k H^i(A,\cO_A)=\binom gi.
\]
In particular, there is no cohomology for $i>g$. 
\end{fct}
The proof involves a Koszul complex computation. It looks like a differential complex. The differential maps are all trivial. See Mumford for the construction of the complex.

We'll only need to use Fact~\ref{fct:787-11-1} for $H^1$. Then Proposition~\ref{pr:787-11-1} follows from the next proposition
\begin{pr}\llabel{pr:787-11-1'}
$T_e\av\cong H^1(A,\cO_A)$ as $k$-vector spaces. ($T_e\av$ is the tangent space at the identity.)
\end{pr}
Because $A$ is smooth, 
this implies $\dim \av=g$. Now as $k$-vector spaces, letting $S=\Spec k[\ep]/\ep^2$, 
\begin{align*}
T_e\av &= \set{
\phi\in \Hom(\Spec k[\ep]/\ep^2,\av)
}{\phi|_{\Spec k}=e_{\av}}\\
&=\ker (\Pic (A\times S)/p_2^* \Pic (k[\ep]/\ep^2) \to \Pic(A)/p_2^* \Pic(k)\\%def. 11.1?
&=\ker(H^1(A\times S,\cO_{A\times S}^{\times})\to H^1(A,\cO_A^{\times}))
\end{align*}
identifying $\Pic$ with $H^1$ cohomology (see Hartshorne~\cite[Exercise III.4.5]{Ha77}) and noting that $\Pic(k[\ep]/\ep^2)$ is trivial. 

As a topological space, $A\times S=A$. To find the kernel of the above map, we form a long exact sequence. 
\[
1\to 1+\ep \cO_A\to \cO_{A\times S}^{\times} \to \cO_A^{\times}\to 1.
%locally make sense epsilon
\]
This is like looking at $R\ot_k k[\ep]/\ep^2$. %invertible elt, \ep times st + 1. 
We get a long exact sequence 
\[
\xymatrix{
& H^0(\cO_{A\times S}^{\times}) \sj{r} & H^0(\cO_A^{\times}) \ar[lld]\\
H^1(1+\ep \cO_A) \ar[r] & H^1(\cO_{A\times S}^{\times})\ar[r] & H^1(\cO_A^{\times}).
}
\]
Note that as sheaves of abelian groups , we have an isomorphism
\[
\cO_A\xra{s} 1+\ep\cO_A,\qquad a\mapsto 1+\ep a.
\]
(This is just like a map $R\ot_k k[\ep]/\ep^2\to R$.) 

We conclude that $H^1(A,\cO_A)\cong T_e\av$ as (abstract) groups. It is left as an exercise to upgrade this to an isomorphism as $k$-vector spaces. 
%scalar mult by k happening. 
\end{proof}
Now we give the proof of Proposition~\ref{pr:787-11-2}.
\begin{proof}
Again, we need some input from cohomology. Let $A\in \abk$ and $L\in \Line(A)$. We define the Euler characteristic by
\[
\chi(L)=\chi_A(L):=\sum_{i\ge 0} (-1)^i \dim H^i(A,L)\in \Z.
\]
Studying the Euler characteristic is much easier than studying cohomology spaces, and it behaves nicely.
\begin{fct}[Black box 2]
\begin{enumerate}
\item
We have that
\[
\chi_{A\times \av} (\cal P_A)=(-1)^g.
\]
\item For $f:A\to B$ isogeny, $M\in \Line(B)$, 
\[
\chi_A(f^*L)=(\deg f) \chi_B(L).
\]
\end{enumerate}
\end{fct}
This is proved in Mumford~\cite[\S13.1--2]{Mu70}. In fact the cohomology is concentrated in degree $g$.:
\[
H^i(A\times \av,\cal P_A)=\begin{cases}
k &\text{if }i=g\\
0& i\ne g.
\end{cases}
\]


Now recall the definition of $f^{\vee}:B^{\vee}\to \av$. For all $T\in \schk$ (or over $S$, in general), %moduli interpretation.
%dual defined over arb base
we defined $f^{\vee}$ to make the diagram below commute.
\beq{eq:965-11-1}
\xymatrix{
B^{\vee}(T)\ar[d]_{f^{\vee}} \aq{r} & \Hom(T,B^{\vee})\ar[d]^{f^{\vee}\circ (\bullet)} \ha{r} & \Pic_e(B\times T)\ar[d]^{(f\times 1)^*}\\
\av(T) \aq{r} & \Hom(T,\av) \ha{r} &\Pic_e(A\times T). }
\eeq
\fixme{Notes}

%what's a good way to go between AxT, BxT: just pull back.
%yoneda:
\begin{lem}\llabel{lem:787-11-2}
For all $\la:A\to \av$, we have the commutative triangle
\[
\ctr{A}{(\av)^{\vee}}{\av}{K_A}{}{}{\la}{}{\la^{\vee}},
\]
i.e., $\la=\la^{\vee}\circ K_A$. 
\end{lem}
\begin{proof}
Plug into~\eqref{eq:965-11-1}. We have $B=\av$, $f=\la$, and $T=A$.
\end{proof}
Now we have a diagram
\[
\xymatrix{
K_A\ar@{|->}[r] \ar@{}[d]|-{\rotatebox[origin=c]{-90}{$\in$}}
\ar@/_4pc/[ddd]
& \tau^* \cal P\ar@{}[d]|-{\rotatebox[origin=c]{-90}{$\in$}}\ar@/^4pc/[ddd]\\
\Hom(A,(\av)^{\vee}) \ar[d]^{\la^{\vee}\circ \bullet}\ha{r} & \Pic_e(\av\times A)\ar[d]^{(\la\times 1)^*}\\
\Hom(A,\av)\ha{r}& \Pic_e(A\times A)\\
\la^{\vee}\circ K_A\mt{r} \inl\aq{d}& (\la\times 1)^* \tau^* \cal P_A \inl\aq{d}^{\sim}\\
\la^{\vee} \mt{r} &(1\times\la)^*\cal P_A
}
\]
\begin{cor}
If $\la:A\to \av$ is an isogeny then $\la^{\vee}$ is an isogeny. 
\end{cor}
\begin{proof}
Since the dimensions of $A$ and $\av$ are the same, to show $\la$ is an isogeny it suffices to show $\la$ is surjective.

If $\la $ is surjective, then $\la^{\vee}$ is also surjective by Lemma~\ref{lem:787-11-2}.
\end{proof}
\begin{lem}\llabel{lem:2poincare-bundle}
\[
(1\times f^{\vee})^* \cal P_A\cong (f\times 1)^*\cal P_B.
\]
where
\[
\xymatrix{
& A\times B^{\vee} \ar[ld]_{1\times f^{\vee}} \ar[rd]^{f\times1}&\\
A\times \av && B\times B^{\vee}
}
\]
We have $\cal P_A$ lives on the LHS and $\cal P_B$ lives on the  RHS. We now have
\[
\xymatrix{%forget rigid if trivial along id section
\id_{\av} \ar@{|->}[d]\ar@{}[r]|-{\in} & \Hom(\av,\av) \ha{r} \ar[d]^{()\circ f^{\vee}} & \Pic_{e,A/k}(\av) \ar@{}[r]|-{\ni} \ar[d]^{(1\times f^{\vee})^*}& \cal P_A\ar@{|->}[d]\\%func of Pic on top, how pic constructed
f^{\vee}& \Hom(B^{\vee},A^{\vee})\ha{r} & \Pic_{e,A/k} (B^{\vee}) & \text{same}\\
\id_{B^{\vee}} \ar@{}[r]|-{\in}\ar@{|->}[u] & \Hom(B^{\vee},B^{\vee}) \ar[u]_{f^{\vee}\circ ()} \ha{r} & \Pic_{e,B/k} (B^{\vee}) \ar[u]_{(f\times 1)^*} \ar@{}[r]|-{\ni} & \cal P_B \ar@{|->}[u]
}
\]
\end{lem}
\fixme{Top comes from def pic, bottom from EQ. Same on rhs.}
\begin{cor}
If $f$ and $f^{\vee}$ are isogenies, then $\deg f=\deg f^{\vee}$. 
\end{cor}
\begin{proof}%$A\times B^{\vee}$. 
%base change on isog, degree remains same. Think in terms of comm alg.
We have
\[
\chi((1\times f^{\vee})^* \cal P_A) =\chi((f\times 1)^*\cal P_B).
\]
The LHS is
\[
\deg(1\times f^{\vee})\chi(\cal P_A)=\deg(f^{\vee})\chi(\cal P_A)
\]
while the RHS is
\[
\deg(1\times f^{\vee})\chi(\cal P_B)=\deg(f)\chi(\cal P_B).
\]
But $\chi(\cal P_A)=\chi(\cal P_B)$ (why) so $\deg f=\deg f^{\vee}$. 
\end{proof}
%%
%\fixme{
\begin{cor}
%cor 3
$K_A$ is an isomorphism.
\end{cor}
\begin{proof}
\[
\ctr{A}{(\av)^{\vee}}{\av}{K_A}{}{}{\la_L}{}{\la_L^{\vee}}.
\]

Choose an ample line bundle $L\in \Line(A)$. Then $\la_L$ is an isogeny because $K(L)$ is finite over $k$. So $\la_L^{\vee}$ is an isogeny. $K_A$ is also an isogeny, since $\ker K_A\subeq \ker \la_L$. Now 
\[
\deg(\la_L)=\deg(\la_L^{\vee}\circ K_A)=\deg(\la_L^{\vee})\deg(K_A)
\]
think in terms of algebras, this is like $B$ locally free rank $m$ over $A$ and $C$ locally free rank $n$ over $B$, then $C$ locally free rank $mn$ over $A$.
Hence $\deg(K_A)=1$, and $K_A$ is an isomorphism.
\end{proof}
%doesn't follow from lem 1: we're talking about map between 2 different varieties A\to B. If B like \av already have answer.
If remains the prove the following.
\begin{lem}
If $f$ is an isogeny, then $f\vee$ is an isogeny. 
\end{lem}
\begin{rem}
(maybe better) show $\ker f^{\vee}$ is Cartier dual  $\ker f$.
\end{rem}
Here's an exercise: The following commutes.
\[
\commsq{A}{(\av)^{\vee}}{B}{(B^{\vee})^{\vee}}{f}{K_A}{(f^{\vee})^{\vee}}{K_B}
\]
I.e. ``$f=(f^{\vee})^{\vee}$". Since the across maps are iso, if $f$ is an isogeny then $f$ is surjective, $\dim A=\dim B$, $(f^{\vee})^{\vee}$ is surjective. If $f^{\vee}:B^{\vee}\to A^{\vee}$ not isogeny then not surjective. Then $f^{\vee}(B^{\vee})\subeq A^{\vee}$, subabelian variety of dimension less than $g$. 
image proper group subvariety.

But dimension less than $g$ can't surject to dimension $g$, contradiction.
\[
B^{\vee}\to f^{\vee}(B^{\vee}) \to A^{\vee}
\]
where $f^{\vee}(B^{\vee})$ has dimension less than $g$.
\[
(B^{\vee})^{\vee} \leftarrow ()^{\vee} \leftarrow (\av)^{\vee}
\]
dim $<g$ in middle, $(f^{\vee})^{\vee}$ surjective. dim $g$ on left. Contradiction, end proof of Prop. 2. 
\end{proof}%}
%%
