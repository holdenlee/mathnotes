\lecture{Thu. 9/27/12}


\subsection{Isogenies}
Recall the definition of an isogeny: a morphism $f:A\to A'$ in $(\text{Ab}/S)$ is an \textbf{isogeny} if $f$ is quasi-finite and surjective on points.

We showed $f$ quasi-finite and surjective is equivalent to $f$ being finite faithfully flat (Lemma~\ref{lem:787-6-1}). Note the second set of conditions is can be checked fiberwise, hence is easier to work with. In particular, $f$ is an isogeny iff $f_s:A_s\to A_s'$ is isogeny for all $s\in S$, where $A_s:=A\times_S \Spec k(s)$. 

Equivalently to $f$ being quasi-finite, $\ker f$ is quasi-finite over $S$. %ref
Indeed, if we knew every fiber over 0 were finite, then by group translation the fiber over every point is finite.

Moreover, the fact that $f$ is an isogeny implies $\ker f$ is a finite flat locally free commutative group scheme over $S$ (Lemma~\ref{lem:787-6-2}). We see that finite flat group schemes arise naturally in this context.

We would like a notion of rank or dimension for abelian schemes.
\begin{df}
Define the function $\rank:S\to \Z_{\ge 0}$ by mapping $s\mapsto \rank(\ker f_s)$.
\end{df}
The rank is locally constant in the topological sense. If $S$ is connected, then the rank is constant.
It is natural to ask: what does an isogeny do the the rank?

It is clear that the composition of two isogenies is an isogeny and an isomorphism is an isogeny. We'll now show an isogeny preserves a natural invariant in geometry, the dimension.

\begin{df}
We say $A\in \Abs$ has \textbf{relative dimension} $d\in \Z_{\ge 0}$ over $S$ if for all $s\in S$, $\dim A_s=d$. 
\end{df}
As a reminder, the dimension is the maximum length of closed subsets on the scheme, or the dimension of the corresponding ring in the sense of Krull dimension.

Note that the dimension may not be defined, For instance if $A_1$ has relative dimension 1 over $S_1$ and $A_2$ has relative dimension of dimension 2 over $S_2$. Then $A=A_1\coprod A_2$ over $S=S_1\coprod S_2$ does not have a dimension.

%As with rank, the dimension is locally constant.
\begin{df}
%For $d=1$ we get elliptic curve. In fact, 
An \textbf{elliptic curve} is an abelian variety of dimension 1.
\end{df}
\begin{lem}
let $f:A\to A'$ be an isogeny. For all $s\in S$, $\dim A_s=\dim A_s'$.
\end{lem}
\begin{proof}%have stronger result
Let's import the following fact. 
\begin{lem}\llabel{lem:787-7-1}
For varieties $X$ and $Y$ over a field $k$, if $\phi:X\to Y$ is an open morphism then for all $y\in \phi(X)$, $\dim\phi^{-1}(y)$ is the same. In fact,
\[
\dim X=\dim Y+\dim \phi^{-1}(y).
\]
\end{lem}
\begin{proof}
See~\cite[Th. 14.114 and Pr. 14.102]{GW}.
\end{proof}
%dim X-\dim Y.
%If $f$ is quasifinite, then $\dim f^{-1}(y)=0$, 
Reduce to the case where $S=\Spec k$ and apply to $f:A\to A'$ varieties over $k$. Then $f$ is quasi-finite; it is open because it is flat and locally of finite presentation (\cite{GW}[Th. 14.33]). Now $\dim f^{-1}(y)=0$ because $f$ is quasi-finite. %?
Thus we get $\dim A=\dim A'$.
\end{proof}
What about the converse? If the dimensions are the same, can we say that the map is an isogeny? We'll need some more conditions.
%Trivial homomorphism is not an isogeny.
\begin{lem}\llabel{lem:787-7-2}
Let $f:A\to A'$ be a morphism in $\Abs$. Assume $\dim A_s=\dim A_s'$ for all $s\in S$. Then the following are equivalent.
\begin{enumerate}
\item
$f$ is an isogeny. %q-f and surjective
\item
$f$ is quasi-finite.
\item
$f$ is surjective.
\end{enumerate}
\end{lem}
Recall $f$ is an isogeny if it is quasi-finite and surjective. This lemma says that if the dimensions are the same, we can just check one of the conditions.

\begin{proof}
Since (1) is by definition just (2) and (3) combined, it suffices to prove (2)$\iff$(3).

Reduce to the case where $S=\Spec k$. %, and apply to the imported result~\ref{lem:787-7-1}. We have (2) is equivalent to $\dim f^{-1}(y)=0$ for all $y\in f(A)$, which is equivalent to $\dim A=\dim A'$. \fixme{problem? see details below}
%trivial hom not flat.

Let $B:=\ol{f(A)}\subeq A'$ where $f(A)$ is the scheme-theoretic image of $A$ and $A'$ is irreducible containing $\ol{f(A)}$. We have
\beq{eq:787-7-1}
\dim B=\dim A' \iff B=A'\text{ as a set} \iff f(A)=A'
\eeq
because $f$ is proper. (A morphism between two proper schemes is proper.) %$\ctr{A}{A'}{k}{}{}{}{\text{proper}}{}{\text{proper}}$.
%WHY??????????????

Applying generic flatness~\ref{pr:generic-flatness} to $f:A\to B\subeq A'$, we get that $f$ is flat on the inverse image of an open dense subset $V\subeq B$. %Note $f:A\to B$ has dense image, and is 
%%scheme-theoretic closure
%even surjective (by the same reasoning, because it's proper).

%\fixme{Don't quite get this.} Then there exists open $V\subeq B$, and 
Note $f^{-1}(V)\ne \phi$ because $f$ has dense image in $B$. 
Because $f$ on $f|_{f^{-1}(V)}$ is flat and locally of finite presentation, it is open. \fixme{(Why?)}  This means we can apply our imported result~\ref{lem:787-7-1}.
%We can apply the fact that $f$ is an open morphism such that $f^{-1}(V)\xra{f}\to V$ is flat. 

We have the following chain of equivalences.
\begin{enumerate}
\item $f$ is quasi-finite (on one fiber, or on all fibers).
\item $\dim f^{-1}(V)=\dim V$. (To go between statements 1 and 2, use Lemma~\ref{lem:787-7-1}; quasifiniteness says $\dim f^{-1}(y)=0$.)
\item $\dim A=\dim B$. (Note $\dim A=\dim f^{-1}(V)$ and $\dim B=\dim V$ because these are open dense subsets.)
\item $f$ is surjective. (To go between 3 and 4, use~\eqref{eq:787-7-1}.)
\end{enumerate}
%Note $f^{-1}(V)$ has the same dimension as $V$ and $V$ has the same dimension as $B$. 
%Applying our imported result~\ref{lem:787-7-1}, we have (ii) holds iff $\dim f^{-1}(V)=\dim V$, iff $\dim A=\dim B$ iff (iii) holds.
\end{proof}
Now for an application.
\begin{ex}
To check the relative Frobenius map $\Frob: A\to A^{(p)}=A\times_S S$ (where $S\to S$ is given by the Frobenius map) and $[n]:A\to A$ are isogenies, we can work over an algebraically closed field
(if a morphism is quasi-finite over a field extension, then it is already quasi-finite before base extension) and check it is either quasi-finite or surjective. %since the kernels are finite.
\end{ex}
We summarize how we can check a morphism $f$ is an isogeny.\\

\cpbox{
Checking that $f$ is an isogeny can be checked fiberwise. To check quasi-finiteness, we can just check the fiber over 0, i.e., check $\ker f$ is quasi-finite over $S$. Moreover, if the fibers of $A,A'$ have the same dimension, it suffices to check just either surjectivity or quasi-finiteness.
}
\vskip0.15in
\subsection{Motivation: Classification}
We'll give some motivations, and then talk about line bundles.

When we're studying any type of geometric objects, the first problem is that of classification. We want to classify abelian varieties in some sensible manner.
\begin{enumerate}
\item
Let $A$ be an abelian scheme. We will assume that 
%Let's accept that 
$A[n]:=\ker[n]$ is quasi-finite over $S$. %i.e. [n] is an isogeny.
Then $A[n]$ is a finite locally free commutative group scheme over $S$ by Lemma~\ref{lem:787-6-2}. 
Hence a good first step is to study or classify finite locally free commutative group scheme over $S$. 
\end{enumerate}
Of particular interest to us is the case $S=\Spec k$ (or the Spec over a ring of mixed characteristic such as $\Z_p$). Regarding $A[n]$, we could work with one prime at a time, by considering the $p$-power torsion of the group schemes:
\[
A[p^{\iy}]=\bigcup_{n\ge 1} A[p^n]
\]
over $S$. More precisely, interpret the RHS as a directed system $\varinjlim_nA[p^n]$. We call these objects \textbf{$p$-divisible groups}. $p$-divisible groups are simpler than abelian schemes, so we can study these objects first and then use them to classify abelian schemes.

For number theory, we can try $S=\Spec$ of $\F_{p^r}$, $\ol{\F_{p}}$, $\Z_p$, $\mathbb W(\ol{\F_p})$, and then $\Q_p$, $\ol{\Q_p}$, $\Q$, and $\ol{\Q}$. We're interested in rational points of abelian varieties over number fields; this is still a mysterious topic and we don't know much.

Let's look at the classification problem from a different perspective. By introducing isogenies, we can go about classification as follows.
\begin{enumerate}
\item
The dimension is invariant, so let's fix the dimension.
\item
An isogeny gives an equivalence relation. We classify isogeny classes within each dimension.
\item An isogeny is strictly weaker than isomorphism. Thus we now classify isomorphism classes within each isogeny class.
\end{enumerate}%isog class abel scheme
Related to problem 2 is 
\begin{enumerate}
\item[2$'$.]
Classify isogeny classes of $p$-divisible groups.
\end{enumerate}
It's easier solve the related problem for $p$-divisible groups:
The theory is already very rich here. Then we can use the solution for $p$-divisible groups to solve the problem for abelian varieties.

We could also study $\Hom(A,B)$, $\Isog(A,B)$, and so forth (and do the same for $p$-divisible groups, etc.).
\begin{ex}
Let the dimension be $d=1$, and $S=\Spec \ol{\F_p}$. Then $A$ is an elliptic curve over $\ol{\F_p}$. We classify according to the isogeny type of $A[p^{\iy}]$. We have exactly 2 possibilities, corresponding to 2 isogeny classes:
\begin{enumerate}
\item
$A$ is \textbf{ordinary} if  (as a group)
\[
A[p](\ol{\F_p})\cong \Z/p\Z.
\]
\item
$A$ is \textbf{supersingular} if
\[
A[p](\ol{\F_p})\cong \{0\}.
\]
\end{enumerate}
The isogeny class determine many properties of the elliptic curve.

If we consider isogeny types over $A$ itself, we get a much richer theory. %kummer-tate theory.
%dieudonne manin theory
\end{ex}
\subsection{Line bundles}
Let's study line bundles over abelian varieties. (It's easier to work with Spec of a field.)

We will talk about the following.
\begin{enumerate}
\item
Abelian varieties are projective over fields, Theorem~\ref{thm:ab-var-proj}. (To prove projectivity we need to find ample line bundles.) 
\item Duality theory: This imposes a strict condition on what the torsion group scheme should be.
\item Polarization
\item As a byproduct we'll see that $[n]$ is an isogeny.
\end{enumerate}
%that should be enough motivation!

We're entering Mumford~\cite[\S 6]{Mu70}. We'll try to just work over fields, rather than algebraically closed fields as Mumford does.

\begin{thm}[Theorem of the cube]\llabel{thm:cube}%not most general
Let $X$ and $Y$ be proper varieties %int'l scheme sep fin type /k
over $k$ and $Z$ be a variety over $k$.
Let $x\in X$, $y\in Y$, and $z\in Z$ be scheme-theoretic points. Let $L$ be a line bundle over $X\times_k Y\times_k Z$. If
$L|_{\{x\}\times Y\times Z}$, $L|_{X\times \{y\}\times Z}$, and $L|_{X\times Y\times \{z\}}$ are trivial, then $L$ is trivial. (By $\{x\}\times Y\times Z$ we mean the pullback of the line bundle along the closed immersion $\{x\}\times Y\times Z$, and so forth.)
%trivial if isomorphic to structure sheaf.
\end{thm}
We'll postpone the proof and treat this theorem as a black box for now. 
\begin{cor}
Let $A\in (\text{Ab}/k)$ where $k$ is any field. Define the maps
\[
\xymatrix@R-24pt{
A\times_kA\times_kA \ar[r] & A\\
(x_1,x_2,x_3) \ar@{|->}[r]^{p_{123}} 
\ar@{|->}[rd]^{p_{ij}}
\ar@{|->}[rdd]_{p_{i}}  
& x_1+x_2+x_3\\
& x_i+x_j\\
& x_i.
}
\]
Then $\Te(L):=p_{123}^*L \ot \pa{\bigotimes_{i<j} p_{ij}^*L^{-1}}\ot \pa{\bigotimes_i p_i^* L}$ is trivial. 
\end{cor}
We try to restrict this line bundle to subschemes of the form $\{x\}\times X\times Y$ and show it's trivial. For $x$ we choose the natural distinguished point---the identity. Then by the Theorem of the Cube we can get the line bundle to be trivial over the triple product.
\begin{proof}
We show that $\Te(L)|_{\{e_A\}\times A\times A}$ is trivial (Restrict to $x_1=0$ part), and show the same for the other restrictions. % to $A\times \{e_A\}\times A$. 

We will compute $i_{23}^* \Te(L)$, where $i_{23}: \{e_A\}\times A\times A\hra A\times A\times A$ is the obvious inclusion.

For example, to compute $i_{23}^* p_{123}^* L=(p_{123}\circ i_{23})^* L$, note we have the commutative diagram
\[
\xymatrix{
\{e_A\}\times A\times A\ha{r} \ar[rd]^{\mu}_{(x_2,x_3)\mapsto x_2+x_3} & A\times A\times A\ar[d]^{p_{123}}\\
& A
}
\]

\fixme{Picture!}


We can similarly compute the others, and cancel nicely!
%theorem of cube tells you enough to restrict to x_1=0. 
When restrict to $x_1=0$, $p_{123}=p_{23}$ cancel out, $p_{12}=p_2$ cancel out, $p_{13}=p_3$ cancel out, and $p_1=0$, so everything vanishes.
\end{proof}
\begin{cor}\llabel{cor:787-7-2}
Let $Y\in (\text{Sch}/k)$ and $A\in (\text{Ab}/k)$. Let $f_1,f_2,f_3\in \Hom_{\text{Sch}/k}(Y,A)$. Then define
\begin{align*}
f_{123}&=f_1+f_2+f_3=p_{123}\circ (f_1,f_2,f_3)\\
f_{ij} & =f_i+f_j=p_{ij}\circ (f_1,f_2,f_3).
\end{align*}
(The image has group structure so can make sense of adding---map to $A\times A\times A$ and compose with $p_{123}$.) Let $L\in \Pic(A)$. 
Then 
\[f_{123}^*L \ot \pa{\bigotimes_{i<j} f_{ij}^*L^{-1}}\ot \pa{\bigotimes_i f_i^* L}
\]
is trivial. 
\end{cor}
\begin{proof}
This equals $(f_1,f_2,f_3)^*\Te(L)$ so it is trivial by the previous corollary. %for instance $(f_1,f_2,f_3)^*p_{123}^* L$.
\end{proof}

\begin{cor}\llabel{cor:787-7-3}
We have $[n]^* L=L^{\ot\fc{n(n+1)}{2}}\ot [-1]^* L^{\ot \fc{n(n-1)}{2}}$ where $A$ and $L$ are as before. 
\end{cor}
\begin{proof}
The idea is to use Corollary~\ref{cor:787-7-2} along with induction.

Apply the corollary to $f_1=[n]$, $f_2=[1]$ (the identity map), and $f_3=[-1]:=i_A$. (Note $[n]$ is defined for all $n\in \Z$.) We get that
\[
[n]^*L\ot ([n+1]^* L^{-1} \ot [n-1]^*L^{-1} \ot \underbrace{[0]^*L^{-1}}_{\text{trivial}}) \ot [n]^* L\ot \underbrace{[1]^*L}_{L} \ot [-1]^* L \text{ is trivial.}
\]
In $\Pic(A)$, we have $[n+1]^*L-2[n]^* L+[n-1]^*L=L+[-1]^* L$. Use induction (plug in $n=1,2,3,\ldots$ and $-1,-2,-3,\ldots$) to get the result.
\end{proof}