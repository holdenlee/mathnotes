\lecture{Thu. 11/1/12}

%how use upper part of background board?
Last time we showed that in $\abk$, assuming $(\chr k, \ell)=1$, there is an injection
\[
\Z_{\ell}\ot \Hom_{(\text{Gp}/k)}(A,B)\hra \Hom_{\Z_{\ell}\text{--module}} (T_{\ell}A,T_{\ell}B).
\]
 We have that $\End^0(A)$ is a finite dimensional semisimple algebra over $\Q$: decomposing $A\sim \prod_{i=1}^r A_i^{n_i}$ where $A_i$ is simple, we get
\[
\End^0(A)\cong \prod_{i=1}^r  \cal M_{n_i} (\End^0(A_i)).
\]
Here each $\End^0(A_i)$ is a finite division algebra over $\Q$, so $\End^0(A)$ is finite-dimensional semisimple.

\subsection{Central simple algebras}
\begin{df}
Let $F$ be a field of characteristic 0, such as a finite extension of $\Q$ and $\Q_{\ell}$. 
A \textbf{central simple algebra (CSA)} $D$ over $F$  is a finite-dimensional simple $F$-algebra with center $F$.
\end{df}
%Any matrix algebra.
\begin{ex}
Any matrix algebra $D=\cal M_n(F)$ is a central simple algebra.
In particular, for $n=1$, $D=F$ is a central simple algebra over $F$.
\end{ex}
We have the following facts.
\begin{pr}
The following hold.
\begin{enumerate}
\item
Every central simple algebra over $F$ splits over the algebraic closure of $F$:
\[
D\ot_F\ol{F} \cong \cal M_n(\ol F).
\]
\item
$[D:F]=n^2$ for some $n\in \Z$.
\item
$D\cong \cal M_r(D')$ for some central division algebra $D'/F$, and we have
\[
r[D':F]^{\rc 2}=[D:F]^{\rc 2}.
\]
\end{enumerate}
The central simple algebras are classified by the \textbf{Brauer group}
\[
\Br(F)=\{\text{CSA}/F\}/\sim
\]
with multiplication given by $\ot$ and equivalence relation $D\sim D'$ if $\cal M_r(D)\cong \cal M_{r'} (D')$ for some $r,r'$.
\end{pr}
Just like we have a norm and trace for a field extension $K/F$, we have a norm and trace for a CSA $D/F$. We define the reduced norm and trace and then relate them to general norm and trace functions.

Normally we define the norm and trace as the determinant and trace of the multiplication map considered on the space itself $D$. But the dimension of $D/F$ is $n^2$, which is too large. Hence we define the reduced norm and trace: we consider the multiplication map on $\ol F^n$ instead. We can do this becaue by Proposition~\ref{pr:787-15-1}, $D\ot_F\ol F\cong \cal M_n(\ol F)$, and by Noether-Skolem, any automorphism of $\cal M_n(\ol F)$ is given by an inner automorphism and doesn't change the determinant and trace.
\begin{df}
Define the \textbf{reduced norm} $N_{D/F}^{\circ}$ and \textbf{reduced trace} $T_{D/F}^{\circ}$ by
\begin{align*}
N_{D/F}^{\circ}(a):&=\text{det}_{\ol F}(a\ot 1|_{\ol F^n})\\
T_{D/F}^{\circ}(a):&=\tr_{\ol F}(a\ot 1|_{\ol F^n})
\end{align*}
where $n=[D:F]^{\rc 2}$. 
\end{df}
One can check that $N_{D/F}^{\circ}(a)$ and $T_{D/F}^{\circ}(a)$ are invariant under $G(\ol F/F)$; hence they are in $F$.

%inner aut
%conjugation by an element, doesn't change det and trace.
%We put something smaller than $D$; this is why it's reduced. We skip checking this is well-defined.

\begin{df}
Let $D/F$ be a CSA, and $F/E$ be a separable extension.
\begin{enumerate}
\item
A \textbf{norm form} of $D$ over $E$ is a nonzero polynomial function $N:D\to E$ such that
\[
N(x_1x_2)=N(x_1)N(x_2).
\]
\item
A \textbf{trace form} of $D$ over $E$ is a function $T:D\to E$ that is $E$-linear (in particular $T(x_1+x_2)=T(x_1)+T(x_2)$) and such that $T(xy)=T(yx)$.
\end{enumerate}
\end{df}
%imitate norm and trace in linear algebra.

Why are we interested in norm and trace forms? Recall that we had a degree function on $\End(A)$, and we extended it to $\End^0(A)$. Because degree is multiplicative, it is a norm form. 

By proving a classification theorem for norm forms, we can better understand $\deg$.
\begin{ex}
%usual norm trace for num field, sep ext
Let $D/F$ be a CSA and $F/E$ a separable extension.
\begin{itemize}
\item
$N_{D/F}^{\circ}$ and $T_{D/F}^{\circ}$ are norm/trace forms on $D/F$. 
\item In the situation of the definition,
\begin{align*}
N_{D/E}^{\text{min}}:&=\nm_{F/E}\circ N_{D/F}^{\circ}\\
T_{D/E}^{\text{min}}:&=\tr_{F/E}\circ T_{D/F}^{\circ}
\end{align*}
are norm/trace forms on $D/E$, 
called the \textbf{minimal norm} and \textbf{minimal trace}.

This is a typical way to get a norm and trace form.
\end{itemize}
\end{ex}
As the following shows, all norm and trace forms come from the minimal norm and trace.
\begin{lem}[Classification of norm/trace forms]\llabel{lem:classify-norm}
\begin{itemize}
\item
Any norm form $N$ on $D/E$ is in the form $(N_{D/E}^{\text{min}})^m$, where $m\in \Z$.
\item
Any trace form $T$ on $D/E$ is in the form $\phi\circ T_{D/F}^{\circ}$, where  $\phi:F\to E$ is a $E$-linear map.
\end{itemize}
\end{lem}
\begin{proof}[Proof for the special case $E=F=\ol F$]
%over alg clos csa is mat alg
We have $D\cong \cal M_n(F)$ by Proposition~\ref{pr:787-15-1}(1) so $N$ gives rise to a morphism of group schemes (actually algebraic groups) over $F$ 
\[\GL_n\to \G_m.\] %spec. just write down the polynomial map. 
%group schemes over F.
%Being a norm form gives 
The fact that $N(x_1x_2)=N(x_1)N(x_2)$ means this is a group homomorphism, so a morphism of algebraic groups. %always restrict to $\GL_n$.
We have
\[
N(x_1x_2x_1^{-1}x_2^{-2})=1
\]
by multiplicativity (because the target $\G_m$ is commutative). So $N=1$ on the closed subgroup scheme $\SL_n$ generated by commutators. Thus the morphism factors
\[
\xymatrix{
\GL_n\ar[r]^N \ar[d] \ar[rd]^{N_{D/F}^{\min}}  & \G_m\\
\GL_n/\SL_n \ar[r]^{\det}_{\cong} & \G_m. \ard{u}_{\hat{\,}m,\,m\in \Z}
}
\]
%For algebraic groups
But the only map $\G_m\to \G_m$ is the $m$th power map. A morphism $\G_m\to \G_m$ corresponds to a map $k[t,t^{-1}]\to k[t,t^{-1}]$; to be  a morphism of group schemes it has to be the $m$th power map.

For the trace form, note $T(xy-yx)=0$ for all $x,y\in \cal M_n(F)$, the elements $xy-yx$ generate the trace 0 part in $\cal M_n(F)$ (prove this by a hands-on approach), so we similarly get the trace form factors as follows.
\[
\xymatrix{
\cal M_n(F) \ar[r]^T \ar[d]\ar[rd]^{T_{D/F}^{\circ}} & E\\
\fc{\cal M_n(F)}{\an{xy-yx}}  \ar[r]^-{\tr}_-{\cong} & F.\ard{u}_{\exists E\text{--linear}}
}
\]
All maps in the diagram are $E$-linear, so the map $F\to E$ is $E$-linear.
\end{proof}

\begin{thm}[Mumford~\cite{Mu70}, Theorem 19.4]
Suppose $(\chr k,\ell)=1$, $A\in \abk$ and $\dim A=g$. Suppose $f\in \End(A)\ot_{\Z} \Z_{\ell}$; we have $T_{\ell}(f)\in \End_{\Z_{\ell}} (T_{\ell}A)$. Then the following hold.
\begin{enumerate}
\item
$\deg f=\det T_{\ell}f$
\item $P_f(n):=\deg([n]-f)=\det([n]-T_{\ell}(f)) $ is  a monic polynomial of degree $2g$ with coefficients in $\Z$. %n is the variable. identify n with [n] in endo algebra.
\item
$P_f(f)=0$ in $\End(A)$. 
\end{enumerate}
\end{thm}
%when A finite field?
%top homology group canonically gth exterior power.
\begin{rem}
The philosophy is that because $f$ has geometric origin, $T_{\ell}(f)$ in a sense doesn't depend on $\ell$. 
(2) implies that the $P_f(n)$ are independent of $\ell$, and (1) implies that $\det T_{\ell}(f)$ is independent of $\ell$.

In $\ell$-adic \'etale cohomology, we often see independence of $\ell$. Independence of $\ell$ questions are known in many cases (though not everywhere). You can interpret this theorem in the context of \'etale cohomology.
\end{rem}

\begin{proof}
Part 1 is about the equality of 2 norm forms. Recall that we extended $\deg$ from $\End(A)$ to $\End^0(A)$; it is a norm form
\[
N_1:\End^0(A)\xra{\deg}\Q.
\]
The determinant also gives a norm form
\[
N_2:\End(A)\ot_{\Z}\Z_{\ell} \xra{\det T_{\ell}(\cdot)} \Z_{\ell}.
\]
By Theorem~\ref{thm:deg-poly}, $\deg$ is homogeneous of degree $2g$; $\det \tl(\bullet)$ is also homogeneous of degree $2g$ because  it is the determinant of an endomorphism of rank $2g$-modules.) We get two norm forms that are homogeneous of degree $2g$: %know what to do when extend scalar.
\[
N_1,N_2:\End^0(A)\ot_{\Q} \Q_{\ell} \to \Q_{\ell}.
\]
To prove part 1 we need to show $N_1=N_2$. We do this in 2 steps.\\

\step{1} We show that $N_1$ and $N_2$ have the same $\ell$-adic evaluation:
\[
|N_1(\al)|_{\ell}=|N_2(\al)|_{\ell}\quad\text{ for all }\al\in \End^0(A)\ot_{\Q} \Q_{\ell}.
\]
Write $\al=\ell^{-a}\al_0$ where $\al_0\in \End A\ot \Z_{\ell}$. Because $N_1$ and $N_2$ are homogeneous of the same degree, it suffices to show that 
\[
|N_1(\al_0)|_{\ell}=|N_2(\al_0)|_{\ell}\quad\text{for all $\al_0\in \End(A)\ot_{\Z}\zl$}.
\]
Now write
\begin{align*}
N_1(\al_0)&= \ell^{n_1}\cdot u_1\\
N_2(\al_0)&= \ell^{n_2}\cdot u_2
\end{align*}
where $u_i\in \Z_{\ell}^{\times}$. We show that $n_1=n_2$. %bc dealing with tate modules, good look at l-power torsion of A and how maps work on them.
To show this, consider $\al_0$ as a map $A[\ell^N]\to A[\ell^N]$ for $N$ finite; this gives us a description of $n_1$. The map on the Tate modules $\tl A\xra{\tl(\al_0)} \tl A$ gives us a description of $n_2$. By taking an inverse limit of the maps $A[\ell^N]\xra{\al_0} A[\ell^N]$, we can relate $n_1$ and $n_2$. 
For any $N\ge0$, we have the exact sequences
\[
\xymatrix{
0\ar[r] & \ker\al_0(N) \ar[r] & A[\ell^N] \ar[r]^{\al_0(N)}  & A[\ell^N] \ar[r] & \coker \al_0(N)\ar[r] & 0\\
0 \ar[r] & \ker\al_0(N+1) \ar[r]\ar[u]^{[\ell]} & A[\ell^{N+1}] \ar[r]^{\al_0(N+1)} \ar[u]^{[\ell]} & A[\ell^{N+1}] \ar[r]\ar[u]^{[\ell]} & \coker \al_0(N+1)\ar[r] \ar[u]_{\cong}& 0\\
& \vdots \ar[u] & \vdots \ar[u] & \vdots \ar[u] & &
}
\]
where $\al_0(N):=\al_0|_{A[\ell^N]}$. Here we're implicitly looking at $\ol k$-points, $A[\ell^N]=A(\ol k)[\ell^N]$.

Taking the inverse limit we get the $\ell$-adic Tate modules. Taking $N\gg0$ large enough the kernel stabilizes:
\[
\ker \al_0(N)=\ker \al_0(N+1)=\cdots 
\]
and has order equal to the $\ell$-part of $|\ker \al_0|=\deg \al_0$, which is $\ell^{n_1}$. 
When we take the inverse limit by multiplication by $\ell$, we get 0, because a finite $\ell$-group cannot be infinitely divisible. 

When we take the limit we preserve the size of $\coker$. 
The exact sequences give that 
\[
\fc{|\ker\al_0(N)|}{|A[\ell^N]|}\fc{|A[\ell^N]|}{|\coker \al_0(N)|}=1,
\]
%The Euler characteristic quotient of the exact sequence  is 1 so we get that 
so $\ker$ and $\coker$ have the same order. 
We get that for large enough $N$,
\[
|\coker \al_0(N)|=\ell^{n_1}.
\]
Taking the inverse limit $\varprojlim_{N}$ we get
\[
0\to T_{\ell}A\xra{T_{\ell}(\al_0)} T_{\ell} A\to \coker \tl(\al_0)\to 0
\]
where the cokernel has order equal to $|\coker \al_0(N)|=\ell^{n_1}$. 
On the other hand, %the $\ell$-adic unit acts like an isomorphism, so we have $N_2(\al_0)=\det T_{\ell}(\al_0)=\ell^{n_2}u_2$. 
because $N_2(\al_0)=\det T_{\ell}(\al_0)=\ell^{n_2}u_2$ and the $\ell$-adic unit acts like an isomorphism, we have
\[
|\coker \tl(\al_0)|=|\tl A/\tl (\al_0)\tl A|=\ell^{n_2}.
\]
%so the cokernel has order $\ell^{n_2}=\ell^{n_1}$, as needed. 
Hence $\ell^{n_1}=\ell^{n_2}$, as needed.
This proves step 1. (Note we used homogeneity, not yet the classification of norm forms.)\\
%A any abelian variety 

\step 2 We show $N_1(\al)=N_2(\al)$. Write $ \End^0A\ot_{\Q}\Q_{\ell}\cong \prod_{j=1}^r D_j$, where the $D_j$ are
 finite simple over $\Q_{\ell}$. %norm form restrict to element nontrivial for one j, trivial for all other j's. Need to use N(1)=1.
We decompose $N_i$ into norm forms $N_{i,j}$ as follows:
\[
N_i=\prod_{j=1}^r N_{i,j},\qquad N_{i,j}:D_j\to \Q_{\ell},\,i=1,2.
\]
By Lemma~\ref{lem:classify-norm}, we have
\[
N_{i,j}=(N_{D_j/\Q_{\ell}}^{\min})^{\nu_{i,j}},\qquad \nu_{i,j}\in \Z.
\]
%D_j=D/F_j=F\supset \Q_{\ell}=E
Plug in $\al=(1,\ldots, 1, \underbrace{\al_j}_j,1,\ldots, 1)\in \End^0A\ot_{\Q}\ql\cong \prod_{j=1}^r D_j$ into Step 1 to get
\[
|N_{D_j/\Q_{\ell}}^{\min}(\al_j)|^{\nu_{1,j}}=
|N_{D_j/\Q_{\ell}}^{\min}(\al_j)|^{\nu_{2,j}}.
\]
%This shows all the exponents are equal. 
We are free to choose $\al_j$, so $\nu_{1,j}=\nu_{2,j}$ for all $j$. Hence $N_1=N_2$.  This proves part 1. \\
%Note we extended
%\[
%\xymatrix{
%\End^0(A) \ar[r] \ha{d}& \Q\ha{d}\\
%\End^0(A) \ot \Q_{\ell} \ard{r}^{?} & \Q_{\ell}.
%}
%\]
%To do this we linearly extend the polynomial map  $\Q\phi+\Q\psi \to \Q$ for each $\phi,\psi$.

For part 2, note $P_f(N)$ is the characteristic polynomial of $T_{\ell}(f)$. Hence it is monic of degree $2g$. {\it A priori} the coefficients are in $\Z_{\ell}$; we need to show they are in $\Z$.

First we show the coefficients are in $\Q$; then we show they are also algebraic integers, so they are in $\Z$.

The degree function assumes integer values on $\End(A)$. If $P_f(n)\in \Z$ for all $n\in \Z$, it's not hard to see that $P_f(X)\in \Q[X]$. %plug in sufficiently many integers and solve over $\Q$. 

Note the $\Z$-subalgebra generated by $f$, $\Z[f]\subeq \End(A)$, is a finite commutative $\Z$-algebra. 
($\End(A)$ is finitely generated by Theorem~\ref{thm:787-HomAB}(1), and any subgroup of finitely generated abelian group is finitely generated.) Hence $f$ is integral over $\Z$. $T_{\ell}f$ satisfies the same monic equation that $f$ satisfies, with coefficients in $\Z$. In particular, the eigenvalues of $T_{\ell}\Z$, which are in $\ol{\Q}$, satisfy the same equation. By integrality, the eigenvalues are in $\ol{\Z}$. Hence $P_f(X)\in \ol{\Z}[X]$. Since the coefficients are also in $\Q$, we get $P_f(X)\in \Z[X]$. This proves part 2.\\

Now we prove part 3. If we plug an operator into its own characteristic polynomial, we get 0 (Cayley-Hamilton). Hence
\[
T_{\ell}(P_f(f))= P_f(T_{\ell}(f))=0.
\]
By Theorem~\ref{thm:787-HomAB}(2), $T_{\ell}$ is injective, so
\[
P_f(f)=0.
\]
\end{proof}
%center is in general finite product of fields. <- Semisimple algebra
\begin{df}
$P_f(X)\in \Z[X]$ is called the \textbf{characteristic polynomial} of $f$. Define the trace and norm of $P_f$ to be the following coefficients:
\[
P_f(X)=X^{2g}-\underbrace{a_{2g-1}}_{\tr(f)}X^{2g-1}+\cdots +\underbrace{a_0}_{\nm f=\deg f}.
\]
\end{df}
\begin{cor}
Let $A\in \abk$ have dimension $g$ and be simple. Let $F:=Z(\End^0A)$. Then $d=[\End^0A:F]^{\rc 2}$ and $e=[F:\Q]$. Then $de\mid 2g$.
\end{cor}
This really relies on the classification of norms.
\begin{proof}
By Lemma~\ref{lem:classify-norm}, 
\[
\nm f=(N_{\End^{\circ}A}^{\min}f)^n
\]
with $n\in \Z$. We look at the degree of both sides.  The LHS is a polynomial $f$ of degree $2g$. The degree of the RHS is $de\cdot n$. Hence
\[
de\mid 2g.
\]
\end{proof}
\begin{df}
A $\ol k$-simple abelian variety $A\in \abk$ is of \textbf{CM-type} if $de=2g$.
\end{df}
If $A$ is an elliptic curve (of dimension 1) this says $de=2$. In characteristic 0, it can only happen that $d=1$ and $e=2$. To understand the endomorphism algebra of elliptic curves, we need another fact on the endomorphism algebra: $\End^0(A)$ has a positive involution, called the \textbf{Rosati involution}. This gives us a finer classification of endomorphism algebras. 
In positive characteristic, $\End^0(A)$ can also be a central quaternion algebras over $\Q$, and we get supersingular elliptic cuves.
We'll talk about this and give a classification of complex abelian varieties next time.