%%%This is a science homework template. Modify the preamble to suit your needs. 

\documentclass[12pt]{article}

\makeatother
%AMS-TeX packages
\usepackage{amsmath}
\usepackage{amssymb}
\usepackage{amsthm}
\usepackage{array}
\usepackage{amsfonts}
\usepackage{cancel}
\usepackage[all,cmtip]{xy}%Commutative Diagrams
\usepackage[pdftex]{graphicx}
\usepackage{float}
%geometry (sets margin) and other useful packages
\usepackage[margin=1in]{geometry}
\usepackage{sidecap}
\usepackage{wrapfig}
\usepackage{verbatim}
\usepackage{mathrsfs}
\usepackage{marvosym}
\usepackage{hyperref}
\usepackage{graphicx,ctable,booktabs}

\newtheoremstyle{norm}
{6pt}
{6pt}
{}
{}
{\bf}
{:}
{.5em}
{}

\theoremstyle{norm}
\newtheorem{thm}{Theorem}[section]
\newtheorem{lem}[thm]{Lemma}
\newtheorem{df}{Definition}
\newtheorem{rem}{Remark}
\newtheorem{st}{Step}
\newtheorem{pr}[thm]{Proposition}
\newtheorem{cor}[thm]{Corollary}
\newtheorem{clm}[thm]{Claim}

%Math blackboard, fraktur, and script commonly used letters
\newcommand{\A}[0]{\mathbb{A}}
\newcommand{\C}[0]{\mathbb{C}}
\newcommand{\sC}[0]{\mathcal{C}}
\newcommand{\cE}[0]{\mathscr{E}}
\newcommand{\F}[0]{\mathbb{F}}
\newcommand{\cF}[0]{\mathscr{F}}
\newcommand{\cG}[0]{\mathscr{G}}
\newcommand{\sH}[0]{\mathscr H}
\newcommand{\Hq}[0]{\mathbb{H}}
\newcommand{\cI}[0]{\mathscr{I}}%ideal sheaf
\newcommand{\N}[0]{\mathbb{N}}
\newcommand{\Pj}[0]{\mathbb{P}}
\newcommand{\sO}[0]{\mathcal{O}}
\newcommand{\cO}[0]{\mathscr{O}}
\newcommand{\Q}[0]{\mathbb{Q}}
\newcommand{\R}[0]{\mathbb{R}}
\newcommand{\Z}[0]{\mathbb{Z}}
%Lowercase
\newcommand{\ma}[0]{\mathfrak{a}}
\newcommand{\mb}[0]{\mathfrak{b}}
\newcommand{\fg}[0]{\mathfrak{g}}
\newcommand{\vi}[0]{\mathbf{i}}
\newcommand{\vj}[0]{\mathbf{j}}
\newcommand{\vk}[0]{\mathbf{k}}
\newcommand{\mm}[0]{\mathfrak{m}}
\newcommand{\mfp}[0]{\mathfrak{p}}
\newcommand{\mq}[0]{\mathfrak{q}}
\newcommand{\mr}[0]{\mathfrak{r}}
%Letter-related
\newcommand{\bb}[1]{\mathbb{#1}}
\providecommand{\cal}[1]{\mathcal{#1}}
\renewcommand{\cal}[1]{\mathcal{#1}}
%More sequences of letters
\newcommand{\chom}[0]{\mathscr{H}om}
\newcommand{\fq}[0]{\mathbb{F}_q}
\newcommand{\fqt}[0]{\mathbb{F}_q^{\times}}
\newcommand{\sll}[0]{\mathfrak{sl}}
%Shortcuts for symbols
\newcommand{\nin}[0]{\not\in}
\newcommand{\opl}[0]{\oplus}
\newcommand{\ot}[0]{\otimes}
\newcommand{\rc}[1]{\frac{1}{#1}}
\newcommand{\rra}[0]{\rightrightarrows}
\newcommand{\send}[0]{\mapsto}
\newcommand{\sub}[0]{\subset}
\newcommand{\subeq}[0]{\subseteq}
\newcommand{\supeq}[0]{\supseteq}
\newcommand{\nsubeq}[0]{\not\subseteq}
\newcommand{\nsupeq}[0]{\not\supseteq}
%Shortcuts for greek letters
\newcommand{\al}[0]{\alpha}
\newcommand{\be}[0]{\beta}
\newcommand{\ga}[0]{\gamma}
\newcommand{\Ga}[0]{\Gamma}
\newcommand{\de}[0]{\delta}
\newcommand{\De}[0]{\Delta}
\newcommand{\ep}[0]{\varepsilon}
\newcommand{\eph}[0]{\frac{\varepsilon}{2}}
\newcommand{\ept}[0]{\frac{\varepsilon}{3}}
\newcommand{\la}[0]{\lambda}
\newcommand{\La}[0]{\Lambda}
\newcommand{\ph}[0]{\varphi}
\newcommand{\rh}[0]{\rho}
\newcommand{\te}[0]{\theta}
\newcommand{\om}[0]{\omega}
\newcommand{\Om}[0]{\Omega}
\newcommand{\si}[0]{\sigma}
%Brackets
\newcommand{\ab}[1]{\left| {#1} \right|}
\newcommand{\ba}[1]{\left[ {#1} \right]}
\newcommand{\bc}[1]{\left\{ {#1} \right\}}
\newcommand{\pa}[1]{\left( {#1} \right)}
\newcommand{\an}[1]{\langle {#1}\rangle}
\newcommand{\fl}[1]{\left\lfloor {#1}\right\rfloor}
\newcommand{\ce}[1]{\left\lceil {#1}\right\rceil}
\newcommand{\ve}[1]{\left\Vert{#1}\right\Vert}
%Text
\newcommand{\btih}[1]{\text{ by the induction hypothesis{#1}}}
\newcommand{\bwoc}[0]{by way of contradiction}
\newcommand{\by}[1]{\text{by~(\ref{#1})}}
\newcommand{\ore}[0]{\text{ or }}
%Arrows
\newcommand{\hr}[0]{\hookrightarrow}
\newcommand{\xr}[1]{\xrightarrow{#1}}
%Formatting
\newcommand{\subprob}[1]{\noindent\textbf{#1}\\}
%Functions, etc.
\newcommand{\Ann}{\operatorname{Ann}}
\newcommand{\AP}{\operatorname{AP}}
\newcommand{\Ass}{\operatorname{Ass}}
\newcommand{\Aut}{\operatorname{Aut}}
\newcommand{\chr}{\operatorname{char}}
\newcommand{\cis}{\operatorname{cis}}
\newcommand{\Cl}{\operatorname{Cl}}
\newcommand{\Der}{\operatorname{Der}}
\newcommand{\End}{\operatorname{End}}
\newcommand{\Ext}{\operatorname{Ext}}
\newcommand{\Frac}{\operatorname{Frac}}
\newcommand{\FS}{\operatorname{FS}}
\newcommand{\GL}{\operatorname{GL}}
\newcommand{\Hom}{\operatorname{Hom}}
\newcommand{\Ind}[0]{\text{Ind}}
\newcommand{\im}[0]{\text{im}}
\newcommand{\nil}[0]{\operatorname{nil}}
\newcommand{\ord}[0]{\operatorname{ord}}
\newcommand{\Proj}{\operatorname{Proj}}
\newcommand{\PSL}{\operatorname{PSL}}
\newcommand{\Rad}{\operatorname{Rad}}
\newcommand{\rank}{\operatorname{rank}}
\newcommand{\Res}[0]{\text{Res}}
\newcommand{\sign}{\operatorname{sign}}
\newcommand{\SL}{\operatorname{SL}}
\newcommand{\Spec}{\operatorname{Spec}}
\newcommand{\Specf}[2]{\Spec\pa{\frac{k[{#1}]}{#2}}}
\newcommand{\spp}{\operatorname{sp}}
\newcommand{\spn}{\operatorname{span}}
\newcommand{\Supp}{\operatorname{Supp}}
\newcommand{\Tor}{\operatorname{Tor}}
\newcommand{\tr}[0]{\text{trace}}
%Commutative diagram shortcuts
\newcommand{\fiber}[3]{\xymatrix{#1\times_{#3} #2}\ar[r]\ar[d] #1\ar[d] \\ #2 \ar[r] & #3}
\newcommand{\commsq}[8]{\xymatrix{#1\ar[r]^{#6}\ar[d]^{#5} &#2\ar[d]^{#7} \\ #3 \ar[r]^{#8} & #4}}
%Makes a diagram like this
%1->2
%|    |
%3->4
%Arguments 5, 6, 7, 8 on arrows
%  6
%5  7
%  8
\newcommand{\pull}[9]{
#1\ar@/_/[ddr]_{#2} \ar@{.>}[rd]^{#3} \ar@/^/[rrd]^{#4} & &\\
& #5\ar[r]^{#6}\ar[d]^{#8} &#7\ar[d]^{#9} \\}
\newcommand{\back}[3]{& #1 \ar[r]^{#2} & #3}
%Syntax:\pull 123456789 \back ABC
%1=upper left-hand corner
%2,3,4=arrows from upper LH corner, going down, diagonal, right
%5,6,7=top row (6 on arrow)
%8,9=middle rows (on arrows)
%A,B,C=bottom row
%Other
%Other
\newcommand{\op}{^{\text{op}}}
\newcommand{\fp}[1]{^{\underline{#1}}}
\newcommand{\rp}[1]{^{\overline{#1}}}
\newcommand{\rd}[0]{_{\text{red}}}
\newcommand{\pre}[0]{^{\text{pre}}}
\newcommand{\pf}[2]{\pa{\frac{#1}{#2}}}
\newcommand{\pd}[2]{\frac{\partial #1}{\partial #2}}
\newcommand{\bs}[0]{\backslash}
\newcommand{\ol}[1]{\overline{#1}}
\newcommand{\mmod}[1]{\,(\text{mod}^{\times} #1)}
\newcommand{\nmod}[1]{\,(\text{mod}\, #1)}
\newcommand{\set}[2]{\bc{\left. {#1}\right|{#2}}}
%Matrices
\newcommand{\coltwo}[2]{
\left[
\begin{matrix}
{#1}\\
{#2} 
\end{matrix}
\right]}
\newcommand{\matt}[4]{
\left[
\begin{matrix}
{#1}&{#2}\\
{#3}&{#4}
\end{matrix}
\right]}
\newcommand{\smatt}[4]{
\left[
\begin{smallmatrix}
{#1}&{#2}\\
{#3}&{#4}
\end{smallmatrix}
\right]}
\newcommand{\colthree}[3]{
\left[
\begin{matrix}
{#1}\\
{#2}\\
{#3}
\end{matrix}
\right]}
\newcommand{\iy}[0]{\infty}
%
%Redefining sections as problems
%
\makeatletter
\newenvironment{problem}{\@startsection
       {section}
       {1}
       {-.2em}
       {-3.5ex plus -1ex minus -.2ex}
       {2.3ex plus .2ex}
       {\pagebreak[3]%forces pagebreak when space is small; use \eject for better results
       \large\bf\noindent{Problem }
       }
       }
       {
       }
\makeatother

\usepackage{fancyhdr}
\pagestyle{fancy}
\lhead{Problem \thesection}
\chead{} 
\rhead{\thepage} 
\lfoot{\small\scshape 18.785 Analytic Number Theory} 
\cfoot{} 
\rfoot{\footnotesize PS \# 10} % !! Remember to change the problem set number
\renewcommand{\headrulewidth}{.3pt} 
\renewcommand{\footrulewidth}{.3pt}
\setlength\voffset{-0.25in}
\setlength\textheight{648pt}
\allowdisplaybreaks[1]
%
%Contents of problem set
%    
\begin{document}
\title{18.785 Analytic Number Theory Problem Set \#10}% !! Remember to change the problem set number
\author{Holden Lee}
\date{4/24/11}% !! Remember to change the date
\maketitle
\thispagestyle{empty}
%Example problems
\begin{problem}{\it ($f$ cuspidal iff $\tilde{f}\in L^2(\Ga\bs G)$)}
First suppose $f$ is cuspidal. Then its norm with respect to the Petersson inner product is well defined, i.e.
\[
\iint_{\Ga\bs H} |f(z)|^2 y^m\frac{dxdy}{y^2}
\]
converges. Noting that $d\mu=\frac{dx dy}{y^2}$ and that $y=\Im[g(i)]=|j(g,i)|^{-2}$, this equals 
\[
\int_{\Ga\bs G} |j(g,i)^{-m}f(g(i))|^2\,d\mu=\int_{\Ga\bs G} \tilde{f}(g)\,d\mu
\]
so the latter converges.

Conversely suppose $\tilde{f}\in L^2(\Ga\bs G)$. 
By conjugation we may assume $P=P_0$ with the cusp at infinity. 
Now $\tilde{f}-\tilde{f}_P$ is rapidly decreasing so $\tilde{f}-\tilde{f}_P\in L^2$. By the triangle inequality,
\[
\ve{\tilde{f}_P}\le \ve{\tilde{f}-\tilde{f}_P}+\ve{\tilde{f}}<\iy.
\]
Now $\tilde{f}_P=j(g,i)^{-m}a_0$ where $a_0$ is the constant term at the cusp. 
As above,
\[
\ve{\tilde{f}_P}^2=\int_{\Ga\bs G} |\tilde{f}(g)|^2\,d\mu=|a_0|^2\int_{\Ga\bs H} y^{m}\frac{dxdy}{y^2}.
\]
%where $a_0$ is the constant term at the cusp. 
Taking a fundamental domain, it contains some region in the form $[a,b]\times [c,\iy)$, and the above integral diverges. Therefore $a_0=0$. This shows that $f$ is actually a cusp form.
%thm 9.2.4
\end{problem}
\begin{problem}{\it($\displaystyle \cal A(\Ga,J,\chi)=\bigoplus_{i=1}^q \cal A(\Ga,J_i,\chi)$)}
First suppose $f\in \cal A(\Ga,J,\chi)$. Let
\[
P_j(x)=\prod_{\scriptsize\begin{array}{c} 1\le i\le q\\i\ne j\end{array}}(x-\la_i)^{n_i}.
\]
Since the $P_i$ are relatively prime there exist $g_i$ such that 
\[
\sum_{i=1}^q g_iP_i=1.
\]
Then
\[
f=\sum_{i=1}^q \underbrace{[(g_iP_i)(\cal C)]f}_{\in \cal A(\Ga, J_i,\chi)}\in \sum_{i=1}^q \cal A(\Ga,J_i,\chi).
\]
To see the inclusion, note that $(\cal C-\la_i)^{n_i}(g_iP_i)(\cal C)f=g_iP(\cal C)f=0$.

It is clear that each $\cal A(\Ga,J_i,\chi)\subeq \cal A(\Ga,J,\chi)$. Thus we are left to show the sum is actually a direct sum. Suppose 
\[
f_1+\cdots +f_q=0
\]
where $f_i\in \cal A(\Ga, J_i,\chi)$. Operate by $P_j(\cal C)$, and note this annihilates every term except $f_j$. We get $P_j(\cal C)f_j=0$. However, we also know $(\cal C-\la_j)^{n_j}f_j=0$. Since $\gcd(P_j(x),(x-\la_j)^{n_j})=1$, we get $f_j=0$. Hence $f_1=\cdots =f_q=0$, showing the sum is a direct sum.
\end{problem}
\begin{problem}{\it (Integral over $N$ is 0)}
First we make several reductions.
\begin{enumerate}
\item $f$ can be written as a sum of functions having specific $K$-type, so we may assume $f(gk)=f(g)\chi(k),g\in G,k\in K$ for some character $\chi$. 
\item By problem 2, $f$ is a sum of $f_i\in \cal A(\Ga,J_i,\chi)$, so it suffices to solve the problem for the $f_i$, i.e. we may assume $(\cal C-\la)^m$ annihilates $f$. 
\item Next, we may assume $N,A$ are the groups $\bc{\smatt 1x01|x\in \R}$ and $\bc{\smatt t00{t^{-1}}|t>0}$ since every other $p$-pair is obtained by conjugation. 
\item Finally, note every $g\in G$ can be written as $g=n_ga_gk_g$ with $n_g\in N$, $a_g\in A$, and $k_g\in K$. Then
\[
\int_Nf(ng)\,dn=\int_Nf(na_g)\chi(k_g)\,dn
\]
so it suffices to show that $\int_N f(na)\,dn=0$ for each $a\in A$.
\end{enumerate}

Let
\[
\ph(g)=\int_N f(ng)\,dn
\]
and
\[
\phi(t)=\ph\matt{e^t}{0}{0}{e^{-t}}.
%\int_N f\pa{n\matt{e^t}{0}{0}{e^{-t}}}\,dn.
\]
Since $A=\bc{\smatt{e^t}{0}{0}{e^{-t}}|t\in \R}$, it suffices to show that $\phi(t)\equiv 0$. The fact that $f$ is integrable over $G=NAK$ gives that $\ph$ is integrable over $AK$, so the following is finite:
\begin{align*}
%\int_{G=NAK} |f(g)|\,dg&=\int_{K}\int_A\int_N |f(nak)|a^{-2}\,dn\,da\,dk\\
\int_K\int_A |\ph(ak)|a^{-2}\,da\,dk
&=\int_K\int_A |\ph(a)\chi(k)|a^{-2}\,da\,dk\\
&=\int_A|\ph(a)|a^{-2}\,da \int_K |\chi(k)|\,dk.
\end{align*}
Since the second integral is positive, $\int_A|\ph(a)|a^{-2}<\iy$. But this equals $\int_t \phi(t)e^{-2t}\,dt$. So we have
%
%
%\begin{align*}
%\int_{NA} f(g)\,dg&=\int_{A}\int_N f(na)\,dn\,da\\
%&=\int_{A}\ph(a)\,da\\
%&=
\begin{equation}\label{finite}
\int_t \phi(t)e^{-2t}\,dt<\iy.
\end{equation}
%\end{align*}
%This is finite as $f\in L^1(G)$ by assumption. 

%We claim that $\cal C$ acts on the function 
%$v(g)=\int_A f(na)\,da$ at 
Write $Y\phi(t)$ as shorthand for $(Y\ph)(g)|_{g=\smatt{e^t}00{e^{-t}}}$. %where
%\[
%v(g)=\int_N f(ng)\,dn.
%\]
We claim that
\begin{equation}\label{caction}
\cal C\phi(t)=\pa{\rc2\frac{d^2}{dt^2}-\frac{d}{dt}}\phi(t).
\end{equation}

First we show $H\phi(t)=\frac{d}{dt}\phi(t)$.%, i.e. %Letting 
%$u(g)=f\pa{ng}$, 
%It suffices to show that 
%\[H\ph(g)|_{g=\smatt {e^t}00{e^{-t}}}=\frac{d}{dt}\ph\pa{\matt {e^t}00{e^{-t}}}.\]
%since integrating over $n\in N$ gives the result. 
Indeed,
\begin{align*}
H\phi(t)&=
H\ph(g)|_{g=\smatt {e^t}00{e^{-t}}}\\
&=\left.\frac{d}{dt_1}\ph\pa{
\matt {e^t}00{e^{-t}}
e^{t_1H}}\right|_{t_1=0}\\
&
=\left.\frac{d}{dt_1}\ph\pa{
\matt {e^t}00{e^{-t}}
\matt {e^{t_1}}00{e^{-t_1}}
}\right|_{t_1=0}\\
&
=\left.\frac{d}{dt_1}\ph\pa{
\matt {e^{t+t_1}}00{e^{-(t+t_1)}}}\right|_{t_1=0}
\\
&
=\left.\frac{d}{dt_1}\ph\pa{
\matt {e^{t_1}}00{e^{-t_1}}}\right|_{t_1=t}
\\
&=\frac{d}{dt}\phi(t).
\end{align*}

Next we show $EF\ph(t)=0$. First note $\ph(g)$ is left $N$-invariant since the integral is over $n\in N$ and $n$ appears on the left in the argument of $f$. Note
\begin{align*}
\matt{e^t}{0}{0}{e^{-t}}e^{t_1E}&=\matt{e^t}{0}{0}{e^{-t}}\matt 1{t_1}01\\
&=\matt{e^t}{t_1e^t}0{e^{-t}}\\
&=\underbrace{\matt{1}{t_1e^{2t}}{0}{1}}_{\in N}\matt{e^t}00{e^{-t}}.
\end{align*}
Now
\begin{align*}
EF\ph(g)|_{g=\smatt{e^t}{0}{0}{e^{-t}}}&=\left.\pd{^2}{t_1\partial t_2} \ph\pa{
\matt{e^t}00{e^{-t}}e^{t_1E}e^{t_2F}
}\right|_{t_1=t_2=0}\\
&=\left.\pd{^2}{t_1\partial t_2} \ph\pa{
\matt{1}{e^{2t}t_1}{0}{1}
\matt{e^t}00{e^{-t}}e^{t_2F}
}\right|_{t_1=t_2=0}\\
&=\left.\pd{^2}{t_1\partial t_2} 
\ph\pa{\matt{e^t}00{e^{-t}}e^{t_2F}
}\right|_{t_1=t_2=0}&\text{left $N$-invariance}\\
&=0.
\end{align*}

Putting the above two results together and using $\cal C=\rc2 H^2-H+EF$ gives~(\ref{caction}).

Now $(\cal C-\la)^mf=0$ gives $(\cal C-\la)^m\phi(t)=0$, which turns into the differential equation
\[
\pa{\rc 2\frac{d^2}{dt^2}-\frac{d}{dt}+\la}^m\phi=0.
\]
Since the quadratic $\rc 2x^2-x+\la$ has vertex at 1, its zeros are $1\pm s$ for some $s$. Then
\[
\phi(t)=p(t)e^{t(1+s)}+q(t)e^{t(1-s)}=e^t(p(t)e^{ts}+q(t)e^{-ts})
\]
for some polynomials $p,q$. Since $f$ is $\cal Z$-finite, $K$-finite, and integrable so is $\ph$; hence $\ph$, and hence $\phi$ is bounded by PSet 9 question 3.

We claim one of $p(t)$ or $q(t)$ is 0. 
Else $\phi(t)$ grows at least like $e^t$: Indeed, if $|p(t)e^{ts}|\nsim|q(t)e^{-ts}|$ are of different orders then the larger one is at least constant. If $|p(t)e^{ts}|\sim|q(t)e^{-ts}|$, then $s=is'$ is pure imaginary, the leading terms of $p$ and $q$ are the same,
 and the expression in parenthesis is 
\[2p(t)\cos s'\theta+[q(t)-p(t)]e^{ts}=p(t)\pa{\cos s'\theta +\frac{q(t)-p(t)}{p(t)} e^{ts}}\]
which does not approach 0 as $\cos s'\theta $ is infinitely often 1 and $\frac{q(t)-p(t)}{p(t)}\to 0$.

WLOG, $q(t)=0$. BWOC, $p(t)\neq 0$. Then for $\phi$ to be bounded, $\Re s=-1$ and $p(t)$ is constant. 
We've shown that $\phi(t)e^{-2t}=p(t)e^{t(-1+s)}$ is integrable~(\ref{finite}). But if $\Re s=-1$ and $p(t)\ne 0$, this blows up as $t\to -\iy$. Hence $p(t)=0$ and $\phi(t)=0$, exactly what we need.

\end{problem}
\end{document}