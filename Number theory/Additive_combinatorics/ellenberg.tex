\def\filepath{C:/Users/Owner/Dropbox/Math/templates}

\input{\filepath/packages_article.tex}
\input{\filepath/theorems_with_boxes.tex}
\input{\filepath/macros.tex}
\input{\filepath/formatting.tex}
%\input{\filepath/other.tex}

%\def\name{NAME}

%\input{\filepath/titlepage.tex}

\pagestyle{fancy}
%\addtolength{\headwidth}{\marginparsep} %these change header-rule width
%\addtolength{\headwidth}{\marginparwidth}
\lhead{Polynomial method}
\chead{} 
\rhead{Kakeya conjecture} 
\lfoot{} 
\cfoot{\thepage} 
\rfoot{} 
\renewcommand{\headrulewidth}{.3pt} 
%\renewcommand{\footrulewidth}{.3pt}
\setlength\voffset{0in}
\setlength\textheight{648pt}

\begin{document}

\tableofcontents
\section{Kakeya problem}
\begin{conj}
Let $E\subeq \R^n$ contain a line segment of length 1 in every direction (called a \textbf{Kakeya set}). Then 
\[
\dim_{\text{Haus}}(E)=n.
\]
\end{conj}
Note that Besicovich showed that a Kakeya set $E$ can have $\mu(E)=0$.

For $n>2$ this is a big open problem in harmonic analysis.
%montgomery's conjecture on dirichlet polynomials.

Tom Wolff proposed the discrete problem, over $\F_q^n$. Note here $q$ is the parameter that varies rather than $n$. Think of it as a bigger and bigger discrete grid.

(Proof omitted here. See AMiIT notes.)

%
Comments: In number theory we're inclined to use Lang-Weil. But here you get polynomials by interpolation so have no control over cohomology, etc.
You only have Hilbert-polynomial like invariants like degree. 
All we really used is Bezout's Theorem: the intersection of a line and hypersurface either has $\le d$ points or it's a geometric containment.
{\it What if you know more algebraic geometry; could you do more?}

A lot of people have used the ``polynomial method" since: Tao, Guth, Katz, Quilodran, Dvir, Sharir, Bourgain, Zohl, Solymosi, Zhang, Iosevich,... They include combinatorialists, harmonic analysts, number theorists, etc. 
%We don't know what's accessible!


\section{Kakeya problem, analyst's version}
Analysts like to think of operators on function spaces.

A subset corresponds to a characteristic function. Things that you can prove about subsets you should prove about all functions!

\begin{thm}[Kakeya maximal function conjecture, Ellenberg, Oberlin, Tao 2010]
Let $f:\F_q^n \to \R$. Define $f^*:\Pj^{n-1}(\F_q)\to \R$ by
\[
f^*(\om) = \sup_{\ell\parallel \om}\sum_{x\in \ell} |f(x)|.
\]
%it can't have a dense line in every direction
Then
\[
\ve{f^*}_n< c_n q^{\fc{n-1}{n}} \ve{f}_n
\]
%operator being bounded in $L^n$
%sum of the values of the function is equal to $q$.
\end{thm}
Note that the Kakeya condition on $S$ tells us $\chi_s^*=q$ for all $\om\in \Pj^{n-1}(\F_q)$, so this conjecture gives
\[
q\cdot q^{\fc{n-1}{n}} < c_n q^{\fc{n-1}{n}}\ub{\ve{\chi_S}_n}{|S|^{\rc n}}.
\]
How to use the polynomial method? We use the multiplicity trick (Saraf-Sudan). Given a function $f:\F_q^n\to \Z$, interpolate a $P$ of bounded degree vanishing to degree $f(x)$ at $x$. 
%another linear condition with dimension bound.
This gives $\sim f(x)^n$ linear conditions. 
Thus the $L^n$ norm comes up naturally. 

%Instead of a Kakeya condition, we can say ``at least $n$ points."

\section{Open questions}

Here's an open question.
\begin{prb}[Furstenberg-type problem]
Let $S\subeq \F_q^n$ be a set such that for every $K$-plane $V$, there exists a $k$-plane $W$ parallel to $V$ such that $|W\cap S|>q^n$. Is $|S|>q^{\fc{cn}{k}q}$?
\end{prb}
%C=K, certainly you have positive proportion.
%$k$-plane operator

Line segment intersects in set of Hausdorff dimension at least $\be$.

\begin{thm}
\begin{itemize}
\item (EOT)
For $c=k$, $|S|>q^n(1-q^{1-k})^{\binom n2}$.
\item
(Zhang) For $c=\rc 2$, $\F_q=\F_p$, 
\[
|S|>q^{\fc n2+\fc{c^r}{n^2}}.
\]
\end{itemize}
\end{thm}
%some of the results are sharp for trivial methods; for some the polynomial method is only the first step.

How does the finite field case differ from the Euclidean case? Dvir proves something too much!
\begin{itemize}
\item
Besicovich: There exists a Kakeya set of measure 0.
\item
There is a Kakeya set in $\F_q^n$ has measure $>c_n$. ($2^{-n}$.)
\end{itemize}•
Scale is missing! We can't do multiscale analysis. (Terry Tao writes on graininess, planiness...) Lines can be clustered closely. In $\F$, two things are the same or not. There is no metric with multiple scales. In summary,
\begin{itemize}
\item
$\F_q$ has only one scale.
\item
$\R$ has infinitely many scales.
\end{itemize}
\begin{prb}
Is the Kakeya conjecture true for
\begin{itemize}
\item
$\F_q[[z]]$
\item
$\Z_p$
\item 
$\F_q[t]/t^2$
(This has only 2 different scales, but it's hard! Try it! This is an infinitesimal step towards understanding how to approach the general problem!)
\end{itemize}
\end{prb}
\begin{thm}[Dummit, Hoblicsch]
There exists Kakeya sets in $\F_q[[t]]$ and $\Z_p^n$ of measure 0.
\end{thm}
Their sets do have Minkowski dimension $n$. Thus these rings are rich enough for the Besicovich phenomenon. Their construction is not related to Besicovich's construction; it's very clever. It's iterative, and uses a martingale theorem. The limit of a stochastic process has to converge to 0; a little measure seeps away each step.

\section{Guth-Katz}

Guth and Katz showed:
\begin{thm}[Guth, Katz, 2010, conjecture of Bourgain]
Let $L$ be a set of $N^2$ lines in $\R^3$ so that no more than $N$ lie in any plane. Let $S$ be a set of points such that each line in $L$ has at least $N$ points of $S$. Then 
\[
|S|>cN^3.
\]
\end{thm}
Asymptotically, the lines don't intersect and each line contributes $\succsim N$ points.
\begin{thm}[Guth-Katz, 2010, Erd\H os distance conjecture]
$N$ points in $\R^2$ determine at least $\fc{N}{\ln N}$ distinct distances.
\end{thm}
This is almost sharp, $\fc{N}{\sqrt{\ln N}}$ is attainable. %for Peter, $\sqrt{\ln N}$ is a big deal.
%paid 1000 dollars. No you haven't solved. Graham: I'm giving it anyway.
%taken off $\sqrt{\log N}$...

The Elekes-Sharir setup is as follows (they set up a program for these types of problems): 
\begin{itemize}
\item
few distinct distances mean that 
\item there are many pairs $(x_1,y_1)$ and $(x_2,y_2)$ such that $d(x_1,y_1)=d(x_2,y_2)$.
\item What's a grown-up way of saying this? That there exist $g\in G$, the group of rigid motions, such that $g(x_1)=x_2, g(y_1)=y_2$. 
$G(x_1,x_2)=\set{g\in G}{gx_1=x_2}$ a translate of $\SO_2\in G$.
%codimension 1 question
%these guys intersect more than you might expect.
\item 
There are many poits of intersection between $G(x_1,x_2)$ in $G$, 1-dimensional lines in 3-dimensional space. That is the main theorem in their paper.
\end{itemize}
A lot of problems in extremal combinatorial geometry (distinct areas of simplices, etc.), as long as we can express the problem as orbits of groups.
There's a desire for incidence theory not just for linear subspaces, but about varieties, using their degrees.

Guth-Katz, in solving the Erd\H os problem, uses polynomials to cut $\R^n$ into cells. {\it This seriously uses $K=\R$.}

Guth-Katz in solving Bourgain's conjecture, uses a purely algebraic approach that works over $\C$.

The Bourgain conjecture on $\F_q$ would be: Given $q^2$ lines in $\F_q$, no $q$ of which lie on the same plane, the union of the lines has $cq^3$ points. This is false!

There is a piece that looks like differential geometry: there are three vectors (?) that are planar. This is not true in characteristic $p$! Planes are the enemy, but in characteristic $p$. 
There is a non-plane curve in $\F_p$ all of whose points are flex points! They have degree $\ge p$. 
\begin{thm}[Ellenberg, Hoblicsek, 2014]
Bourgain is true for $\F_p$, $p$ prime.
\end{thm}
%stephanov, $\F_p$ to $\F_{p^2}$ for Riemann over finite fields.

What I'm working on now. We're not satisfied about theorems about sets of points in space. (That's a zero-dimension set in affine space.) We want geometric Kakeya: extremal problems for 0-dimensional subschemes of $\A^n$. 
From Ravi Vakil, to prove stuff about smooth curves, degenerate them until they're as non-smooth as they can be. This looks scary but they're easier. 
Degenerate to fat points at the origin, then recover stuff about smooth schemes. 

Painful combinatorial difficulties can be ameliorated by passing to the maximal degenerate case.

Bombieri-Pila: The theory of unlikely intersections. Real algebraic geometry, transcendental theory. Hardest to show polynomial is not 0.
%High-dimensional 

Bounding number of rational points on curves. %varieties. Heath-Brown upper bounds.
\end{document}