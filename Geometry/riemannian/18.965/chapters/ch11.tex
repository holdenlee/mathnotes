\lecture{Tue. 10/16/12}
\llabel{sec:11}

\subsection{Jacobi fields and curvature}\llabel{sec:jac-curv}
Recall that for a manifold with metric $(M,g)$ where $g$ is symmetric and non-degenerate, we have a Levi-Civita connection $\nb$. Suppose we have a parametrized surface $f$ where $s\mapsto f(s,t)$ is a geodesic. Then the variational field $J=\pd ft$ satisfies the Jacobi equation
\[
J''+R(\ga',J)\ga'=0
\]
where $J''=\cvs \cvs J$.

For instance, consider the exponential map $\exp_p:T_pM\to M$. We can consider it in polar coordinates, so it takes $(r,\te)$ as arguments, where $-\pi<\te\le \pi$ and $0<r$. Then $\pdd{\te} \exp_p=J$ satisfies the Jacobi equation. The Jacobi field measures the ``distortion" of the exponential map; moreover $\ab{\pdd r \exp_p}=1$.

Consider the special case where $M^2$ is a surface. Letting $\ga$ be a geodesic starting at $p$, $\ga(t)=\exp_p(t,\te)$, we have by Gauss's Lemma~\ref{lem:gauss}, 
\beq{eq:965-11-1}
\pdd{\te} \exp_p \perp \ga'.
\eeq
 %figure
Note $\ga''=0$ since $\ga'$ is a parallel vector field along $\ga$. 
The normal $\vec{n}$ to the curve is also parallel along $\ga$ (by parallel translation). %normal along.
By~\eqref{eq:965-11-1}, $J$ is perpendicular to $\ga$ so we can write $J=j \vec{n}$ ,$J'=j'\vec{n}$, and $J''=j''\vec{n}$. The Jacobi equation tells us
\[
j''\vec n + j R(\ga',\vec n) \ga'=0.
\]
Writing $ R(\ga',\vec n) \ga'=k\vec n$, we get
\[
j'' \vec n + jk \vec n=0\iff j''+kj=0.
\]
Now $J(0)=\pdd{\te} \exp_p(0,0)=0$ so $j(0)=0$, and we get $\rb{\pdd{r}\pdd{\te} \exp_p}_{r=0}=\vec n$ gives $j'(0)=1$. Taylor expansion gives
\[
j=j(0)+j'(0)r+\fc{j''(0)}2r^2+\fc{j'''(0)}6r^3+\cdots
\]
where 
\begin{align*}
j(0)&=0\\
j'(0)&=1\\
j''(0)&=-j(0)k(0)=0\\
j'''(0)&=-j'(0)k(0)-j(0)k'(0)\\
&=-k(0).
\end{align*}
Thus
\beq{eq:965-11-2}
j=r-\fc k6 r^3+\pat{higher order terms}.
\eeq
This gives us a way of thinking about the curvature. Suppose we have a surface, and we want to know the curvature at $p$. Consider a sphere (circle) of radius $r$ at 0 in $T_pM$; call it $S_r$. Then $\exp_p(S_r)$ is a curve on the manifold; call it $\pl B_r$. What is the length of $\pl B_r$?

We'd like to compute the length $c(\te)=\exp_p(r,\te)$. We have
\[
\pd c{\te}=\pdd{\te} \exp_p=J.
\]
Hence the length is
\beq{eq:965-11-3}
\ab{\pd c{\te}}\approx r-\fc{k}{6}r^3.
\eeq
Integrating this from $-\pi$ to $\pi$, the length of $\pl B_r$ is
\[
\int_{-\pi}^{\pi} \ab{\pd c{\te}}\,d\te\approx \pa{r-\fc k6 r^3}2\pi =2\pi r-\fc{k}3\pi r^3.
\]
Thus we see
\[
\text{length}(\pl B_r)-\text{length}(S_r)\approx -\fc k3\pi r^3.
\]
For instance, for the sphere, this difference is negative so the curvature is positive.\\
%pic fig?

\cpbox{
The curvature measures the difference between the length of distance spheres in Euclidean space, and the length of the distance spheres on the manifold.
\begin{itemize}
\item
Positive curvatures means that spheres on the manifold have smaller length. Geodesics coming from a point don't spread out as fast.
\item
Negative curvature means the opposite.
\end{itemize}
}
\vskip0.15in
We can get a similar expression in any dimension. %|J|^2, M, \ga, J. 

\subsection{Conjugate points}
Suppose $\ga$ is a geodesic, $\ga:[a,b]\to M$. Let $J$ be a Jacobi field. Given $(v,w)\in T_pM\times T_pM$ where $\ga(a)=p$, there exists a unique Jacobi field with
\[
J(a)=v,\qquad J'(a)=w.
\]
In particular, letting $J$ be the Jacobi field with $J(a)=0$ and $J'(a)=v$, we get a lnear map 
\[F:T_{\ga(a)}M\to T_{\ga(b)} M
\] sending $w\in T_{\ga(a)}M$ to $J(b)$.
\begin{df}
We say that $b$ is a \textbf{conjugate point}  to $a$ along $\ga$ if there is a non-trivial Jacobi field with $J(a)=0$ and $J(b)=0$.
\end{df}
%$F$ is a linear map. For each tangent vector $w\in T_{\ga(a)}M$ gives a point $T_{\ga(b)}M$.  %map nontrivial kernel

The manifold is nicer if we have no conjugate point. %Injective, hence isomorphism. ???


\subsection{Isometric immersions}
Consider a submanifold $M^2\subeq \R^3$ or more generally, any isometric immersion $M^m\hra N^n$. %is an isometric immersion.

The following are natural questions:
\begin{itemize}
\item
How do the connection on $M$ and $N$ relate?
\item
How do the curvatures of $M$ and $N$ relate?
\end{itemize}•
Let $\ol{\nb}$ be the connection on $N$ and $\nb$ be the connection on $M$. The following says that if we want to calculate $(\nb_XY)_p$, we extend $X$ and $Y$ in any way to $N$, calculate the covariant derivative and then take the tangential component.

\begin{pr}
Let $X,Y\in \X(M)$, and $\ol X$, $\ol Y$ are (local) extensions of $X$ and $Y$ to a vector field on $N$, then
\[
\nb_XY=(\ol{\nb}_{\ol{X}}\ol Y)^T
\]
where $T$ means tangential component.
\end{pr}
To see that we can extend the vector field, note that because $M$ is a submanifold of $N$, by the implicit function theorem there is a coordinate system $(x_1,\ldots, x_m,x_{m+1},\ldots, x_n)$ on which the manifold sits like a plane, $(x_1,\ldots, x_m,0,\ldots, 0)$. Extend the vector field trivially in the other directions.
%vector fields tangent.
\begin{proof}
Define ${\nb}_XY$ by $(\ol{\nb}_{\ol X}\ol Y)^T$. 
We need to check $\nb$ is a symmetric compatible connection. %Then by uniqueness of compatible connections~\ref{thm:levi-civita} it must be {\it $\nb$}. 
We check
\begin{enumerate}
\item
$\nb$ is linear in each variable (clear).
\item
If we multiply $X$ by a function it pops out linearly (clear). %\fixme{fix}.
\item
If we multiply $Y$ be a function, the Leibniz rule holds. Extending $f$ to a function on a neighborhood of $M$ near a point, 
\begin{align*}
\nb_X(fY)=(\ol{\nb}_{\ol X} (\ol f \ol Y))^T&=\pa{\ol X(\ol f)\ol Y+ \ol f\ol{\nb}_{\ol X} \ol Y}^T\\
&=X(f)Y+f(\nb_{\ol X} \ol Y)^T.
\end{align*}
\item
Symmetry: We have
\begin{align*}
\nb_XY-\nb_YX&=(\ol{\nb}_{\ol X} \ol Y)^T-(\ol{\nb}_{\ol Y} \ol X)^T\\
&=(\ol{\nb}_{\ol X} \ol Y-\ol{\nb}_{\ol Y} \ol X)^T \\
&=\pa{[\ol X,\ol Y]}^T=[X,Y].
\end{align*}
\item
Compatibility with connection: Given $X,Y,Z\in \X(M)$, we need to check
\[
Zg(X,Y)=g(\nb_ZX,Y)+g(X,\nb_ZY).
\]
%Evaluate at $p$, two things are the same. 
But taking the derivative in direction $Z$ is the same whether we are thinking in $M$ or in $N$. Thus this is equivalent to
\[
\ol Zg(\ol X,\ol Y)=g(\ol{\nb}_{\ol Z}\ol X,\ol Y)+g(\ol X,{\ol{\nb}}_{\ol Z}\ol Y),
\]
which holds.
%inner product with something tangent, normal doesnot matter.
\item Well-definedness: $\nb$ doesn't depend on the extension.
\end{enumerate}
\end{proof}
\subsection{Second fundamental form}
Given $M\hra N$, $X,Y\in \X(M)$ and $\ol X,\ol Y\in \X(M)$ such that $\ol X|_M=X$, $\ol Y|_M=Y$, we claim 
\[
(\ol{\nb}_{\ol X} \ol{Y})^{\perp}(p)
\]
depends only on the value of $\ol X(p)$ and $\ol Y(p)$. %Any other extension only get normal contribution.

%2 vector fields  
%linear in each of 2 var

%Suppose we're looking at $\ol f\pa{\nb_{\ol f\ol X}\ol Y}^T$. 
%Then
We have, for a function $f$ extended in a neighborhood in $N$, 
\[
(\ol{\nb}_{\ol X} (\ol f \ol Y))^{\perp}=(\ol f\ol{\nb}_{\ol X}\ol Y)^{\perp}
+\pa{\ol X(f)\ol Y}^{\perp}
=\ol f\pa{\ol{\nb}_{\ol X}\ol Y}^{\perp} + \underbrace{\ol X(f)(\ol Y(p))^{\perp}}_0
\]
because $\ol Y(p)=Y(p)$ has no perpendicular component. 
Thus $\ol{\nb}^{\perp}$ is linear in both its bottom and top variable; we've seen that anything with such a linearity property (where functions just pop out) just depend on values at $p$. Thus $(\ol{\nb}_{\ol X}\ol Y)^{\perp}(p)$ depends only on $X$, $Y$ at $p$.
\begin{df}
The bilinear map $B:T_pM\times T_pM\to \R$ is defined by
\[
B(X,Y):=(\ol{\nb}_{\ol X}\ol Y)^{\perp}(p).
\]
and is called the \textbf{second fundamental form}.
\end{df}

We claim that $B$ is also symmetric. We have
\[
B(Y,X)=(\ol{\nb}_{\ol Y} \ol X)^{\perp}
=(\ol{\nb}_{\ol X}\ol Y)^{\perp}+([\ol X,\ol Y])^{\perp}=B(X,Y)
\]
%vf supposed to be tangent to M.
because the Lie bracket of two vectors in the tangent space of $M$ is still in the tangent space, and hence has orthogonal component 0.

%Shay operator, fiber surface, Gauss equation.