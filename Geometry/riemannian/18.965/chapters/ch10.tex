\lecture{Thu. 10/11/12}

\subsection{Jacobi fields}
%Last time we discussed Jacobi fields. 
If we have a manifold $M$ with a symmetric connection $\nb$, then the curvature is defined by 
\[
R(X,Y)Z=\nb_Y\nb_XZ-\nb_X\nb_Y Z+\nb_{[X,Y]}Z.
\]
(It was initially defined for vector fields, but it really only depends on tangent vectors.) We proved that if $f:[a,b]\to [-\ep,\ep]$ is a parameterized surface (i.e. smooth map), and $V$ is a vector field along $f$, then (Lemma~\ref{lem:vf-ps})
\[
\cvd \cvs V-\cvs \cvd V=R\pa{\pd fs,\pd ft}V.
\]
We also showed (Proposition~\ref{pr:covar-commute})
\[
\cvs \pd ft=\cvd \pd fs.
\]
If $f$ is a parametrized surface, and $s\mapsto f(s,t)$ for each fixed $t$ is a geodesic, then 
\begin{align}
\nonumber
0&=\cvs \pd fs\\
\nonumber
\implies 0&=\cvd \cvs \pd fs\\
\nonumber&=R\pa{\pd fs,\pd ft}\pd fs + \cvs\cvd \pd fs&\text{Lemma~\ref{lem:vf-ps}}\\
&=R\pa{\pd fs,\pd ft}\pd fs + \cvs\cvs \pd ft&\text{Proposition~\ref{pr:covar-commute}}
\llabel{eq:965-10-1}
\end{align}
Fix $t$, say $t=0$. Denote the map $s\mapsto f(s,0)$ by $\ga(s)$; it is a geodesic by assumption. Define
\[
J=\pd ft
\]
to be the \textbf{variational vector field}.
%map from rectangle into manifold. 
Think of $f(s,0)$ as  a single curve, sitting inside a whole family of curves given by $f(s,t)$. We say $f(s,t)$ is a {\it variation} of curves in the $t$-direction. Now putting in $J=\pd ft$ in~\eqref{eq:965-10-1} gives
\[
0=R(\ga',J)\ga'+\cvs \cvs J=R(\ga',J)\ga'+J''.
\]
\begin{df}
If $\ga$ is a geodesic, and $J$ is a vector field along $\ga$, then $J$ is said to be a \textbf{Jacobi field} if 
\[J''+R(\ga',J)\ga'=0.\]
\end{df}
We proved that Jacobi fields naturally occur: if we take a variation of geodesics, then the variational vector field is a Jacobi field. We'll see how the Jacobi equations gives us the first explanation for a geometric notion of curvature.

%Riemannian, so LC
Now we make some calculations. Let $\ga$ be a geodesic. Then $\ga'$ is a parallel vector field. Let $E_1,\ldots, E_{n-1}$ be orthonormal parallel vector fields along $\ga$ such that each is orthogonal to $\ga'$. 

\ig{10-2}{1}

At each point along $\ga$, we have that $\ga'$, $E_1, \ldots,E_{n-1}$ is an orthonormal basis along $T_{\ga(s)}M$. Suppose $J$ is a vector field along $\ga$. Write 
\[J=j_0\ga'+j_1E_1+\cdots +j_{n-1}E_{n-1}\]
 where the $j_i$ are functions of $s$. Note $J_0=g(\ga',J)$, $J_i=g(J,E_i),\,i>0$; it is clear that the $j_i$ are smooth functions.

Now $E_i'=0$ and $E_i''=0$ so (using the fact $\ga$ is a geodesic),
\begin{align}
\nonumber
J' &= j_0'\ga'+\cancelto0{j_0\ga''}+j_i'E_i+\cancelto0{j_iE_i'}\\
\nonumber
&= j_0'\ga'+j_i'E_i\\
\llabel{eq:965-10-2}
J''&=j_0''\ga'+j_i''E_i'.
\end{align}
Recall that the sectional curvature of a 2-plane $\Pi$ was defined by 
\[
K(\Pi)=\fc{g(R(v_1,v_2)v_1,v_2)}{|v_1\wedge v_2|^2}
\]
where $|v_1\wedge v_2|^2=|v_1|^2|v_2|^2-g(v_1,v_2)^2$. In the particular case where $v_1,v_2$ is an orthonormal basis, the denominator is 1 so
\[
K(\Pi)=g(R(v_1,v_2)v_1,v_2).
\]
%If $J=j_0\ga' +j_iE_i$, then $J''=j_0''\ga'+j_i''E_i$. Filling this
Now substituting~\eqref{eq:965-10-2} into the equation for the Jacobi field $J''+R(\ga',J)\ga'=0$ we get
\begin{align*}
j_0''\ga' +j_i''E_i+\cancelto 0{j_0R(\ga',\ga')\ga'}+j_i R(\ga',E_i)\ga'&=0\\
j_0''\ga' +j_i''E_i+j_i R(\ga',E_i)\ga'&=0
\end{align*}
Now $R(\ga',E_i)\ga'$ is a vector field along $\ga$. By Proposition~\ref{pr:965-9-1}, this vector field is orthogonal to $\ga$:
\[
0=g(R(\ga',E_i)\ga',\ga').
\]
%(This is because $g(X,X)=0$ for any $X$.) 
Write $R(\ga',E_i)\ga'=R_i^k E_k$. Then we can write the Jacobi equation as
\begin{align*}
j_0''\ga'+j_i'' E_i+j_iR_i^kE_k&=0\\
j_0''\ga'+j_i'' E_i+j_kR_k^iE_i&=0.
\end{align*}
This is true iff it is zero componentwise:
\begin{align*}
j_0''&=0\\ 
j_i''&=j_kR_k^i.
\end{align*}
This is a system of ordinary differential equations. The solution is unique given initial data. %Completely determined. One and only one solution.

We have that $j_0$ is a linear function, so $j_0=ds+e$ for some constants $d$ and $e$. Usually we look at Jacobi fields that are orthogonal to the geodesic. In the case where $(M,g)$ is 2-dimensional, we can write $J=j_0\ga'+j_1E_1$. We have $j'=j_0'\ga'+g_1'E_1$. %$J$ is a Jacobi field iff $J(0)\perp \ga$ and $J'(0)\perp \ga'$, i.e. $j_0(0)=j_0'(0)=0$, i.e. $j_0\equiv 0$.

%original field orht to geodeisc, then remain orthogonal to geodesic. special jf oftne look at
%orthog to \ga', but 1st component is \ga'
%only 1 vector orthogonal

Let $J$ be a Jacobi field with $J(0)\perp \ga$ and $J'(0)\perp \ga'$. Let $J=j_1E_1$; we write this in short as $J=jE$. The Jacobi equation is $J''+R(\ga',J)\ga'=0$ which becomes
\[
j''E+jR(\ga',E)\ga'=0.
\]
Now 
\[R(\ga',E)\ga'=kE\text{ where }
k=g(R(\ga',E)\ga',E).
\]
For a surface, the sectional curvature is the Ricci curvature (under the correct normalization). We get
\[
j''E+jk E=0\iff j''+kj=0.
\]
This is the Jacobi equation for a 2-dimensional manifold. 
Consider 3 cases when $k$ is constant.
\begin{itemize}
\item
If $k=0$ then $j''=0$ %and $j=ds_a$. Now Jacbo 
and $j=(ds+e)E$.
\item
When the curvature equals 1 everywhere, i.e. $k\equiv1$, then we get $j''+j=0$.\footnote{This is the 1-dimensional Schr\"odinger equation.} The only solutions are $j=d\cos s+e\sin s$. 
%all straight line go thru origin, paramby unit seep 

%Take $t$-derivative, get linearly growning. Variation where curves are geodesics. Variational field is Jacob field.

%Can get both $eE$ and $dsE$; Jcobi equation is linear so.

For instance, the unit sphere has constant curvature 1. 
%Surface with curvature 1. Round unit sphere. $K=1$. 
%Fig 4
Its geodesics are the great circles are geodesics. Think of a family (variation) of great circles going through the north and south poles, with each great circle parametrized by unit speed. Then it makes sense that $j=e\sin s$ (it vanishes at $s=0$ and $\pi$, and has minimum absolute value in the middle; we have the geodesics are together at $s=0,\pi$ and farthest apart in the middle). %lookat things initially parallel.
\item
When the sectional curvature is constantly $-1$, the Jacobi equation is $j''-j=0$. We also know what the solution is in this case; the general solution is $j=d\cosh s+e\cosh s$. 
\end{itemize}

Suppose $J_1$ and $J_2$ are Jacobi fields along $\ga$. Let
\begin{align*} 
f(s)&=g(J_1',J_2)-g(J_1,J_2').
\end{align*}
Then
\begin{align*} 
f'&=g(J_1'',J_2)-g(J_1,J_2'')\\
&\quad+\cancel{g(J_1',J_2')}-\cancel{g(J_1',J_2')}\\
&=-g(R(\ga',J_1) \ga',J_2)+g(J_1,R(\ga',J_2)\ga')=0
\end{align*}
%use other symmetries of curve oper. If switch then nothing change.
%Extra structure much more general
%
%1-dimensional Schr\"odinger eqution $j''+j=0$.
%symplectic structure

In other words, $f$ is constant along a geodesic. Note $\ga'$ is a Jacobi field %parallel field
since $\ga''+R(\ga',\ga')\ga'=0$.\footnote{Think of $\ga$ as a family of geodesics, sliding forward along itself like a snake.} Thus specializing to $J_1=\ga'$, this equation says $g(\ga',J_2')$ is constant along a geodesic.

%\fixme{This section has been omitted. Prof. Colding will redo the calculations a bit differently next lecture.}
%Suppose that $\ga$ is a geodesic and $J$ is a Jacobi field along $\ga$. Then 
%\[
%|J|^2=g(J,J).
%\]
%Consider the function $f(s)=|J|^2$. Now using the Leibniz rule and symmetry of $g$,
%\begin{align*}
%f'&=2g(J',J)\\
%f''&=2g(J',J')+2g(J'',J)\\
%g(J'',J)&=-g(R(\ga',J)\ga',J).(*)\\
%&=-|J|^2K(\ga',J).
%%not on surfae, don't need?
%\end{align*}
%(*)
%Assuming originally orthogonal, Jacobi field remains orthogonal: $J\perp \ga$ implies $|J\wedge \ga'|^2=|J|^2$. (one is a unit vector.geodesic assumed unit speed.)
%
%$f(s)=|J|^2$. Now let's look at the Taylor expansion of $f$. We assume $J(0)=0$. Variation of geodesics all with unit speed originating at same point. Then $J(0)=0$ and $J\perp \ga'$. We want to evaluate $f(0)$, $f'(0)$, $f''(0)$, and $f'''(0)$. 
%\begin{align*}
%f(0)&=0\\
%f'(0)&=0\\
%f''(0)&=2g(J',J')=2|J'(0)|^2\\
%f'''(0)&=
%\end{align*}
Let's revist our geometric intuition for curvature. Consider (for simplicity) the case of a 2-dimensional surface. %(but works in all dimensions, don't need look atnorm squared, can looked at Jf first. Calculations simpler for surface. 
Fix a point $P$. We give a geometric definition of the sectional curvature at $P$. Consider the image under the exponential map of a small sphere of radius $\ep$ at the origin of the tangent space.

\ig{10-4}{1}

Geometrically, as $\ep\to 0$, the sectional curvature is first nontrivial coefficient of the Taylor expansion of the length of the the image. This is why we wanted to look at $f$. %The Jacobi fields by what we did in the beginninof thes class. Jf Derivative of exponential. 
I.e. the sectional curvature measures the distortion of geodesics. Next time we will derive the geometric description of the curvature from our original definition.
%Relate to exponential, clean up next time. Look at norm, not norm squared