\lecture{Thu. 12/6/12}

Today is the last day of class.

Last time we looked at a minimal surface $\Si\in \R^3$. We assumed that $\Si$ was stable, which meant that for all $\phi$ with compact support on $\Si$, the variation of area is nonnegative:
\[
-\int_{\Si} \phi L\phi\ge 0.
\]
Here $L$ is the Laplacian,
\[
L\phi = \De_{\Si} \phi+|A|^2\phi,
\]
and $|A|^2=\ka_1^2+\ka_2^2=-2K$ where $\ka_1,\ka_2$ are the principal curvatures. 

Last time we proved, using the Gauss-Bonnet Theorem, that for $p\in \Si$, 
\[
\text{Area}(\cal B_r(p))\le \fc 43 \pi r^2.
\]
When we did this calculation, we assumed there are no cut points, so $\exp_p:B_r(0)\to \cal B_r(p)$, $B_r(0)\subeq T_pM$, is a diffeomorphism.

When you have an operator 
\[
L\phi=\De_{\Si} \phi+V \phi
\]
for some ``potential" $V$, the eigenfunctions of $L$ are those such that 
\[
L\phi+\la\phi=0
\]
for some constant $\la$. We say $\phi$ has eigenvalue $\la$. (Note the sign convention.) If $\Si$ is compact %or we look at some compact subset of the surface
then one can prove $L$ is a compact operator, so there is a basis of eigenfunctions. We can order the eigenfunctions $\phi_i$ where the associated eigenvalues satisfy
\[
\la_1\le \cdots \le \la_i\le \cdots,\qquad \la_i\to \iy.
\]
All we need to know is that if we take the eigenfunction $\phi$ with lowest eigenvalue $\la$, then $\phi$ cannot change sign:
\[
|\phi|>0.
\]
This is easy to prove; we'll come back to it. By replacing $\phi$ by $|\phi|$ we may assume $\phi$ is positive.

If $\Si$ is stable, then
\[
-\int_{\Si} \phi L\phi\ge 0
\]
where $L\phi +\la_1\phi=0$, $\phi>0$. We get 
\[
0\le -\int_{\Si}\phi L\phi=\la\int_{\Si}\phi^2;
\]
the lowest eigenvalue is nonnegative. We obtain
\[
L\phi=-\la\phi\le 0.
\]

If we have a Schr\"odinger operator $\De+V$, and $\phi>0$ that is a solution to $(\De+V)\phi=0$, then for all $\phi$ with compact support,
\[
\int \psi(\De+V)\psi\ge 0.
\]
(We looked at $\ln \phi$.) It doesn't need to be a solution, it just needs to be a {\it supersolution}, i.e., satisfy $L\phi+\la \phi\le 0$. 
At some point we used an absorbing inequality; it still holds.

Now if we have $p\in \Si\subeq \R^3$ with $\Si$ stable, $\ka_1=-\ka_2$, and $K\le 0$, then Hadamard's Theorem gives that $\exp_p:T_p\Si\to \Si$ is a covering map. (Here we are assuming $\Si$ is complete noncompact.) Note that in our inequality we assumed there were no cut points. 
%complet noncompact 
%if boundary, then on finite set.
We have $\Si\subeq \R^3$; pulling back the metric we get a covering map (locally an isometry). The composition
\[
T_p\Si\xra{\exp_p} \Si\subeq \R^3
\]
is an immersion. But in $T_p\Si$ the exponential map is actually a diffeomorphism; there is no cut point. By going to the cover, we can assume there is no cut point. We don't have the assumption that $T_pM$ is stable, but we show this is true. Consider the operator $L=\De_{\Si}+|A_{\Si}|^2$. Take the eigenfunction corresponding to the smallest eigenvalue, we may assume $\phi>0$. Then $L_{\Si}\phi\le 0$. Composing with the exponential map, we may consider it on $T_pM$:
\[
\wt{\phi}=\phi\circ \exp_p.
\]
Let $\wt{\Si}=T_pM$ be $T_pM$ with the pullback metric. We have %bc pointwise the same.
\[
L_{\wt{\Si}}\phi\le 0.
\]
%$\De_{\Si}$ is the same because it is given by the metric; $A_{\Si}$ is the same because tells how surface sits inside.
This implies $\wt{\Si}$ is stable.
We have an inequality for all functions. On the cover there are many more functions than pullback functions, so we have to prove something more.

We've removed the cut point assumption; we always have the inequality on area; we have the inequality for $\wt{\Si}$, so we clearly also have it on $\Si$. The area of the corresponding ball on $\Si$ is smaller than the pullback area. %don't need assumption on conjugate points. Use Hadamard theorem.

\subsection{Curvature estimate}

We're aim to prove a curvature estimate. 
\begin{thm}
If $\Si$ is stable, then if $\cal B_r(p)\subeq \Si\bs \pl \Si$, then
\[
\sup_{\cal B_{\fc r2}}|A|^2\le Cr^{-2}
\]
where $C$ is a constant independent of $r$ and $p$.
\end{thm}
If we have a minimal surface, its image under any isometry is a minimal surface. In $\R^3$, any scaling of a minimal surface is a minimal surface. Thus we obtain a whole family of minimal surfaces. For instance, we can make the neck of a catenoid as small as we want.
The first surface is stable iff the image is. (Actually the index is the same.)

A catenoid cannot be stable. Take a huge ball (let $r\to \iy$); the half ball has almost 0 curvature. But in the middle there's curvature. Thus a catenoid is not stable. The same argument works for the helicoid.

To prove our theorem, we need the following.
\subsubsection{Logarithmic cut-off trick}
%\begin{df}
Let $\Si$ be a minimal surface, and suppose we have a quadratic area bound 
\[\text{Area}(\cal B_r(p))\le cr^2.\] 
We show that if this holds for all $r$, then we can find a function that is 1 on the unit ball centered at the point, and has small energy.

Define
\[
\phi=\begin{cases}
1&\text{on }\cal B_1(p)\\
1-\fc{\ln s}{\ln r}&\text{on }\pl \cal B_s(p),\, 1\le s\le r\\
0 &\text{otherwise.}
\end{cases}
\]
The function decays from 1 to 0 from 1 to $r$. 
%\end{df}

\ig{24-1}{1}

We calculate that
\[
|\nb \phi|=
\begin{cases}
0&\text{on }\cal B_1(p)\cup (\Si\bs \cal B_r(p))\\
\rc{s\ln r}&\text{on }\pl\cal B_r(p),\,1<s<r.
\end{cases}•
\]
%not diffble but Lipschitz, so fine. If want, approx by smooth func. Lipschitz bounded everywhere. Converge to blah. Doesn't really matter
Note $\phi$ has compact support because it dies at $r$. We show that if $r$ is large, $\phi$ has small energy:
\begin{align*}
\int |\nb \phi|^2
&=\int_0^{\iy}\int_{\pl \cal B_r(p)} |\nb \phi|^2\,d\ell ds\\
&=\int_1^r \prc{s\ln n}^2\,d\ell ds\\
&=\rc{(\ln r)^2} \int_1^r \fc{\ell (s)}{s^2}\,ds.
\end{align*}
(Integrate over the distance spheres.)
Using 
\[
\text{Area}(\cal B_r(p))=\int_0^r \ell(s)\,ds
\]
we have
\[
\ddd s\text{Area}(\cal B_r(p))=\ell(r).
\]
We integrate by parts because we don't have a bound for $\ell$ but we have a bound for area. We get
\begin{align*}
\int |\nb \phi|^2
&=\rc{(\ln r)^2} \int_1^r \fc{\ell(s)}{s^2}\,ds\\
&=\rc{(\ln r)^2} \pa{
\ba{\fc{\text{Area}(\cal B_s)}{s^2}}^r_1 - 2\int_1^r \fc{\text{Area}(\cal B_s(p))}{s^3}\,ds
}\\
&=\rc{(\ln r)^2} \pa{\pa{\fc{\text{Area}(\cal B_r(p))}{r^2}-\text{Area}(\cal B_r(p))}-2\cdots}
\end{align*}
As $r\to \iy$, the first term goes to 0. %the last term is always negative. Is energy, nonneg. As $r\to \iy$ only first term positive.
Since energy is nonnegative and only the first term is positive, we get that it goes to 0.

%complete without boundary with assumption on area. can construct function 1 on unit ball and have compact support, and as small energy as you want. 

We've proven that if $\Si\subeq \R^3$ is stable and complete without boundary, then $\text{Area}(\cal B_r(p))\le \fc 43 \pi r^2$. We'd like to prove $\sup |A|^2\le cr^{-2}$. 
If $\Si$ doesn't have any boundary, then this holds for all $r$. Taking $r\to \iy$, 
\[
|A|^2(p)\le \sup_{\cal B_{\fc r2}(p)} |A|^2\le cr^{-2}\to 0.
\]
(This result is by Schoen in 1982.)
Then the second fundamental form is 0 at every point, so it must be a plane:
The derivative of the normal is 0, so the normal is constant; hence the surface must be in a plane orthogonal to this constant normal.
We get $|A|^2\equiv 0$, so $n$ is constant, and $\Si=n^{\perp}$.

%We want to prove the inequality in general. %ball doesn't intersect boundary. 
%Let's do the case there's no boundary in all. If can prove for all $r$
%surface is the graph of a function
\begin{thm}[Bernstein Theorem, 1911]
If $\Si\subeq \R^3$ is a stable minimal surface in $\R^3$ without boundary, then $\Si$ is a plane.
\end{thm}
%Then 
\begin{proof}
Let $\Si\subeq \R^3$ be stable. Then $\text{Area}(\cal B_r(p))\le \fc 43 \pi r^2$. Then there exist $\phi_r$ so that $\int |\nb \phi_r|^2\to 0$, $\phi_r$ has compact support, and $\phi_r=1$ on $B_1(p)$.

Now the inequality $0\le -\int_{\Si}\phi L\phi$ ($L\phi=\De_{\Si} \phi+|A|^2\phi$) becomes, after integrating by parts and using the fact that $\phi$ is compactly supported,
\begin{align*}
\int \phi L\phi &=\int \phi(\De\phi+|A|^2 \phi)\\
&=-\int|\nb \phi|^2+\int |A|^2\phi^2
\end{align*}
Since $\Si$ is stable,
\[
\int|\nb \phi|^2\ge \int|A|^2\phi^2.
\]
Inserting $\phi_r$, we get (noting $\phi_r$ is 1 on the unit ball)
\[
\int |\nb \phi_r|^2\ge \int|A|^2\phi_r^2\ge \int_{\cal B_1(p)}|A|^2.
\]
Now the LHS goes to 0, so $|A|^2(p)=0$. This proves the Bernstein Theorem.
\end{proof}
The catenoid is the surface of revolution of hyperbolic cosine. Topologically it is a cylinder. It is complete without boundary. The catenoid can't be stable, because if it were stable it would have to be a plane. The same goes for the helicoid.

How do we prove the more general statement? This is quite useful. 
\begin{thm}[Schoen, 1982]\llabel{thm:schoen}
Let $\Si\subeq \R^3$ be stable. Suppose $\cal B_r(p)\subeq \Si\bs \pl \Si$. Then
\[
\sup_{\cal B_{\fc r2}(p)}|A|^2(p)\le cr^{-2}
\]
for some constant $c$ independent of $p$ and $r$.
\end{thm}
We use the following.
\begin{thm}[Choi-Schoen]\llabel{thm:choi-schoen}
There exists an $\ep>0$ such that if $p\in\Si\subeq \R^3$ is a minimal surface and $\cal B_r(p)\subeq \Si\bs \pl \Si$, then 
$\int_{\cal B_r(p)} |A|^2<\ep$  implies $\sup_{\cal B_{\fc r2}}|A|^2\le r^{-2}$.
%interior estimate
\end{thm}
This is key.
\begin{proof}[Proof of Theorem~\eqref{thm:schoen} using Theorem~\eqref{thm:choi-schoen}]
Imagine we have a ball $\cal B_r(p)$. We can scale it so the ball is very large. Now $\int_{\cal B_r(p)}|A|^2$ is invariant under scaling. (The second fundamental form goes down and area goes up; they cancel each other out.) %think of \cal B_r(p)$ as very large ball.

We just need to prove the bound for the second fundamental form in the center. Then you can do it everywhere. %\fixme{(fig 2)}

But if the ball is very large, we can find a function $\phi$ that is 1 on the unit ball and has very small area. We use quadratic area bounds proved in this setting. We have $\int|\nb \phi|^2< \ep$ and by stability,
\[
\int|\nb\phi|^2\ge \int|A|^2\phi^2\ge_{\cal B_1}|A|^2.
\]
The integral is small because of point mass bounds (from Choi-Shoen), so we have the theorem. %in half of ball have point mass estimate. Point mass estimate reduce to Choi-Schoen.
We used the quadratic area bound.
\end{proof}

These theorems are examples of a general type of theorem common in nonlinear differential equations. 
The second fundamental  form is like energy. 
If you have an energy inequality, then you get a point mass estimate. (If the energy is above a certain threshold, then we don't get an estimate.)

Consider the second fundamental form on the catenoid: we have $\int|A|^2<\iy$. This condition is called ``finite total curvature." This is not small, so we don't have a pointwise estimate. On the other hand , the helicoid is not bounded. 

As $|A|^2=-2K$, the condition $\int|A|^2<\iy$ is called ``finite total curvature." %$|A|^2=-2K$, so the $\int|A|^2<\iy$ condition is called ``finite total curvature."

You can find all this in~\cite{CM}.

\begin{proof}[Proof sketch of Theorem~\ref{thm:choi-schoen}]
We first show Simon's inequality
\[
\De|A|^2\ge -2|A|^4.
\]
Let $u:\R^n\to \R$. Consider $I(r)=r^{1-n}\int_{\pl B_r(0)} u$. We have by Stokes's Theorem%something area/volume constant?
\[
I'(r)=r^{1-n}\int_{\pl B_r(0)}\dd us = r^{1-n}\int_{\cal B_r(0)}\De u.
\]
If $u$ is harmonic this is constant. 
%Taking a harmonic function, we have the mean value equality
%\[
%\]
Then $I(r)=\lim_{s\to 0} I(s)=\Vol(\pl B_1)u(0)$. We have a mean value equality
\[
u(0)=\rc{\Vol(\pl B_r(0))}\int_{\pl B_r(0)}u.
\]
If $u$ is subharmonic, we get inequality in one direction, if supharmonic, we get inequality in other direction.
If we have an eigenfunction (or subsolution) $\De u +\la u=0$ we get a mean value equality, with some constant depending on $\la, r$.

Simon's inequality is a nonlinear inequality. There's a simple way of arguing by contradiction so we can actually assume $|A|^2(p)=1$. We get $\sup_{\cal B_r(p)}|A|^2\le 4$. Then trivially we can replace one $|A|^2$ by 4, and now have a linear inequality. The inequality says $|A|$ is a subsolution to an eigenvalue equation. Using the mean value inequality, $A$ at the center is bounded by the mean. But we assumed $\int |A|^2$ is really small. Thus we get contradiction. The reduction is simple, and nothing to do with minimal surface; it's just a calculus fact about functions. 
\[
\De|A|^2\ge -2|A|^4\ge -2|A|^2|A|^2\ge -8|A|^2.
\]
Now locate the point where $F=(r-d)^2|A|^2$ is maximal. %Scale, point where maximal. Recall, get this point, ball, $\sup$ 4 thing.
\end{proof}