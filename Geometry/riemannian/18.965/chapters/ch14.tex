\lecture{Thu. 10/25/12}

Absent.

We covered the Hopf-Rinow Theorem. See~\cite[p. 144-149]{dC}.

\begin{df}
A Riemannian manifold $M$ is \textbf{extendible} if there exists a Riemannian manifold $M'$ such that $M$ is isometric to a proper open subset of $M'$.

A Riemannian manifold $M$ is (geodesically) \textbf{complete} if for all $p\in M$, $\exp_p$ is defined for all $v\in T_pM$, i.e., any geodesic $\ga(t)$ starting at $p$ is defined for all $t\in \R$.
\end{df}

\begin{pr}
If $M$ is complete, then $M$ is non-extendible.
\end{pr}

We give $M$ a metric space structure by letting $d_M(p,q)$ be the infimum of lengths of all curves joining $p$ and $q$. (This is the same as the original metric space structure.)

\begin{thm}[Hopf-Rinow Theorem]\llabel{thm:hopf-rinow}
Let $M$ be a Riemannian manifold and let $p\in M$. Then the following are equivalent.
\begin{enumerate}
\item
$\exp_p$ is defined on all $T_pM$.
\item
The closed and bounded sets of $M$ are compact.
\item
$M$ is complete as a metric space.
\item
$M$ is geodesically complete.
\item
There exists a sequence of compact subsets $K_n\subeq M$, $K_n\sub K_{n+1}$ such that if $q_n\nin K_n$ then $d(p,q_n)\to \iy$.
\end{enumerate}
Any of these statements implies
\begin{enumerate}[resume]
\item
For any $q\in M$ there exists a geodesic $\ga$ joining $p$ and $q$ with $\ell(\ga)=d(p,q)$.
\end{enumerate}
\end{thm}