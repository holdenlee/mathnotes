\def\filepath{C:/Users/Owner/Dropbox/Math/templates}

\input{\filepath/packages_article.tex}
\input{\filepath/theorems_with_boxes.tex}
\input{\filepath/macros.tex}
\input{\filepath/formatting.tex}
%\input{\filepath/other.tex}

%\def\name{NAME}

%\input{\filepath/titlepage.tex}

\pagestyle{fancy}
%\addtolength{\headwidth}{\marginparsep} %these change header-rule width
%\addtolength{\headwidth}{\marginparwidth}
\lhead{Statistics}
\chead{} 
\rhead{} 
\lfoot{} 
\cfoot{\thepage} 
\rfoot{} 
\renewcommand{\headrulewidth}{.3pt} 
%\renewcommand{\footrulewidth}{.3pt}
\setlength\voffset{0in}
\setlength\textheight{648pt}

\begin{document}
\tableofcontents
\section{Introduction}
\subsection{Probability distributions}

\begin{tabular}{|c|c|c|c|c|c|}
\hline 
Name & Distribution & Mean & Variance & MGF & Notes\tabularnewline
\hline 
Normal & \textbf{$\frac{1}{\sqrt{2\pi}}e^{-\frac{(x-\mu)^{2}}{2\sigma^{2}}}$} & $\mu$ & $\sigma^{2}$ &  & \tabularnewline
\hline 
Exponential & $\lambda e^{-\lambda x}$ & $\frac{1}{\lambda}$ & $\frac{1}{\lambda^{2}}$ &  & \tabularnewline
\hline 
Gamma & $\frac{\beta^{\alpha}}{\Gamma(\alpha)}x^{\alpha-1}e^{-\beta x}$ & $\frac{\alpha}{\beta}$ & $\frac{\alpha}{\beta^{2}}$ & $(1-\frac{t}{\beta})^{-\alpha}$ & \tabularnewline
\hline 
Beta & $\frac{\Ga(\al)+\Ga(\be)}{\Ga(\al+\be)}{x^{\alpha-1}(1-x)^{\beta-1}}$ & $\frac{\alpha}{\alpha+\beta}$ & $\frac{\alpha\beta}{\alpha+\beta-2}$ &  & \tabularnewline
\hline 
Binomial & $\binom{n}{k}p^{k}(1-p)^{n-k}$ & $pn$ & ${p(1-p)n}$ &  & $\to N(pn,{p(1-p)n})$\tabularnewline
\hline 
Poisson & $\frac{\lambda^{k}e^{-\lambda}}{k!}$ & $\lambda$ & $\lambda$ &  & \tabularnewline
\hline 
 &  &  &  &  & \tabularnewline
\hline 
\end{tabular}

The moment generating function is defined as $\E(e^{tX})=\int e^tx\,d\mu$. Here is a sketch of uniqueness.
\begin{thm}[Billingsley, Theorem 30.1]
\begin{enumerate}
\item
Let $\mu$ be a probability measure
 with moments of all orders:
\[
\al_k=\int_{-\iy}^{\iy} x^k\,d\mu.
\]
If $\sum_{k\ge 0}\frac{\al_kr^k}{k!}$ has a positive radius of convergence, then $\mu$ is the only probability measure with those moments.
\item If the MGF of $\mu$ is $g$ and $g$ is analytic at 0, then $\mu$ is the unique probability measure with MGF $g$.
\end{enumerate}
\end{thm}
\begin{proof}[Proof sketch]
To see (1) from (2), note $g^{(k)}(0)=\al_k$.

\begin{enumerate}
\item We have uniqueness of characteristic functions (if we took $\E(e^{itX})$ instead) basically by the theorem for Fourier transforms.
\item
Let $\be_k = \int_{-\iy}^{\iy} |x|^k\,d\mu$ and suppose the series converges at radius $r$. Show that $\frac{\be_kr^k}{k!}=0$.
\item 
We want to show the moments determine the characteristic function. Letting $\ph$ be the characteristic function $\ph$ of $\mu$, we would like it to be true that
\[
\ph(t+h)=
\sum_{k=0}^\iy \frac{h^k}{k!}\int_{-\iy}^{\iy} (ix)^k e^{itx}\,d\mu ,\qquad |h|\le r.
\]
as the sum comes from just expanding $e^{itx}$ in power series. Show the partial sums converge by bounding by $\be_{n}$'s.

For $\ph^{(k)}(0)$ give the moments, and we get uniquenes son an interval $(-r,r)$. Repeat near the boundary of the interval. 
\end{enumerate}
\end{proof}

Like characteristic functions (Fourier transforms), MGF's turn convolution (addition of random variables) into multiplication.

Some notes about the distributions.
\begin{enumerate}
\item The normal distribution is ubiquitous because by the central limit theorem, the sum of many (nice) iid random variables is close to normal.
\item The exponential distribution is memoryless.
\item The sum of gamma random variables $(\al_n,\be)$ is $(\sum\al_n,\be)$ because MGF's turn addition of rv's into multiplication.
\item The Poisson distribution can be derived from the following axioms. It is the probability that $k$ events occur in an interval of length 1 when
\begin{enumerate}
\item
events occurring in disjoint intervals are independent and 
\item 
equal for intervals of the same length.
\item
the probability of an event occurring $I$ of length $\to 0$ goes $\to 0$.
\end{enumerate}
The probability of $k$ events in an interval of length 1 is then 
\[
\lim_{n\to \iy}\binom nk \pf{\la}{n}^k\pa{1-\frac{\la}{n}}^{n-k}= \frac{\la^k e^{-n}}{k!}.
\]
\end{enumerate}

\subsection{Statistical distributions}

Here are distributions common in statistics. They'll be explained later.

\begin{tabular}{|c|c|c|c|}
\hline 
Name & is distribution of... & Notes & \tabularnewline
\hline 
\textbf{$\chi_{n}^{2}=\Ga(\fc k2,\rc2)$} & $\sum_{i}Z_{i}^{2},\, Z_{i}\sim N(0,1)$ &  & \tabularnewline
\hline 
$t_{n}$ & $\frac{X}{\sqrt{\rc nS}},\,X\sim  N(0,1), \,S\sim \chi_{n}^{2}$ &  & \tabularnewline
\hline 
$F_{m,n}$ & $\frac{A/m}{B/n},\,A\sim \chi_{m}^{2},\,B \sim\chi_{m}^{2}$ &  & \tabularnewline
\hline 
 &  &  & \tabularnewline
\hline 
 &  &  & \tabularnewline
\hline 
\end{tabular}

\subsection{Multidimensional normal distribution}

For $A\in \Mat_{n\times n}(\R)$, if $x$ is iid standard normal, then $Ax$ is defined to have distribution $N(0,AA^T)$. Note that if the covariance of $x$ is $D$ ($D=I_n$ here), then the covariance of $Ax$ is $ADA^T$, so the covariance.

Why is this well-defined? (I.e., if $AA^T=BB^T$ then $Ax,Bx$ are identically distributed.) Assume $A$ is full rank (else we need to restrict to a subspace) We calculate the distribution. By change of variables it's
\[
\rc{(2\pi)^{\fc n2}|\det A|}e^{-\frac{|A^{-1}x|^2}{2}}\,dx = 
\rc{(2\pi)^{\fc n2}|\det A|}e^{-\frac{x^T(AA^T)^{-1}x}{2}}\,dx.
\]

Define the shifted distribution $N(\mu,A),\mu\in \R^n$ in the obvious way.

\subsection{Statistics}

In statistics, we want to estimate some function\footnote{``statistic." For instance, mean, standard deviation} of a distribution $F$\footnote{$F$ refers to cdf, $f$ refers to pdf} parameterized by $\te'\in \Te$, given that $F\in \cal F$ (some family), when we observe samples drawn from $F$.

\begin{enumerate}
\item
If $\cal F$ (i.e., $\Te$) is finite-dimensional, we're doing parametric statistics. (For example, we assume $F$ is normal---normal distributions are parametrized by $(\mu,\si)\in \R\times \R_{\ge0}$.)
\item
If $\cal F$ is infinite-dimensional, we're doing nonparametric statistics.
\end{enumerate}

The two kinds of statistics are\footnote{for simplicity, suppose we're dealing with $\cal F$ finite; for $\cal F$ infinite (as is usually the case), replace $\Pj$ with probability density and $\sum$ with $\int$.}
\begin{enumerate}
\item Bayesian: We assume a distribution on priors. Roughly speaking, given observed event (ex. the sample drawn) $B$, the likelihood that it came from distribution $F\in \cal F$ is 
\[
\Pj(F|B) = \frac{\blu{\Pj(F)} \Pj(B|F)}{\Pj(B)}.
\]
Our best guess for $F$ is $\arg\max_F \Pj(F|B)$. We can calculate the expected value for $\te(F)$; it will be $\sum_F \Pj(F|B) \te(F)$.
\item Frequentist: We assume no distribution on priors $\Pj(F)$ is known. In this case we simply maximize
\[
\Pj(B|F).
\]
\end{enumerate}

Frequentist vs. Bayesian axioms:

\noindent
\begin{tabular}{|c|c|c|}
\hline 
 & Frequentist & Bayesian\tabularnewline
\hline 
Probability is  & limiting relative frequency. & degree of belief.\tabularnewline
\hline 
Parameters  & are fixed unknown constants. & can be talk about probabilistically.\tabularnewline
\hline 
Statistical procedures & good long run frequency properties. & involve a distribution on $\theta$.\tabularnewline
\hline 
\end{tabular}

``To combine prior beliefs with data in a principled way, use Bayesian inference.
To construct procedures with guaranteed long run performance, such
as confidence intervals, use frequentist methods. Generally, Bayesian methods
run into problems when the parameter space is high dimensional."

\section{Basic inference}
Suppose $(X_1,\ldots, X_n)$ is the observed sample, and our estimate for the statistic $\te$ is $\wh{\te_n}=g(X_1,\ldots, X_n)$. (Example: $\te$ is one component of $\te'$. For example, $\te'=(\mu,\si)$ and $\te$ is just $\mu$ or $\si$.)\footnote{This is suboptimal notation, but $\te$ is used for both in the whole parameter and a function of it in the literature, and that's confusing.}


\subsection{Estimators}



\subsubsection{Error}

\begin{df}
Define the \textbf{standard error} by \[\se=\sqrt{\Var_\te(\wh{\te_n})}.\]
Note we can't find this directly because we don't know the actual distribution $F$ so don't know $\te(F)$. 
Suppose $\se=s(\te)$. The \textbf{estimated standard error} comes from estimating $F$ (i.e., $\te$) first and plugging that value of $\te$ into the formula for se: $\hse = s(\hten)$.

Define \textbf{bias} by\footnote{Warning: $\E_\te$ means average given $\te$, not over $\te$.}
\[
\bias(\wh{\te_n})=\E_\te(\wh{\te_n})-\te.
\]
(Note we must be given the actual value of $\te$ to calculate the bias, so this is a function of $\te$.)

The \textbf{mean standard error (MSE)} is 
\bal
\E_\te[ (\wh{\te_n}-\te)^2] 
&= (\ol{\wh{\te_n}}-\te)^2 + \E_\te (\hten-\ol{\te_n})^2\\
&= \bias(\hten)^2 + \Var_{\te}(\hten)
\end{align*}
\end{df}
Bias measures how much the average estimate is from the actual value, the second part measures how much the estimate is from the average estimate.

\begin{ex}
\begin{enumerate}
\item
Bernoulli($p$). $\se=\sfc{p(1-p)}{n}$ and $\hse = \sfc{\wh p (1-\wh p)}{n}$ where $\wh p = \rc n \sum_i X_i$.
\item
$N(\mu,\si)$. $\se=\sfc{n-1}{n}\si$ and $\hse = \sfc{n-1}n \hat{\si}=\sfc{n-1}n\sfc{n}{n-1}\sqrt{\E\Var_\te\{X_1,\ldots,X_n\}} =\sqrt{\E\Var_\te\{X_1,\ldots,X_n\}} = \fc{\sum(X_i-\ol X)^2}{n-1}$.

If we knew $\se$, then the distribution of $\wh\mu$ is
\[
\fc{\wh \mu - \mu}{\se}\sim N\pa{0,\fc{\si}{\sqrt n}}.
\]
However, if we use $\hse$ instead of $\se$ we get a $t$-distribution rather than a normal distribution. For $n$ large, the $t$-distribution is approximately the same.
\end{enumerate}
\end{ex}

\subsubsection{Properties}
What properties do we want for an estimator?
\begin{enumerate}
\item
Unbiased: For every $F\in \cal F$, $\E_{X_1,\ldots, X_n\sim \cal F}\wh{\te_n}=\te$. (This is actually not so important!)
\item Consistent: $\wh{\te_n}\xra{P}\te$ as $n\to \iy$.
\item Asymptotically normal (stronger than consistent): $\fc{\wh{\te_n}-\te}{\se}\leadsto N(0,1)$. \fixme{Warning: sometimes we care about this quantity with $\hse$!}
\item Asymptotic optimal/efficient: among all well-behaved estimators, the MLE has smallest variance.
%or \rightsquigarrow
\end{enumerate}

\begin{exr}
Explain why the sample variation is given by $\wh{\si}^2=\frac{\sum (X_i-\ol X)^2}{N-1}$.
\end{exr}
\begin{proof}[Solution.]
This is an unbiased estimator, while $\frac{\sum (X_i-\ol X)^2}{N}$ is a biased estimator.

For simplicity, consider the discrete case. Let $(X_i)_{i=1}^n$ denote the sample. We have (see exercise below)
\bal
\E \Var((X_i)) &= \E(X_i^2)-(\E(X_i))^2\\
&= \frac{\sum X_i^2}{n} - \frac{\sum X_i^2}{n^2} - \frac{\sum_{i\ne j} X_iX_j}{n^2}\\
&= \fc{n-1}{n} \pa{\frac{\sum X_i^2}{n}-\frac{\sum_{i\ne j} X_i X_j}{n(n-1)}}\\
&= \frac{n-1}{n} (\E (X_i^2)-(\E X_i)^2) = \frac{n-1}{n}\Var\{X_1,\ldots,X_n\}.
\end{align*}
\end{proof}

\subsection{Maximal likelihood estimators}

\begin{df}
Let $\te$ be the parameter for a distribution and $x^n=(x_1,\ldots,x_n)$ the sample. The \textbf{likelihood} of $x^n$ given $\te$ is
\[
\cal L(\te) :=\Pj(x^n|\te) =\prod_{i=1}^n f(x_i;\te)
\]
where $f(x;\te)$ is the probability density of $x$ given $\te$.
The \textbf{log-likelihood} is
\[
\ell(\te):= \ln \Pj(x^n|\te) = \sum_{i=1}^n \ln f(x_i;\te).
\]
(We usually aren't too careful with constants, and may drop them.)
The \textbf{maximal likelihood estimator} is
\[
\arg_{\te}\max \cal L(\te) = \arg_{\te}\max \ell(\te).
\]
\end{df}
\subsubsection{Examples}
We calculate the MLE for several distributions.
\begin{enumerate}
\item
Bernoulli. Here each $X_i$ is 0 or 1. We have
\[
\cal L(\te) = p^{\sum_i X_i} (1-p)^{n-\sum_i X_i}, \qquad \ell(\te) = (\sum X_i) \ln p + (n- \sum X_i)\ln (1-p).
\]
Setting $\pdt{p}=0$ gives $\wh p=\frac{\sum X_i}{n}$ as MLE.
\item
Normal. Finding the MLE for $\mu$ means maximizing $\ell(\mu,\si)=C-\sum \frac{(x_i-\mu)^2}{2\si^2}$, which is minimizing the sum of squares 
\[
\sum (x_i-\mu)^2.
\]
The MLE is $\wh{\mu} = \rc n \sum x_i$.\footnote{It does NOT make sense to find the MLE for $\si$, which is $\iy$.}
\item 
(Uniform distribution)
Given $\cal F=\{U_{[0,l]}\}$, the estimated $l$ is $\min\{X_1,\ldots, X_n\}$. (Suppose the buses in an unknown city are labeled 1 to $N$. Assuming no knowledge of a prior distribution on number of buses, and you see buses numbered $b_1,\ldots, b_n$, your best guess for the number of buses is $\max\{b_1,\ldots, b_n\}$.)
\item
Linear regression. We assume $x_i$ are fixed, and $y_i=\be_0+\be_1x_i+\ep$ where $\ep\sim N(0,\si^2)$ is error. Then we want to minimize
\[
\sum_i (y_i-\be_0-\be_1x_i)^2
\]
(minimize least squares). Setting $\pdt{\be_0}=\pdt{\be_1}=0$ gives the system $\matt{n}{\sum X_i}{\sum X_i}{\sum X_i^2}=\coltwo{\sum Y_i}{\sum Y_iX_i}$ which has solution
\[
\be_1 = \frac{\ol{XY}-\ol X\cdot\ol Y}{\ol{X^2}-\ol{X}^2}\qquad \be_0=\ol Y-\be_1\ol X.
\]
\item Multivariate linear regression. Here $Y=X\be + \ep$ where $X$ is $n\times p$. The MLE is again given by least squares, which is given by the projection $\wh \be = (X^TX)^{-1}X^TY$ (assume $X$ has full rank; this is necessary).
\end{enumerate} 


%\subsection{Examples}
%
%\begin{enumerate}
%\item
%(Uniform distribution)
%Given $\cal F=\{U_{[0,l]}\}$, the estimated $l$ is $\min\{X_1,\ldots, X_n\}$. (Suppose the buses in an unknown city are labeled 1 to $N$. Assuming no knowledge of a prior distribution on number of buses, and you see buses numbered $b_1,\ldots, b_n$, your best guess for the number of buses is $\max\{b_1,\ldots, b_n\}$.)
%\item 
%(estimate for normal distribution)
%Let $\wh\mu=\rc n\sum X_i$ and $\wh{\se}=\wh\si = \frac{\sum (X_i-\ol X)^2}{n-1}$. 
%\end{enumerate}•

\subsubsection{Properties of MLE}
Define some quantities first.
\begin{df}
\begin{enumerate}
\item
\textbf{KL distance}
\[
D(f,g)=\int f(x)\ln \pf{f}{g}\,dx.
\]
Why do we care about this? Maximizing $\ell_n(\te)$ is equivalent to maximizing 
\[
M_n(\te)=\rc n\sum_i\ln \fc{f(X_i;\te)}{f(X_i;\te_*)}
\]
which has the nice property that the maximum is 0. (Without the $\rc n$ it would blow up.) By LLN the expected value of this is exactly $-D(\te_*,\te)$.
\item 
\textbf{score function} $s(X;\te)=\pd{\ln f}{\te}$.

Important property: $\E s=\int_{-\iy}^{\iy} s(X;\te) f\,dx=(\int_{-\iy}^{\iy}f\,dx)_{\te}=0$. 
\item
\textbf{Fisher information} $I(\te)=\Var_\te(s(X;\te))$, $I_n(\te)=nI(\te)$.
I.e., $I(\te)=-\E((\ln f)_{\te\te})$.
\end{enumerate}
\end{df}

\begin{enumerate}
\item
\begin{thm}[Convergence of MLE]
Suppose 
\begin{enumerate}
\item
$\sup_{\te\in \Te}|M_n(\te)-M(\te)|\xra{P}0$,
\item
for all $\ep>0$, $\sup_{|\te-\te_*|\ge\ep} M(\te)<M(\te_*)$.
\end{enumerate}
Then the MLE $\wh{\te_n}\xra P\te_*$.
\end{thm}
\begin{proof}
First show that $M(\te_*)-M(\wh{\te_n})\xra P0$. Then use continuity of $M$.
\end{proof}
\item
\begin{thm}[Asymptotic normality of MLE]
\begin{enumerate}
\item
$\se\sim \sfc1{nI(\te)}$ and $\fc{\wh{\te_n}-\te}{\se}\to N(0,1)$.
\item \fixme{$\wh{\se}=\sfc{1}{nI(\wh{\te_n})}$: why are we redefining $\wh{\se}$? We defined it a different way before. Do these definitions coincide?}
$\fc{\wh{\te_n}-\te}{\wh{\se}}\to N(0,1)$.
\end{enumerate}•
\end{thm}
\begin{proof}
\begin{enumerate}
\item
Linearize to find that
\[
\ell'(\wh \te)-\ell'(\te)\approx (\wh \te-\te)(\ell''(\te))\implies -\fc{\ell'}{\ell''}(\te)\approx \wh{\te}-\te.
\]
Now
\[
\sqrt n(\wh{\te_n}-\te)=\fc{\rc{\sqrt n}\ell'(\te)}{-\rc n\ell''(\te)}\to \fc{N(0,I(\te))}{I(\te)}\to N(0,1),
\]
the top in distribution, the bottom in probability. (The top uses CLT on $\sum (\ln f)_\te$; the bottom uses LoLN on $\sum -(\ln f)_{\te\te}$.)
\item
Show that $\sfc{I(\wh{\te_n})}{I(\te)}\xra P 1$.
\end{enumerate}
\end{proof}
\item Think of this as a chain rule.
\begin{thm}
If $\tau=g(\te)$ and $g'(\te)\ne 0$, then $\fc{\wh{\tau_n}-\tau}{\wh{\se}(\wh{\tau})}\to N(0,1)$ where $\wh{\tau_n}=g(\wh{\te_n}),\wh{\se}(\wh{\tau_n})=|g'(\wh{\tau})|\wh{\se}(\wh{\tau_n})$.
\end{thm}
Proof: just expand $g$ using $g'$.
\item (Equivariance) If $\tau=g(\te)$ is 1-to-1, then $\wh{\tau_n}=g(\wh{\te_n})$. Follow definitions!
\end{enumerate}

Write $x^n\lra y^n$ if $f(x^n;\te)=cf(y^n;\te)$ as functions of $\te$. $T$ is sufficient if $T(x^n)=T(y^n)\implies x^n\lra y^n$.
$T$ is minimally sufficient if it is also a function of every other sufficient statistic. Factorization gives that $f(x^n;\te)=g(t(x^n);\te)h(x^n)$.

\fixme{Some stuff on exponential families at end of \S9}.
\section{Hypothesis testing}

%Let $\cal B$ be the space of possible $B$ (ex. samples). We look for a region $\cal A\subeq \cal B'$ for which 
%\[
%\Pj(B|\te\in\Te)\le \al,
%\]
%the probability that we observe $B$ given $\te \in \Te$ is small. This way, if we observe 

We are given hypothesis $H_0:\te\in \Te_0$ and $H_1:\te\in \Te_1=\Te_0^c$ and want to know which is true. (For example, $H_0=\{\te_0\}$ and $H_1=\R\bs \{\te_0\}$.) $H_0$ is the \textbf{null hypothesis}; we reject or fail to reject it. In other words, we have a decision function $\de$ that given $X^n$ gives 0 or 1.
There are two kinds of errors:
\begin{enumerate}
\item reject ($\de=1$) when $H_0$ is true
\item fail to reject ($\de = 0$) when $H_1$ is true.
\end{enumerate}

\begin{df}
The \textbf{power} is 
\[
\be(\te) = \Pj(\de=1|\te),
\]
the probability of rejecting given $\te$. We want to maximize power for $\te\in H_1$ (i.e., minimize $1-\be(\te)$, which is the type 2 error).

The \textbf{size} is
\[
\sup_{\te\in H_0}\be(\te) = \sup_{\te\in H_0}\Pj(\de=1|\te).
\]
The is the maximum probability of a false rejection (type 1 error). For example, when $H_0$ is $\{\te_0\}$ this is just $\Pj(\de=1|\te_0)$.
We say a test has level $\al$ if it has size $\ge \al$.

Given rejection regions $R_\al$ of size $\al$, the \textbf{$p$-value} is $\inf\set{\al}{T(x^n)\in R_{\al}}=\sup_{\te\in \Te_0}(\Pj_{X^n}(T(X^n)\ge T(x^n)|\te))$ where $T$ is the statistic used in the test. 
\end{df}

\subsection{Basic tests}

\prbox{
\begin{exr}
\begin{enumerate}
\item
Give a hypothesis test for the normal distribution (z and t test).
\item
Give a hypothesis test for comparing 2 means of normal variables when 
\begin{enumerate}
\item
2 groups are independent
\item 
the elements are paired.
\end{enumerate}
Do the same for Bernoulli.
\item Compare the variances of 2 normal distributions.
\end{enumerate}
\item Give a test comparing 2 variances of normal distributions.
\end{exr}
}

Let $z_{\al}$ be the value of $z$ such that $\int_z^{\iy}N(x)\,dx=\al$ where $N$ is the standard normal.

For the examples below we take $H_0$ consisting of a single point, so the tests are double-tailed. For $H_0=\{\te\le \te_0\}$ and $\{\te\ge \te_0\}$ we use single-tailed tests.
\begin{enumerate}
\item
For the normal distribution:
\begin{enumerate}
\item
We know that $\lim_{n\to \iy}\fc{\wh{\mu_n}-\mu_0}{\wh{\se}}\to N(0,1)$. Thus we can use the $z$-test which of size $\al$:\footnote{It really doesn't matter for these approximate tests whether we use $\wh{\si}$ or $\wh{\se}=\sfc{n-1}n\wt{\si}$ because $\sqrt{n-1}\sim \sqrt{n}$. For the exact tests we have to make the distinction.}
\[
\sqrt{n}\af{\wh\mu-\mu_0}{\wh{\si}}
>z_{\al/2}.
\]
The power is approximately $1-\Phi(\fc{\mu_0-\mu}{\wh{\si}}+z_{\al/2})+\Phi(\fc{\mu_0-\mu}{\wh{\si}}-z_{\al/2})$. 
% or n-1?
\item
That is only an approximation, good when $n$ large. The real distribution is the $t$-distribution with $n-1$ degrees of freedom. 

Now we use:
\begin{lem}
Let $X_1,\ldots, X_n\sim N(0,1)$. Then 
\[
\sqrt{n-1}\fc{\wt{\mu}}{\wh\si} \sim t_{n-1}.
\]
\end{lem}
\begin{proof}
$\sqrt n\wh{\mu}=(\rc{\sqrt n}\ldots \rc{\sqrt n})x$. Let $A$ be an orthogonal matrix with first row $A_1=(\rc{\sqrt n}\ldots \rc{\sqrt n})$. Let $y=Ax$. We have $y^Ty=x^Tx$ so 
\[n\wh{\si}=x_1^2+\cdots +x_n^2-\ub{y_1^2}{n\ol{X}^2}=y_2^2+\cdots +y_n^2.\]
This is distributed as $\chi^2_{n-1}$ and independent of $y_1$. Thus the distribution is $\sqrt{n-1}\fc{N(0,1)}{\chi_{n-1}^2}=t_{n-1}$.
\end{proof}
In summary we have
\[
\sqrt{n-1}\fc{\wh\mu-\mu_0}{\wh{\si}} = 
\sqrt{n-1} \cdot 
\ub{\fc{\sqrt n(\wh{\mu}-\mu_0)}{\si}}{\sim N(0,1)}
/ \ub{\fc{\sqrt n \wh{\si}}{\si}}{\sim \chi_{n-1}^2} \sim t_{n-1}.
\]

The more accurate test is
\[
\sqrt{n-1}\af{\wh\mu-\mu_0}{\wh{\si}}>t_{n-1,\al/2}.
\]
\end{enumerate}
\item Let $\mu_x=\E_{i=1}^m X_i$ and similarly for $y$.
\begin{enumerate}
\item
First, the normal approximation. We have $\fc{\wh{\mu_x}-\mu_x}{\wh{\si_x}}\approx N(0,1)$ and similarly for $y$; clear denominators, add, and divide by standard deviation (noting variances add) to get
\[
\fc{\wh{\mu_x}+\wh{\mu_y}-\mu_x-\mu_y}{\sqrt{n\wh{\si_x}^2+m\wh{\si_y}^2}} \approx N(0,1);
\]
do the $z$-test.

For an exact test, we find
\[
\fc{\wh{\mu_x}+\wh{\mu_y}-\mu_x-\mu_y}{\sqrt{n\wh{\si_x}^2+m\wh{\si_y}^2}} = \fc{N(0,\fc{\si_x^2}{m})+N(0,\fc{\si_y^2}{n})}{\sqrt{\si_x^2\chi_{m-1}^2+\si_y^2\chi_{n-1}^2}}
\]
which simplifies only when $\si_x=\si_y$ (assumption of equal variances) to get
\[
\sfc{m+n}{mn}\fc{N(0,1)}{\sqrt{\chi_{m+n-2}^2}}.
\]
Thus test
\[
\ab{\sfc{mn(m+n-2)}{m+n}\fc{\wh{\mu_x}+\wh{\mu_y}}{\sqrt{n\wh{\si_x}^2+m\wh{\si_y}^2}}}>t_{m+n-2,\al}
\]
For unequal variances, use the Satterthwaite approximation (see 18.443 lecture 7).
\item
Use the $t$-test on the pairs $x_i-y_i$.
\end{enumerate}
For Bernoulli, let $\wh{\si}=\sfc{\wh p (1-\wh p)}{m}$ and use the normal approximation to the binomial to test.
\item
Suppose we have $m$, $n$ samples respectively. Now if both variances equal $\si^2$,
\[
\fc{\wh{\si_m}^2}{\wh{\si_n}^2}\sim \fc{\rc m\chi_{m-1}^2}{\rc n\chi_{n-1}^2}.
\]
So take
\[
\frac{m(n-1)}{n(m-1)}\fc{\wh{\si_m}^2}{\wh{\si_n}^2}
\]
as the $F$-test statistic.
\end{enumerate}
\subsection{$\chi^2$ goodness of fit}

%\prbox{
%\begin{exr}
%\item
%Give Pearson's $\chi^2$ test.
%\item
%Give 
%\end{exr}
%}

\begin{enumerate}
\item
How to test multinomial distributions? Use the following
\[
\sum_{j=1}^r \fc{(X_j-E_j)^2}{E_j}
\]
as a test statistic for $\chi_{r-1}^2$, where $E_j=np_j$ is the expected number in category $j$.
\begin{proof}
The covariance matrix $A$ for the variables $\pf{X_i-np_i}{\sqrt{np_i}}_i$ has diagonal $(1-p_i)_i$ and off-diagonal entries $-\sqrt{p_ip_j}$. Note that $(\sqrt{p_i})_i$ spans the kernel. Note $A-I$ has rank 1 so all other eigenvectors have value 1. Let $U$ diagonalize $A$; let $y=Ux$. Then 
\[
\sum_j \fc{(X_j-E_j)^2}{E_j} = x^Tx = y^Ty.
\]
Because $y$ has covariance matrix $\matt 000{I_{r-1}}$, this is $\chi_{r-1}^2$.
\end{proof}
\item
For goodness-of-fit of a sample to a distribution $f$, split the domain of $f$ into intervals (where $\int f=\rc{n}$, say), and run Pearson's test on them.

Careful: here $\wh{p_j}$ should be the MLE for the grouped distribution, the maximum for $\arg_\te\max \Pj(I_1|\te)^{v_1}\cdots$, rather than the MLE for the distribution, then grouped. (See 18.443 L12.)

The degrees of freedom should be $r-\dim(\Te)-1$.

\item
Give tests for independence and homogeneity.

These are the same!

Suppose there are $N_{ij}$ observations for $(i,j)$. We want $\max \prod p_i^{N_{i\bullet}}\prod_j q_j^{N_{\bullet j}}$. Take logs; the gradient should be in the span of $(1,\ldots,1,0,\ldots,0)$ and $(0,\ldots,0,1,\ldots,1)$ (Lagrange multipliers) so all the $N_{i\bullet}/p_i$ are the same, and similarly for $q$.  df$=ab-(a-1)-(b-1)-1=(a-1)(b-1)$.

\end{enumerate}

\subsection{Nonparametric testing}

For any distribution, let $F_n(x)$ be the estimated distribution function after getting $n$ samples. By LLN, for a particular $x$, $F_n(x)\to F(x)$: $\sqrt{n}(F_n(x)-F(x))\xra d N(0,F(x)(1-F(x)))$.

The basis of nonparametric tests using $F_n$ is the following: 
\begin{thm}
$\sup|F_n-F|$ does not depend on $F$. 
\end{thm}
\begin{proof}
Let $y=F_n(x)$ so that $x=F_n^{-1}(y)$; putting things in terms of $y$ makes this into the problem for the uniform distribution. (Warning: some finesse is required because $F_n$ can have jumps.)
\end{proof}

The Kolmogorov-Smirnov test uses
\[
\Pj(\sqrt n\sup_{x\in \R} |F_n(x)-F(x)|\le t)\to H(t)=1-2\sum_{i=1}^{\iy} (-1)^{i-1} e^{-2i^2t}.
\]
There are variations; see L14.

\subsection{Multiple testing}

If we do $N$ tests at level $\al$, we can expect $pN$ false rejections. We can test at level $\fc{\al}N$ but often this is too stringent. A better way in practice is to order $p$-values from smallest to largest. If the $i$th $p$-value falls below $\fc{i\al}{m}$ then reject all tests below that.

\section{Bayesian statistics}

%\chapter{Motivation}

%We'll see lots of analysis in action.
\section{Dirac delta function}

What is the derivative of the Dirac delta?

You may have seen the mysterious ``\textbf{Dirac delta function}" defined by 
\beq{eq:dist0-1}
\int_{-\iy}^{\iy} \de(x-x_0)f(x)\,dx=f(x_0), \qquad \de(x)=0,\,x\ne 0.
\eeq
This emerged from Fourier's classical treatise on heat. It was there implicitly in his work. Cauchy and Dirac noticed it. It is used in math, applied math, physics, engineering.
%does the job we're asking it

But there is no function in any sense of the word that does this job! It makes no mathematical sense!

%Let's look at~\eqref{eq:dist0-1} in an abstract sense. 
However, looking at~\eqref{eq:dist0-1} in an abstract sense, the ``process" which takes $\de(x-x_0)$ and the ``nice" function $f(x)$, and spits out $f(x_0)$ is well-defined.
%There's some $\de(x-x_0)$, given $f(x)$, some process happens and spits out $f(x_0)$. 

However, people don't just talk about the delta function, they also talk about its derivative! Trying to differentiate something that doesn't exist...? %We'll put on our applied maths hat on and try to define the derivative.

How can we define the derivative $\de'(x-x_0)$? A first attempt might be (assuming $f$ is nice)
\begin{align}
\nonumber
\int \de'(x-x_0)f(x)\,dx&=\lim_{h\to 0} \int 
\pf{\de(x-x_0+h)-\de(x-x_0)}{h}f(x)\,dx\\
\nonumber
&=\lim_{h\to 0} \fc{f(x_0-h)-f(x_0)}{h}\\
\nonumber
&=-f'(x_0)\\
\llabel{eq:dist0-2}
&=-\int \de(x-x_0)f'(x)\,dx.
\end{align}
The equality~\eqref{eq:dist0-2} suggests that we could have simply integrated by parts
\[
\int \de'(x-x_0)f(x)\,dx=-\int \de(x-x_0)f'(x)\,dx+\ub{\text{boundary term}}{0}.
\]
This function that doesn't exist seems to follow the usual rules of calculus! 
This suggests there is a way of interpreting all the integrals in a consistent way
We can make all this rigorous using the theory of distributions. %looks like the usual rules of integral calculus can be applied to $\de(x-x_0)$.

\section{Fourier transforms of polynomials}

The \textbf{Fourier transform} is defined by (abbreviate $\int:=\iiy$)
\[
\cal F:f\mapsto \wh f(x)=\int e^{-i\la x}f(x)\,dx.
\]
This makes sense if $f$ is absolutely integrable:
\[
\ab{
\int e^{-i\la x}f(x)
}\le \int |e^{-i\la x}f(x)|\,dx=\int |f(x)|\,dx<\iy.
\]
What if $f\nin L^1$, in particular, what if $|f|\not\to 0$ as $|x|\to \iy$? You may have seen
\beq{eq:dist0-3}
\de(\la)=\rc{2\pi}\int e^{-i\la x}\,dx.
\eeq
There is a way of interpreting this so that it makes sense. This seems to suggest that the Fourier transform of $\rc{2\pi}$ is equal to $\de(\la)$. What if $f(x)=x^n$? Can we take the Fourier Transform?
\bal
\int e^{-i\la x}x^n\,dx&=\int \pa{i\pd{}{\la}}^n e^{-i\la x}\,dx\\
&=\pa{i\pd{}{\la}}^n \int e^{-i \la x}\,dx\\
&=2\pi e^{-i \pi n/2}\de^{(n)}(\la).
\end{align*}
If we can make sense on the derivatives of $\de$, then we can define the Fourier transform of polynomials.

An alternative way of defining Fourier transform of $f(x)=x^n$ would be to use Parseval's Theorem, which states
\beq{eq:dist0-4}
\int f(x)\wh g(x)\,dx =\int \wh f(\la)g(\la)\,d\la
\eeq
for all ``nice" functions $f$ and $g$. We could define $\hat f(\la)$ where $f(x)=x^n$, as the function for which
\beq{eq:dist0-5}
\int x^n\hat g(x)\,dx=\int \hat f(\la)g(\la)\,d\la\text{ for all nice functions } g
\eeq
we could say that $\hat f(\la)$ is the Fourier transform of $f(x)=x^n$. Note $\hat g(x)$ decays quickly, so this makes sense. This can be done rigorously using the theory of distributions. %If we can find $\hat f(x)$

``Everything's easy when you know the answer." It's only perfectly natural when you've been shown its perfectly natural. To prove consistency is not quite so easy.

\section{Discontinuities to Differential Equations}
%we want to see things like that happen.

There are some important genuinely important physical scenarios in which we would like a solution to a PDE to have discontinuities. For example, in acoustics we want the pressure $p(x,t)$ to solve the wave equation
\beq{eq:dist0-6}
p_{xx}-p_{tt}=0,
\eeq
but for $p$ to jump either side of a shock wave.
%propagate out?
%pavilion g

Is there any meaningful way to say that the function 
\[
u(x,y)=\al(x-y)+\be(x+y)
\]
is a solution to the wave equation $u_{xx}-u_{yy}=0$ if $\al,\be \nin C^2$? Assume $\al,\be\in C^2$ and $u_{xx}-u_{yy}=0$ so that for any nice function $f(x,y)$ (say $f=0$ when $x^2+y^2$ is sufficiently large),
\bal
0&=\int\int f(x,y)\pa{\pd{{}^2u}{x^2}-\pd{{}^2u}{y^2}}\,dx\,dy\\
&=\int\int u(x,y) \pa{\pd{{}^2}{x^2}-\pd{{}^2f}{y^2}}\,dx\,dy
\text{integration by parts twice}.
\end{align*}
If we can find $u(x,y)$ such that
\beq{eq:dist0-7}
\int\int u(x,y) \pa{\pd{{}^2}{x^2}-\pd{{}^2f}{y^2}}\,dx\,dy=0
\text{ for all nice functions }f,
\eeq
we say that $u=u(x,y)$ is a \textbf{weak solution} to the PDE $u_{xx}-u_{yy}=0$. We can use distribution theory to study weak solutions to PDE's.
%3 reasons want formalize

\section{Summary}

In each motivating example we introduced a family of ``nice" functions that allowed us to extend classical definitions to a wider class of problems. This is the underlying idea in distribution theory. Given a vector space $V$ of ``nice" functions we define the distributions on $V$ to be the topological dual $V^*$, which consists of all the continuous linear forms $V\to \C$. 

For example, if $V=C(\R)$ then we can define Dirac delta by the linear form\footnote{You may be more familiar with the notation $\an{x,y}=x\cdot y$ for finite-dimensional vector spaces.}
\beq{eq:dist0-8}
\an{\de_{x_0},f}=f(x_0).
\eeq
%linear map.
In general, any $u\in V^*$ is linear, so $\an{u, \al f+\be g}=\al\an{u,f}+\be \an{u,g}$ for arbitrary constants $\al,\be$ and $f,g\in V$. We need functional analysis because we require continuity. (The algebraic dual is too big to be interesting. Hence we supplement $V$ with a topology, i.e. a notion of convergence $f_n\to V$ in $V$. This is the motion of $w^*$-convergence, $\an{u,f_n}\to \an{u,f}$.

\blu{Lecture 2}
\chapter{Distributions}

Recap:
\begin{itemize}
\item
Delta function doesn't make sense.
\item
Way to define distributions is to first define a ``nice" space of functions $V$ (having all the properties we want it to have) and define distributions as continuous linear maps from $V$ to $\C$. 
\end{itemize}
%V has all properties we want it to have.

We'll always work with continuous functions, so we can define continuity of functions $V\to \C$ very explicitly.

\section{Notation and preliminaries}

An element of $\R^n$ will be written $x,y,z,\ldots$ so that 
\[
x=(x_1,x_2,\ldots, x_n)
\]
and we will use $dx=dx_1\,dx_2\,\cdots \,dx_n$ to denote the volume element in $\R^n$. Capitals $X,Y,Z$ will always denote open subsets of $\R^n$ and $K$ will always denote a compact (closed and bounded) subset of $\R^n$. Integrals over all $\R^n$ or over $X\subeq \R^n$ will be denoted by $\int [\cdot]\,dx$ and $\int_X[\cdot]\,dx$, respectively.
We will use multi-index notation $\al,\be,\ga$ (Greek letters) will denote multi-indices $\al=(\al_1,\ldots, \al_n)\in \Z_+^n=\{0,1,2,3,\ldots\}^{\times n}$. Multi-index notation reads as follows.
\bal
\pl^{\al} &\equiv\pa{\pdd{x_1}}^{\al_1}\cdots \pa{\pdd{x_n}}^{\al_n},&x^{\al}&\equiv x_1^{\al_1}\cdots x_n^{\al_n}\\
\al!&\equiv \al_1!\cdots \al_n!&|\al|&\equiv \al_1+\cdots +\al_n.
\end{align*}
%functions for which it's switched on.
We will often write $\pl^{\al}_x$ to make it clear what we're differentiating with respect to. We will also use $D=-i\pl$ when we do Fourier analysis. Define the \textbf{support} of a function $f$ by 
\[
\Supp(f)=\ol{\set{x\in \R^n}{f(x)\ne 0}}.
\]
We will often want to take limits inside integrals. To do this we refer to the dominated convergence theorem: (See for instance~\url{https://dl.dropboxusercontent.com/u/27883775/math\%20notes/18.125.pdf}, Theorem 15.1.)
\begin{thm}[Dominated convergence theorem]\llabel{thm:dct}
Given a sequence of absolutely integrable functions $\{f_m\}_{m\ge 1}$ such that $f_m(x)\to f(x)$ for each $x$ and $|f_m(x)|\le g(x)$, where $g$ is absolutely integrable, then 
\[
\lim_{m\to \iy}\int_Xf_m(x)=\int_X\ba{\lim_{m\to \iy} f_m(x)}\,dx=\int_X f(x)\,dx.
\]
\end{thm}
If $f$ is absolutely integrable on $X$, i.e.,
\[
\int_X |f|\,dx<\iy,
\]
we say that $f\in L^1(X)$.

\section{Test functions and distributions}
We need to define our first vector space of test functions.
\begin{df}
The space $D(X)$ consists of all the smooth functions from $X$ to $\C$ that have compact support. We say that a sequence $\{\ph_m\}_{m\ge1}$ tends to zero in $D(X)$ if there exists some compact set $K\subeq X$ such that $\Supp(\ph_m)\subeq K$ and $\pl^{\al}\ph_m\to 0$ uniformly for each multi-index $\al$.

The space $D(X)$ is often written $C^{\iy}_c(X)$.
\end{df}
(For convergence, the function is not allowed to have its mass moving away to infinity.)
%integrate by parts, evaluated on the boundary, derivatives shift to the other side.

Since the functions in $D(X)$ vanish at the boundary of $X$, we have
\[
\int_X\ph \pl^{\al}\psi\,dx=(-1)^{\al}\int_X\psi\pl^{\al}\ph\,dx,
\qquad \ph,\psi\in D(X)
\]
by integration by parts $|\al|$ times. We have Taylor's Theorem to any order
\beq{eq:taylor}
\ph(x+h)=\sum_{|\al|\subeq N}\fc{h^{\al}}{\al!} \pl^{\al}\ph(x)+R_N(x,h)
\eeq
where the remainder is $o(|h|^N)$ uniformly in $x$. %treat it as  ncice polynomial plus remainder.

%we have space of tests. distributions will be defined as linear maps. We want them to be continuous. $u(\ph)\equiv \an{u,\ph}\in \C$. If $\ph_m\to 0$ in $D(X)$, $\an{u,\ph_m}\to 0$ in $\C$.
\begin{df}\llabel{df:distribution}
A linear map $u:D(X)\to \C$ is a \textbf{distribution} (on $X$) if
for each compact $K\subeq X$, there exist $C,N$ such that 
\beq{eq:dist1-1}
|\an{u,\ph}|\le C\sum_{|\al|\le N} \sup|\pl^{\al}\ph|
\eeq
%vs sequential continuity. 
for each $\ph\in D(X)$ with $\Supp(\ph)$ with $\Supp(\ph)\subeq K$. The space of all such maps is denoted $D'(X)$. If we can use the same $N$ for all $\ph\in D(X)$, we call the least such $N$ the order of $u$, denoted $\ord(u)$.
\end{df}
%in the back of mind have this example, go back to familiar example.
\begin{rem}
Thinking of $D(X)$ as a locally convex space (in fact, Fr\'echet space) with seminorms defined by $\sup|\pl^{\al}\ph|$, we have $D'(X)=D(X)^*$. See Example 2.2.4 in Functional Analysis\footnote{\url{https://dl.dropboxusercontent.com/u/27883775/math\%20notes/part_iii_functional.pdf}}.
The example there is actually a larger space ($\cal E(X)$ of the next chapter), but it contains our $D(X)$.
\end{rem}

\begin{ex}
\begin{enumerate}
\item
We check that the Dirac delta $\de_x$ is a distribution. $\de_x$ is defined by
\[
\an{\de_x,\ph}=\ph(x)\qquad\ba{\int \de(x-y)\ph(y)\,dy=\ph(x)}.
\]
We want to check if~\eqref{eq:dist1-1} holds:
\[
\ab{\an{\de_x,\ph}}=|\ph(x)|\le \sup|\ph|
\] 
so~\eqref{eq:dist1-1} holds with $C=1$, $N=0$,  no matter what $\ph$ we choose. So $\de_x$ is a distribution of order 0.
\item
Here is a more useful example. Consider the linear map $T$ on $D(X)$ defined by 
\[
\an{T,\ph}:=\sum_{|\al|\le M} \int_X f_{\al}\pl^{\al} \ph\,dx,
\]
$f_{\al}\in C(X)$. Now for $\ph\in D(X)$, with $\Supp(\ph)\subeq K$. By definition of $T$, 
\bal
|\an{T,\ph}|&=\ab{
\sum_{|\al|\le M}\int_K f_{\al} \pl^{\al}\ph\,dx
}\\
&\le \sum_{|\al|\le M} \int_K |f_{\al}||\pl^{\al} \ph|\,dx\\
&\le \sum_{|\al|\le M} \sup_{|\al|\le M}|\pl^{\al}\ph| \int_K |f_{\al}|\,dx\\
&\le \pa{\max_{\al}\int_K |f_{\al}|\,dx}\sum_{|\al|\le M} \sup |\pl^{\al} \ph|
\end{align*}
Note that the test functions have compact support, so it doesn't matter if $f_{\al}$ blows up at the boundary. So we have estimate~\eqref{eq:dist1-1} with 
\[
C=\max_{|\al|\le M}\int_K |f_{\al}|\,dx,\qquad N=M.
\]
Note that $C=C_K$. We could have done this only assuming that $\{f_{\al}\}$ were locally integrable on $X$ (integrable on every compact subset of $X$), written $f_{\al}\in L_{\text{loc}}^1(X)$.

Note here the constant $C$ depends on the support test function. $N$ can also depend on it.
%$N$ can also depend on the support of the test function.
\end{enumerate}
\end{ex}
\begin{lem}\llabel{lem:dist1-1}
A linear map $u:D(X)\to \C$ belongs to $D'(X)$ if $\an{u,\ph_m}\to 0$ for every sequence $\{\ph_m\}_{m\ge 1}$ in $D(X)$ that tends to 0.
\end{lem}
\blu{Lecture 3 (27 Jan)} 
\begin{rem}
That Definition~\ref{df:distribution} and Lemma~\ref{lem:dist1-1} are equivalent conditions for continuity is a special case of Lemma 2.2.5 in the functional analysis notes.
\end{rem}
\begin{proof}
\begin{itemize}
\item
$\implies$: If $u\in D(X)$ and $\ph_m\to 0$ in $D(X)$ then
\[
|\an{u,\ph_m}|\le \sum_{|\al|\le m} \sup|\pl^{\al} \ph_m|\to 0.
\]
\item
$\Leftarrow$: Assume not, so there exists a compact set $K\subeq X$ such that estimate~\eqref{eq:dist1-1} does not hold for any $C,N$. In particular, it doesn't hold for $C=N=m$. So there exist $\phi_m\in D(X)$ such that $|\an{u,\phi_m}|>m\sum_{|\al|\le m}\sup |\pl^{\al} \phi_m|$. WLOG, we can assume that $\an{u,\phi_m}=1$, by setting $\wt{\phi_m}=\fc{\phi_m}{\an{u,\phi_m}}$. This implies
\bal
\implies \sum_{|\al|\le m} \sup|\pl^{\al}\phi_m|&<\rc m\\
\implies \sup|\pl^{\al}\phi_m|&<\rc m,&|\al|\le m\\
\implies \phi_m\to 0& \text{ in }D(X).
\end{align*}
But this is a contradiction.
\end{itemize}
\end{proof}
\section{Limits in $D'(X)$}
Often we have  sequence of distributions $\{u_m\}_{m\ge 1}$. If there exist some $u\in D'(X)$ such that $\an{u_m,\ph}\to \an{u,\ph}$ for all $\ph\in D(X)$, then we say that $u_m\to u$ in $D'(X)$.

Limits in $D'(X)$ often look strange. 
\begin{ex}
For instance, define the distribution $u_m\in D'(\R)$ by the locally integrable function
\[
u_m(X):=\sin (mx).
\]
Then $u_m\to 0$ in $D'(\R)$. 

%compact support, ibp no boundary terms.
Proof: We have using \blu{integration by parts}
\bal
\an{u_m,\ph}&=\int \sin(mx)\ph(x)\,dx\\
&=\rc{m} \int \cos(mx)\ph'(x)\,dx
\end{align*}
Hence $u_m\to 0$ in $D'(\R)$. 
\end{ex}
%think about Fourier expansion of $\ph$
\begin{thm}[Closure under pointwise convergence]\llabel{thm:dist1-1}
If $u_m\in D'(X)$ is such that $\lim_{m\to \iy}\an{u_m,\ph}$ exists for every $\ph\in D(X)$, then the linear map
\[
\an{u,\ph}:=\lim_{m\to \iy}\an{u_m,\ph}
\]
is an element of $D'(X)$. 
\end{thm}
It's obvious that the LHS will satisfy the estimate or the definition. Use the principle of uniform boundedness (See the Banach-Steinhaus Theorem, 4.2.7 in FA notes).

\section{Basic operations}

\subsection{Differentiation and multiplication by smooth functions} 
If $u\in C^{\iy}(X)$, then $\pl^{\al}u$ defines an element of $D'(X)$ for every multi-index $\al$ by
\bal
\an{\pl^{\al}u,\ph}&=\int_{X}\pl^{\al}u\ph\,dx\\
&=(-1)^{|\al|} \int_X u\pl^{\al} \ph\,dx&\text{integration by parts}\\
&=\an{u,(-1)^{|\al|}\pl^{\al}\ph}.
\end{align*}
%add in f
%Makes sense for any distribution, so it's a good definition.
The RHS is well-defined for any $u\in D'(X)$. We arrive at the following.
\begin{df}\llabel{df:dist1-3}
For $f\in C^{\iy}(X)$, $u\in D'(X)$ and any multi-index $\al$ we define $\pl^{\al}(fu)$ by 
\[
\an{\pl^{\al}(fu),\ph}:=\an{u,(-1)^{|\al|} f\pl^{\al}\ph}.
\]
We call $\pl^{\al}u$ the \textbf{distributional derivatives} of $u$.
\end{df}
\begin{ex}
Take the Dirac delta $\de_x$. Then $\pl^{\al}\de_x$ is defined by
\[
\an{\pl^{\al}\de_x,\ph}:=\an{\de_x,(-1)^{|\al|} \pl^{\al}\ph} = (-1)^{|\al|} \pl^{\al}\ph(x).
\]

Consider the Heaviside function
\[
H(x)=\begin{cases}
1,&x>0\\
0,&x\le 0.
\end{cases}
\]
Then this defines an element of $D'(\R)$. We compute $H'$:
\[
\an{H',\ph}:=\an{H, -\ph'}=\int_0^{\iy} -\ph'(x)\,dx = -\ph|^{\iy}_0=\ph(0)=\an{\de_0,\ph}.
\]
Hence $H'=\de_0$. In general if $\an{u,\ph}=\an{v,\ph}$ for all $\ph\in D(X)$ then we say $u=v$ in $D'(X)$. 
%gradient infinite.
\end{ex}
Now we ask: how does the calculus for normal functions carries over to calculus for distributions?
\begin{lem}
If $u'=0$ in $D'(\R)$ then $u$ is a constant.
\end{lem}
\begin{proof}
Note that $u'=0$ means 
\[0=\an{u',\psi}=-\an{u,\psi'}.\]
%every function that is a total derive.
%{\it Note that }

{\it \blu{Idea: we'd like to say that given $\ph$, we can write $\ph=\ddd x\int_{-\iy}^x \ph(y)\,dy=\psi'$, and use the above to conclude $0=\an{u,\ph}$, so $u$ is constant. The problem is that when we integrate a test function, we don't necessarily get a test function. We need to adjust our function so that the integral is 0 for large $x$.}}

Fix $\te\in D(\R)$ with $\an{1,\te}=\int \te\,dx=1$. For arbitrary $\ph\in D(\R)$ write 
\bal
\ph&=(\ph-\an{1,\ph}\te)+\an{1,\ph}\te\\
&\equiv \ph_A+\ph_B
\end{align*}
Then 
\[
\an{1,\ph_A} =\an{1,\ph}-\an{1,\ph}\an{1,\te}=0.
\]
This is helpful because 
\[
\psi_A(x)=\int_{-\iy}^x \ph_A(y)\,dy\in D(\R).
\]
%
We have $\ph_A=\psi_A'$. So 
\bal
\an{u,\ph}&=\an{u,\ph_A}+\an{u,\ph_B}\\
&=\an{u,\psi_A'}+\an{1,\ph}\ub{\an{u,\te}}b=0+c\an{1,\ph} =\an{c,\ph}.
\end{align*}
This implies that $u$ is a constant.
\end{proof}
\subsection{Reflection and translation}
\begin{df}
For $\ph\in D(\R^n)$ then we can define its \textbf{translation} by $h\in \R^n$ by 
\[
(\tau_h\ph)(x):=\ph(x-h)
\]
and \textbf{reflection}
\[
\check \ph(x)=\ph(-x).
\]
\end{df}
(Motivation: $\an{u,\ph}=\int u\ph\,dx$ gives
$\an{\tau_h u,\ph}=\int u(x-h)\ph(x)\,dx=\int u(x)\ph(x+h)\,dx=\an{u,\tau_{-h}\ph}$.) By duality, the definitions of these operations on $u\in D'(\R^n)$,
\bal
\an{\tau_hu,\ph}&:= \an{u,\tau_{-h}\ph}\\
\an{\check u,\ph}&:=\an{u,\check{\ph}}.
\end{align*} 
\begin{lem}\llabel{lem:dist1-3}
For $u\in D'(\R^n)$ define
\[
v_h=\fc{\tau_{-h} u-u}{|h|}.
\]
Then $v_h\to n\cdot \pl u$, where $\lim_{h\to 0} \fc{h}{|h|}=n\in S^{n-1}$. 
\end{lem}
\begin{proof}
We have
\bal
\an{v_h,\ph}&=\an{\fc{\tau_{-h}u-u}{|h|},\ph}\\
&=\an{u,\fc{\tau_h\ph-\ph}{|h|}}.
\end{align*}
By Taylor's Theorem,
\bal
\tau_h \ph(x)-\ph(x)&=-h\pl \ph(x)+\ub{R(x,h)}{o(|h|)\text{ in }D(\R^n)}\\
\an{u,\fc{\tau_h\ph-\ph}{|h|}}&=\an{u,-\fc{h}{|h|}\pl \ph}+\ub{\an{u,\fc{(R(\cdot ,h))}{h}}}{\to 0\text{ as $|h|\to 0$}}\\
&=n\cdot \an{\pl u, \ph}.
\qedhere
\end{align*}
\end{proof}
This shows the distributional derivative coincides with the normal notion of derivative as difference quotient.

\blu{Lecture 4 (29 Jan)}

\subsection{Convolution between $D(\R^n)$ and $D'(\R^n)$}
If we combine the operations of reflection and translation, we get 
\[
(\tau_x\check{\ph})(y)=\check{\ph} (y-x)=\ph(x-y).
\]
If $u\in D(\R^n)$, we define the \textbf{convolution} of $u$ and $\ph\in D(\R^n)$ with
\[
(u*\ph)(x):=\int u(x-y)\ph(y)\,dy=\int \ph(x-y)u(y)\,dy=(\ph*u)(x).
\]
Using $\tau_h$ and $\check{\bullet}$, we can write 
\[
u*\ph(x)=\an{u,\tau_x \check{\ph}}.
\]
This is well defined for all $u\in D'(\R^n)$.
\begin{df}\llabel{df:dist1-5}
For $u\in D'(\R^n)$ and $\ph\in D(\R^n)$, define their convolution by 
\[u*\ph(x)=\an{u,\tau_x\check{\ph}}.\]
\end{df}
It is clear that $u*\ph(x)$ is just some function of $x\in \R^n$. It is actually smooth.
\begin{lem}\llabel{lem:dist1-4}
Define $\Phi_x(y)=\phi(x,y)$ where $\phi\in C^{\iy}(\R^n\times \R^n)$ and $\phi(\cdot, y)=0$ for $y$ outside some compact $K\subeq \R^n$. Then for $u\in D'(\R^n)$, 
\[
\pl_x^{\al} \an{u,\Phi_x}=\an{u, \pl^{\al}_x \Phi_x}.
\]
\end{lem}
We can take the derivatives inside the bracket.
\begin{proof}
By Taylor's theorem,  
%left with is linear map on $H$.
%use defn of $D'$
\[
\Phi_{x+h}(y)-\Phi_x(y)=\sum_i h_i \pd{\phi}{x_i} (x,y)+R_x(y,h)
\]
It is not difficult to show that $R_x(y,h)=o(|h|)$ in $D(\R^n)$ for each $x\in \R^n$. Hence
\[
\an{u,\Phi_{x+h}}-\an{u,\Phi_x}=\sum_i h_i \an{u,\pl{\Phi_x}{x_i}}+\an{u,R_x(\cdot, h)}.
\]
%sequence of test functions tend to 0, then continuity means $u$ acting on that sequence tends to 0.
Since $R_x(\cdot, h)=o(|h|)$ in $D(\R^n)$, dividing by $|h|$ and taking $h\to 0$ gives ($u=\fc{h}{|h|}$)
\[
n\cdot \an{u, \Phi_x}=n\cdot \an{u,\pl_x\Phi}.
\]
So the result follows.
\end{proof}
\begin{cor}\llabel{cor:dist1-1}
If $u\in D'(\R^n)$ and $\ph\in D(\R^n)$ then $u*\ph$ is smooth and $\pl^{\al}(u*\ph)=u*\pl^{\al}\ph$. 
\end{cor}
\begin{proof}
By Lemma~\ref{lem:dist1-4}, $\pl^{\al}(u*\ph)=\pl^{\al}_x\an{u,\tau_x \check{\ph}}=u*\pl^{\al}\ph$.
\end{proof}
%dist are normal functions, just take a derivative at the end.
\subsection{Density of $D(\R^n)$ in $D'(\R^n)$}
We have just seen the following.\\

\cpbox{
If $u\in D'(\R^n)$ and $\ph\in D(\R^n)$ then $u*\ph$ is smooth.}
\vskip0.15in
This is extremely useful. No matter how wild $u$ is, $u*\ph$ is nice. For this reason, $u*\ph$ is often called a  \textbf{regularisation} of the distribution $u$.
%wild creatures. give me any wild distribution, convolute with anything, get a smooth function. This is what we'll use to prove the density result.
We will use this fact to prove $D(\R^n)$ is dense in $D'(\R^n)$, i.e., for each $u\in D^1(\R^n)$ there exists a sequence of test functions $\{\ph_m\}_{m\ge 1}$ in $D(\R^n)$ such that $\ph_m\to u$ in $D'(\R^n)$, i.e., $\an{\ph_m,\te}\to \an{u,\te}$ for all $\te\in D(\R^n)$.

\blu{How to apply this: Suppose we have a problem about a distribution we don't know anything about. We know for some sequence of $\phi_m$, $\ph_m=u*\phi_m\to u$. We replace $u$ with $\ph_m$, do manipulations there, run the argument and take a limit.}


We need a technical lemma.
\begin{lem}\llabel{lem:dist1-5}
For $\ph,\psi\in D(\R^n)$ and $u\in D'(\R^n)$ we have
\[
(u*\ph)*\psi=u*(\ph*\psi).
\]
%insert dummy var
\end{lem}
\begin{proof}
The LHS is
\bal
(u*\ph)*\psi(x)&=\int (u*\ph)(x-y)\psi(y)\,dy\\
&=\int \an{u,\tau_{x-y} \wh \ph}\psi(y)\,dy\\
&=\int\an{u(z),\ph(x-y-z)\psi(y)}\,dy&\text{$z$ is ``dummy" variable}\\
&=\lim_{h\to 0} \sum_{m\in \Z^n}\an{u(z), \ph(x-z-hm)\psi(hm)h^n}&\text{Riemann sum}\\
&=\lim_{h\to 0}\an{u(z), \sum_{m\in\Z^n}\ph(x-z-hm)\ph(hm)h^n}
\end{align*}
(We want to take the $\int$ inside the $\an{\cdot}$, and we do this by turning it into a Riemann sum. Note the sum only has finitely many terms for each $m$, so this is legal.)
%sum eventually dead, only finitely many terms.
It is not hard to show that 
\[
\sum_{m\in \Z^n}\ph(x-hm)\psi(m)h^n\to \ph*\psi(x)\text{ in }D(\R^n)\text{ as }h\to 0.
\]
Continuing the above calculation,
\bal
&=\an{u(z),\ph*\psi(x-z)}\\
&=\an{u,\tau_x (\ph*\psi)\check{\,}}\\
&=u*(\ph*\psi)(x).
\end{align*}
%compact support, LHS has compact support, by fixed compact set, can differentiate
\end{proof}
\begin{thm}\llabel{thm:dist1-2}
$D(\R^n)$ is dense in $D'(\R^n)$. 
\end{thm}
We would like to say
\[
u(x)=\int\de(x-y)u(y)\,dy\stackrel?= \lim_{m\to \iy} \int \de_m(x-y)u(y)\,dy
\]
The $\de_m(x)$ are such that $\int \de_m(x)\,dx=1$ and get squashed closer and closer to $\de$. %(``a family of good kernels").
\begin{proof}
Fix $\psi\in D(\R^n)$ with $\int \psi\,dx=1$ and set
\[
\phi_m(x)=m^n \psi(mx)\qquad \pa{\int \phi_m\,dx=\int \psi\,dx=1}.
\]
%convolution of look-like de with original
Also introduce the bump function $\chi\in D(\R^n)$ with $\chi=1$ on $|x|<1$ and $\chi=0$ on $|x|>1$. Now set $\chi_m\pf xm$ and
\[
\ph_m=\pur{\chi_m(x)}(u*\phi_m)(x).
\]
%why can't $\ph$ of this guy?
(The purpose of $\pur{\chi_m(x)}$ is to make $\ph_m$ have compact support.)
Choose $\an{\ph_m,\te}$ for $\te\in D(\R^n)$ arbitrary, giving
\bal
\an{\ph_m,\te}&=\an{u*\phi_m, \chi_m\te}\\
&=(u*\phi_m)*(\chi_m\te)\check{\,}(0)\\
%&=u*\check{\ph}(0)\\
&=u*(\phi_m*(\chi_m \te)\check{\,})(0)\\
\phi_m*(\chi_m\te)\check{\,}(x)&=\int\phi_m (x-y)\chi_n(-y)\te(-y)\,dy\\
&=\int m^n \phi(m(x-y))\chi\pf{-y}{m}\te(-y)\,dy& y'=m(x-y)\implies y=x-\fc{y'}m\\
%Make the substitution 
&=\int\phi(y)\chi\pa{\fc y{m^2}-\fc xm}\te\pa{\fc ym -x}\,dy\\
&=\te(-x)+\ub{\int \phi(y)\chi\pa{\fc y{m^2}-\fc xm}\ba{
\te\pa{\fc ym-x}-\te(-x)
}\,dy}{=:R_m(-x)}\\
&=(\check{\te}+\check{R_m})(x).
\end{align*}
We can show $R_m\to 0$ in $D(\R^n)$. So \[\an{\ph_m,\te}=\an{u,\te}+\ub{\an{u,R_m}}{\to 0\text{ as }m\to \iy}.\] Hence $\an{\ph_m,\te}\to \an{u,\te},\te\in D(\R^n)$, giving $\ph_m\to u$ in $D'(\R^n)$ i.e. $D(\R^n)$ dense in $D'(\R^n)$. 
\end{proof}

 
%\bibliographystyle{plain}
%\bibliography{refs}
\end{document}