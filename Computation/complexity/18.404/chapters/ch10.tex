\lecture{Thu. 10/11/12}

Midterm Thu. 10/25 in walker (up through next week's material. Everything on computability theory but not complexity theory.)

Homework due next week.

Handout: sample problems for midterm

Last time we talked about
\begin{itemize}
\item
reducibility
\item mapping reducibility
\end{itemize}
Today we will talk about 
\begin{itemize}
\item
Post Correspondence Problem
\item
LBA's
\item 
Computation history method
\end{itemize}
We have been proving undecidability. First we proved $A_{TM}$ is undecidable by {\it diagonalization}. Next, by {\it reducing} ATM to another problem, we show that if the other problem were decidable so is $A_{TM}$; hence the other problem must also be undecidable.

Today we'll look at a fancier undecidable problem. It is a prototype for undecidable problems that are not superficially related to computability theory. All proofs for these problems use the method we'll introduce today, the {\it computation history method}.

Ours is a toy problem, with no independent interest. But it is nice to illustrate the method, and it is relatively clean. 

Even the solution to Hilbert's tenth problem uses the computation history method (though there are many complications involved).
\subsection{Post Correspondence Problem}
Given a finite collection of dominoes
\[
P=\bc{
\coltwo{u_1}{v_1},\coltwo{u_2}{v_2},\cdots,\coltwo{u_k}{v_k}
}
\]
a \textbf{match} is a sequence of dominoes from $P$ (repetitions allowed) where 
\[
u_{i_1}\cdots u_{i_{\ell}}=v_{i_1}\cdots v_{i_{\ell}}
\]
The question is: Is there a match in $P$?

For example, if our collection of dominoes is
\[
P=\bc{
\coltwo{aa}{aba},\coltwo{ab}{aba},\coltwo{ba}{aa},\coltwo{abab}{b}
}
\]
then we do have a match because 
\[
\left|
\begin{array}{cccccccccccc}
a&b\mid&a& a\mid&b&a\mid&a&a\mid&a&b&a&b\\
a&b&a\mid&a&b&a\mid&a&a\mid&a&b&a\mid&b.
\end{array}
\right|
\]

Formally, define
\[
PCP=\set{\an{P}}{P\text{ has a match}}.
\]
We will show PCP is undecidable. (Note it is Turing recognizable because for a given arrangement it's easy to see if it's a match; just enumerate all possible arrangements. If we find a match, accept.)

This is an old problem, the first shown undecidable by Post in 1950's.\footnote{Don't confuse the Post Correspondence Problem with probabilistic checkable proofs, also abbreviated PCP.}

Let's modify the PCP so that the match has to start with the starting domino (to see how to fix this, see the book).

%Given $P_{M,w}$ let the starting domino be
\begin{thm}\llabel{PCP}
PCP is undecidable.
\end{thm}
The proof has two ideas. Each takes some time to introduce. Instead of doing them at once (they intertwine), we'll defer the proof and prove a different theorem that uses only one of the ideas. Then we prove this theorem and use both ideas.

%Postpone the proof (put it on the stack). 
To introduce the first idea we'll go back to a problem in computation. 
\subsection{Computation Histories and Linearly Bounded Automata}
\begin{df}
A \textbf{linearly bounded automaton} (LBA) is a modified Turing machine, where the tape is only as long as the input string.\footnote{The LBA doesn't have a limited tape. (Then it would be finite automaton.) The tape is allowed to be enough to grow just enough to fit the input, but it can't grow any further. }
\end{df}
The head doesn't move off the left or right hand sides. The machine limited in how much memory it has: if you have an amount of tape, the amount of memory is linearly  bounded in terms of the size of the input (you might have a constant better memory because you're allowed a larger tape alphabet, but you can't for instance have $n^2$ memory). 
Now let the acceptance and empty problems be
\begin{align*}
A_{LBA}&=\set{\an{M,w}}{\text{LBA $M$ accepts $w$}}\\
E_{LBA}&=\set{\an{M}}{\text{LBA $M$ and }L(M)=\phi}
\end{align*}
Even though $A_{TM}$ was undecidable, $A_{LBA}$ is decidable.
%can't use tape outside input.
\begin{thm}\llabel{ALBA}
$A_{LBA}$ is decidable.
\end{thm}
This is a dramatic change in what we can do computationally!

The key difference that makes $A_{LBA}$ decidable is that linearly bounded automata have finitely many configurations (see below).

As a side remark, $A_{LBA}$ is not decidable by LBA's. In the theorem, by decidable we mean it's decidable by ordinary Turing machines. It is decidable but only by using a lot of memory.
\begin{proof}
Define a configuration to be a total snapshot of a machine at a given time:
\[
(q,t,p)
\]
where $q$ is the state, $t$ is the tape contents, and $p$ is the head position.
 For an input size, the number of configurations of a LBA %(combination of states, tape, and head)
is finite.

If we run the LBA for certain amount of time $T$, then it has to repeat a configuration.  If it has halted by time $T$, then we know the answer. If the machine hasn't halted, it's in a loop and will run forever. The reject.

(We don't even have to remember configurations.)

%Becase 
%configurations

For a string $w$ of length $n$, the number of configurations of length $n$ is 
\[
|Q|\cdot |\Ga|^{n}\cdot n.
\]

We now write down the TM that decides $A_{LBA}$.

``On input $\an{M,w}$,
%if M runs very long time, then start to get worried.
\begin{enumerate}
\item
Compute $|Q|\cdot |\Ga|^n \cdot n$.
\item
Run $M$ on $w$ for that many steps.
\item
\ul{Accept} if accepted.\\
\ul{Reject} if not yet accepted after that many steps.
\end{enumerate}
\end{proof}
Note that to write the number $|Q||\Ga|^nn$ down requires on the order of $n\ln n$ length tape, so intuitively $A_{LBA}$ is not  decidable by LBA's. The same diagonalization method can prove that $A_{LBA}$ is not decidable by LBA's. In general, we can't have a class of automata which decide whether automatons of that class accept.

In contrast with $A_{LBA}$, $E_{LBA}$ is still undecidable.
\begin{thm}\llabel{ELBA}
$E_{LBA}$ is undecidable.
\end{thm}
We prove this in a totally different way. Before, to prove $E_{TM}$ is undecidable (Theorem~\ref{thm:etm}), we showed $A_{TM}$ reduces to $E_{TM}$. We also have $A_{LBA}$ reduces to $E_{LBA}$, but doesn't tell us anything because $A_{LBA}$ is decidable!

Instead we reduce $A_{TM}$ to $E_{LBA}$. This is not obvious! Assume $R$ decides $E_{LBA}$. We'll construct $S$ deciding $A_{TM}$. (This is our standard reduction framework.) This is hard because we can't feed Turing machines into $E_{LBA}$: it only accepts LBA's as input, not general Turing machines.

We use the idea of computation history. 
\begin{df}Define an \textbf{accepting computation history} of a TM $T$ on input $w$ to be 
\[
C_1,C_2,\ldots, C_{\text{accept}}
\]
where $C_1$ is the start configuration, each $C_i$ leads to $C_{i+1}$, and $C_{\text{accept}}$ is in an accepting state.
\footnote{The computation history stores a record of all motions the machine goes through, just as a debugger stores a snapshot of what all the registers contain at each moment.}
\end{df}
If a Turing machine does not accept its input, it does not have an accepting computation history. 
An accepting configuration history exists iff the machine accepts.


It's convenient to have a format for writing this down (this will come up later in complexity theory). Write down $(q,t,p)$ as \[t_1qt_2.\] This means: split the tape into 2 parts, the part before the head is $t_1$, the part after the head is $t_2$ and $q$ points to the first symbol in $t_2$. All I'm doing here is indicating the position of the head by inserting a symbol representing the state in between the two parts.
%very first place in $t_2$. indicate look at very next symbol to right.

Write the computation history as a sequence of strings like this separated by pound signs.
\[
C_1\#C_2\#\cdots \# C_{\text{accept}}.
\]
Here $C_1$ is represented by $q_0w_1\cdots w_n$.

%Debugger store a snapshot of what all the registers contains. All motions the machine goes through. A record.

%The LBA will have $T$ built in, is going to accept strings that constitute
\begin{proof}[Proof of Theorem~\ref{ELBA}]
Let $S=$``on input $\an{T,w}$,
\begin{enumerate}
\item
Construct LBA $M_{T,w}$=``On input $z$,
\begin{enumerate}
\item
test if $z$ is an accepting computation history for $T$ on $w$. 
\item
\ul{Accept} if yes.\\
\ul{Reject}  if not.
\end{enumerate}
\end{enumerate}
Note $M_{T,w}$ does not simulate $T$. It simply checks if the input is a valid computation of $T$, in the form $C_1\#C_2\#\cdots \#C_{\text{accept}}$. %Given whole computation, it just checks if the computation is okay.

\ig{10-2}{1}

Why is this a LBA? It doesn't actually simulate $T$, it just checks the computation; this doesn't require running off the tape.

What does $C_1$ need to look like? It must look like $q_0w_1\cdots w_n$. How do we check $C_2$? This is a delicate point. %whole string built in. We don't know whether this whole string exists. Instead check steps of computation carried out configuration.
See whether $C_1$ updated correctly to $C_2$. Zigzag back and forth between $C_1$ and $C_2$ to check that everything follows legally. If anything is wrong, reject. 
It can put little markers on the input.
It repeats all the way to $C_{\text{accept}}$, then check that $C_{\text{accept}}$ is in an accept state.

%LBA itself doesn't have input. 

%Input could be large, but your job is just to write the LBA description down. Does the LBA accept any string? 
Now the only string our LBA $M_{T,w}$ could possibly accept, by design, is an accepting computation history. The point is that {\it checking a computation is a lot easier than doing it yourself}; a LBA is enough to check a TM's computation.

Now we have 0 or 1 string is in $M_{T,w}$. If $T$ does not accept, then $L(M_{T,w})$ is empty. If $T$ accepts, there is exactly one accepting computation history, namely the correct $C_1\#\cdots \# C_{\text{accept}}$. %keep everything configuration
Each configuration forces the next all the way to the accepting configuration. (We've built the Turing machine based on the particular Turing machine $T$ and string $w$.)
\emph{Hence $M_{T,w}$ is empty if and only if $T$ does not accept $w$.}

%Mechanism to give LBA enough tape to do the computation.
%Checking computation history can be done.
%Linear in $z$ not in $w$!
%Giving the guy a huge input!
%limited size given input so give huge input.
%Fake things.
%
%(dots on symbols keep track where are, everything looks same except around head)
We've reduced $A_{TM}$ to $E_{LBA}$.  
This proves $E_{LBA}$ is undecidable. %\fixme{add concluding statements}
\end{proof}

Note the computational history method is especially useful to show the undecidability of a problem that has to do with {\it weaker} models of computation.

\subsection{Proof of undecidability of PCP}
\begin{proof} (This is slightly rushed; see the book for a complete treatment.) 
The idea is to construct a collection of dominoes where a match has to be an accepting computation history.  (For simplicity, we'll deal with the modified PCP problem where we designate a certain domino to be the starting domino.)

%A mess that's just barely manageable.

Given $T$, $w$, we construct a PCP problem $P_{T,w}$ where a match corresponds to an accepting computation history. We construct the dominoes in order to force any match to simulate a computation history. 
\begin{itemize}
\item
Let the start domino be
\[
\coltwo{
\#}{
\#q_0w_1\ldots w_n\#
}.
\]
\item
If in $T$ %(getting down to nitty-gritty, look into guts of Turing machine, transition fucntion) 
we have $\de(q,a)=(r,b,R)$, put the domino
\[
\coltwo{qa}{rb}
\]
in. %Why is this a good thing? 

Similarly we have a domino for left transitions (omitted).
%Left too. Is messy, omit.
\item
For all tape symbols $a\in \Ga$, put in $\coltwo aa$. %(copy. problematic because it itself match, modify problem)
\end{itemize}


Consider a concrete example, $w=011$, and $\de(q_0,0)=(q_5,2,R)$. We have the domino $\coltwo{q_00}{2q_5}$. (The construction has one simple idea really: the transition dominoes, like $\coltwo{qa}{rb}$, force each configuration to lead to the next configuration.) We start with
\[
\left|
\begin{array}{cccccccc}
\# | & q_0 &0|&&&&&\\
\# & q_0&0& 1& 1& \#|& 2& q_5| 
\end{array}
\right|
\] 
We've managed to push forward the match on one more domino. Now we have to copy everything. We use $\coltwo 11$, $\coltwo 00$, $\coltwo 22$, $\coltwo{\#}{\#}$:
%reduce ATM to PCP.
\[
\left|
\begin{array}{ccccccccccc}
\# | & q_0 &0|&1|&1|&\#|&&&&\\
\#| & q_0& 0&1& 1& \#|& 2& q_5| & 1| & 1| & \#|
\end{array}
\right|
\] 
At end, computation history is done, but match isn't. We add dominoes $\coltwo{q_{\text{accept}}c}{q_{\text{accept}}}$, $\coltwo{cq_{\text{accept}}}{q_{\text{accept}}}$, and $\coltwo{q_{\text{accept}}\#\#}{\#}$.

These dominoes ``eat" the tape one symbol at time, around the accept state. Putting in one last domino finishes it off.

The start domino is a technicality.

%How to do using PCP solver. Convert Turing machine problem into PCP problem. 
We've reduced $A_{TM}$ to PCP. Therefore, PCP is undecidable.
%The only way the match could possibly accept is 
\end{proof}
%Inspiration for NP-completeness.\\

\cpbox{
A Turing machine accepts an input if and only if it has an accepting computation history for that input. Thus the problem of whether $T$ accepts $w$ can be formulated as: does $T$ have an accepting computation history for $w$?

This formulation is more concrete, and it is much easier to check whether a computation history is correct, then ask whether $T$ accepts $w$.}
\vskip0.15in
\cpbox{
To show a problem that isn't related to computability theory is undecidable, find a way to simulate/encode an undecidable problem (such as $A_{TM}$) with that problem. It is useful to encode computation histories.
}
%doesn't 