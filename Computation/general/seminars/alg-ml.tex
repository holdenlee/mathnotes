\def\filepath{C:/Users/Owner/Dropbox/Math/templates}

\input{\filepath/packages_article.tex}
\input{\filepath/theorems_with_boxes.tex}
\input{\filepath/macros.tex}
\input{\filepath/formatting.tex}
%\input{\filepath/other.tex}

%\def\name{NAME}

%\input{\filepath/titlepage.tex}

\pagestyle{fancy}
%\addtolength{\headwidth}{\marginparsep} %these change header-rule width
%\addtolength{\headwidth}{\marginparwidth}
\lhead{Algorithms and Machine Learning Reading Group}
\chead{} 
\rhead{} 
\lfoot{} 
\cfoot{\thepage} 
\rfoot{} 
\renewcommand{\headrulewidth}{.3pt} 
%\renewcommand{\footrulewidth}{.3pt}
\setlength\voffset{0in}
\setlength\textheight{648pt}

\begin{document}

\tableofcontents

\section{Linear coupling: an ultimate unification of gradient and mirror descent}

Linear Coupling: An Ultimate Unification of Gradient and Mirror Descent

I am talking about the paper by Allen-Zhu and Orecchia \url{http://arxiv.org/abs/1407.1537} on explaining Nestorov's accelerated gradient descent. The paper is very elegant. The original abstract is here. 

First-order methods play a central role in large-scale convex optimization. Even though many variations exist, each suited to a particular problem form, almost all such methods fundamentally rely on two types of algorithmic steps and two corresponding types of analysis: gradient-descent steps, which yield primal progress, and mirror-descent steps, which yield dual progress. In this paper, we observe that the performances of these two types of step are complementary, so that faster algorithms can be designed by linearly coupling the two steps. 
In particular, we obtain a simple accelerated gradient method for the class of smooth convex optimization problems. The first such method was proposed by Nesterov back to 1983, but to the best of our knowledge, the proof of the fast convergence of accelerated gradient methods has not found a clear interpretation and is still regarded by many as crucially relying on "algebraic tricks". We apply our novel insights to construct a new accelerated gradient method as a natural linear coupling of gradient descent and mirror descent and to write its proof of convergence as a simple combination of the convergence analyses of the two underlying descent steps. 
We believe that the complementary view and the linear coupling technique in this paper will prove very useful in the design of first-order methods as it allows us to design fast algorithms in a conceptually easier way. For instance, our technique greatly facilitates the recent breakthroughs in solving packing and covering linear programs [AO14, AO15].

\subsection{Basic convex optimization}

We want to find $\min f(x)$ such that $x\in Q$, where $f$ is convex. We won't worry about the constraint $Q$ in this talk.

In the first-order method, we are given access to $\nb f(x)$, but we can compute it at every point. The complexity measure is the number of queries.

Let $\ved$ be a norm, and $\ved_*$ be the dual norm. In most cases the norms will both be the Euclidean norm $\ved$. 
\begin{df}
We say $f$ is $L$-smooth with respect to $\ved$ as follows if for all $y,x$,
\[
f(y)\le f(x)+\nb f(x)^T (y-x) +\fc L2 \ve{y-x}^2.
\]
\end{df}
I.e., we can upper-bound $f$ by a quadratic function (with squared-term coefficient $\fc L2$). %We want to do Taylor expansion cleverly.

\begin{df}
We say $f$ is $\mu$-strongly convex if for all $y,x$, 
\[f(y)\ge \ub{f(x)+\nb f(x)^T(y-x)+\fc \mu2\ve{y-x}^2}{\Phi_x(y)}.\]
\end{df}
Without the last term this inequality is always true for convex functions. This says we can lower-bound $f$ by a quadratic function, rather than just a linear function tangent to it.

The gradient method is a primal method; the mirror is a dual method.

We review the gradient descent method. Start at some point, and use the update rule
\[
x_{t+1} = x_t-\rc L \nb f(x_t).
\]
%L is smoothness
where $L$ is the smoothness.
\begin{lem}
We have
\[
f(x_{t+1}) \le f(x_t)-\rc{2L} \ve{\nb f(x_t)}_*^2
\]
\end{lem}
If we have a sharp gradient, we must make some progress.
For the $L^2$ norm this is easy to see. Using the definition of the smoothness,
\bal
f(x_{t+1}) &\le f(x_t) + \nb f(x_t)^T (x_{t+1}-x_t) + \fc L2 \ve{x_{t+1}-x_t}_2^2\\
&= f(x_t)+\nb f(x_t)^T \pa{-\rc L \nb f(x_t)}\\
&=f(x) -\rc{2L}\ve{\nb f(x_t)}_2^2 &x_{t+1}-x_t=-\rc L \nb f(x_t).
\end{align*}
(Note it's also okay for $x_{t+1}-x_t=\de_t$ for some $\de_t$ correlated with the gradient (the angle is nontrivial). Note $\nb{x_{t+1}-x_t}^2$ is quadratic, but $\nb f(x_t)$ is linear, so you can tune it to be larger. In other cases you may only have approximate/stochastic queries.) 

%approximte a function locally using $\Phi_x(y)$.
How do you get the dual norm if you work with non-Euclidean norms? If you use a general norm, then the update rule is 
\[
x_{t+1}=\amin_y \Phi_{x_t}(g).
\]
It's not clear whether you can do it in polynomial time; the norm should be nice. 
Then the calculation above becomes
\bal
f(x_{t+1})&\le \Phi_{x_t}(x_{t+1})\\
& \le \Phi_{x_{t}}(y)\\
&=f(x_t) + \nb f(x_t)^T (y-x_t) + \fc L2 \ve{y-x_t}^2\\
&\quad \text{choose }y-x_t = -\te v\\
&\quad \text{such that }\nb f(x_t)^Tv = \ve{\nb f(x_t)}_* \ve{v},\ve{v}=1\\
&\quad \nb f(x_t)^T (y-x_t) =-\te\ve{\nb f(x_t)}_*^2\\
%exists y such that small
%term to be negative, and term not too large.
%highly correlated.
&\quad \rc2 \ve{y-x_t}^2 =\rc L2 \te^2\ve{v}^2 =\fc L2 \te^2\\
&=f(x_t)-\te \ve{\nb f(x_t)}_* + \fc{L}2\te^2\\
&\quad \te=\rc L \ve{\nb f(x_t)}_*\\
&\quad \ve{\nb f(x_t)}_* = \max\set{
\an{\nb f(x_t),v}}{\ve{v}=1}%can use this as update rule
\end{align*}

We won't work with dual norms for the rest of the talk.

\subsection{Mirror descent}

\begin{df}
Define a \vocab{distance generating function} (DGF) with respect to $\ved$  as a 1-strongly function $w(x)$ such that 
\[
w(y)\ge w(x) + \nb w(x)^T (y-x) + \rc 2\ve{y-x}^2
\]
%the most general definition
The \vocab{Bregman distance} is 
\bal
V_x(y) &= w(y)-w(x) - \nb w(x)^T (y-x)\\
V_x(y) &= \rc2 \ve{y-x}^2.
\end{align*}
(Note it's asymmetric. Think of $x$ as a reference point; we measure the distance to $x$.)
\end{df}

\begin{ex}
Take $w(x)=\rc 2 \ve{x}_2^2$. The Bregman distance is
\[
V_x(y)=\rc2 \ve{y-x}_2^2.
\]
\end{ex}
It generates the norm square. %does it satisfy other properties of the norm/
%picture: quadratic tangent to line, 
$V_x(y)$ is distance between the linear approximation and the upper quadratic approximation.

\begin{ex}
Consider a simplex
\[
S=\set{x}{\ve{x}_1=1,x\ge 0}.
\]
This is strongly convex with respect to the $L^1$ norm. 
Let $w(x)=\sum x_i\ln x_i$. Then 
\[
V_x(y)=\sum y_i \ln \fc{y_i}{x_i} = D_{KL} (y||x)\ge \rc2\ve{y-x}^2_{TV}.
\]
by Pinsker's inequality.
%Then 
%\[
%V_x(y)\ge \rc2 \ve{y-x}_1^2
%\]
%solving SDP with triangle inequality.
\end{ex}
What mirror descent does is the following. Set
\[
x_{t+1} 
=
\amin_y \{V_{x_t}(y) + \al \nb f(x_t)^T (y-x_t)\}.
\]
We don't assume smoothness. Suppose we have a convex function. We don't want to minimize $f(x_t)+\nb f(x_t)^T (y-x_t)$ in $y$ because that would be $-\iy$.
We want to find some point not too far from $x_t$. We believe the linear funciton is a good approximation, but it is only a good approximation when it is close to $x$. Hence we add in a regularizer $V_{x_t}(y)$. 

\begin{ex}
Let $w(x)=\rc 2\ve{x}_2^2$. 
\[
x_{t+1} = \amin \{\rc 2 \ve{y-x_t}^2+\al \nb f(x_t)^T (y-x_t)\}
=x_t-\al\nb f(x_t).
\]
We have a different step size $\al$ which in some cases can be chosen larger.
\end{ex}

$w$ is a convex function and $w(x)$ is a constant so $V_x(y)$ is convex. The minimizer is where the gradient is 0.
\[
\nb V_{x_t}(y) + \al \nb f(x_t)=0
\]
Plugging in $x_{t+1}$ we should get 0:
\bal
\nb V_{x_t}(x_{t+1}) + \al \nb f(x_t)&=0\\
\nb w(x_{t+1}) &= \nb w(x_t) - \al \nb f(x_t).
\end{align*}
%can move $\al$ to $\rc{\al}$. Balance regularizer...
Hence we get another way to state the update rule for mirror descent,
\[
\nb w(x_{t+1}) = \nb w(x_t) - \al \nb f(x_t).
\]

We have a primal space $X$, and we have a map $\nb w$ which sends $x_t\in X$ to $\nb w(x_t)$. We take a gradient step in the dual space $-\al \nb f(x_t)$ to get $\nb w(x_{t+1})$ and map it back using the inverse map $\nb w^{-1}$ to get $x_{t+1}$. This is where the name ``mirror" descent comes from.

%unique inverse? 
%strongly convex means gradient is invertible.
%gradient is strictly increasing.
For strongly convex we roughly have a lower bound $|\nb w(x)-\nb w(y)|\ge \mu \ve{x-y}$ so that the gradient is invertible.
%Right hand side primal norm, left hand side dual norm.
\begin{lem}[Main lemma for mirror descent]
\bal
\al [f(x_k)-f(x^*)]&\le \al[\nb f(x_k)^T (x_k-x^*)]\\
%no smoothness, only bound yuo can have.
&\le \fc{\al^2}2 \ve{\nb f(x_k)}_*^2 + V_{x_k}(x^*) - V_{x_{k+1}}(x^*)
\end{align*}
Summing we get a telescoping sum. Summing over $k$,
\bal
\al \sum_{k=1}^{T-1} (f(x_k) - f(x^*)) &\le 
\sum_{k=0}^{T-1} \fc{\al^2}2 \ve{\nb f(x_k)}_*^2 + V_{x_0}(x^*) - V_{x^T}(x^*)\\
&\le \sum_{k=0}^T \fc{\al^2}2 \ve{\nb f(x_k)}_*^2 + V_{x_0}(x^*)
\end{align*}

\end{lem}

\section{Improved noisy population recovery, and reverse Bonami-Beckner inequality
for sparse functions
}


Authors: Shachar Lovett, Jiapeng Zhang

Abstract:
The noisy population recovery problem is a basic statistical inference
problem. Given an unknown distribution in $\{0,1\}^n$  with support of size k,
and given access only to noisy samples from it, where each bit is flipped
independently with probability 1/2,
estimate the original probability up to an additive error of $\ep$. We give
an algorithm which solves this problem in time polynomial in $(k^{\log\log k},
n, 1/\ep)$ .
This improves on the previous algorithm of Wigderson and Yehudayoff [FOCS
2012] which solves the problem in time polynomial in $(k^{\log k}, n, 1/\ep)$ .
Our main technical contribution, which facilitates the algorithm, is a new
reverse Bonami-Beckner inequality for the L1 norm of sparse functions.

The setup is the following. 
\begin{prb}
There an unknown distribution $\pi$ over $\{0,1\}^n$ with $|\Supp(\pi)|=k$. (Think of this as a mixture of $k$ distributions.)

The input are noisy samples from $\pi$,
\begin{enumerate}
\item
$X\sim \pi$
\item
flip each $x_i$ with probability $\rc2-\mu$.
\end{enumerate}

Output the distribution $\wh\pi$ close in total variation distance:
\[
TV(\wh{\pi},\pi)\le \ep.
\]
%k as large, part of input.
\end{prb}
Ideally an algorithm has runtime $\poly(k,n,\rc{\ep},\rc \mu)$.
\begin{enumerate}
\item
This problem was first considered by Kearns et al., who gave an algorithm in time $\poly(n,\rc{\ep})\exp(k)$.
\item
YW do this in time in $\poly(n,\rc{\ep})k^{\ln k}$.
\item This paper does this in time $\poly(n,\rc{\ep})k^{\ln\ln k+\ln\prc{\nu}}$.
\end{enumerate}
A related probem is lossy population recovery, where we are given
\begin{enumerate}
\item
$X\sim \pi$.
\item Replace each $x_i$ with `?' with probability $1-\nu$. 
\end{enumerate}
This is easier; Moitra and Saks show a polynomial time algorithm for constant $\nu$.
%stronger access to underlying distribution.
%are any of this tight compared to sample complexity. 
%bound come from sample complexity.
%no lower bound known. possible it can be solved in polynomial time.

First we make some simple claims. 

First claim: If you want to solve the noisy population recovery problem, we can assume we know the support of the distribution. Suppose we have an algorithm that works given the support. Now run it on increasing prefixes. First run it on the prefixes 0, 1. What are the probability of strings with $x_1=0,1$? Now run the algorithm on $x_1x_2$, etc.

%prob 
%
Suppose $\Supp(\pi) = \{x_1,\ldots, x_k\}$, and we want to find the probabilities $p_1,\ldots, p_k$. We use the maximum likelihood estimator. Draw $m$ samples $S_m=\{y_1,\ldots, y_m\}$. Calculate
\bal
p(y_1|p_1,\ldots, p_k)
 &= \sum_i p_i \pa{\rc 2-\mu}^{\De(x_i,y_i)} \pa{\rc2+\mu}^{n-\De(x_i,y_i)}\\
&= \sum_i p_i e^{-\De(x_i,y_i)C_\mu}.
\end{align*}
We want to find
\[
\max_{\sum_ip_i=1} \sum_{y_i\in S_m}\ln \pa{\sum_i p_i e^{-\De(x_i,y_i)C_\mu}}.
\]
%for prefix knows
%extend by 1 character. each support fork off into 2 strings.
%know more going into problem: support organizes pairs of strings only differ in one symbol at the end. approx prob of each pair. does that help?
How many samples do we need to approximate this? We want to distinguish distributions which are $\ep$ apart: $TV(\pi_1,\wt\pi_2)>\ep$. However we don't have perfect $\pi_1,\pi_2$?
%asymptotically achieve information bound. how bad is sample complexity?
We want to characterize $TV(\wt{\pi_1},\wt{\pi_2})>a$. YW built a graph structure rather than using the MLE.

Given $\pi_1:\{0,1\}^n\to \R$, the Fourier expansion is \[\pi_1(x) = \sum_{S\subeq [n]}\wh{\pi_1}(G)\chi_S(x).\]
Now
\bal
\wt{\pi_1}(x) &= \E_e[\pi_1(x+e)]&e_i\sim B(\rc 2-\mu)\\
&=T_{2\mu} \pi_1(x)  =\sum_{S\subeq [n]} (2\mu)^{|S|} \wh{\pi_1}(S) \chi_S(x).
\end{align*}
Given $f:\{0,1\}^n\to \R$, $f=\rc2(\pi_1-\pi_2)$, $\ve{f}_1\ge \ep$, $\ve{f}_1=\E_x[|f(x)|]$, $\ve{T_{2\mu}f}_1\ge a$,
%this will dictate sample complexity.
\begin{thm}
Given $f:\{0,1\}^k\to \R$, $|\Supp(f)|=k$, 
\[
\ve{T_{2\mu}f}_1\ge k^{-O(\ln\ln n + \ln \prc{\mu})}\ve{f}_1.
\]
\end{thm}
This is a reverse hypercontractivity inequality because a hypercontractive inequality goes the other way.
%TV distance, any test of the 2 distributions is $>\ep$.

The idea is as follows. Given $\Supp(x)=\{x_1,\ldots, x_n\}$,
%\rc2-\mu
$\wt f(x)= \sum_i v_i e^{C_\mu d(x,x_i)}\le \rc{k}\sum_i v_i$ if not close to any. If it's close to one it will get a reasonably high value. This means that looking at balls of radius $\ln k$ around each value, the balls will contain a reasonble fraction of the $L^1$ mass, and one will contain at least $O\prc k$ of them. %support bounded within radius of $\ln k$. For such functions, it will be much easier to bound; we still get $\ln \ln k$.
%$v_1$ largest. Reasonable fraction.

Our goal is to bound $\ve{T_{2\mu}f}_1$, $T_{2\mu}(f)=\sum_{S\subeq [n]} \mu^{|S|} \wh f(S) \chi_S(x)$. We use the trivial bound (for any $S$)
\bal
\ve{T_{2\mu}(f)}_1 & \ge |\wh{T_{2\mu}}f(S)|\\
& = (2\mu)^{|S|} |\wh f(S)|.
\end{align*}
%(To see this, divide strings into $\chi_S(x)=1$ and $\chi_S(x)=-1$. Look at the $L^1$ norm of $x$. Write $f(x)=\wh f(S)\chi_S(x)+f'(x)$.
To see this note $\wh f(S)=|\E \chi_S(x)f(x)|\le\ve{f}_1$. Our goal is just to bound one Fourier coefficient.

\begin{lem}
Define a function $g(x)=f(x)T_{2\mu}h(x)$, where $h:\{0,1\}^n\to \R$  is a given function with $\ve{h}_{\iy}\le 1$.

Then 
\[
\ve{T_{2\mu}f}_1\ge \mu^{|S|} |\wh g(S)|.
\]
\begin{proof}
We find
\bal
\wh g(S) &= \E_x [\chi_S(x) g(x)] \\
&= \E_x [\chi_S(x)f(x)T_{2\mu} h(x)]\\
&= \E_x [h(x)T_{2\mu} \chi_S(x)f(x)]\\
&\le \E_x [|T_{2\mu}\chi_S(x) f(x)|]\\
&=\ve{T_{2\mu}\chi_S(x)f(x)}_1\\
&\le \rc{(2\mu)^{|S|}} \ve{T_{2\mu} f(x)}_1\\
T_{2\mu} \chi_S(x) f(x) &= \E_e [\chi_S(x+e)f(x+e)],& e_i\sim B(\rc2-\mu)\\
&=\E_{e_S} [\chi_S(x+e) \E_{e_{\ol S}}[f(x+e_S+e_{\ol S})]\\
%massage f into function function bounded support radius r.
%instead of doing it directly take $g$, $h$ is an indicator function of strings which are in one of the balls. some kind of convolution? No - pointwise product. Like a kernel. Cut off outside. 
%see most difference
%easier prove low degree coeff of large.
&\le \E_{e_S} \ba{
\ab{
\E_{\ol S} f(x+e_S+e_{\ol S})
}
}\\
%var dist between noisy dist
&\le \ve{\E_{e_S} f(x+e_S+e_{\ol S})}_1
\le \fc{\ve{T_{2\mu}f}_1}{(2\mu)^{|S|}}\\
f(x+e_S+e_{\ol S}) &= \sum_{T\subeq [n]}(2\mu)^{|S|}\fc{(2\mu)^{|S\bs T|}}{(2\mu)^{|S|}} \wh f(T)\chi_S(T)\\
&=\fc{T_{2\mu}f}{(2\mu)^{|S|}}.
%g ignore most of function f.
\end{align*}
The same formula is true even if the noise  probability is different for different coordinates. %T_{2\mu} \pi_1
%choose H to be indicator function, make g in ball radius r?
%not just sparse support but of bounded radius?
%low degree F coefficient for $g$.
\end{proof}
\end{lem}

%prob convex? convex with noise?
%k: doesn't try to gamify and does well

%helpful if all mass within a small ball.

%|f(x_1)|$ is max
Choose $r\approx \fc{\ln k}{\mu^2}$, $h(x)=1_{E}(x)$, 
\[
E=\set{y\in \{0,1\}^n}{d(y,x_i)<d(y,x_2)\text{ for all }x_i\text{ such that }d(x_i,x_1)\ge r}.
\]
For $g(x)=f(x)T_{2\mu} h(x)$, $|g(x_1)| \ge \rc{2k}$, $g(x_i) \le \ve{f}_1 e^{\mu^2 d(x_1,x_i)}$, and
\bal
g(x_1)&=f(x_1)\E_eh(x_1+e)\\
&=f(x_1)\Pj_e [x_1+e\in E]\\
\Pj(x_1+e\nin E)&\le e^{-\mu^2 r}\\
%strt at $x_i$ and flip?
g(x_1)&\ge \fc{f(x_1)}{2}.
\end{align*}
%close to function whose support bounded. 
We prove the existence of a large low-degree coefficient. Take $g\approx g^*$, $|\Supp(g^*)|\le k$, $\Supp(g^*) \subeq B(r,x)$.
\begin{lem}
For $g^*:\{0,1\}^n\to \R$, $|\Supp(g^*)| \le k$, $\Supp(g^*)\subeq B(n,r)$, (1)
\[
\ve{T_{2\mu} g^*}_1\ge k^{-\ln (4x)}\ve{g^*}_1.
\]
\end{lem}
For $\ve{T_{2\mu}g^*}_1\ge \mu^{|S|} |g^*(S)|$,
\begin{clm}
There exists a low-degree polynomial $p(x)=\sum_sp_s$, $\deg(P)\le \ln k$,
\begin{enumerate}
\item
$P(x)=g^*(x)$ for all $x\in \Supp(g^*)$
\item
$\ve{p}_1 = \sum_S|p_S|\le kr^{\ln k} \ve{g^*}_1$.
\end{enumerate}•
\end{clm}

(For $x_1=1$, $f(x)=\pf{1-\mu}2^r\pf{1+\mu}2^{n-r}$, $\ve{\wt f}_1=\sum_i \pf{1-\mu}2^r\pf{1+\mu}{1+\mu}^{r}\binom nr$.)

How claim implies 1:
\bal
\an{g^*,p} &= \sum_x g^{*2}(x) = \sum_x g^{*2} \ge \fc{\ve{g^*}_1^2}{k}\\
\an{g^*,p} &= \sum_S \wh{g^*}(S)p_S\le \max_S|\wh g(S)|\ve{f}_1.
\end{align*}
%indicator of close to $x_1$.
%low deg coeff of large value
\begin{proof}
$C_1(K), C_2(K,w)$. $C_1(K)$ minimal value such that for all $n\ge 1$, $\forall x\subeq B(n,x)$, $\forall f:X\to \R$ there exists $P(x)$ such that 
\begin{enumerate}
\item
$\deg(p)\le \ln k$
\item
$P(x)=f(x)$, $\forall x\in X$
\item 
$\ve{P}_1\le c_1(k) \ve{f}_1$.
\end{enumerate}
$C_1(k,w)$ is defined similarly but for $w(x)\le w$
%sparse sets of low weight
%g(k)$ bounded by $k$ times $r^{\ln k}$
Claim: 
\[
G(k,w)\le \max_{1\le a-\fc{k}2} (G(k,w-a)+G(a))
\]
Proof: For $f,w(x)=w$, 
\bal
f(x)=f_1(x_1,\ldots, x_{n-1})+x_nf_2(x_1,\ldots, x_{n-1})\\
f(x)=f_1(x_1,\ldots, x_{n-1})+x_nf_2(x_1,\ldots, x_{n-1})1(x\nin E).
\end{align*}
Partition $\Supp(x)$ into $x_n=0,1$. At most $k/2$ with $x_n=1$. $E$ is intersection with $x_1=0$. $E\subeq \Supp(f_2)$. 

The latter is a function on $n-1$ variables with support $\le \fc k2$. If the support size is $a$, %then the support on the second is weight $\l
\bal
G(k,w) &\le \max_{i-a\le \fc k2} G(k,w-a)+G(a).
G(k)&=G(k,kx) \\
&\le \max_{1\le a_1,\ldots, a_t\le \fc k2,\sum a_i\le kx} \sum_i G(a_i).
\end{align*}
Induction hypothesis $G(a_i)\le a_ir^{\ln a_i}$, $G(k)\le kr^{\ln k}$.
\end{proof}
%weight of set is going down. bounded weight.

Question: Given $f:\{0,1\}^n\to \R$, $|\Supp(f)|=k$, does that there exists $S$, $|S|\le \ln k$, 
\[
\wh f(S)\ge \rc{\poly(k)}.
\]
Then we have a polynomial time algorithm. %$k^{\ln \ln k}$.
%just know $\rc{k^{\ln \ln k +\ln \fc n{\mu}}}$.

%reduction lose polylog k factors. Support clustered not good idea?
%difficult case within small bal.
%1/k on k strings. 
%difference are 2 string with weight 1/k on them. 

%if f unif on set of size k what is S?

%boil down to understand $L^1$ norm on noise operator. 
%mu small means large noise?
%noisy dist overlap a lot?
%exp close by choosing mu?

%what are bad examples?

%2 sets of $k$ strings...
%need exp dependence on $\mu$ in general.

\section{Hardness of signaling for Bayesian games by Shaddin Dughmi}

Say we're driving to New York or Philadephia. There are many paths. It's the time via GW bridge, Lincoln tunnel, etc. But people say some route is fastest, then it becomes slower because more people go to that. 

Start with the Bayesian setting $\{G_\te\}_{\te\in \Te}$. Without any knowledge, players play $\E_{\te\in \Te,\te\sim \la}G_{\te}$ and converge to Nash equilibrium. (Nature decides $\te$, but is unkown to both players.)
%\la is prior dist on nature)

Signalling means a distribution $\ph(\te,\si)$, $\si\in \Si$. For example, $\Si$ could be a partition of $\Te$. Here we consider symmetric signalling; both players get the same information. 
A nondeterministic signal could do better.

The goal is to have the induced distribution on $\te$ be given by $\si$, $\sum_{\si} \al_\si \ph(\te|\si)=\la$, where $\al_\si$ is the probability of picking $\si$ as signal. Decomposing the prior distribution into a convex combination.

In this setting the goal is to maximize player 1's expected payoff. It could be welfare, or any monotone function on the pair.
%0's.
%minimax. trying to maximize payoff in minimax.
%why not create a correlated equilibrium?
There is a quasi-polytime algorithm that works for any game, with $\ep$-Nash equilibrium (subject to ...). (Markokis-Mehta: you can get $\ep$-Nash.) This is based on the LMM algorithm. The idea is to create an $\ep$-net.
This is for a broader class of objective functions and games.

In our setup, $\{G_\te\}_{\te\in \Te}$ is a zero-sum game. (All equilibria have the same value by the minimax theorem.)

The whole point is that signalling is as hard as finding a hidden clique of size $k=\om((\ln n)^2)$. This matches the upper bound given by the LMM algorithm. %assuming no good algorithm for hidden clique.

Define a new problem called Hidden Clique Cover ($G(n,p,k,r)$) problem, which is the following. We state it not as a decision problem but as a search problem. Let $G\sim G(n,p)$ a Erd\H os-Renyi random graph. Plant a clique of size $k$ $r$ times.
%pick $r$ random $k$-size subsets; not disjoint. 
%plant them. Picking $r$ a random. $\Te(n/k)$: all vertices participating.
%decision version is hard? almost equivalent?
%but easy counting of degree.

Our goal is to recover a constant fraction of these cliques.
%list of cliques, return $\al r$ cliques.

If you can solve the search version you can make progress on the regular version. Plant the correct one at one point and plant $r-1$ garbage cliques. Every time you recover some constant fraction; hope that the algorithm will output the correct one at some point. Thus the problem is HC-hard. (HC$=$hidden clique.)

Given an instance $G\sim G(n,p,k,r)$, $T\subeq V$, our goal is to output a list of cliques $S$ such that $|S\cap T|\ge \ep|T\cup S|$. Given an instance, and a set of vertices, then we can recover all cliques which have a large intersection with $T$ in polynomial time.

%T almost same as $k$? Try taking $T$ to be the whole graph, but $T\cup S$ is too big.

It's a randomized algorithm: pick some samples. Suppose you know what $|S\cap T|$ is. Sample $\ln n$ samples from $T$; a constant fraction of them is uniformly distributed from $S\cap T$. Look at the vertices that have it as a common neighbor (?). Now you have a larger subset; cut out the vertices which don't have at least $k$ neighbors. This returns the list of cliques with high probability.

If you have a subset guaranteed to have constant fraction intersection with clique $S_i$ then you can fully recover $S$.

The game is as follows. Given $G\sim G(n,p,k,r)$, a state is $\te\in V$. There are 2 players, attacker and defender. The attacker wants to destroy edges (ex. network from a server). He can hit the vertex or one of its neighbors. Hitting a vertex severs all incident edges.

The attacker picks $v\in V$. If $(v,\te)\in E$, then the attacker gets a point.

The defender picks $D\subeq V$, defends edges incident on $D$. Actually this isn't quite true: the defender picks a vertex with weight $d$; then mixed strategies include all subsets of size $d$.
%turn into mixed strategy?
%If $z_i$ is the strategy vector, $\sum_i z_i\le d$, $d=|D|$.

If the attacker picks an edge but the defender defended the edge, then the attacker gets $-\rh$, where $\rh d=\Te(k)$.  
This rules out FPTAS but not PTAS.

The attacker's payoff is
\[
A^{\te}(a,D) = |\{(\te,a)\}\cap E| - \rh|D\cap \{\te,a\}|.
\]
The subsets are mixed strategies over points. 
%, just represent them s mixed strategy.

Observations:
\begin{itemize}
\item
There exists a signalling scheme that gives $\ge 0.8$ payoff for the attacker.

The hidden clique cover gives a partition of the graph, defined as follows. Letting $S_i$ be the $i$th planted clique, define
\bal
\wh{S_i}&=S_i\bs \bigcup_{j=1}^{i-1}S_i,\\
\wh{S_0}&=V\bs \bigcup_{i=1}^r S_i.
\end{align*}
$\te$ falls into one of these partitions. Send the corresponding set in the partition to both players.

Note that $\wh{S_0}$ has to be small if $kr=\Te(n)$. Most of the vertices are in one of the good partitions. It has to be a clique. There the attacker wins if he picks any vertex---he only cares about whether the defender defends successfully; it doesn't happen that much: if $d=|D|$ the probability of successful defense is
\[
\Pj(a\in D)+\Pj(\te\in D) \le \fc{2d}{|S_i|}.
\]
This is small in expectation. %only $-0.2$.
\item
Given a signalling with payoff $\ge 0.5+\ep$, you can output $\Om(\ep)$ planted cliques in $G$ (with high probability).

This is harder. We want to massage the signal into behaving like in the first observation.

Given input $A_G$ and $\ph$, 
\begin{enumerate}
\item
for each $\si\in \Si$, compute minimax strategy $y_\si$ (defender) and $x_\si$ (attacker).
%pairs 0 to k
%$\ep$ not a constant.
%The gap between 0 and 1 is not constant.
%\rh d fixed. 
%you can make $d$ large. Then it changes the theorem. 
%implicitly described yet tractable?
%set d small, \rh large.
%$n^{1+\ep}$ is not ruled out for the zero-sum.
%We want it to return a small subset but it might be crazy.

Let
\bal
x_\si&=\begin{cases}
x_\si,&\text{if }x_\si\le \fc{2}{\rh d}\\
0,&\text{otherwise}
\end{cases}\\
\wh{y_{\si}}&=...
\end{align*}
%suppot of $z_\si$, some constant fraction has constant fraction intersection.
\item
Let $z_\si$ be the solution to the linear program
%not? best response for defender.
\[
\max_{\ve{z_\si}_\iy\le \fc{2}{\rh d}, \sum z_{\si}(v)=1,z_\si(v)\ge0} (\wh{x_\si})^TA_Gz_\si.
\]
\end{enumerate}
Depending on the attacker's strategy, guess what $\te$ is; get a distribution on nature. Tke the support on nature. If the attacker does well there is a way to predict nature, and you can do well.
%$\rc2$ is what you expect in background.
\item
Return $\Supp(z_\si)$ for all $\si$.
\end{itemize}
%You don't lose so much in 1 and 2.
%x is concentrated on something. 
$\wh{x_\si}$ not behaves like some subset, and $z_\si$ also behaves like a subset. It finds $z$ that is correlated with $x$ on the graph. If the payoff is larger than what's expected, there has to be a constant fraction of the planted cliques.

The rest is calculation.

An open problem is the following. %Even in worst-case assumption, setting the objective function to ...
ETH-hard to get a good $\ep$-approximate Nash.
%Not clear whether you can 
Come up with a zero-sum game from worst-case settings, so it's hard to distiguish between
\begin{itemize}
\item
YES: there exists a good signal.
\item
NO: for all signals the payoff is bad.
\end{itemize}
%It's not clear how you can have from worst-case a zero-sum game. Zero-sum is easy. 
%

\section{ETH hardness of DkS (with perfect completeness}

We show that, assuming the (deterministic) Exponential Time Hypothesis,
distinguishing between a graph with an induced $k$-clique
and a graph in which all $k$-subgraphs have density at most $1-\epsilon$,
requires $n^{\tilde{\Omega}(\log n)}$ time.
Our result essentially matches the quasi-polynomial algorithms of Feige
and Seltser [fs97] and Barman [Barman14] for this problem,
and is the first one to rule out an additive PTAS for Densest
$k$-Subgraph. We further strengthen this result by showing that our lower
bound continues to hold when, in the soundness case,
even subgraphs smaller by a near-polynomial factor ($k' = k \cdot
2^{-\tilde \Omega (\log n)}$) are assumed to be at most
$(1-\epsilon)$-dense.

Our reduction is inspired by recent applications of the ``birthday
repetition" technique [AIM14,BKW15]. Our analysis relies
on information theoretical machinery and is similar in spirit to analyzing
a parallel repetition of two-prover games in which the provers
may choose to answer some challenges multiple times, while completely
ignoring other challenges.

\begin{df}
DkS, the \vocab{densest $k$-subgraph} problem, is as follows. Given a graph $G=(V,E)$, define 
$\text{den}(S)=\fc{|(S\times S)\cap E|}{|S\times S|}$. 
The decision problem is to decide between the following 2 cases. ($C>S$)
\begin{enumerate}
\item YES: there exists $S$, $|S|=k$ such that $\text{den}(S)\ge C$. 
\item NO: for all $|S|=k$ such that $\text{den}(S)\le s$.
\end{enumerate}
In the algorithmic (search) version, actually find the $S$.
\end{df}
Upper bound: 
The state of the art gives a $O(n^{\rc 4})$-approximation (BCCFV) using SDP. This is the best that can be obtained using LP.

Lower bound: 
\begin{itemize}
\item
DkS is $\pa{1+\rc{c(k)}}$-approximable in time $2^{(\lg n)^k}$ assuming RETH. [Khot 2001]
\item
Any constant inapproximable assuming variant of SSE (small set expansion) [RS10]
\item
Any constant inapproximable assuming Feige's conjecture/hidden clique [Alon et al, 2011].
\end{itemize}
%This is a spectrum: you get stronger results assuming stronger conjectures.

Here it's not distinguishing between dense and less dense graph; it's distinguishing between cliques and less dense cliques.

Our result is the following. It is ETH-hard to distinguish in $n^{\wt O(\lg n)}$ time the following.
\begin{itemize}
\item
%rules out additive approx algorithm
YES: $\exists S$, $|S|=n^{1-o(1)}$ such that $\text{den}(S)=1$,
\item
NO: $\forall S$, $\text{den}(S)\le 1-\ep$.
\end{itemize}•
This matches the upper bound: there exists a dynamic programming algorithm in $n^{O(\ln n)}$ time [Feigel, Setter 97]. Barak 15 gives an additive approximation [Barak 15].

The reduction is simple, but the analysis is hard.

The idea is birthday repetition. With high probability, two $O(\sqrt n)$-size subsets of a $n$-sized set share an element.
%check constraint or consistency.

Start with Dinur's instance of 2CSP: Given a 3SAT instance, there is a CSP with gap $(1-c,1)$. 
Each vertex corresponds to assignments on $\sqrt n$ variables. The number of vertices is $|V|=\binom{n}{\sqrt n}|\Si|^{\sqrt n}$.
Edges $(u,v)$ are between vertices that
\begin{enumerate}
\item
are equal on the variables they share (consistency), and
\item 
they do not violate any 2CSP constraints.
\end{enumerate}•
%$u\cup v$ satisfies both assignments.
There are clusters labeled by variables we're giving assignments to.
%\chapter{Motivation}

%We'll see lots of analysis in action.
\section{Dirac delta function}

What is the derivative of the Dirac delta?

You may have seen the mysterious ``\textbf{Dirac delta function}" defined by 
\beq{eq:dist0-1}
\int_{-\iy}^{\iy} \de(x-x_0)f(x)\,dx=f(x_0), \qquad \de(x)=0,\,x\ne 0.
\eeq
This emerged from Fourier's classical treatise on heat. It was there implicitly in his work. Cauchy and Dirac noticed it. It is used in math, applied math, physics, engineering.
%does the job we're asking it

But there is no function in any sense of the word that does this job! It makes no mathematical sense!

%Let's look at~\eqref{eq:dist0-1} in an abstract sense. 
However, looking at~\eqref{eq:dist0-1} in an abstract sense, the ``process" which takes $\de(x-x_0)$ and the ``nice" function $f(x)$, and spits out $f(x_0)$ is well-defined.
%There's some $\de(x-x_0)$, given $f(x)$, some process happens and spits out $f(x_0)$. 

However, people don't just talk about the delta function, they also talk about its derivative! Trying to differentiate something that doesn't exist...? %We'll put on our applied maths hat on and try to define the derivative.

How can we define the derivative $\de'(x-x_0)$? A first attempt might be (assuming $f$ is nice)
\begin{align}
\nonumber
\int \de'(x-x_0)f(x)\,dx&=\lim_{h\to 0} \int 
\pf{\de(x-x_0+h)-\de(x-x_0)}{h}f(x)\,dx\\
\nonumber
&=\lim_{h\to 0} \fc{f(x_0-h)-f(x_0)}{h}\\
\nonumber
&=-f'(x_0)\\
\llabel{eq:dist0-2}
&=-\int \de(x-x_0)f'(x)\,dx.
\end{align}
The equality~\eqref{eq:dist0-2} suggests that we could have simply integrated by parts
\[
\int \de'(x-x_0)f(x)\,dx=-\int \de(x-x_0)f'(x)\,dx+\ub{\text{boundary term}}{0}.
\]
This function that doesn't exist seems to follow the usual rules of calculus! 
This suggests there is a way of interpreting all the integrals in a consistent way
We can make all this rigorous using the theory of distributions. %looks like the usual rules of integral calculus can be applied to $\de(x-x_0)$.

\section{Fourier transforms of polynomials}

The \textbf{Fourier transform} is defined by (abbreviate $\int:=\iiy$)
\[
\cal F:f\mapsto \wh f(x)=\int e^{-i\la x}f(x)\,dx.
\]
This makes sense if $f$ is absolutely integrable:
\[
\ab{
\int e^{-i\la x}f(x)
}\le \int |e^{-i\la x}f(x)|\,dx=\int |f(x)|\,dx<\iy.
\]
What if $f\nin L^1$, in particular, what if $|f|\not\to 0$ as $|x|\to \iy$? You may have seen
\beq{eq:dist0-3}
\de(\la)=\rc{2\pi}\int e^{-i\la x}\,dx.
\eeq
There is a way of interpreting this so that it makes sense. This seems to suggest that the Fourier transform of $\rc{2\pi}$ is equal to $\de(\la)$. What if $f(x)=x^n$? Can we take the Fourier Transform?
\bal
\int e^{-i\la x}x^n\,dx&=\int \pa{i\pd{}{\la}}^n e^{-i\la x}\,dx\\
&=\pa{i\pd{}{\la}}^n \int e^{-i \la x}\,dx\\
&=2\pi e^{-i \pi n/2}\de^{(n)}(\la).
\end{align*}
If we can make sense on the derivatives of $\de$, then we can define the Fourier transform of polynomials.

An alternative way of defining Fourier transform of $f(x)=x^n$ would be to use Parseval's Theorem, which states
\beq{eq:dist0-4}
\int f(x)\wh g(x)\,dx =\int \wh f(\la)g(\la)\,d\la
\eeq
for all ``nice" functions $f$ and $g$. We could define $\hat f(\la)$ where $f(x)=x^n$, as the function for which
\beq{eq:dist0-5}
\int x^n\hat g(x)\,dx=\int \hat f(\la)g(\la)\,d\la\text{ for all nice functions } g
\eeq
we could say that $\hat f(\la)$ is the Fourier transform of $f(x)=x^n$. Note $\hat g(x)$ decays quickly, so this makes sense. This can be done rigorously using the theory of distributions. %If we can find $\hat f(x)$

``Everything's easy when you know the answer." It's only perfectly natural when you've been shown its perfectly natural. To prove consistency is not quite so easy.

\section{Discontinuities to Differential Equations}
%we want to see things like that happen.

There are some important genuinely important physical scenarios in which we would like a solution to a PDE to have discontinuities. For example, in acoustics we want the pressure $p(x,t)$ to solve the wave equation
\beq{eq:dist0-6}
p_{xx}-p_{tt}=0,
\eeq
but for $p$ to jump either side of a shock wave.
%propagate out?
%pavilion g

Is there any meaningful way to say that the function 
\[
u(x,y)=\al(x-y)+\be(x+y)
\]
is a solution to the wave equation $u_{xx}-u_{yy}=0$ if $\al,\be \nin C^2$? Assume $\al,\be\in C^2$ and $u_{xx}-u_{yy}=0$ so that for any nice function $f(x,y)$ (say $f=0$ when $x^2+y^2$ is sufficiently large),
\bal
0&=\int\int f(x,y)\pa{\pd{{}^2u}{x^2}-\pd{{}^2u}{y^2}}\,dx\,dy\\
&=\int\int u(x,y) \pa{\pd{{}^2}{x^2}-\pd{{}^2f}{y^2}}\,dx\,dy
\text{integration by parts twice}.
\end{align*}
If we can find $u(x,y)$ such that
\beq{eq:dist0-7}
\int\int u(x,y) \pa{\pd{{}^2}{x^2}-\pd{{}^2f}{y^2}}\,dx\,dy=0
\text{ for all nice functions }f,
\eeq
we say that $u=u(x,y)$ is a \textbf{weak solution} to the PDE $u_{xx}-u_{yy}=0$. We can use distribution theory to study weak solutions to PDE's.
%3 reasons want formalize

\section{Summary}

In each motivating example we introduced a family of ``nice" functions that allowed us to extend classical definitions to a wider class of problems. This is the underlying idea in distribution theory. Given a vector space $V$ of ``nice" functions we define the distributions on $V$ to be the topological dual $V^*$, which consists of all the continuous linear forms $V\to \C$. 

For example, if $V=C(\R)$ then we can define Dirac delta by the linear form\footnote{You may be more familiar with the notation $\an{x,y}=x\cdot y$ for finite-dimensional vector spaces.}
\beq{eq:dist0-8}
\an{\de_{x_0},f}=f(x_0).
\eeq
%linear map.
In general, any $u\in V^*$ is linear, so $\an{u, \al f+\be g}=\al\an{u,f}+\be \an{u,g}$ for arbitrary constants $\al,\be$ and $f,g\in V$. We need functional analysis because we require continuity. (The algebraic dual is too big to be interesting. Hence we supplement $V$ with a topology, i.e. a notion of convergence $f_n\to V$ in $V$. This is the motion of $w^*$-convergence, $\an{u,f_n}\to \an{u,f}$.

\blu{Lecture 2}
\chapter{Distributions}

Recap:
\begin{itemize}
\item
Delta function doesn't make sense.
\item
Way to define distributions is to first define a ``nice" space of functions $V$ (having all the properties we want it to have) and define distributions as continuous linear maps from $V$ to $\C$. 
\end{itemize}
%V has all properties we want it to have.

We'll always work with continuous functions, so we can define continuity of functions $V\to \C$ very explicitly.

\section{Notation and preliminaries}

An element of $\R^n$ will be written $x,y,z,\ldots$ so that 
\[
x=(x_1,x_2,\ldots, x_n)
\]
and we will use $dx=dx_1\,dx_2\,\cdots \,dx_n$ to denote the volume element in $\R^n$. Capitals $X,Y,Z$ will always denote open subsets of $\R^n$ and $K$ will always denote a compact (closed and bounded) subset of $\R^n$. Integrals over all $\R^n$ or over $X\subeq \R^n$ will be denoted by $\int [\cdot]\,dx$ and $\int_X[\cdot]\,dx$, respectively.
We will use multi-index notation $\al,\be,\ga$ (Greek letters) will denote multi-indices $\al=(\al_1,\ldots, \al_n)\in \Z_+^n=\{0,1,2,3,\ldots\}^{\times n}$. Multi-index notation reads as follows.
\bal
\pl^{\al} &\equiv\pa{\pdd{x_1}}^{\al_1}\cdots \pa{\pdd{x_n}}^{\al_n},&x^{\al}&\equiv x_1^{\al_1}\cdots x_n^{\al_n}\\
\al!&\equiv \al_1!\cdots \al_n!&|\al|&\equiv \al_1+\cdots +\al_n.
\end{align*}
%functions for which it's switched on.
We will often write $\pl^{\al}_x$ to make it clear what we're differentiating with respect to. We will also use $D=-i\pl$ when we do Fourier analysis. Define the \textbf{support} of a function $f$ by 
\[
\Supp(f)=\ol{\set{x\in \R^n}{f(x)\ne 0}}.
\]
We will often want to take limits inside integrals. To do this we refer to the dominated convergence theorem: (See for instance~\url{https://dl.dropboxusercontent.com/u/27883775/math\%20notes/18.125.pdf}, Theorem 15.1.)
\begin{thm}[Dominated convergence theorem]\llabel{thm:dct}
Given a sequence of absolutely integrable functions $\{f_m\}_{m\ge 1}$ such that $f_m(x)\to f(x)$ for each $x$ and $|f_m(x)|\le g(x)$, where $g$ is absolutely integrable, then 
\[
\lim_{m\to \iy}\int_Xf_m(x)=\int_X\ba{\lim_{m\to \iy} f_m(x)}\,dx=\int_X f(x)\,dx.
\]
\end{thm}
If $f$ is absolutely integrable on $X$, i.e.,
\[
\int_X |f|\,dx<\iy,
\]
we say that $f\in L^1(X)$.

\section{Test functions and distributions}
We need to define our first vector space of test functions.
\begin{df}
The space $D(X)$ consists of all the smooth functions from $X$ to $\C$ that have compact support. We say that a sequence $\{\ph_m\}_{m\ge1}$ tends to zero in $D(X)$ if there exists some compact set $K\subeq X$ such that $\Supp(\ph_m)\subeq K$ and $\pl^{\al}\ph_m\to 0$ uniformly for each multi-index $\al$.

The space $D(X)$ is often written $C^{\iy}_c(X)$.
\end{df}
(For convergence, the function is not allowed to have its mass moving away to infinity.)
%integrate by parts, evaluated on the boundary, derivatives shift to the other side.

Since the functions in $D(X)$ vanish at the boundary of $X$, we have
\[
\int_X\ph \pl^{\al}\psi\,dx=(-1)^{\al}\int_X\psi\pl^{\al}\ph\,dx,
\qquad \ph,\psi\in D(X)
\]
by integration by parts $|\al|$ times. We have Taylor's Theorem to any order
\beq{eq:taylor}
\ph(x+h)=\sum_{|\al|\subeq N}\fc{h^{\al}}{\al!} \pl^{\al}\ph(x)+R_N(x,h)
\eeq
where the remainder is $o(|h|^N)$ uniformly in $x$. %treat it as  ncice polynomial plus remainder.

%we have space of tests. distributions will be defined as linear maps. We want them to be continuous. $u(\ph)\equiv \an{u,\ph}\in \C$. If $\ph_m\to 0$ in $D(X)$, $\an{u,\ph_m}\to 0$ in $\C$.
\begin{df}\llabel{df:distribution}
A linear map $u:D(X)\to \C$ is a \textbf{distribution} (on $X$) if
for each compact $K\subeq X$, there exist $C,N$ such that 
\beq{eq:dist1-1}
|\an{u,\ph}|\le C\sum_{|\al|\le N} \sup|\pl^{\al}\ph|
\eeq
%vs sequential continuity. 
for each $\ph\in D(X)$ with $\Supp(\ph)$ with $\Supp(\ph)\subeq K$. The space of all such maps is denoted $D'(X)$. If we can use the same $N$ for all $\ph\in D(X)$, we call the least such $N$ the order of $u$, denoted $\ord(u)$.
\end{df}
%in the back of mind have this example, go back to familiar example.
\begin{rem}
Thinking of $D(X)$ as a locally convex space (in fact, Fr\'echet space) with seminorms defined by $\sup|\pl^{\al}\ph|$, we have $D'(X)=D(X)^*$. See Example 2.2.4 in Functional Analysis\footnote{\url{https://dl.dropboxusercontent.com/u/27883775/math\%20notes/part_iii_functional.pdf}}.
The example there is actually a larger space ($\cal E(X)$ of the next chapter), but it contains our $D(X)$.
\end{rem}

\begin{ex}
\begin{enumerate}
\item
We check that the Dirac delta $\de_x$ is a distribution. $\de_x$ is defined by
\[
\an{\de_x,\ph}=\ph(x)\qquad\ba{\int \de(x-y)\ph(y)\,dy=\ph(x)}.
\]
We want to check if~\eqref{eq:dist1-1} holds:
\[
\ab{\an{\de_x,\ph}}=|\ph(x)|\le \sup|\ph|
\] 
so~\eqref{eq:dist1-1} holds with $C=1$, $N=0$,  no matter what $\ph$ we choose. So $\de_x$ is a distribution of order 0.
\item
Here is a more useful example. Consider the linear map $T$ on $D(X)$ defined by 
\[
\an{T,\ph}:=\sum_{|\al|\le M} \int_X f_{\al}\pl^{\al} \ph\,dx,
\]
$f_{\al}\in C(X)$. Now for $\ph\in D(X)$, with $\Supp(\ph)\subeq K$. By definition of $T$, 
\bal
|\an{T,\ph}|&=\ab{
\sum_{|\al|\le M}\int_K f_{\al} \pl^{\al}\ph\,dx
}\\
&\le \sum_{|\al|\le M} \int_K |f_{\al}||\pl^{\al} \ph|\,dx\\
&\le \sum_{|\al|\le M} \sup_{|\al|\le M}|\pl^{\al}\ph| \int_K |f_{\al}|\,dx\\
&\le \pa{\max_{\al}\int_K |f_{\al}|\,dx}\sum_{|\al|\le M} \sup |\pl^{\al} \ph|
\end{align*}
Note that the test functions have compact support, so it doesn't matter if $f_{\al}$ blows up at the boundary. So we have estimate~\eqref{eq:dist1-1} with 
\[
C=\max_{|\al|\le M}\int_K |f_{\al}|\,dx,\qquad N=M.
\]
Note that $C=C_K$. We could have done this only assuming that $\{f_{\al}\}$ were locally integrable on $X$ (integrable on every compact subset of $X$), written $f_{\al}\in L_{\text{loc}}^1(X)$.

Note here the constant $C$ depends on the support test function. $N$ can also depend on it.
%$N$ can also depend on the support of the test function.
\end{enumerate}
\end{ex}
\begin{lem}\llabel{lem:dist1-1}
A linear map $u:D(X)\to \C$ belongs to $D'(X)$ if $\an{u,\ph_m}\to 0$ for every sequence $\{\ph_m\}_{m\ge 1}$ in $D(X)$ that tends to 0.
\end{lem}
\blu{Lecture 3 (27 Jan)} 
\begin{rem}
That Definition~\ref{df:distribution} and Lemma~\ref{lem:dist1-1} are equivalent conditions for continuity is a special case of Lemma 2.2.5 in the functional analysis notes.
\end{rem}
\begin{proof}
\begin{itemize}
\item
$\implies$: If $u\in D(X)$ and $\ph_m\to 0$ in $D(X)$ then
\[
|\an{u,\ph_m}|\le \sum_{|\al|\le m} \sup|\pl^{\al} \ph_m|\to 0.
\]
\item
$\Leftarrow$: Assume not, so there exists a compact set $K\subeq X$ such that estimate~\eqref{eq:dist1-1} does not hold for any $C,N$. In particular, it doesn't hold for $C=N=m$. So there exist $\phi_m\in D(X)$ such that $|\an{u,\phi_m}|>m\sum_{|\al|\le m}\sup |\pl^{\al} \phi_m|$. WLOG, we can assume that $\an{u,\phi_m}=1$, by setting $\wt{\phi_m}=\fc{\phi_m}{\an{u,\phi_m}}$. This implies
\bal
\implies \sum_{|\al|\le m} \sup|\pl^{\al}\phi_m|&<\rc m\\
\implies \sup|\pl^{\al}\phi_m|&<\rc m,&|\al|\le m\\
\implies \phi_m\to 0& \text{ in }D(X).
\end{align*}
But this is a contradiction.
\end{itemize}
\end{proof}
\section{Limits in $D'(X)$}
Often we have  sequence of distributions $\{u_m\}_{m\ge 1}$. If there exist some $u\in D'(X)$ such that $\an{u_m,\ph}\to \an{u,\ph}$ for all $\ph\in D(X)$, then we say that $u_m\to u$ in $D'(X)$.

Limits in $D'(X)$ often look strange. 
\begin{ex}
For instance, define the distribution $u_m\in D'(\R)$ by the locally integrable function
\[
u_m(X):=\sin (mx).
\]
Then $u_m\to 0$ in $D'(\R)$. 

%compact support, ibp no boundary terms.
Proof: We have using \blu{integration by parts}
\bal
\an{u_m,\ph}&=\int \sin(mx)\ph(x)\,dx\\
&=\rc{m} \int \cos(mx)\ph'(x)\,dx
\end{align*}
Hence $u_m\to 0$ in $D'(\R)$. 
\end{ex}
%think about Fourier expansion of $\ph$
\begin{thm}[Closure under pointwise convergence]\llabel{thm:dist1-1}
If $u_m\in D'(X)$ is such that $\lim_{m\to \iy}\an{u_m,\ph}$ exists for every $\ph\in D(X)$, then the linear map
\[
\an{u,\ph}:=\lim_{m\to \iy}\an{u_m,\ph}
\]
is an element of $D'(X)$. 
\end{thm}
It's obvious that the LHS will satisfy the estimate or the definition. Use the principle of uniform boundedness (See the Banach-Steinhaus Theorem, 4.2.7 in FA notes).

\section{Basic operations}

\subsection{Differentiation and multiplication by smooth functions} 
If $u\in C^{\iy}(X)$, then $\pl^{\al}u$ defines an element of $D'(X)$ for every multi-index $\al$ by
\bal
\an{\pl^{\al}u,\ph}&=\int_{X}\pl^{\al}u\ph\,dx\\
&=(-1)^{|\al|} \int_X u\pl^{\al} \ph\,dx&\text{integration by parts}\\
&=\an{u,(-1)^{|\al|}\pl^{\al}\ph}.
\end{align*}
%add in f
%Makes sense for any distribution, so it's a good definition.
The RHS is well-defined for any $u\in D'(X)$. We arrive at the following.
\begin{df}\llabel{df:dist1-3}
For $f\in C^{\iy}(X)$, $u\in D'(X)$ and any multi-index $\al$ we define $\pl^{\al}(fu)$ by 
\[
\an{\pl^{\al}(fu),\ph}:=\an{u,(-1)^{|\al|} f\pl^{\al}\ph}.
\]
We call $\pl^{\al}u$ the \textbf{distributional derivatives} of $u$.
\end{df}
\begin{ex}
Take the Dirac delta $\de_x$. Then $\pl^{\al}\de_x$ is defined by
\[
\an{\pl^{\al}\de_x,\ph}:=\an{\de_x,(-1)^{|\al|} \pl^{\al}\ph} = (-1)^{|\al|} \pl^{\al}\ph(x).
\]

Consider the Heaviside function
\[
H(x)=\begin{cases}
1,&x>0\\
0,&x\le 0.
\end{cases}
\]
Then this defines an element of $D'(\R)$. We compute $H'$:
\[
\an{H',\ph}:=\an{H, -\ph'}=\int_0^{\iy} -\ph'(x)\,dx = -\ph|^{\iy}_0=\ph(0)=\an{\de_0,\ph}.
\]
Hence $H'=\de_0$. In general if $\an{u,\ph}=\an{v,\ph}$ for all $\ph\in D(X)$ then we say $u=v$ in $D'(X)$. 
%gradient infinite.
\end{ex}
Now we ask: how does the calculus for normal functions carries over to calculus for distributions?
\begin{lem}
If $u'=0$ in $D'(\R)$ then $u$ is a constant.
\end{lem}
\begin{proof}
Note that $u'=0$ means 
\[0=\an{u',\psi}=-\an{u,\psi'}.\]
%every function that is a total derive.
%{\it Note that }

{\it \blu{Idea: we'd like to say that given $\ph$, we can write $\ph=\ddd x\int_{-\iy}^x \ph(y)\,dy=\psi'$, and use the above to conclude $0=\an{u,\ph}$, so $u$ is constant. The problem is that when we integrate a test function, we don't necessarily get a test function. We need to adjust our function so that the integral is 0 for large $x$.}}

Fix $\te\in D(\R)$ with $\an{1,\te}=\int \te\,dx=1$. For arbitrary $\ph\in D(\R)$ write 
\bal
\ph&=(\ph-\an{1,\ph}\te)+\an{1,\ph}\te\\
&\equiv \ph_A+\ph_B
\end{align*}
Then 
\[
\an{1,\ph_A} =\an{1,\ph}-\an{1,\ph}\an{1,\te}=0.
\]
This is helpful because 
\[
\psi_A(x)=\int_{-\iy}^x \ph_A(y)\,dy\in D(\R).
\]
%
We have $\ph_A=\psi_A'$. So 
\bal
\an{u,\ph}&=\an{u,\ph_A}+\an{u,\ph_B}\\
&=\an{u,\psi_A'}+\an{1,\ph}\ub{\an{u,\te}}b=0+c\an{1,\ph} =\an{c,\ph}.
\end{align*}
This implies that $u$ is a constant.
\end{proof}
\subsection{Reflection and translation}
\begin{df}
For $\ph\in D(\R^n)$ then we can define its \textbf{translation} by $h\in \R^n$ by 
\[
(\tau_h\ph)(x):=\ph(x-h)
\]
and \textbf{reflection}
\[
\check \ph(x)=\ph(-x).
\]
\end{df}
(Motivation: $\an{u,\ph}=\int u\ph\,dx$ gives
$\an{\tau_h u,\ph}=\int u(x-h)\ph(x)\,dx=\int u(x)\ph(x+h)\,dx=\an{u,\tau_{-h}\ph}$.) By duality, the definitions of these operations on $u\in D'(\R^n)$,
\bal
\an{\tau_hu,\ph}&:= \an{u,\tau_{-h}\ph}\\
\an{\check u,\ph}&:=\an{u,\check{\ph}}.
\end{align*} 
\begin{lem}\llabel{lem:dist1-3}
For $u\in D'(\R^n)$ define
\[
v_h=\fc{\tau_{-h} u-u}{|h|}.
\]
Then $v_h\to n\cdot \pl u$, where $\lim_{h\to 0} \fc{h}{|h|}=n\in S^{n-1}$. 
\end{lem}
\begin{proof}
We have
\bal
\an{v_h,\ph}&=\an{\fc{\tau_{-h}u-u}{|h|},\ph}\\
&=\an{u,\fc{\tau_h\ph-\ph}{|h|}}.
\end{align*}
By Taylor's Theorem,
\bal
\tau_h \ph(x)-\ph(x)&=-h\pl \ph(x)+\ub{R(x,h)}{o(|h|)\text{ in }D(\R^n)}\\
\an{u,\fc{\tau_h\ph-\ph}{|h|}}&=\an{u,-\fc{h}{|h|}\pl \ph}+\ub{\an{u,\fc{(R(\cdot ,h))}{h}}}{\to 0\text{ as $|h|\to 0$}}\\
&=n\cdot \an{\pl u, \ph}.
\qedhere
\end{align*}
\end{proof}
This shows the distributional derivative coincides with the normal notion of derivative as difference quotient.

\blu{Lecture 4 (29 Jan)}

\subsection{Convolution between $D(\R^n)$ and $D'(\R^n)$}
If we combine the operations of reflection and translation, we get 
\[
(\tau_x\check{\ph})(y)=\check{\ph} (y-x)=\ph(x-y).
\]
If $u\in D(\R^n)$, we define the \textbf{convolution} of $u$ and $\ph\in D(\R^n)$ with
\[
(u*\ph)(x):=\int u(x-y)\ph(y)\,dy=\int \ph(x-y)u(y)\,dy=(\ph*u)(x).
\]
Using $\tau_h$ and $\check{\bullet}$, we can write 
\[
u*\ph(x)=\an{u,\tau_x \check{\ph}}.
\]
This is well defined for all $u\in D'(\R^n)$.
\begin{df}\llabel{df:dist1-5}
For $u\in D'(\R^n)$ and $\ph\in D(\R^n)$, define their convolution by 
\[u*\ph(x)=\an{u,\tau_x\check{\ph}}.\]
\end{df}
It is clear that $u*\ph(x)$ is just some function of $x\in \R^n$. It is actually smooth.
\begin{lem}\llabel{lem:dist1-4}
Define $\Phi_x(y)=\phi(x,y)$ where $\phi\in C^{\iy}(\R^n\times \R^n)$ and $\phi(\cdot, y)=0$ for $y$ outside some compact $K\subeq \R^n$. Then for $u\in D'(\R^n)$, 
\[
\pl_x^{\al} \an{u,\Phi_x}=\an{u, \pl^{\al}_x \Phi_x}.
\]
\end{lem}
We can take the derivatives inside the bracket.
\begin{proof}
By Taylor's theorem,  
%left with is linear map on $H$.
%use defn of $D'$
\[
\Phi_{x+h}(y)-\Phi_x(y)=\sum_i h_i \pd{\phi}{x_i} (x,y)+R_x(y,h)
\]
It is not difficult to show that $R_x(y,h)=o(|h|)$ in $D(\R^n)$ for each $x\in \R^n$. Hence
\[
\an{u,\Phi_{x+h}}-\an{u,\Phi_x}=\sum_i h_i \an{u,\pl{\Phi_x}{x_i}}+\an{u,R_x(\cdot, h)}.
\]
%sequence of test functions tend to 0, then continuity means $u$ acting on that sequence tends to 0.
Since $R_x(\cdot, h)=o(|h|)$ in $D(\R^n)$, dividing by $|h|$ and taking $h\to 0$ gives ($u=\fc{h}{|h|}$)
\[
n\cdot \an{u, \Phi_x}=n\cdot \an{u,\pl_x\Phi}.
\]
So the result follows.
\end{proof}
\begin{cor}\llabel{cor:dist1-1}
If $u\in D'(\R^n)$ and $\ph\in D(\R^n)$ then $u*\ph$ is smooth and $\pl^{\al}(u*\ph)=u*\pl^{\al}\ph$. 
\end{cor}
\begin{proof}
By Lemma~\ref{lem:dist1-4}, $\pl^{\al}(u*\ph)=\pl^{\al}_x\an{u,\tau_x \check{\ph}}=u*\pl^{\al}\ph$.
\end{proof}
%dist are normal functions, just take a derivative at the end.
\subsection{Density of $D(\R^n)$ in $D'(\R^n)$}
We have just seen the following.\\

\cpbox{
If $u\in D'(\R^n)$ and $\ph\in D(\R^n)$ then $u*\ph$ is smooth.}
\vskip0.15in
This is extremely useful. No matter how wild $u$ is, $u*\ph$ is nice. For this reason, $u*\ph$ is often called a  \textbf{regularisation} of the distribution $u$.
%wild creatures. give me any wild distribution, convolute with anything, get a smooth function. This is what we'll use to prove the density result.
We will use this fact to prove $D(\R^n)$ is dense in $D'(\R^n)$, i.e., for each $u\in D^1(\R^n)$ there exists a sequence of test functions $\{\ph_m\}_{m\ge 1}$ in $D(\R^n)$ such that $\ph_m\to u$ in $D'(\R^n)$, i.e., $\an{\ph_m,\te}\to \an{u,\te}$ for all $\te\in D(\R^n)$.

\blu{How to apply this: Suppose we have a problem about a distribution we don't know anything about. We know for some sequence of $\phi_m$, $\ph_m=u*\phi_m\to u$. We replace $u$ with $\ph_m$, do manipulations there, run the argument and take a limit.}


We need a technical lemma.
\begin{lem}\llabel{lem:dist1-5}
For $\ph,\psi\in D(\R^n)$ and $u\in D'(\R^n)$ we have
\[
(u*\ph)*\psi=u*(\ph*\psi).
\]
%insert dummy var
\end{lem}
\begin{proof}
The LHS is
\bal
(u*\ph)*\psi(x)&=\int (u*\ph)(x-y)\psi(y)\,dy\\
&=\int \an{u,\tau_{x-y} \wh \ph}\psi(y)\,dy\\
&=\int\an{u(z),\ph(x-y-z)\psi(y)}\,dy&\text{$z$ is ``dummy" variable}\\
&=\lim_{h\to 0} \sum_{m\in \Z^n}\an{u(z), \ph(x-z-hm)\psi(hm)h^n}&\text{Riemann sum}\\
&=\lim_{h\to 0}\an{u(z), \sum_{m\in\Z^n}\ph(x-z-hm)\ph(hm)h^n}
\end{align*}
(We want to take the $\int$ inside the $\an{\cdot}$, and we do this by turning it into a Riemann sum. Note the sum only has finitely many terms for each $m$, so this is legal.)
%sum eventually dead, only finitely many terms.
It is not hard to show that 
\[
\sum_{m\in \Z^n}\ph(x-hm)\psi(m)h^n\to \ph*\psi(x)\text{ in }D(\R^n)\text{ as }h\to 0.
\]
Continuing the above calculation,
\bal
&=\an{u(z),\ph*\psi(x-z)}\\
&=\an{u,\tau_x (\ph*\psi)\check{\,}}\\
&=u*(\ph*\psi)(x).
\end{align*}
%compact support, LHS has compact support, by fixed compact set, can differentiate
\end{proof}
\begin{thm}\llabel{thm:dist1-2}
$D(\R^n)$ is dense in $D'(\R^n)$. 
\end{thm}
We would like to say
\[
u(x)=\int\de(x-y)u(y)\,dy\stackrel?= \lim_{m\to \iy} \int \de_m(x-y)u(y)\,dy
\]
The $\de_m(x)$ are such that $\int \de_m(x)\,dx=1$ and get squashed closer and closer to $\de$. %(``a family of good kernels").
\begin{proof}
Fix $\psi\in D(\R^n)$ with $\int \psi\,dx=1$ and set
\[
\phi_m(x)=m^n \psi(mx)\qquad \pa{\int \phi_m\,dx=\int \psi\,dx=1}.
\]
%convolution of look-like de with original
Also introduce the bump function $\chi\in D(\R^n)$ with $\chi=1$ on $|x|<1$ and $\chi=0$ on $|x|>1$. Now set $\chi_m\pf xm$ and
\[
\ph_m=\pur{\chi_m(x)}(u*\phi_m)(x).
\]
%why can't $\ph$ of this guy?
(The purpose of $\pur{\chi_m(x)}$ is to make $\ph_m$ have compact support.)
Choose $\an{\ph_m,\te}$ for $\te\in D(\R^n)$ arbitrary, giving
\bal
\an{\ph_m,\te}&=\an{u*\phi_m, \chi_m\te}\\
&=(u*\phi_m)*(\chi_m\te)\check{\,}(0)\\
%&=u*\check{\ph}(0)\\
&=u*(\phi_m*(\chi_m \te)\check{\,})(0)\\
\phi_m*(\chi_m\te)\check{\,}(x)&=\int\phi_m (x-y)\chi_n(-y)\te(-y)\,dy\\
&=\int m^n \phi(m(x-y))\chi\pf{-y}{m}\te(-y)\,dy& y'=m(x-y)\implies y=x-\fc{y'}m\\
%Make the substitution 
&=\int\phi(y)\chi\pa{\fc y{m^2}-\fc xm}\te\pa{\fc ym -x}\,dy\\
&=\te(-x)+\ub{\int \phi(y)\chi\pa{\fc y{m^2}-\fc xm}\ba{
\te\pa{\fc ym-x}-\te(-x)
}\,dy}{=:R_m(-x)}\\
&=(\check{\te}+\check{R_m})(x).
\end{align*}
We can show $R_m\to 0$ in $D(\R^n)$. So \[\an{\ph_m,\te}=\an{u,\te}+\ub{\an{u,R_m}}{\to 0\text{ as }m\to \iy}.\] Hence $\an{\ph_m,\te}\to \an{u,\te},\te\in D(\R^n)$, giving $\ph_m\to u$ in $D'(\R^n)$ i.e. $D(\R^n)$ dense in $D'(\R^n)$. 
\end{proof}

 
If the prover is limited to choosing at most 1 vertex from each cluster, then the constant gap is inherented into the instances.
%for a lot of these reductions, those are what's actually causing the reductions to fail
%$k=\binom{n}{\sqrt n}$.

%constant gap: constant losses

Choosing 2 random clusters, you have likely have to check something: consistency or constraints.
%2 vertices in the same cluster?

%1 vertex from each cluster then done.
%The whole analysis is saying 
Idea: If you choose 2 vertices from the same cluster they can never give a match; you lose some edges.
%cheating strategy if choose 2 vertices from the same cluster. How is there a constant gap.

The main technique/observation is that in this setting. %total entropy has to be $\lg k$, because it's a distribution on $k$ distinct elements. 

Fix a subset of vertices $S$ of size $k$. Choose a vertex randomly from $S$, and define the following random variables:
 %each vertex corresponds to a different value of $(X_i,Y_i)$.
%The total entropy on the following induced distribution is $\lg k$: 
$X_i=1$ if the $i$th variable participates, and $Y_i$ is the assignment.
The entropy of $(X,Y)$ is 
%renaming of the vertices
%$\log k$ is the assignment.
$H(X,Y) = \lg k$ because the variables $X,Y$ give a full description of the vertex; every vertex corresponds to a unique value of $(X,Y)$.
\[
\lg k = H(X,Y) = \sum_i H(X_i|X_{<i}Y_{<i}) +\sum_i H(Y_i|X_{\le i}Y_{<i}).
\]
The first sum is the entropy from the choice of variables, and the second sum is the entropy from the assigment. 
%Can also try moshkovitz-raz
If one is large, the other is small. 
\begin{enumerate}
\item
If the second sum is large, then one must lose from consistency.
(You're choosing multiple assignments for a given set, so you're losing in the consistency checks.)

This is trickier. We'll sketch the proof below. %$O\pf{\lg k}{\lg n}$.
\item
If the first sum is large, then the 2CSP value must carry over.
(This is exactly what we want. Each variable occurs as it should be with some slack. If they occur around where they should be, the constraints appear as they should be so carry over the value of the 2CSP.)
%log k times constant.
%Proof idea: Constraints between $(u,v)$  
\end{enumerate}
Assume $\sum H(Y|X_{\le i},Y_{\le i})>5\ep_1\pf{\ln k}{\ln n}$. 
\begin{df}[Good variables]
$i$ is good if
%good amount of entropy for both first and second term.
\bal
H(X_i|X_{<i},Y_{<i}) & \ge H(\Pj[X_i=1])-2\Pj[X_i=1]\ln |\Si|\\
H(Y| X_{\le i},Y_{<i}) & \ge \rc2 \ep_1\Pj[x_i=1].
\end{align*}
\end{df}
%$\ep_1$ depends on...
%first split up the sum into those that satisfy the first and second connditions. Look at the ones satisfying the second condition, and you're done.
%entropy concave
We get
\[
\sum_{\text{good $i$'s}} \Pj[X_j=1]^2 \ge \fc{\pa{
\rc 5\ep_1 k
}^2 }{n(\ln |A|)^2}
%density: squre has to come in
\]
%pick 2 random vertices
%Go through all the $i$'s and look at inconsistencies from the $i$th variables. Use Fano's inequality. The bound is bad. If $X_i$ is inconsistent, you shouldn't count when $X_j$ is inconsistent. We actually want to count the new inconsistency introduced by the $i$th variable.
We say $\om_{i-1}$ is good if 
\begin{itemize}
\item
$i$ is good,
\item
%neighborhood: count the new inconsistencies. For each vertex, we don't want to count the ones we have already counted inconsistencies before.
%look at the vertices inconsistent up to variable $i$.
$\sum_{\si_{i-1}\in N(\om_{i-1})} \Pj[X_i=1|\si_{i-1}]\ge (1-\ep_2)\Pj[X_i=1]$.
%set of variables ocnsistent up to variable $i$.
There is substantial mass on the neighborhoods.
%need good neighbors: a good chunk are cut up.
\item
$H(Y_i|\om_{i-1})\ge \ep_3\Pj[X_i=1|\om_{i-1}]$.
%introduce $i$th variable, this could be large. but inconsistencies could have been counted before. 
%intro $i$: fresh violations? 
\end{itemize}

$\text{den}(S)<1-\de_1$, $\text{den}(S)<1-\de_2$.

%sparse? open problem.
%simpler proof using more entropies.

We tried applying this to small set expansion Here $k=\binom{n}{\sqrt n}$, $|V|=\cdots |\Si|^{\sqrt n}$. But in small set expansion we need $k$ to be linear.

Bounding integrality gaps: the amount you can cheat if you assign something to a local set. Nice technique for ``pick $k$" problems. Sparse PCA? But in the hidden-clique based reduction, we rely heavily on the background noise being random. 

\section{Bitcoin and game theory}

I'll speak tomorrow about some game theoretic attacks against bitcoin from these two papers (and give a brief high-level overview of bitcoin first):

\url{http://arxiv.org/abs/1311.0243}

\url{http://arxiv.org/pdf/1411.7099v2.pdf}

Bitcoin is ``just" a ledger. (Don't think of bitcoins as digital objects moving. Because people agree on a ledger, the coins have value.)

Every node $i$ has a ledger $L_i$.

Goal: get all nodes to agree on $L$.

%

\begin{enumerate}
\item
If two parties want to transact, they announce it to the world.
\item
At every time $t$, the network selecs a random node $i$ (miner) and this node proposes a list of transactions and a pointer to $L_i$.
\item Every other node validates all these transactions. Update $L_j$ to be the ledger that has been updated the most. (Accept updates from the ``correct" node.) (Longest blockchain.)
\end{enumerate}

We want to prevent the double-spend attack. 
\begin{enumerate}
\item
Assume at least $51\%$ of the network is honest.
\item
Wait $k$ steps after a transaction is proposed before considering it safe.
\end{enumerate}
In order to undo a transaction from time $t$, for some $C\ge k$, the attacker gets chosen more often than the rest of the network.
%if never happens for any $C$, keep building on history.

``Proof of work." We replace the above steps by the following.
\begin{enumerate}
\item[0.] Everyone agrees on a collision-resistant hash function $H$ with 256-bit output.
\item[2'.] At any time any node can propose updates$|L_i|$nonce.
\item[3'.] Accept updates if $H\pat{updates|L|nonce}<2^c$ for some agreed $c$.
\end{enumerate}•

For some agreed $C$. Set $L_j=$most updated ledger.

If you successfully propose transactions, you get to add a special transaction creates coins for you. This is called ``mining."

\subsection{Selfish mining}
A selfish update his own blockchain without releasing it until later, so the rest of the network is wasting its time. Then you have a higher proportion of effective power in the network.

If you are in a position where you can always win a tie: immediately report when someone else reports.
%use energy to get bitcoin, hurt its value?
%goldfinger attack

Define lead$=l-k$.
There are 2 ways to have a lead of 0: (0.) you agree, or ($0'.$) you found both found the same number of blocks.

Let $\al$ be your fraction of computing power.
From 0 you can get to 1. 
Let $\ga=\Pj\pat{choose to extend yours}$. 
\begin{itemize}
\item
$0\xra{1-\al}0$.
\item
$1\xra{1-\al}0'$.
\item
$1\xra{\al}2$.
\item
$2\xra{\al}1$.
\item
$0'\to 0$: $\ga(1-\al)$ win, $(1-\ga)(1-\al)$ lose, $\al$ you found another block and win.
\end{itemize}•
Find the stationary probabilities; how much does each person get?

The fraction of revenus is $>\al$, if $\al>\fc{1-\ga}{3-2\ga}$. 
%more incentives for people to work with them. More rewards then... Larger than $50\%$?

%not happen? What is $\ga$?

%can detect, decrease value of BTC.

\section{Lifelong learning}

\subsection{Intro}

In PAC learning for boolean functions, we consider (feature domain) $X=\{0,1\}^n$, $Y=\{0,1\}$, distribution $D$ over $X$, $t:X\to Y$. Given $\set{x_i}{t(x_i)}_i\sim D$ iid, we want $h(x)$ such that 
\[
\Pj(h(x)\ne t(x))
\]
is small.

In lifelong learning, we have a sequence $(t_i,D_i)_{i=1}^m$ (arriving in online fashion), $t_i$ sharing meta-features $\{m_j\}$ (common structure).

There are different learning tasks with common structure. There may be common features involved in identifying dogs and identifying cats. 

The goal is to learn the meta-features $\{m_j\}$ and the target function $\{t_j\}$.

%neural network lit

\begin{ex}
$t_i$ are conjunctions. There exist $k$ monomials such that $t_i$ is a conjunction of some subset of the monomials.

For example, $t_1=x_1x_2x_3$, $t_2=x_2x_3x_4$, and $m_1=x_1x_2,m_2=x_2x_3,m_x=x_4$. Here $t_1=m_1m_2$, $t_2=m_2m_3$.
%metafeatures may not be unique
\end{ex}

\begin{ex}
$t_i$ is a linear separator, and $\{t_i\}$ are in a $k$-dimensional subspace. 
%common things among different learning tasks. 
%nice way capture metafeatures.
\end{ex}
Because the set of meta-features is not unique, it suffices to output $\{\wt{m_j}\}$ that has equal representational power.

\subsubsection{Conjunctions}

First consider the \vocab{consistency} problem: given $\{t_i\}_{i=1}^m$, find $k$ monomials $\{\wt{m_j}\}$. This is the set basis problem which is NP-hard.
%There are $k$ monomials, and you want to find 
Thus we make the \vocab{anchor variable} assumption: for all $m_j$, there exists $y_j\in m_j$, $y_j\nin m_{j'}$ for $j'\ne j$.
%approx ration of set-basis problem.

%given a set of examples, is there a target function that is satisfied for all examples.

Once you've found the anchor variables it's easy to find: let $\wt{m_j} = \bigcap_{t\ni y_j} t$.

The algorithm is as follows. Let $TS=\{t_i\}_{i=1}^m$,
\bal
NS(TS,Z)&=\set{t\in TS}{z\in t}\\
H(t)&=\bigcup \set{\wt{m_j}}{\wt{m_j}\subeq t}.
\end{align*}
Repeat:
\begin{enumerate}
\item
Set $t=0$, $\wt M=\phi$ (metafeatures selected so far).
\item
While $t\in TS$ cannot be represented by the current metafeatures, $t\ne H(t)$,
\begin{enumerate}
\item
Choose $t$ with minimal index.
\item
Minimal: Choose $z_{k+1}$ from $t\bs H(t)$ 
there doesn't exist $z'\in t\bs H(t)$, $N(TS,z')\subeq N(TS,z_{k+1})$. %contained by fewest metavariables
%not nec anchor var.
%anchor var for current situation, not at beginning.
\end{enumerate}
\item 
Set $\wt{m_{k+1}}=\bigcup_{T\ni z_{k+1}}T$.
%if not anchor variable, acts like anchor variable
%find with same representational power.
\end{enumerate}

\begin{df}
Say $\wt m\equiv m$ if 
\begin{enumerate}
\item
(larger than ground truth metafeature)
$m\subeq \wt m$ 
\item
(not too large)
for all $t\in TS$, if $t\supeq m$, then $t\supeq \wt m$.
%still contained in conjunction $t$, holds for any $t$.
%if for each ground truth metafeature...
%why we define $\wt{m_j}$ in this way. 
%take intersection over all conjunctions.
\end{enumerate}
\end{df}

The guarantee of the algorithm is the following.
\begin{lem}
Let $\{\wt{m_j}\}$ be the output of the algorithm.
For $\wt{m_j}$ there exists $m_{t_j}$, 
\begin{enumerate}
\item
$\wt{m_j}\equiv m_{t_j}$ (each constructed feature maps to a ground truth metafeatures)
\item
$m_{t_j}\ne m_{t_{j'}}$, $j\ne j'$.
\end{enumerate}
(In particular, the number of ground truth metafeatures is at most $k$.)
\end{lem}
\begin{proof}
Induct. The base case is clear.

Assume that it's true for $i'\le i$. Now we show the theorem for $i+1$.

We have 
\[
\wt{m_{i+1}}=\bigcap_{z_{i+1}\in T} T.
\]
$m_{t_{i+1}}$ is the $m$ containing $z_{i+1}$, $m\subeq t$.
\begin{enumerate}
\item[2.]
It's not covered by the previous: OK.
\item[1.]
Problem: $z_{i+1}$ may not be an anchor variable.
Suppose $z_{i+1}$ is not an anchor variable. Then $y_{t_{i+1}}\in \wt{m_{i'}}$. We claim that $m_{t_{i+1}}\subeq \wt{m_{i'}}$.
%
Now $\wt{m_{i'}}=\bigcup_{z_{i'}\in T}T$, so $y_{t_{i+1}}\in T$. %not contained in any other
So $m_{t_{i+1}}\subeq T$.
%We claim the corresponding anchor var isn't covered yet


We claim $N(TS,z_{i+1})=N(TS,y_{t_{i+1}})$. Choose the minimal variable in the sense that if you want to construct the metafeatures...; pick anchor variable at each step and construct a metaf equivalent to ground truth metaf.
%(*(/)*)
%
\end{enumerate}
\end{proof}
%how many problems need to see to be confident have all metaf?

Given such a subroutine, each time you receive a new learning task, try to construct it from the previous metafeatures. If you succeed with error 0, go to the next task. Otherwise, take all the learning tasks up to now, and run this algorithm again. Whether this algorithm is efficient depends on whether you can learn the conjunction under the previous metafeatures. Under some query models you can do this efficiently.

Can you learn the new conjunction based on the previous metafeatures? %learn new func as conjunction of previous metafeatures? Otherwise learn by some more expensive means and ...

%each time you get a new learning task, 
%is there a conjunction you can fit now. This is just a conjunction learning problem.

%put conjunction together with all previous to learn.
%keep set of basis of constructed hypothesis.

%technical error: error accumulates a little additively.

%metaf as hidden layers
\subsubsection{Neural nets}
Consider the $t_i$ as images. Metafeatures are ANDs of corresponding variables. If there is a pattern in the image, the corresponding feature is activated.
%when $t_i$ contains all $x_1,x_2,x_3$. 
Then take OR of the metafeatures.
%DNF

%To reconstruct $t_i$ from the $m_j$...
Think of the metafeatures as the hidden layers; this is the corresponding boolean autoencoder.
%union vars in corresponding metafeature.
Here the hidden layer is the same size. What if it's large, but we require sparse coding. We can only use a few metafeatures to reconstruct. Each $t_i$ is a conjunction of at most $k$.

Sparse boolean autoencoder: $\{t_i\}_{i=1}^m$, $\{m_j\}_{j=1}^L$, $L=O(n)$, $t_i$ is a conjunction of $\le k$ of the $m_j$. They consider $|m_j|\le C$. %run in $n^C$.

The assumption is that for all $m_j$, there exists $y_j\subeq m_j$, such that for all $t\supeq y_j$, $t\supeq m_j$, $|y_j|\le C$.
Any containing this set must contain the whole metafeature. This is more general than the anchor variable assumption (for which $|y_j|=1$).  $|y|\le c, \wt{m_y}=\bigcup_{y\in T}$. This corresponds to learning polynomials.
(Anchor words assumption is stronger than the unique neighbor property. Fix $m_j$, there is $x_i$ only appearing in it.)

%y_j only contain in m_j

%if both on, something else should be on...

%more of a philosophical connection.

\section{Noise in computation}

We'll talk about the circuit computation model. (There might be other models that are appropriate but this is the most obvious.)

Some motivations:
\begin{itemize}
\item
The relation of communication to computation.
\item
Understanding computation.
\item 
Practical considerations.
\end{itemize}

The short history: people (von Neumann) cared about it in the 50's, people made gates that have high reliability and stopped thinking about it. In practice most architectures assume the computation is infallible. However there is room for improvement in terms of power consumption.

The basic setup is that we have a circuit with some gatewise noise.
Define
\[
\text{NAND}_\ep(x,y) = \begin{cases}
\neg x \wedge y,&wp\,1-\ep\\
x \wedge y,&wp\,\ep
\end{cases}
\]
What/how is this good for?

Given a circuit $C$ of NAND's is there a $C'$ of $NAND_\ep$'s that does the same thing and is there a meaningful expansion parameter? We can allow $C'$ plus $o(|C|)$ noiseless gates (or allow some noise). Spielman in the 90's had the following solution: store each word as an error correcting code. Do operations on these words. This causes an expansion of $\ln |C|$ (the length of the word).
%should stay within the radius.
You have to allow a threshold gate at the bottom. 

Can we get constant in some meaningful regime? There is no inherent reason why you shouldn't be able to.

We start with the communication connection. We can ask this problem in any setting. One successful setting is data transmission and storage. (We can't assume long-term memory is noiseless!) There is the concept of channel capacity. The capacity
\[
Cap(BSC_\ep)=1-H(\ep).
\]
This tells us how much more communication we need given an amount of noise. The profound fact is not that this is the conversion factor, but that there is one. This is the big discovery of Shannon. %What you use the channel for is independent
You can separate the encoding from the application; it wasn't obvious in the 40's.
Is this true here?

A more accessible model is the interactive error-correcting codes: you can assume there is a lot of communication, but each communication is small. You carry a constant loss.
This is smaller than one-way capacity. The capacity hasn't been formally defined. Given a generic pointer problem, what's the factor you need to blow up? It's a constant factor.

This is connected via the Kharmarcher-Wigderson games. This is a connection between boolean formulas and interactive protocols.
A boolean formula $f(x_1,\ldots ,x_n)$ corresponds to the interactive protocol: $A$ is given $x\in f^{-1}(0)$ and $B$ is given $y\in f^{-1}(1)$. Output $i$ such that $x_i\ne y_i$. 
This is still the most promising attack to get lower bounds.
If you prove a lower bound on the game you prove a lower bound on the depth.
Starting at a disagreement at the bottom of the circuit (output), move up. Each bit they jump 1 layer by sending the index of the 1 or 0 in the previous layer (depending on whether it's AND or OR). 

This protects against short-circuit errors (Kalai, L, R 2013). A short-circuit error means that we can have $AND(0,1)=1$ but not $AND(0,0)=1$. It doesn't protect against flips.

Interactive error correction is just an inspiration for the kind of results we want.

We don't know the error for the simplest channels. It's bounded by above by the one-way capacity; we don't know how big the gap is.

Defining a constant and asking whether it's positive is interesting. 

The problem we gave is analogous to channel capacity. This correspond to asking does the conversion to $NAND_\ep$ have constant rate. %(A superconstant rate is still interesting.)

What is the analogue of entropy? Given $T$, a circuit $C$, what does $\fc{C(T^n)}{n}$ approach? What is going on for multiparty computation. This would be a great deal: try to break out of discreteness. Coding theory and information theory are more successful than complexity theory because they are not discrete.

%This is a concrete problem %positive rate

Next problem: 
The more abstract problem is that given $f:[0,1]\times [0,1]\to [0,1]$, when can you use it for computation?
%given by noisy gate
Given $f:[0,1]\times [0,1]\to M[0,1]$. (The output is a distribution on the interval.)
%f(x,y)\pm 0.1
Is it meaningful to talk about the log factor?
The relationship between error and power (energy consumption of a circuit) is $e^{-\Te(p^2)}$.

For a lot of time the limiting cost was hardware but this is changing now: hardware is cheap relative to power.
We cannot hope to reduce it by much. Reducing it from $10^{-18}$ to $10^{-9}$ is a $30\%$ reduction. (Error in lifetime vs. error per second.)
Checksum for interesting computation.
Ex. A multiplier that can withstand three errors. You're not allowed to blow up the size of the circuit. Ideally verify the computation in sublinear amount of gates compared to the original computation. 
%locally constructible codes?
1 error: checksum.
PSPACE solver. Usually call more than once. Call the solver once. Allowed sublinear overhead. Only need to correct few errors, not blackbox. Multiplier takes superlinear number of gates.

In terms of approachability, the three most approachable directions are
\begin{enumerate}
\item
abstract: compute capacity (analogue of channel capacity)
\item
specific problems: checking for low complexity classes. Merge PCP theory with low-level complexity classes?
\item 
practical stuff. (Dead for a while, but revived because of the power problem.)
\end{enumerate}•
Solved problem in wrong regime. 
%problems face in practice, memory addressing, not what can help with?
%In particular situations there is interesting stuff.

%log^{O(1)} \fc{|C|}{\ep}
%prover verifier games more useful? overhead sublinear.

%interactive coding: hopeless about exact constant. Dependence $1-\Te\prc{\ep}$.

%interactive coding.
%IT: cc brick walls. can't just keep Shannon

Seems unrelated: A simple proof of the following fact. A cellular automata can simulate a Turing machine. Can you do it robustly? 
In one dimension you can keep one bit of information (Gacs, 70s). With a more modern understanding, simplify?
%Godel's theorem
Godel's theorem for logic of computation is a trivial fact (there's pain in proving the arithmetic version).
1-dimension: store a bit. Trying to fight mutation. Build an immune system. Mutation with mechanism for doing exactly what you're trying to do: once in $2^{-n}$ steps have a mutation of size $n$. In $2+$ dimensions you can do robust computation.
%2002 wolfram. 1200 page book. 1/2 page to noise.
Two states that don't mix. Kill all neighbors not like you, etc. What if overpower, etc.

%cell automata: discrete PDE, nasty math. Rule simple not right measure of complexity. Right measure is memory. Fluid mechanics hardest applied math. Can't prove have solutions.

\section{Dictionary learning with VERY few examples}

%Title: Teaching and compressing for low VC-dimension
%http://arxiv.org/abs/1502.06187
%
%Authors: Shay Moran, Amir Shpilka, Avi Wigderson, Amir Yehudayoff
%
%Abstract:
%
%In this work we study the quantitative relation between VC-dimension and two other basic parameters related to learning and teaching. We present relatively efficient constructions of {\em sample compression schemes} and {\em teaching sets} for classes of low VC-dimension. Let C be a finite boolean concept class of VC-dimension d. Set k=O(d2dloglog|C|). 
%We construct sample compression schemes of size k for C, with additional information of klog(k) bits. Roughly speaking, given any list of C-labelled examples of arbitrary length, we can retain only k labeled examples in a way that allows to recover the labels of all others examples in the list. 
%We also prove that there always exists a concept c in C with a teaching set (i.e. a list of c-labelled examples uniquely identifying c) of size k. Equivalently, we prove that the recursive teaching dimension of C is at most k. 
%The question of constructing sample compression schemes for classes of small VC-dimension was suggested by Littlestone and Warmuth (1986), and the problem of constructing teaching sets for classes of small VC-dimension was suggested by Kuhlmann (1999). Previous constructions for general concept classes yielded size O(log|C|) for both questions, even when the VC-dimension is constant.

Authors: Klye Luh, Van Vu in FOCS 2015

We consider the equation
\[
Y=AX
\]
where $Y$ is $n\times m$, the input matrix; $A$ is $n\times n$. %More variables
For dictionary learning, $X$ is sparse and $A$ are the items in the dictionary. When $X$ is extremely sparse (each column has constant number of zeros) this problem is also called sparse coding.

\begin{thm}
If $A$ is full rank and $X_i$ are iid sampled from 
$\smatt{1-\te}{\te}{0}{\xi}$ where $\xi$ is such that $\E\xi =0$, $\te\in [\rc{\sqrt n},\rc{\sqrt n}]$. $\Pj(\xi\ge t)\le e^{-\fc{t^2}2}$.
%With probablity roughly $\rc{\sqrt n}$ nonzero.
\end{thm}
There exists a polynomial time algorithm that exactly recovers $A,X$ when $m\ge n(\ln n)^4$.
Information theoretically we can go to $n\ln n$ because we need that many equations to certify the variables. 

($n^2$ many variables, $\ln n$ bits of entropy from choice in column.)

$\spn(Y')=\spn(X')$. Each row of $X$ is a sparse vector.
Dimension much higher than number of vectors.
It makes sense that the sparsest vectors in the span are the rows of $X$. If we can solve
\[
\min_{b\in \R^n} \ve{b^TY}_0,
\]
$b\ne 0$, $b^Ty X^ic$. We can recover at least one $X^i$. But these are nonconvex constraints. The first constraint is easy: the $L_0\to L_1$ relaxation,
\[
\min_{b\in \R^n} \ve{b^TY}_1.
\]
The second constraint is problematic. (Dropping it we get $b=0$ as the solution.) We restrict $\an{b,v}=1$. This is a trick from Spielman. $v$ is a column of $Y$. 
%TM: view as $L_{\iy}$ norm.

Ideally we want $b^TY=X^i$. $b^TAX=X^i$. We should have $b=(A^T)^{-1} e_i$. It's natural to put $v=A^Te_i$.
%larger value will not be min
We cannot really get this $v$ because we don't know $A$. The analogue of $v$ is the column $Y_j=A X_j$. The $j$th column is quite sparse, so it's like $e_i$. If we choose $v=Y_j$, can we solve the problem?

The naive approach is to minimize $\ve{b^T Y}$ such that $\an{Y_j,b}=1$. Make a change of variables $b=(A^T)^{-1}r$. Then equivalently we want $\min\ve{r^Tx}$ such that $\an{x_j,r}=1$.

Observation: for all $r$,
\[
\ve{r^TX}_1\approx mf(\ve{r}_1).
\]
$f$ is monotonically increasing. If $X$ is 1-sparse, equal to 1-norm. If $X$ is sparse, roughly proportional. $m$ summations of inner product, grows linearly as $m$.

For the simple case, consider $\ve{r^T x}_1  = mf(\ve{r}_1)$. Then the convex program is equivalent to 
\[
\min \ve{r}_1
\]
such that $\an{r,X_j}=1$. $X_j$ is a sparse column vector with $\pm 1,0$.
Putting $\rc2$ on $1,0$... This convex program does not recover ...

The trick: don't naively consider $\an{r,X_j}=1$. Consider $\an{r,X_p+X_q}=1$.
Similar argument with unique neighbor: with decent problem have unique common nonzero place. With high probability same sign. A 2. This gives $b_i$. You can put $\rc2$ on the $2$. Norm can be as small as $\rc2$. This is robust to approximately equal. Suppose only have approximately equal. 
We want to minimize
\[
\min \sum (1\pm S_i) |r_i|.
\]
such that $2r_1+r_3-r_6=1$ (for example). No matter what $\de$ is, still unique solution: put $\rc2$ at $r_1$. %lose 2 gain $\de$.

Final algorithm: go through $p,q$, 
\[
\min \ve{b^TY},\an{b,Y_p,Y_q}=1.
\]
$X_{pq}=b_{pq}^T Y$. Add $X_{pq}$ from $\ve{X_{pq}}_0$ smallest to largest.
...sparsest until rank is $n$.
Recover $X$ can recover $A$ because whp $X$ is right-invertible. 
Actually recover coefficients of dictionary first, use the representation to recover the dictionary. 
Until rank$=n$.

Observation: for $\ve{r}_0\le \sqrt n$ (sparse $r$), $\E\ve{X^T r} = mf(\ve{r}_1)$. Consider $X$ is 1-sparse, just 1 norm of $r$. If $\sqrt n$ sparse...

Also using unique neighbor property hit one value.

For dense $r$, do some calculation, still increasing function of $r$. Expectation calculation.

Step 2: for all $r$, $\ve{\ga^TX}_1$ concentrate to $\E\ve{r^Tx}_1$. (Spielman used Bernstein, this is fancier.)
%for every $r$, concentrate to expectation. For every $r\in \R^n$, have to take union bound.

\[
\Pj(\exists r\ve{\ga^Tx}_1-\E\ve{r^Tx}\ge \de\E)\le o(1).
\]

%if proportional doesn't hold for non-sparse, cheating solution, don't want.

target solution is 1-sparse, first step rule out $\sqrt n$. How prove theorem? 

How did Spielman prove this theorem? $\ve{r^Tx}=\sum_{i=1}^m|\an{r,x_i}|$. 
\[
\Pj(\ve{r^Tx}_1-\E\ve{r^Tx}_1) \le \exp\pa{-\fc{t^2}{\Var \ve{r^ix}_1}},
\]
\[
\Var\ve{r^Tx} = \sum_{i=1}^m \Var|\an{r,X_i}| = \sum_{i=1}^m \an{r,X_i}=\sum_{i=1}^m\sum_{j=1}^n \E(X_{ij}^2r_j^2).
\]
Square, cross term cancel. With prob $\te$ nonzero, once nonzero
\[
=\sum_{i=1}^n \te\sum_{j=1}^n r_j^2.
\]
Why old thing cannot give optimal.
Because expression in $\Pj$ is scaling-invariant, if true for 1-norm $=1$, true for all $r$.
Here if the 1-norm $=r$, then value $<1$. 
\[
\le \te rn.
\]
Then
\[
\Pj(...\ge t)\le \exp\pf{-t^2}{\te m}.
\]
How small can the expectation be? $\E\ve{r^Tx}_1 = m\E\ve{r_i,x_i}$. $(\rc{\sqrt n}\cdots)()$. WHen nonzero iid $\pm1$. %\ve{r}_1.
Sum of $\pm 1$s are $\sqrt{\te n}$. 
\[
\min \E\ve{r^Tx}_1 = m\sfc{\te}{n}.
\]
$t=\sfc{\te}n$. 
\[
\exp\pf{-t^2}{\te m}\le \exp(-m/n).
\]
Union bound: $e^m$ many $r$'s. Only consider $r$s differing by $\ge\rc{n^3}$. $\rc{n^3}$ $\ep$-net, contain $e^n$? many points.

$r_i=\fc{k}{n^4}$ where $k\in \N$. %If some point 
How many with $\ve{r}_1=1$? $e^n$. $=2^{n^4}$. %(n^4)^n.

Union $e^ne(-\fc mn)\approx o(1)$ so get $n^2$. %
%r_j^2<1
%$\E$: lower bounded. 

Variance bound only tight when 1-sparse: that's how they improve analysis.

When take union bound, $\Pj(A\cup B)=\Pj(A)+\Pj(B)-\Pj(A\cap B)$. For a union bound we ignore the last term. But it's large for many different $r$'s. Two close by points would be close in value whp. 
\[
\ab{\ve{r_1^Tx}_1-\ve{r_2^Tx}_1}\le \ve{(r_1-r_2)^T x}_1
\]
Consider this term, apply concentration on this term. $r_1-r_2$ has property that each term is small. Concentrate better than naive approach. 

Why does the intersection help here? When apply concentration bound...
singular vectors is 2-norm. When 2-norms only shaves log at best. Like ARV. Now the difference is $\rc{n^4}$. 

Coarser net $2\ep$-net. 
\[
\Pj(\exists \text{bad point})= \Pj(\exists 2\ep\text{-bad point}) + \Pj(\exists \text{bad point}|\exists 2\ep\text{ points all good}).
\]
Much smaller than union bound. just consider 2 adjacent. Repeat again and again. Make sure the second term is very small. 

%Finally 
Amazing property of 1-norm:
if you build an $\al$-net on $\ve{r}_1=1$, $r_1=\al \Z$, then the number of net points is $\exp\pf{\ln}{\al}$. If $\al=\rc2$, only polynomially many points. $L_{\iy}$ net because each index. View $L_{\iy}$ net on $L_1$ norm, get exponentially decrease in number of points. ...
When net becomes coarse you eventually get polynomially many.

What about $L_2$? $L_2$ $\ep$-net. 

From $\ep$ net $\ve{r_1-r_2}_{\iy}\le \al$. Goal:
\[
\Pj(\ve{r_1^Tx}_1-\ve{r_2^Tx}\ge m\sfc{\te}n) \le \exp\pa{
-\fc{100\ln n}{\al}
}.
\]
Summation from $\rc{n^4}$ to 1 still small. 
To show the inequality, cut the net according to expectation. Concentrate to expected value. Expectation $m\sfc{\te}n$ to $m\te$.
Observe
\[
\E(\ve{r_1^Tx}_1-\ve{r_2^Tx}_1)\ll m\sfc{\te}n.
\]
%blow up net by poly factor
\[
\ve{r_1^Tx}-\ve{r_2^Tx}_1\le \ve{(r_1-r_2)^Tx}_1.
\]
Why use $L_{\iy}$ $\ep$-net. Want to use $\al$.
\[
\sum_j^n(r_1-r_2)_j^2<\al^2 n.
\]
%2 norm less than $\iy$-norm.
Move the summation to 1-norm; know 1-norm is 1. Use $L_2$ $\ep$-net. 
Now we get 
\[
\Pj(\ve{r_1^Tx}-\ve{r_2^Tx}\ge m\sfc{\te}n)
\le \exp\pf{\pa{m\sfc{\te}n}^2}{\te m \al}=\exp\pf{m}{n\al}
%-100 ...
\]
$L^2$ gives $\al^2$. The number of points is $\rc{\al}$, the concentration is also $\rc{\al}$. Real paper have to bound the maximum. 

%random gaussian, look at one with largest inner product. Talagrand's theory.

%template file: https://github.com/holdenlee/templates

\section{High-rate locally-correctable and locally-testable codes with
sub-polynomial query complexity}

Abstract:

Locally-correctable codes (LCCs) and locally-testable codes (LTCs) are
special families of error-correcting codes that admit extremely
efficient sub-linear time algorithms for error correction and
detection, respectively, while making a few queries to the received
word. These codes were originally studied in complexity theory because
of their close relationship to program checking and
probabilistically-checkable proofs (PCPs), but subsequently they were
extensively studied for their own sake.

In this work, we construct the first locally-correctable codes and
locally-testable codes with constant rate, constant relative distance,
and sub-polynomial query complexity and running time. Specifically, we
show that there exist binary LCCs and LTCs with block length n,
constant rate (which can even be taken arbitrarily close to 1),
constant relative distance, and query complexity and running time
$\exp(\sqrt{\log n})$. Previously such codes were known to exist only
with query complexity and running time $n^{\beta}$ (for constant $\beta
> 0$), and there were several, quite different, constructions known.

Along the way, we also construct LCCs and LTCs over large (but
constant) size alphabets, with the same query complexity and running
time $\exp(\sqrt{\log n})$, which additionally have the property of
approaching the Singleton bound: they have almost the best-possible
relationship between their rate and distance. This has the surprising
consequence that asking for a large alphabet error-correcting code to
further be an LCC or LTC does not require any sacrifice in terms of
rate and distance! Such a result was previously not known for any
sub-linear query complexity.

If time permits I will also discuss a very recent work that
substantially improves the query complexity of LTCs to
quasi-polylogarithmic function of the form $(\log n)^{O(\log \log n)}$.

Joint work with Swastik Kopparty, Or Meir and Shubhangi Saraf

An error-correcting code is a map $C:\Si^k\to \Si^n$. $\Si$ is the alphabet, $k$ is the message length, and $n$ is the codeword length.

Two basic parameters are the rate $\fc kn$ and the distance
\[
\min_{u,v\in C} \{d(u,v):=\Pj_{i\in [n]}(u_i\ne v_i)\}
\]
It equals twice the proportion of errors the code can correct (picture).

There are 3 operations associated with  codes.
\begin{itemize}
\item
Encoding: compute map $C$. %assume know map and image
\item
Testing/detecting errors: test $u\in C$.
\item 
Correcting: find $c\in C$ nearest to $u$.
\end{itemize}•
We're interested in sub-linear time algorithms that do these operations. We want them to run in time $\ll n$. They can't even make 1 pass over the codeword! We need several assumptions.
\begin{itemize}
\item
Random access: they can access each coordinate with some cost.
\item
Probabilistic (has a small chance of failure).
\end{itemize}
The first family is locally testable codes (LTC's), which have sublinear time testing algorithms.
\begin{df}
A \vocab{local tester} is given as input $u\in \Si^n$ and outputs accept/reject. It satisfies the following:
\bal
u\in C&\implies \Pj\pat{accept}=1\\
u\nin C&\implies \Pj\pat{reject}\ge 0.99 d(u,C).
\end{align*}
The rejection should depend on the distance, because if it just differs in one location, you probably won't find it.
%(We are mostly interested in the constant fraction of errors regime.)
\end{df}
Locally correctable codes are codes with sublinear time correcting algorithms. 
\begin{df}
A \vocab{local corrector} is given as input $u\in \Si^n$ such that $d(u,C)<\rc2d(C)$ and $i\in [n]$ and is required to output the $i$th coordinate of codeword closest to $u$ with probability $\ge 0.99$.
\end{df}
\begin{df}
A \vocab{local decoder} is given as input $u\in \Si^n$ such that $d(u,C)<\rc2d(C)$ and $i\in [k]$ and is required to output the $i$th coordinate of the message $m$ such that $C(m)$ is the codeword closest to $u$ with probability $\ge 0.99$.
\end{df}
LDC's seem more natural, but LCC's imply LDC's.
If you can decode every coordinate of the codeword, you can decode every coordinate of the message.
\begin{fct}
For linear codes, LCC's imply LDC's.
\end{fct}
Our main goal is to minimize the number of queries and the running time. 
%In coding theory, reject in the rate . coding and correcting errors.
%only computational perspective.
%what running time, parameters

\subsection{Previous work and motivations}
Most of the research has focused on a constant number of queries and constant distance. The goal is to maximize the rate.  People studied this regime because it would be powerful; sometimes you can do amazing stuff with 2--3 queries (PCP). Studying this regime has applications in complexity theory. They are connected to property testing, hardness of approximation, etc.

In practice, all known constructions of such codes have vanishing rate: as $n\to 0$, rate goes to 0. This is very bad for applications in data transmission and storage.

For LTC's,
\begin{itemize}
\item
the best constructions have rate $\rc{\polylog n}$, due to Ben Sannon-Sudan 2005, Dinur 2007, Meir 2009, and Videman 2013.
\item It is not known if we can achieve a constant rate. This is the biggest open problem in this area. 
\end{itemize}• 

There is an implicit connection to PCPs. Every time someone gets a better LTC, with the same techniques people get a better PCP, and vice versa. However there is no formal connection.

For LCC's, 
\begin{itemize}
\item
we cannot get constant rate (Katz-Trevisan 2000). 
\item
Reed-Muller codes give $\exp(-n^{\ep})$, and there are LDC's with rate $\exp(-\exp(\sqrt{\ln n \ln\ln n}))$ by Yekhanin 2007 and Efremenko 2009.
\end{itemize}•
\subsection{High-rate regime}

We are interested in constant rate and constant distance, and we want to minimize the number of queries. Ex. $\log n$ queries is fine.

The practical motivation is to encode massive data.
\begin{enumerate}
\item
Solution 1: break into chunks and encode each part separately.

Pro: The recovery time is small. We only have to read bits that are close.

Con: in the presence of errors, the failure probability is the number of pieces times $\exp(-|\text{small}|)$.
\item
Encode the message in one big chunk.

Pro: Failure probability is $\exp(-|\text{big}|)$.

Con: The recovery time is |big|.
\end{enumerate}
%Encode big chunk at once. In order to recover a coordinate, only need to read a few.
What is known? Up to a few years ago, the only known examples were Reed-Muller. For all $\ep>0$, you can have a code with number of queries is $n^\ep$ and the rate is $\exp(-\rc{\ep})$. This was believed to be best possible.

Actually you can do better. For every $\ep>0$, you can have a code with queries $n^\ep$ and rate $1-\ep$.

Here are known constructions.
\begin{itemize}
\item
The first codes were multiplicity codes (LCC's), generalization of Reed-Muller codes. In Reed-Muller codes you evaluate polynomials; here you evaluate polynomials and their derivatives (Kopparty-Saraf-Yekhanin 2011).

Algebraic flavor.
\item
Tensor codes (LTC's): BenSasson-Sudan 2006, Videman 2011.

Combinatorial flavor.
\item
Lifted Reed-Solomon codes (both LTC's and LCC's), due to Guo-Kopparty-Sudan 2013.
\item
Expander codes (LCCs), due to Hemenway-Ostrovsky-Wooters 2013.

Graph-theoretic.
\end{itemize}•
\subsection{Our results}
We show that we can improve the number of queries to be subpolynomial. We give high-rate LTCs/LCCs with subpolynomial number of queries
\begin{thm}
For every $\ep>0$ there is a LCC and a LTC with number of queries equal to $\exp(\sqrt{\ln n \ln \ln n})$ and rate $1-\ep$.
\end{thm}
\begin{thm}
For all $\ep>0$ there is a LTC with the number of queries $(\ln n)^{O(\ln \ln n)} = \exp((\ln \ln n)^2)$.
\end{thm}
Additionally, we get a code which approaches the singleton bound with distance $\sim \ep$.

\subsection{Proof overview}
\begin{enumerate}
\item
Component 1: Get high-rate LTC's/LCC's with subpolynomial queries with subconstant distance.
\item
Component 2: Apply distance amplification (due to Alon-Luby 1996).
\end{enumerate}

\begin{itemize}
\item
For the first part, to get LTCs with $\exp(\sqrt{\ln n})$ queries, we use tensor codes of high dimension. 
\item
For LCCs with $\exp(\sqrt{\ln n})$ queries, we use multiplicity codes of high dimension. %previously people not interested in these parameters
\item
For LTCs with $(\ln n)^{O(\ln \ln n)}$ queries, use an iterative construction. Alternate tensor product and distance amplification.
%results in coplexity theorem that use this approach.
\end{itemize}

This is similar in spirit to techniques/results in complexity theory:
\begin{itemize}
\item
Zigzag product (Reingold-Vadhan-Wigderson)
\item
$SL=L$ (Reingold)
\item
Gap amplification (Dinur)
\end{itemize}
Rather than just use tensor products, alternate tensor product and amplification.

We use ideas from combinatorial constructions of LTC's due to Meir 2009. %(how to make the constructions more combinatorial).

Component 2 is based off of Alon-Luby 1996, based on expanders. Given a LTC/LCC with the following parameters:
\begin{itemize}
\item
distance $\de'=o(1)$,
\item
rate $r$,
\item
number of queries $q$,
\end{itemize}
for every $\de,\ep$, we obtain a code with the following parameters 
\begin{itemize}
\item
distance $\de'=\Om(n)$, $\de\gg \de'$,
\item
rate $r(1-\de-\ep)$,
\item
number of queries $q\poly(\rc{\de'},\rc{\ep})$.
\end{itemize}
Previously they used this to increase the distance. We actually amplify distance of local codes from subconstant to constant.

%extractors and dispersers? no.
%MVC and disperser - Gopi.
%first application: average case to worst case reductions.
%constant rate not important? usually want $\lg n$ overhead, don't care about rate. Not regime of interest for hardness?
The application is more to classical data transmission and storage. For appications to complexity, we usually want a constant number of queries.

%pick subset of coord, as long as get 2/3 can decode? Locality. Real system (distributed system), don't want to wait for last things to arrive. If small subset, query each...?

%\log n times as big set. As long as get $90\%$ of answers back can. query enough times... local . apply polylog n times. 

%7 data centers and one fail. Different patterns copy.
%Smoothed LCCs. Query lots of disjoint. THings lost but cannot flipped?

%encoding sparse, sublinear time? each bit of codeword depends on constant number of bits? avg to worst-case?



\section{Evolution and dynamical systems/Markov chains}

Look at dynamical systems that arise in nature.

We'll find relationships with
\begin{enumerate}
\item
Markov chains/mixing time
\item algorithms/optimization
\item Fourier analysis on the Boolean cube
\item computational complexity
\end{enumerate}

How can we model and understand molecular evolution? It involves
\begin{itemize}
\item
replication
\item
selection
\item muations, variability
\end{itemize}
Think of viruses. This happens at a fast time scale.

What are the origins of life? Apparatus, machines which combines, comes together, to self-replicate. 

Outside of TCS and math, people have multiple points of view.

Evolution of language also follows simple principles.

A dynamical system (discrete, continuous) is
\bal
x(t+1)&=f(x(t))\\
x'(t)&=f(x).
\end{align*}
A flow that tells you where to go.

Ex. a slime mold's dynamical system gives it the ability to compute. We want to explain how, why, and how fast.

\subsection{Equations of life}

Consider a situation where you have a population with $m$ different types, $[m]$. Let $x_i(t)$ be the proportion of type $i$ at time $t$. We have $\sum_{i=1}^mx_i(t)=1$. 

We consider a simple model. We need to introduce more parameters. Let $a_i>0$ be the fitness of $i$. A first attempt is to let $x_i(t+1)=a_ix_i(t)$. However, nature puts the constraint $\sum_{i=1}^mx_i(t)=1$. We renormalize by setting
\[
x_i(t+1)=\fc{a_ix_i(t)}{\sum_{j=1}^m a_jx_j(t)}.
\]
We've introduced reproduction and selection.
%fitness could depend on composition.

We need to introduce mutations (errors). Let $Q_{ij}$ be the fraction of unit mass that mutates to $j$.

There are 2 time scales: start with a population and see how it changes, vs. environment changes.

Let $A=\diag(a_1,\ldots, a_m)$. Let 
\[
x_i(t+1) = \fc{\sum_{j} a_jx_j(t)Q_{ji}}{\sum a_j x_j(t)},
\]
so
\[
x(t+1) = \fc{QAx(t)}{\ve{Ax(t)}_1}.
\]
Why does this have a steady state? If $Q,A>0$, then it has a  dominant right eigenvector by Perron-Frobenius.
%can do all normalization at end.

So the unnormalized values at time $t$ is $(QA)^tx(0)$.

We look at dynamic systems $f:\De_m\to \De_m$. This subclass is more special and interesting. No matter where you start from, you will converge to a fixed point. Spectral ratio tells you how many iterations you need.

Do all dynamical systems of this type $f$ have a unique fixed point? Can more complex behavior happen?
There can be multiple fixed points. The first hurdle is existence of multiple fixed points. 
\begin{itemize}
\item
Multiple fixed points.
\item Periodic orbits/limit cycles.
\item
Chaos. (Let's not discuss this.)
\end{itemize}
%chaos in cs?
These do come up in evolution of languages and origins of life.

Stability is important when we're talking about equilibria: nearby trajectories converge.

\subsection{Finite populations}

We thought of populations as infinite. Here we have actual particles, say $N$ of them. We introduce stochasticity in evolution. We don't get a dynamical system, but a Markov chain. Each particle of type $i$ replicates to $a_i$ particles. Use the stochastic matrix to make the mutations. 
We have replication and mutation. Now do selection: sample $N$ points with replacement. $M$ and sampling are stochastic.

What is the number of states of the Markov chain?
Think of $N=1000$ as large, $m=50$. The number of states is $m^N$ (or $\sim N^m$ since order doesn't matter). The state space is large.

These models are used to simulate how viruses evolve. 
Track certain parameters. How much time does it take to reach a steady state?

Are these methodologies sound?

In theoretical computer science we don't simulate. Do they converge quickly?

Mutagenic drugs try to increase the stochasticity and remove genetic info from the system. How many life spans evolve before it loses genetic information? If it's much more than your life span...
\begin{itemize}
\item
computationally
\item
feasibility of %explosion
\item
could life have evolved quickly enough? (rapidly mixing) %explosion of state space.
\end{itemize}

Mutation vs. steady state.
Let $(Q,A,M)$ be the parameters of $f$. How do solutions change when we change $Q,A,M$? Let's focus on $Q$. Assume $Q_{ii}=1-\mu$ and $Q_{ij}=\fc{\mu}{m-1}$: replicate correctly with probability $1-\mu$ and make a random error else. 
What happens as $\mu$ increases?
For this system (proposed by Eigen and Schuster in the 70's, Eigen was a Nobel laureate in chemistry), as $\mu$ increases from 0 (not far from 0), there is a point (depending on $m$) where the vector goes thorugh a phase transition, called the \vocab{error threshold}. 
%high vs. low fitness.
$\mu$ decreases with $m$. This observation is the basis of mutagenic strategies.

Their motivation was not to model viruses. They were trying to understand the information capacity of a system. How much genetic information can a system maintain? Assume $\mu \sim \rc{m}$. %approx right
If $\mu>\rc m$ then the output is junk/random/no genetic information. If $\mu<\rc m$ then it preserves genetic information. We want $\mu<\rc m$; a drug designer wants $\mu>\rc m$. 

Go back 40 million years ago. $\mu$ is a thermodynamic parameter. It turns out, based on simplistic models, that the largest bit content is 50. How long is DNA? Billions, $10^9$. You will hit a bottleneck less than a billion. %$Q_{ij}$ is not uniform. Hypercube, bitwise mutations.

$(1-\mu)^L\sim \rc{L}$. 

Theorem!

So many creatures, why only 1 way to store information? If everything started from scratch, organisms should have a different way to store information! The more you think the more you stop believing (simplistic accounts of?) evolution.

\subsection{Hypercycle}

%manifold?

Suppose there are $N$ types. The differential equations are 
\[
\dot{x}_i= k_ix_ix_{i+1} - x_i\sum_j k_jx_jx_{j+1},
\]
$\sum \dot{x}_i=0$.
There are mechanisms in stars which have this form (Bethe-Weierstrass cycles). Autocatalytic cycles. 
How can we bypass the error threshold? Why don't we see alternate forms of information-organizing systems?

Think of each system as a collection of proteins, RNA's which have nice properties. Another collection of things that have nice properties. Can use A to do B, B to do C. Functional linkages. For this to survive, you must justify there is a solution to this dynamical system which lives inside the simplex. If one type goes to 0 it kills the cycle. Conjectured there is a limit cycle inside the interior, unique limit cycle. Proved in late 1980's by mathematicians. There are still open questions.

Mathematically, for this version, we understood how much time to get to the steady state.

Not obvious. Important: things have to stabilize in some sense. This was studied by Papadimitriou and Visheet. Use techniques from computational complexity. There is something special about these dynamical systems. If can choose coefficients, then PSPACE-complete. (Given the parameters of the dynamical system, give a point close to the orbit. cf. Nash equilibirum.) For specific dynamical systems, what can we do? We are in complete lack of mathematical techniques. Part of the interest is not explaining biology and science, but also: are there mathematical techniques that we still lack. %problem seems natural.

In a (recently terminating) series of results DSV12, V15, PSV15, we proved a large class of evolutionary dynamical systems mix rapidly. (Finite sampling.)

%nice math, science

\subsection{Language}

Here's a problem which bothers us. When a baby is born, how does it learn how to speak? We know how to teach a baby a language.

Chomsky proposed universal grammar: a baby is born with a set of grammars. It tries to learn from examples.  
Uniform distribution of grammars that slowly converges.

Question: How big a grammar? This depends on how good the learning is. Let's model it.

Novak et. al. proposed a model in 2001. Suppose there are 10 grammars $Q_1,\ldots, Q_m$. What is the relation between the $i$th and $j$th grammar. What is the probability that a sentence/word from $i$ makes sense in $j$? Keep track of populations $x_1,\ldots, x_m$: confidence in grammars. 
The model is: define $b_{ij}=\fc{F_{ij}+F_{ji}}2$, pairwise compatibility of 2 grammars. Fitness.

Probability of confusion is (mutation)
$Q_{ij}=$,
probability that trying to learn $i$, picking up something from $j$.

We can define the fitness of a grammar 
\[
a_i = \sum_j b_{ij} Q_{ji}x_j:
\]
knowing $i$, how well you can communicate with rest of population?
$a_i$ has to do with the learning. $Q_{ii}=1-\mu_i$, $Q_{ij}=\fc{\mu}{m-1}$.

Quadratic $f$, $x(t+1)=f(x(t))$. Study the steady state vector as a function of $\mu$. As you increase the mutation rate, all points close to corners are stable. They slowly disappear and only the point in the center is stable. The Tower of Babel.
They call it the coherence threshold.

%learn a grammar without knowing logic?

Batch learning, etc. Heuristic arguments: how big the grammar has to be as a function of the learning environment. cf. we got some bound on $M$ given $\mu$. Recently we're looking at how the mixing time behaves. Because of multiple equilibria, can prove phase transition. Not a good idea to have multiple fixed points in dynamical systems where we're studying finite versions. 
Parameters fine-tuned slowly. Prefers dynamical systems with nice equilibria?

Different models of learning, do we need the notion of universal grammar. Here's a learning mechanism for which the size of the grammar is small. Less information have to be born with, ability to teach how to speak.

Techniques for quadratic dynamical systems. Fixed point. Move in parameter space. Determined by Jacobian, eigenvalues. Smooth functions. All eigenvalues real part negative. Converge from every direction. Change parameters, then these properties violated. Hopf fibration (?).
Can understand convergence from close-by points. 

Algorithmic: finite sampling. 

Now can do linear programming via slime. What optimization? Riemannian manifold.
%science or math
%do things without boundaries.
%think about science q's not math constrained.

\section{Dynamical systems and space-bounded computation}

A dynamical system is  function $f:X\to X$. We care about compact $X\subeq \R^n$. This is the most general model for physical systems. $X$ is the state space, $f$ is the transition. Very simple functions $f$ can give rise to chaotic behavior.

Mathematicians study chaos: 2 starting states that start off almost the same diverge quickly. For example the logistic map $f(x)=4x(1-x)$ is chaotic. The general goal is to understand iterates.

An \vocab{invariant measure} is a measure fixed under $f$: For all measurable $A\subeq X$,
\[
\mu(A) = \mu(f^{-1}(A)).
\]
Can we compute invariant measures?

No. You can embed a universal Turing machine inside a smooth dynamical system.

Given a nice $A$, e.g., a box $A=[a_1,b_1]\times \cdots \times [a_n,b_n]$, what is $\mu(A)$ to within precision $\de$? If the Turing machine halts, all the mass of the Turing machine is on one set; if not, it's on the complement.

We expect reasonable systems to be computable. What's present in real life but not here is noise.

Consider a random perturbation of a dynamical system, $x\mapsto f(x)+(\ep\text{-noise})$. Think of the noise as uniform on an $\ep$-ball or Gaussian. 

For the results on invariant measures to be ``physical", they must be robust. Physical computers are robust. These mathematical objects are not physical. Noncomputable phenomenon do not survive under noise. 

Encode a Turing machine as 2 decimal numbers representing what's to the left/right on the tape. If I shake the system, erase all but $\log\ep$ window around the head. The Turing machine dissipates under noise; all the good simulators are high-dimensional.

BGR show that invariant measures of random perturbations are computable. Originally you have infinite memory, it gets erased.
By invariant we mean 
\[
\Pj_\mu(X_{t+1}\in A) = \Pj_\mu(X_t\in A).
\]
Think of $f$ as acting on measures. The invariant measures are those that are fixed under the operation.
%mark, christobal...


How might you compute it? Divide space into $A$ regions. Assume it's roughly constant on each. The problem transforms to computing the stationary  distribution of a $A$-state Markov chain. This can be done in polynomial time in $A$. If you want to compute to precision $\de$, each should have size depending $\de$; take $A=\poly\prc{\de}$.

Should depend primarily on $\ep$ and not so much on the precision: $O_\ep(\poly\prc{\ep})$.

This makes sense when $\ep\ll \de$: the precision is much smaller than $\ep$.

%For reasonable functions $f$ you can
With noise the measure is almost constant on each piece: you jump and scatter. The probability that you are close by is the same. There is smoothing. Without noise there is no guarantee. (You can get weird invariant measures, supported on fractals, etc.) The noise smoothens all the bumps.
%A bump that tries to jump over the fence

We'll do something similar but more clever.

Divide space into $A$ regions $a_1,\ldots, a_n$. In each region choose center points $x_1,\ldots, x_n$ arbitrarily. We'll write the density function as a Taylor series around the $x_i$.

Decompose $\mu_t(x) = \sum_{i=1}^A \one_{x\in a_i} \sum_{k=1}^{\iy} p_{i,k}(t) (x-x_i)^k$. \fixme{(We assume $f$ and noise are nice, i.e., smooth. Gaussian noise smooths things out.)} We can express any smooth measure in terms of Taylor coefficients. The nice thing is that the dynamical system acts linearly on these coefficients. Start with arbitrary $\mu_0$, evolve the system to get $\mu_1,\mu_2,\ldots, \mu_t,\ldots$.

There is an explicit linear map $Pp^{(t)}=p^{(t+1)}$. $P$ is infinite-dimensional but we can fix this by truncating at level $N$. We stopped at level 0 in the picture before.

Let $P_N=P$ truncated to degree $\le N$.
%noisy dynamical systems (and raising
%matrices to very high powers in polylogarithmic space).

This is an $AN\times AN$ matrix.
We lose something. To get $\ve{P_N^T\mu - \pi}\le \de$, set 
\[
T=O(\ln \prc{\de} \exp(\ep^{-1})),\quad T=O(\ln \prc{\de} \poly(\ep^{-1})).
\]
We can compute the map in polynomial time. The total time is $O(\ln \prc{\de}\poly\prc{\ep})$.
This gives an efficient natural time bound. 

\subsection{Space complexity}

The argument of why these should be computable is from space complexity: you can't store so much in one step because of noise.

We use a similar technique. 
\begin{thm}
For nice $f$ and Gaussian noise, we can compute the invariant measure in space complexity 
\[
O(\poly(\ln \prc{\ep} + \ln \ln \prc{\de})).
\]
It's polylog in the size of the input and output.
\end{thm}
The model is the typical: There is a read-only input tape, a work tape of the above size, and a write-only output tape.
This is tight: any Turing machine with this space complexity, can be embedded in a dynamical system.
%invariant measure = stationary distribution?
%local noise: space result like this. 
%with probability $\ep$ just randomly to random point. weird behavior.
$\ep$-noise is like distinguishing $\rc{\ep}$ space. You should behave like a memory $\lg\rc{\ep}$ machine. %This is the strongest theorem you can have.
This is a characterization of dynamical systems as a computational device.

The algorithm is essentially no different. We still want to 
\begin{enumerate}
\item
compute $P_N$,
\item
raise $P_N^T$.
\end{enumerate}

Computing $P_N$ boils down to computing a bunch of integrals. It's relatively easy to do. The second part is trickier.

Now ignore all these dynamical system stuff. It's just a problem of raising matrices to large powers in polylog space. Given an $n\times n$ matrix all of whose entries are integer multiples of $2^{-\poly(n)}$ (i.e., numbers specified to that precision), compute in $O(\polylog(n))$ space $M^{2^n}$. We want an exponential power because $T$ can be exponential iin $\exp(\ep^{-1})$ or $\exp(\ep^{-2})$. Another way of thinking of this is that I give you a Markov chain

Each entry has $\poly(n)$ bits. You cannot even store 1 entry/number in the graph. (Cf. reachability: you can't store the graph. Recall what you can do in log space. 
\begin{enumerate}
\item
In $O(\lg n)$ space you can add $n$-bit numbers (add bit by bit and remember the carry), 
\item multiply $n$-bit numbers (add $n$ $n$-bit numbers at once.)
\item
divide $n$-bit numbers (harder, resolved in 90s). Division can be done in $TC^0$.
\item
work with real numbers that are multiples of $2^{-\poly(n)}$,
\item
Multiply 2 matrices (follows from multiplying and adding)
\end{enumerate}
When you compose, space adds. (You can use recursion, pretend like you wrote the intermediate result; call the first program instead, you need stack space for both)

In polylog space you can 
\begin{enumerate}
\item
Compute $M^{\poly(n)}$. By iterated squaring you can do this with $\lg(n)$ operations. This takes $O((\lg n)^2)$ space. This is essentially Savitch's Theorem: replace the matrix with the adjacency matrix.
\item
Compute $\det(M)$ and the characteristic polynomial of $M$ in time $O((\lg n)^2)$.
\item
Find roots of a real/complex polynomial to within precision $2^{-n}$ in time $O((\lg n)^5)$. 
%to highest precision you can hope fore
%analytic, univariate. find complex roots.
\end{enumerate}
%binary search? can't do!
But we need to compute to exponential-sized exponents. Repeated squaring would take time greater than $n$.

First consider raising a number to an exponential power $x^{2^n}$. Repeated squaring takes too much space. Rewrite as
\[
x^{2^n} = \exp(2^n\ln x)
\]
which you can approximate in logspace using Taylor series, to precision $2^{-\poly(n)}$. 

If we can diagonalize $M$ then we are happy.
If you can diagonalize $M=U^{-1}DU$, then $M^{2^n}=U^{-1}D^{2^n}U$, and you can do this in polylog time. But no one knows how to diagonalize matrices in polylog time.

What you can do is compute eigenvalues, because they are roots of the characteristic polynomial. Compute $\la_1,\ldots, \la_n$ or $M$. 
%paper by Nef.
%division in $TC^0$. amazing result?
%collapse of counting hierarchy.
Then compute $(\la_i^{2^n})$. By Lagrange interpolation construct a degree $n-1$ polynomial $p(\la_i)= \la_i^{2^n}$.
%Hermite interpolation: knowing derivatives too. Approximations to eigenvalues? Division by 0 somewhere. Not like algebraic fields. too close? Not robust. 
%compute eigenvectors? But can only find eigenvalues approximately.
%in the limit break down because th approx only works at that level. 
%raising to power $T$ does one thing...
%norm of $P_N$ not 1 anymore?
In the denominator you get $\la_i-\la_j$. We need a lower bound on the distance between 2 roots, which we have by $2^{-n}$ precision. Now compute $p(M)=M^{2^n}$.
%don't need Cayley hamilton. poly operators commute.
If there are repeated eigenvalues, perturb $M$. One of $n+1$ perturbations will succeed. A small amount of perturbation to a univariate polynomial will have distinct roots. $(1-t)M+E$: $\prod (\la_i-\la_j)=P(M_{ij})$. 


\section{Formulas resilient to short-circuit errors}

(Kalai, Lewleo, Rao, FOCS12)

\begin{thm}
Every formula $F$ can be efficiently converted to $E$ where $E$ is resilient to $(\rc{10}-\ep)$-fraction of short-circuit noise at any input-to-output path, with $|E|=O_\ep(|F|)$.
\end{thm}

What is short-circuit noise?

Consider the von Neumann noise model: There is a BSC channel at the output: the right answer $g$ with probability $1-\ep$ and $\wh g $ with probability $1-\ep$. 

Short circuit noise connects one of the inputs to the output. It can be adaptive: it can look at the input and choose which to output. What's clear is that $g(0,0)=0$ and $g(1,1)=1$. Short-circuit noise is slightly weaker. In the short-circuit csae we can deal with worst-case error. In the von Neumann model, if the output gate is wrong, the output is wrong.

Consider what happens when there are multiple outputs. You can't beat $(1-\ep)^m$, while in the short-circuit model you can. Short-circuit errors are actually realistic.

You can have a whole layer of gates that is short-circuited, disconnecting the input and output.
We assume the negations are pushed to the bottom.

We prove the theorem in 3 steps.
\begin{enumerate}
\item
Convert $F$ into an interactive protocol $P$ (Karchmer-Wigderson)
\item
Convert $P$ to $P'$ resilient to noise.
\item
Conver from $P'$ into formula $W$ (KW).
\end{enumerate}•
\subsection{Interactive protocol}

A formula is a circuit with fan-out 1. Assume we just use AND and OR gates. (Assume negation at the bottom is noiseless.) I.e., just consider monotone formulas.

A protocol: Alice gets input $X$ and Bob gets $Y$. Alice sends a bit; people react based on what they see. This creates a tree, until you get to the output.

Karchmer-Wigderson says these are equivalent. (They both involve trees!) 
Alice gets $x\in f^{-1}(0)$ and Bob gets $y\in f^{-1}(1)$. Their task is to find $i$ such that $x_i\ne y_i$.

Look at the formula: every time there is an AND gate, Alice speaks and says which input evaluates to 0. Bob speaks when there is OR, and says which input evaluates to 1.

What happens when there is a short circuit? An AND is short-circuited. In the protocol, no matter what Alice sends, Bob receives 0. Bob believes that they are in the other branch of the tree. Alice is aware that Bob got 0.
This is exactly errors with feedback.
The conversion goes both ways.
Formula with short circuit is equivalent to interactive protocols with feedback. 
Because the protocol continuation is given by Bob's branch; by definition of the tree representation of the protocol, Alice knows.

If $P$ with feedback is resilient to $e$ errors then $F_{(P)}$ is resilient to $e$-short circuits.

%if we don't allow feedback, it doesn't go?
%adaptive protocols?
%Alice thinks we're here, 
%interactive protocol know how to do.
%problem is converting back to formula
%circuit cannot adaptively change.

\subsection{Making interactive protocol resilient to noise}

EGH15 in ICTS: $\rc3-\ep$ noise over 3-ary alphabet, $\rc5-\ep$ noise over binary. Communicate $\{0,1,\leftarrow\}$ (go back). Every time we lose 3 rounds; this leads to $\rc3$ noise. If we have a protocol with $n$ rounds, with $n+3e$ many we can resist $e$ errors:
\[
\fc{e}{n+3e} =\fc{\fc{n}{\ep}}{n+\fc{3n}{\ep}}>\rc 3-\ep.
\]
Take $e=\fc n{\ep}$.

For binary, $\leftarrow$ needs 2 rounds to communicate. We need $n+5e$. This is not tight.

We can get $\rc3-\ep$ over binary (2015).

For every noise, pay 3 or 5 times.
Size is linear in $\rc{\ep}$: $e\sim \fc{n}{\ep}$, $n+\fc n{\ep}=O(\prc{\ep}n)=O_\ep(n)$. 
%poly blowup in size
%start with balanch.
\begin{thm}[KLR12]
For every $\ep>0$ there is an efficient compiler that takes fan-in 2 $F$ of depth $d$ and generates $E_2$ or $E_3$ of depth  $4d+5e$ or $2d+3e$ respectively resilient to $e$ short-circuit errors, fan-in 3. The 2 is to convert $F$ into and/or's.
\end{thm}
What they get is $\rc{10}$, 10 and 6 respectively.
The fan-in is the alphabet size. $\rc{10}-\ep\to \rc5-\ep$. $\rc6-\ep\to \rc3-\ep$.

Use the new result (sophisticated) and put it in, should get $\rc3$ in fan-in 2.

%black box reduction? %no adaptiveness, etc. 

What happens with circuits, fan-out?

Make the circuit become a formula. Into formula of exponential size. Want noise-resistant of poly size.

What if there is no feedback? Protocol can diverge; how to convert it to a circuit? The motivation is that we have results that are conceptually harder than the feedback for noise-resilient protocols.

%duality between circuits and branching programs.
%protocol branching program.
%now allow remember entire transcript, remember some memory. Noise resilience has never been worked out. tree codes don't apply, but don't need because don't need to protect against so many bad things.
%protocol model has not been studied. on the fce of it, just solve a decision/branching program which corresponds to modifying rather than appending to the transcript.
%increase size at most polynomially.
%circuits in practice are long and thin, because the depth dimension is time. a million gates, ten billion time.
%circuits with feedback.
KW were introduced to study NC1 lower bounds.

%circuit and protocol branching program. step backwards, because CC can get results. introduce size. separating L from P? not hopeless but harder than NC1, P.

%\bibliographystyle{plain}
%\bibliography{refs}
\end{document}