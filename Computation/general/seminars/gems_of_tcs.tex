\def\filepath{C:/Users/Owner/Dropbox/Math/templates}

\input{\filepath/packages_article.tex}
\input{\filepath/theorems_with_boxes.tex}
\input{\filepath/macros.tex}
\input{\filepath/formatting.tex}
%\input{\filepath/other.tex}

%\def\name{NAME}

%\input{\filepath/titlepage.tex}

\pagestyle{fancy}
%\addtolength{\headwidth}{\marginparsep} %these change header-rule width
%\addtolength{\headwidth}{\marginparwidth}
\lhead{Gems of theoretical computer science}
\chead{} 
\rhead{} 
\lfoot{} 
\cfoot{\thepage} 
\rfoot{} 
\renewcommand{\headrulewidth}{.3pt} 
%\renewcommand{\footrulewidth}{.3pt}
\setlength\voffset{0in}
\setlength\textheight{648pt}

\begin{document}
\tableofcontents
\section{List decoding of Reed-Solomon codes}
\subsection{Introduction}
An error-correcting code is $C\subeq \Si^n$, where $\Si,|\Si|=q$ is the alphabet and $n$ is the encoding length. 

Define the normalized Hamming distance
\[
d(x,y)=\rc n|\set{i}{x_i\ne y_i}|.
\]
We want 
\[
\de = \min_{x,y\in C,x\ne y} d(x,y)
\]
to be as large as possible (constant as $n\to \iy$). Imagine balls of radius $\de/2$ around each point.
We can send one of $|C|$ messages by mapping $[|C|]\to \Si^n$; they can withstand $\de/2$ errors. The rate of the code is $\fc{\lg|C|}{n}$ (how many bits of actual information are sent compared with encoding length); we want the rate and distance to be high, but there are fundamental limits to what can be achieved.

$\de/2$ is the unique decoding radius. Unique decoding is not possible beyond $\de/2$. %; they don't intersect. 

In many cases, even if there is not a unique codeword within a given radius ($\fc{\de}{2}+\ep$, say), there may be a only a small number of codewords.
\begin{df}
$C$ is $(\de,L)$-list decodable if for all $x\in \Si^n$, $|B(x,\de)\cap C|\le L$.
\end{df}
%random code, with high probability
If you choose your code randomly---that is, choose a codeword, removing a $\fc{\de}2$ ball around it, and repeat---with high probability it will have good list decodability, up to distance $(1-\ep)\de$. %; in fact, most of the time it can correct close to $\de$. %The code is good becuase most of the time it can correct $\de$.
List decoding is a better way to handle talk about how good the code is than stochastically, because it's worst-case.
%up to $(1-\ep)\de$
%People previously tried to model stochastically.
%This code is good becuase most of the time it can correct $\de$.
%List decoding is a better way to handle this than stochastically because it's worst-case.

The maximum $\de$ is the list-decoding radius. Here $L(n)$ is a function of $n$. The ideal setting is $L(n)=\poly(n)$.  Informally the list decoding radius is the maximum $\de$ such that $L(n)=\poly(n)$.

What does a random code give us; what is possible and not possible? What is the capacity of list decoding?
%minimum distance

\begin{thm}
Let $0<\de<1-\rc{q}$. (Above this error, the noisy word is essentially random, so we can say nothing.) Then there exists a $(\de,L)$-list decodable code with rate
\[
1-H_q(\de)-\rc{L}.
\]
Here $H_q$ is defined so that
\[
|B(0^n,\de)|\sim q^{H_q(\de)n}.
\]
\end{thm}
%\Si^n$.
\begin{proof}
Choose $M=q^{Rn}$ codewords at random. We want
\[
\Pj(C\text{ not }(\de,L)\text{-list decodable})<1.
\]
Do a union bound. The probability is at most, by the union bound,
\[
\binom{M}{L+1}q^n\pf{|B(x,\de)|}{q^n}^{L+1}<1.
\]
($q^n$ is the number of points; the probability of $L+1$ points in a ball centered at a point is $\le \binom{M}{L+1}\pf{|B(x,\de)|}{q^n}^{L+1}$.)
\end{proof}
This is almost tight. 
\begin{thm}
If a code is $(\de,L)$-list decodable with 
\[
R\ge 1-H_q(\de)+\ep
\]
then $L(n)\ge q^{\fc{\ep n}2}$.
\end{thm}
%1-H_q(\de)+\ep$
A random ball will contain too many points.
\begin{proof}
\bal
\E_{x\in_R \Si^n} [|B(x,\de)\cap C|]
&=
|C|\fc{|B(0^n,\de)|}{q^n}\\
&\ge 
2^{Rn}q^{n(H_q(\de)-1)}
%q^{n(1-H_q(\de)+\ep)}%%/\ep^n
\\
&\ge q^{\ep n/2}.
\end{align*}
\end{proof}
When $q\to \iy$, $H_q(\de)\to \de$. We can achieve rate $R\ge 1-\de-\ep$ with alphabet of size $2^{O\prc{\ep}}$  and list size $O\prc{\ep}$. 
%1/L
%all add up to at most $\rc \ep$. Plug in param.
%alphabet grows, can achieve any rate up to.

Graph: the achievable rates are below the line $\de+R=1$.

This is what is known existentially.

The current known explicit codes can achieve $R\ge 1-\de-\ep$ with alphabet of size $2^{\poly\prc{\ep}}$ and list size $n^{O\prc{\ep}}$.
%
The idea is a folded Reed-Solomon code concatenated with an asymptotically good list-decodable code.
%any code - at least this much list decodable.

\begin{thm}[Johnson bound]
Given $C\subeq \Si^n$ with min-distance $\de=1-\ep$, then $C$ is $(1-\sqrt{\ep},\poly(n))$ list-decodable.
%increase by $\sqrt{\ep}-\ep$.
\end{thm}
The Johnson bound says that if you slightly increase the radius, there cannot be an exponential number of codewords.

\begin{thm}[Singleton bound]
For any codewith distance $\de$, $R\le 1-\de$.
\end{thm}
\vocab{MDS (maximal distance separable) codes} are where $R+\de=1$, $\de=1-R$, R$=\ep$. MDS codes are $(1-\sqrt{R},\poly(N))$ list-decodable.
%if don't care about constant, simple. if want constant, need more careful calculation.

\subsection{Reed-Solomon Codes}
We introduce the Reed-Solomon code $RS_{k,n,\F}$, $k< n\le q$.

These are MDS codes. They achieve $R+\de=1$ so are $(1-\sqrt{R}, \poly(n))$-list decodable.

Let $\Si=\F_q=:\F$. Let
\[
C=\{\text{evaluations of degree $k$ polynomials on }\{\al_1,\ldots, \al_n\}\subeq \F\}
\]
where $\al_1,\ldots, \al_n$ are distinct and fixed.
Codewords correspond to degree $k$ polynomials in $\F[x]$.
2 distinct degree $k$ polynomials can only agree on $k$ points, so $\de=\fc{n-k}n=1-\fc kn$. 
%2 degree $k$ polynomials, distinct.
The rate is $\fc{\log_q(q^{k+1})}{n}=\fc{k+1}n$. 
The Reed-Solomon code is $(1-\sfc{k}n, \poly(n))$-list decodable. 

What is the algorithm? Given a set of points, we need to find all polynomials passing through enough of those points.
% Given a word $x$, give a list of the polynomials.
\begin{prb}
Given $(\al_1,y_1),\ldots, (\al_n,y_n)$, find all polynomials of degree $\le k$ such that $p(\al_i)=y_i$ for at least $\sqrt{nk}$ indices $i\in[n]$. 
\end{prb}
Madha Sudan (90's) showed that you can do this with $\sqrt{2nk}$. 
\begin{thm}
There is a polynomial-time algorithm (given in the proof) that given $n$ points as above, finds all degree $\le k$ polynomials agreeing on $\ge 2\sqrt{nk}$ points. 
\end{thm}
\begin{proof}
The algorithm is as follows.
Note this algorithm will work even if the $\al_i$ are not distinct.

Define the $(1,k)$-weighted degree of a polynomial $Q(x,y)$ as $\deg Q(X,Y^k)$.
The strategy is as follows.

We will find a low $(1,k)$-weighted degree polynomial $Q(x,y)$ of degree $D$ such that $Q(\al_i,y_i)=0$ for all $i$.

Look at $R(x)=Q(x,p(x))$ where $p\in L$. We have
\[
\deg R\le D.
\]
For all $i$ such that $p(\al_i)=y_i$, 
\[
R(\al_i)=Q(\al_i,p(\al_i))=Q(\al_i,y_i)=0.
\]
Suppose $R$ has at least $t$ roots and $\deg R\le D$.  If we arrange so that $t>D$, then $R(x)\equiv 0$ identically. Then $Q(x,p(x))\equiv 0$ implies $y-p(x)\mid Q(x,y)$.
%$Q(x,y+p(x))\equiv 0$, take $y=0$.
%t<q.
%cant have more roots than degree.
(There is a deterministic algorithm to factor bivariate polynomials.) Factor $R$ to find all factors in the form $y-p(x)$; then output those polynomials $p(x)$.

Now we just need to find the polynomial $R(x)$; this is polynomial interpolation. 

The number of coefficients in $Q(x,y)$ is $\rc{k}\binom{D+2}2$. We need to satisfy $n$ linear constraints; they are linear homogeneous equations in the coefficients. If $\fc{\binom{D+2}2}k>n$ then there is a nonzero solution: a nonzero $\ph(x,y)$ of $(1,k)$ weight degree $\le D$ with $\ph(\al_i)=y_i$ for all $i$.

If the number of roots is $t>D\approx \sqrt{2kn}$, then we can find all polynomials with agreement $t$.
%\item
%\end{enumerate}
\end{proof}
Guruswami and Sudan improved this to $\sqrt{kn}$.

\fixme{
Consider $k=1$, $\F=\R$, find all lines which pass through at least 3 points. If you can interpolate a degree 2 polynomial through all these points, get all lines as lines within the curve. The degree 2 curve will be 2 lines.

Let's look at a small example to see how to improve the bound.

Consider the following picture (see notebook). Here $n=10$ and $t=4$. Here we can choose $k=4$. However it can only have 4 linear factors. The maximum $D$ is 3. %D to small to contain all lines.
Peculiar: through each point there are 2 lines. An algebraic curve passing all points should vanish at the points with multiplicity 2. Fit a polynomial which vanishes with degree 2 at the points.}

First define multiplicity.
\begin{df}
$Q(x,y)$ vanishes with multiplicity $r$ at $(\al,\be)$ if $Q(x+\al,y+\be)$ doesn't have any monomial of degree $\le r$.
\end{df}
\begin{lem}
Let $Q(x,y)$ be with $(1,k)$ degree $\le D$, vanishing at $(\al_i,y_i)$ with multiplicity $r$ for all $i\in [n]$. Let $P$ be a degree $k$ polynomial with agreement $t>D/r$. Then $y-p(x)\mid Q(x,y)$.
%deg 5, should still contain line.
\end{lem}

\begin{proof}
Let $p\in L$, $P(\al_i)=y_i$. 
Define 
\[
Q^i(x,y) = Q(x+\al_i,y+y_i);
\]
it has no monomials of degree $<r$.
Then
\bal
R(x) &= Q(x,P(x))\\
&=Q^i (x-\al_i, P(x)-y_i)\\
&=Q^i (x-\al_i,\ub{P(x)-p(\al_i)}{x-\al_i\mid P(x)-P(\al_i)}).
\end{align*}
$Q^i$ has no monomials of degree $<r$. Thus $(x-\al_i)^r\mid R(x)$. The number of linear factors is $tr>D$. Thus $R(x)\equiv 0$ and $y-p(x)\mid Q(x,y)$. 
\end{proof}
The number of coefficients in $Q(x,y)$ of $(1,k)$-degree $D$ is $\rc k\binom{D+2}D$.
%no monom degree $r$. 
The number of homogeneous linear equations is $n\binom{r+1}2$. So a nonzero $Q$ exists if $\rc k\binom{D+2}2>n\binom{r+1}2$.

Choose $D=\sqrt{knr(r+1)}$. Let $t=\fc Dr = \sqrt{kn(1+\rc r)}$.

%find all polynomils. 
Make $r$ large enough, we approach the Johnson bound.

The number of polynomials in the list is at most $\fc Dk=\sfc{nr(r+1)}k$, the $y$-degree. 

Conclusion: 
\begin{enumerate}
\item
If $t>\sqrt{kn}$ the list size is $\le n^\ep$ and we can find all of them.
\item
The Reed Solomon code of rate $R$ can be list decoded up to $1-\sqrt{(1+\ep)R}$ errors with best size $\rc{\ep\sqrt R}$.
\end{enumerate}
%no more than $y$ degree

This method (the polynomial method) is very flexible. We give another application, the list recovering problem. Given $x\in C$, 
%replace each alphabet with small set
a noisy channel turns each coordinate into a set $s_i$ and spits out $s_1,\ldots, s_n$, $|s_i|\le \ell$. We have $|s_i|\le l $ such that for at least $1-\de$ fraction of $i$'s, $x_i\in s_i$. Find all $x$ such that $x_i\in s_i$ for at least $1-\de$ fraction of $i$'s. 
\[
\ab{
\pa{
\bigcup_{y\in S_1\times \cdots \times S_n} B(y,\de)
}\cap C
}\le L.
\]
Reed-Solomon codes are also good list-recoverable codes.

If $t>\sqrt{knl}$, where $t$ is the number of $i$ such that $x_i\in S_i$, then the Reed-Solomon code is $(1-\sqrt{kl},l,O(n^2l^2)=L)$ list recoverable.
%we want optimal paramters.

We only achieve $1-\sqrt R$. We want to attain $(1-R,L)$. Guruswami and Rudra came up with folded Reed-Solomon codes. 
Take a generator $\ga $ of $\F_q^{\times}$, $n=q-1$. $\al_1,\ldots, \al_n$ are $S=\{1,\ga,\ldots, \ga^{q-1}\}$. \fixme{Consider blocks. $\F^m$ by $m$. $P(1),\dots P(r),\ldots, P(r^{q-1})$.

It's an open problem to make $n^{O\prc{\ep}}$ independent of $n$. 
Use concatention to get length down. 

You can concatenate a list-recoverable code with a list-decodable codes to get a list-decodable code. $C_{\text{out}}\circ C_{\text{in}}$.}

%$C_{}$
%small list-decodable.
%folding: alphabet grow large

%S
%GS: RS (1-\srqt k)
%PVbetter than 1-R for small enough $R$
%FRS 1-R

%Extractors, expanders etc.

%to be list decodable don't need dep on n?
%codes achieve both poly n /poly n.

\section{$SL=L$}

%picture of Z.

Reingold, Vadhan, and Wigderson.

There are many stories in this problem.

One of the most fundamental questions in theoretical CS is the following.
\begin{clm}
Randomness is useless.
\end{clm}
This claim comes from very recent years. 20 years ago people actually believe it's useful (Papadimitriou gave $BPP\ne P$ as a homework exercise in his book).

There are 2 questions: is randomness useful in time and in space?
\begin{enumerate}
\item
$BPP\stackrel ? P$
\item
$RL\stackrel ? L$
\end{enumerate}
We focus on the second problem. The first problem is wide open: we don't know $BPP\stackrel?{\in}DTIME(2^{o(n)})$. We know more about the second question: Savitch proves
\[
BPP\subeq DSPACE(\ln^2n).
\]
We can do better: Saks and Zhou in 1999 show 
\[
RL\subeq DSPACE(\ln^{\fc 32}n).
\]
(They actual show this for BPP.)
Think of RL as (undirected) $s-t$ nonconnectivity problem. $coRL$ is interesting because it contains $s-t$ connectivity. (RL doesn't have a complete problem.) For $P\stackrel?=NP$, just look at a complete problem; here we don't have one.

Here we show $SL=L$. Reingold showed this in 2004. 

For simplicity, just think of SL as one problem, undirected $s-t$ connectivity.

Consider a graph. 


Jieming starts from a vertex and wants to find a way to Nanjing. Jieming has very little memory, and cannot remember the whole structure of the graph. He can't remember much more than the name of a city.

The standard way to solve this is to take a uniform random walk on the graph. If the graph has $|V|=n$, after $n^2\ln n$ steps, there is a high probability that he will have visited Nanjing.

But for a general graph, it's necessary to flip $n^2\ln n$ coins: Consider 2 complete graphs with a bridge: It takes order of $n$ time to hit the bridge vertex, and it has $O\prc n$ chance of crossing the bridge.

For which graphs can we do this randomized routing faster than the worst case? A natural family is the family of expaner graphs.

Expander graphs have rapid mixing under a random walk.
There are 2 definitions.
\begin{df}
A graph $G=(V,E)$ is a \vocab{$\la$-edge expander} if for all sets $S\subeq V$, $|S|\le \fc{|V|}2$, 
\[
\fc{E(S,\ol S)}{\Vol(S)}\ge \la.
\]
Here, $\Vol(S)=\sum_{v\in S}\deg(v)$. (We'll focus on the simple case when the graph is $d$-regular, i.e, for all $v\in V$< $\deg(v)=d$.) (Pictorially, a subset of $G=(V,E)$ is very spiky, like a sea urchin.)

A graph $G=(V,E)$ is a \vocab{$\la$-spectral expander}if $\la_2(L(G))\ge \la$. The Laplacian $L(G)$ is defined from the adjacency matrix $A_G$. $A_G$ is defined by $(A_G)_{ij}=1$ if there is a edge between $i$ and $j$. $M$ is the random walk matrix $\rc d A_G$, and 
\[
L=I-M.
\]
(Thinking of this matrix as an operator, $(Lx)_i=\rc d\sum_{(i,j)\in E}x_{ij}$, it flows from a vertex to adjacent vertices.)
\end{df}
The Laplacian originates in differential geometry, $L=\div \nb$.

These 2 definitions are quite close.
\begin{thm}
Let $h(G),\la(G)$ denote the edge and spectral expansions of $G$. Then
\[
\la(G)\le h(G)\le \sqrt{2\la(G)}.
\]
\end{thm}
The LHS is trivial, the RHS is Cheeger's inequality.

People in CS use a third definition that is more useful.
\begin{df}
A graph is a $\la$-expander if for all $x\perp u=\colthree{1}{\vdots}1,x\ne 0$, 
\[
\ab{\fc{x^TMx}{x^Tx}}\le \la.
\]
\end{df}
(For edge and spectral expansion we want $\la$ to be large. For random walk expansion we want $\la$ to be close to 0.) %For the other definitions we want $\la$ to be close to 1.)
$\la(L)$ range from 0 to 2, while $\la(M)$ range from 1 to $-1$.

Note that 
\[
\ab{\fc{x^T(I-L)x}{x^Tx}}=\ab{1-\fc{x^TLx}{x^Tx}}
\]
so spectral and random-walk expansion are not quite the same. For bipartite graphs, $\la(M)=1$: random walks are not mixing at all. The side you're on depends on the parity.

For the zig-zag product they consider the third definition.

\subsection{Zig-zag product}

Think of $G_1$ having large size and degree $(N_1,D_1)$, $G_2$ having small size and degree $(N_2,D_2)$, and suppose $D_1=N_2$. 
\[
G_3=G_1\zz G_2
\]
has large size $N_1N_2$ and small degree, $D_2^2$.

Let $\la_M(G_i)=\la_i$ denote the random walk expansion. Then  (a naive bound)
\[
\la_M(G_3)\le \la_1+\la_2+\la_2^2.
\]
It's easy to construct a large degree expander. It's easy to construct small-size expanders (try all possible graphs). It's easy to construct $G_1,G_2$, but an expander like $G_3$ is hard to construct. It takes as input 2 easy-to-construct expanders and outputs a harder to construct graph that's an expander.

Replace each vertex of the original graph with a copy of the second graph $G_2$. Its size is hence $N_1N_2$.

Each vertex in $G_3$ is labeled $(v,u)\in G_1\times G_2$. The $(i,j)$th neighbor is defined as follows. Take $u'$ the $i$th neighbor of $u$ in $G_2$, $u\xra i u'$. Take the $u'$th neighbor of $v$ in $G_1$; move to the cloud of $w$; $v\xra{u'} w$ in $G_2$. Now consider $\wt u$ such that $w\xra{\wt u}v$ in $G_1$; we move to $(w,\wt u)$. Now move $\wt u \xra j \wt u'$ in $G_2$. Define
\[
(v,u)\xra{(i,j)} (w,\wt u').
\]
%(Use $G_1$ inside the cloud, $G_2$ outside.)

What's the magic of the zig-zag product? We claim
\[
A_3:=A(G_3) = \wt{A_2}\wt{A_1}\wt{A_2}
\]
where $\wt{B_2}=I\ot A_2$ has $N_1$ blocks and each is a copy of $A_2$, and $\wt{A_1}$ is a permutation (actually matching) matrix where $(\wt{A_1})_{((v,u),(w,u'))}=1$ if 
\[
v\xra{u,G_1} w \xra{u',G_1}v.
\]

Consider when $(u,v)\xra{(i,j)} (w,\wt u')$:
\bal
u&\xra{i,G_2}u'\\
v&\xra{u',G_1}w\xra{\wt u,G_1} v\\
\wt u & \xra{j}\wt u'
\end{align*}
What is the action of $\wt{A_2}\wt{A_1}\wt{A_2}$ on $e_{(v,u)}$? It goes to $\sum_{uu'\in G_2}e_{(v,u')}=e_v\ot A_2e_u$. ``Stay in the block of $v$, move to all the neighbors of $u$."

In the second step, we move across blocks as given by 
$\wt{A_1}$.

%(If you sum over $u$, you get $A_1$.)
%want converge in log n steps
%replace tree - length log n.
%log n lg d.

We need to show
\[
%\la_m(C_{i_3}) = \max_{x\perp u} 
\forall x\in \R^{N_1\times N_2}, x\perp u \implies  \fc{x^TM_3x}{x^Tx}\le \la_1+\la_2+\la_2^2.
\]
Decompose $x$ as a vector that is uniform on each block, 
\[x=\ub{\al\ot u}{\al^{\parallel}} + \ub{x'}{\al^{\perp}}\] where $x'$ has blocks $\wt{\al_i}\perp u_i$. Then (noting $\wt{A_2}\al^{\parallel}=\al^{\parallel}$),
\bal
\an{x,\wt{A_2}\wt{A_1}\wt{A_2}x} 
&=\an{\wt{A_2}x, \wt{A_1}\wt{A_2}x}\\
&=\an{\wt{A_2}(\al^{\parallel}+\al^{\perp}), \wt{A_1}\wt{A_2}(\al^{\parallel}+\al^{\perp})}\\
&=\an{\al^{\parallel}+\wt{A_2}\al^{\perp}, \wt{A_1}(\al^{\parallel} + \wt{A_2} \al^{\perp})}\\
&=\an{\al^{\parallel}, \wt{A_1}\al^{\parallel}}
+\an{\al^{\parallel}, \wt{A_1}\wt{A_2}\al^{\perp}}
+\an{\wt{A_2}\al^{\perp}, \wt{A_1}\al^{\parallel}}
+\an{\wt{A_2}\al^{\perp}, \wt{A_1} \wt{A_2}\al^{\perp}}\\
&\le \la_1+2\la_2+\la_2^2
\end{align*}
%\wt{A_2}\al^{\parallel}=\al^{\parallel}
Because $\al_1+\cdots +\al_{N_1}=0$ we can use the expansion properties of $G_1$ to get the $\la_1$ bound for the first term.

The idea: 
\begin{enumerate}
\item
Suppose $G$ has parameters $(N,D)$ and expansion $\la$. (say $1-\rc{ND}$)
\item
$G^2$ has parameters $(N,D^2)$, expansion $\la^2$.
\item
Take $H$ with parameters $(D^2,\sqrt{D}), \la_2$. Then 
\[
G^2\zz H=G_{\text{new}}
\]
has parameters $(ND^2,D)$ expansion $\la(G_{\text{new}}) \le \la^2 + 2\la_2+\la_2^2 \le \la^2+\ep$.
\end{enumerate}
We go from $G$ with parameters $(N,D,\la)$ to $G_{\text{new}}$ with parameters $(ND, D,\la^2+\ep)$. Call this operator $Z$. Doing this operator $t$ times, $Z^tG=(ND^{2t}, D, \la^{2^t}+\ep')$. Then set $t=\lg \pf{ND}2$ to get $Z^tG$ with parameters $(N^2D, D, \rc e +\ep)$. (We're cheating a little, using the naive bound. We need a better bound to get $\ep$ small: $G_3$ has expansion $\rc2(1-\la_2^2)\la_1+\rc2 \sqrt{(1-\la_2^2)^2\la_1^2+t\la_2^2}$. 
%$\la_2=0$ this is 0
%if slightly bigger, constant $\la_1$ plus something. $(1-\la_2^2)\la_1+\la_2
%sum of norm squares equals 1. Just that problem. Just n
To go from our proof to the real proof, you just need to note $\al^{\parallel}\perp \al^{\perp}$, $\al^{\parallel}\perp \wt{A_2}\al^{\perp}$. Plug in this picture into the formula.

%($\ep$ should be $\rc{N}$).
%D^2 vertices and ... edges. 
%Start with 2 constant size expanders and blow up. The real constructor: start with arbitrary bound, but need better bound.

\section{$O(\sqrt{\log n})$ approximation for sparsest cut by Arora, Rao and Vazirani}

The outline of the talk is as follows.
\begin{enumerate}
\item
Define the problem.
\item
Give a LP relaxation $(L,R)$ giving a $O(\ln n)$ approximation.
\item
$L_1$ metrics and sparsest cut
\item 
Give the SDP relaxation due to Arora, Rao, and Vazirani. 
\end{enumerate}•

\subsection{Problem}

\begin{df}
Given a graph $G=(V,E)$, a capacity function $c_e:E\to \R^+$, and $d:V\times V\to \R^+$, find
\[
\min_{S\subeq V} \fc{\sum_{(i,j)\in \de(S)} c_{(i,j)}}{\sum_{i,j\in \de(S)??} d_{(i,j)}}.
\]
\end{df}
For simplicity, we consider uniform sparsest cut $c_e=1,d_{ij}=1$, so we want
\[
\phi(S)=\min_{S\subeq V} \fc{\de(S)}{|S||V-S|}.
\]
This is closely connected to the edge expansion $\al(G)=\min_{S\subeq V,|S|\le \fc n2} \fc{\de(S)}{|S|}$. Approximating $\phi,\al$ are equivalent up to a constant factor.

It's NP-hard to find the sparsest cut exactly. %what about constant factor approximation?
%constant hardness for SoS algorithms
\subsection{Metrics}
\begin{df}
A \vocab{metric} is $(X,d)$ with
\[
d(x,z)\le d(x,y)+d(y,z),
\]
$d(x,y)=0$ iff $x=y$. (A semimetric only needs satisfy $d(x,y)=0$ if $x=y$.)
\end{df}

There is a natural semimetric one can define on the space of cuts. Assign a distance of 1 if they are separated by 1:
\[
d_S(x,y)=|1_S(x)-1_S(y)|.
\]
There is now a nice interpretation
\[
\phi(G)=\min_S\fc{\sum_{e\in E,(i,j)}d_S(i,j)}{\sum_{i,j}d_S(i,j)}.
\]
We can instead relax this to minimizing over all semimetrics $d$, rather than just metrics of the form $d_S$.
Call this $LR(G)$. 
We show
\[
LR(G)\le \phi(G)\le O(\ln n)LR(G).
\]
This can be formulated as a LP.

Find 
\[
\min\sum_{(i,j)\in E} d(i,j)
\]
subject to $d(i,k)\le d(i,j)+d(j,k)$., $\sum_{i,j}d(i,j)=n^2$, $d(i,j)\ge 0$, $d(i,i)=0$, $d(i,j)=d(j,i)$.
This can be solved in polynomial time.

Now we prove the inequality.

A finite $L_1$ metrics is $X\subeq \R^d$ with $d(x,y)=\ve{x-y}_1$. Given any metric $d$ we say there is an embedding $f:d\to L_1$ with distortion $\al$ if
\[
\ve{f(x)-f(y)}_1\le d(x,y) \le \ve{f(x)-f(y)}_1\al.
\]
$L_1$ metrics capture the sparsest cut problem.

Given any embedding $f$ to $L_1$, there exists $S,d_S$,
\[
\fc{\sum_e d_S(i,j)}{\sum_{i,j}d_S(i,j)}
\le 
\fc{\sum_e \ve{f(x)-f(y)}_1}{\sum_{i,j}\ve{f(i)-f(j)}_1}.
\]
%if you can embed with distortion 1 you get an exact algorithm
RHS is
\bal
%\fc{\sum_{i'}\sum_e |f_i
\ge \min_i \fc{\sum_e |f_i(x)-f_i(y)|}{\sum_{i,j}|f_i(i)-f_i(j)|}.
\end{align*}
$f_i(i)=g_i$. You can scale and shift so that $|\max g(i)- \min g(i)|=1$. Fix a threshold $d$, take the embedding on a line, and take the cut defined by that threshold.
Let $i^*$ be where the minimum is attained. It equals
\[
\fc{\E_t\sum_{e=\{i,j\}} d_{s_t}(i,j)}{\E_t \sum_{i,j} d_S(i,j)}
\ge  \fc{\sum d_{s_t}(i,j)}{\sum_{i,j}}.
\]
%move $g_i$ locally?

Now we use
\begin{thm}[Bourgain]
For every semimetric $d$ there is an embedding $f:d\to L_1$ with distortion $O(\ln n)$. 
\end{thm}
The tight example comes from analyzing the LP. A constant degree expander.
%dual multicommodity flows. Use lower bounds for those to give a lower bound for Bourgain's theorem.

Then 
\[
\ve{f^*(x)-f^*(y)}_1\le d(x,y) \le O(\ln n) \ve{f^*(x)-f^*(y)}_1.
\]
We get
\[
\le \fc{\sum_e \ve{f^*(x)-f^*(y)}_1}{\sum_{x,y} \ve{f^*(x)-f^*(y)}_1}\le O(\ln n) \fc{\sum_ed(x,y)}{\sum_{x,y}d(x,y)}
\]
Cone generated by semimetrics contains every $L_1$ metric.

Another relaxation is the spectral relaxation. 
Consider the normalized adjacency matrix $A$. The spectral gap is $1-\la_2$. 
The spectral gap is
\[
\sum_{x:V\to \R}\fc{\sum_C |x(i)-x(j)|^2}{\sum_{i,j} |x(i)-x(j)|^2}.
\]
%cheeger's theorem gives an approximation for this.
If we embed $x:V\to \R^n$ we use the $L_2$ norm. 

What if we combine these two relaxations: impose the triangle inequality constraint, etc. We ask that $\ve{x(i)-x(k)}^2\le \ve{x(i)-x(j)}^2 + \ve{x(j)-x(k)}$. 
\[
ARV(G)=\sum_{x:V\to \R^n,\triangle \text{ inequalities}}\fc{\sum_C \ve{x(i)-x(j)}^2}{\sum_{i,j} \ve{x(i)-x(j)}^2}.
\]
ARV showed
\[
ARV(G)\le \phi(G)\le O(\sqrt{\ln n}) ARV(G).
\]
The lower bound is
\[
\phi(G)\ge \Om(\ln \ln n)ARV(G).
\]

%look at canonical examples 
A cycle makes Cheeger's inequality tight. If you take a cycle and use the $L_1$ relaxation, it's not tight. (?) Tight ones are constant degree expanders. What is the balance between these two? ARV shows the union does give you some improvement. 

We prove $\phi(G)\le O(\sqrt{\ln n}) ARV(G)$. We show the inequality $\phi(G)\le O((\ln n)^{\fc 23})ARV(G)$ as it contains the main ideas but is similar.
%linear degree SoS hardness.

We show ARV can be computed in polynomial time.

Assign vectors $x(i)\in \R^n$ to minimize:
\[
\min \sum_{e}\ve{x(i)-x(j)}^2
\]
such that 
\[
\sum_{i,j}\ve{x(i)-x(j)}^2 = n^2 
\]
and for all $i,j,k$,
\[
\ve{x(i)-x(k)}^2\le \ve{x(i)-x(j)}^2 + \ve{x(i)-x(k)}^2.
\]
Find a cut such that the sparsity of the cut is at most $O(\sqrt{\ln n})$.

The rounding strategy is as follows. Suppose we can find an embeding $f:V\to \R$
\[
\fc{\sum_e |f(i)-f(j)|}{\sum_{i,j} |f(i)-f(j)|}\le O(\sqrt{\ln n}) ARV(G).
\]
Then we're done.
The strategy for finding such an embedding: Let $v_i$ the vector associated with vertex $i$ by the SDP. Look at the metric
\[
d(i,j) = \ve{v_i-v_j}^2.
\]
The strategy is to find a set $S$ such that defining $f(i)=d(i,S)=\min_{j\in S}d(i,j)$.
%vertices into $\R^n$

An easy inequality by the triangle inequality.
\[
f(i)-f(j)\le \ve{v_i-v_j}^2
\]
\[
ARV(G)=\fc{\sum_e\ve{v_i-v_j}^2}{\sum_{i,j} \ve{v_i-v_j}^2}.
\]
Numerator inequality ok. Average distortion. Need to show 
\[
\sum_{i,j}f(i)-f(j)\ge \fc{\sum\ve{v_i-v_j}^2}{\sqrt{\ln n}}.
\]

%given $L^2$ can embed in $L^1$?
%always embed with $O(\sqrt{\ln n})$ distortion.
%if true then 
later proved by ALN (2008): $L_2^2\to L_1$ with distortion $O(\sqrt{\ln n }\ln \ln n)$.

%\ell_2\to \ell_1
$L_2^2\to L_2\to L_1$. Second one is isometric.
%poly many.

Here we only need average case distortion.

We're trying to produce 2 disjoint sets $L$ and $R$ of vertices both of size $\Om(n)$ separated by $\De=\rc{\sqrt{\ln n}}$. Two sets that are far away. Let's do this for $(\ln n)^{\fc 23}$. (Separated in $L_2^2$ metric.) Then the equation is satisfied: look at summation between pairs coming from $L$ and $R$. 
%on average is 1
(Define $S=L$.) %one is 0.

Supposer there exists some vertex $i^*$ such that $|B(i^*,\rc4)|\ge \fc n4$. Set $S=B(i^*,\rc 4)$. Then the inequality holds:
\bal
n^2 &= \sum_{i,j}d(i,j)\\
& \le \sum_{i,j} d(i,i^*) + d(j,i^*) \\
&= 2n \sum_i d(i,i^*)\\
&\le 2n (\sum_i d(i,S)+\rc 4).
\end{align*}
Gives $\sum_i d(i,S)\ge \fc n4$. This gives
\[
\sum_{j\in S} d(j,S)\ge \fc n4. 
\]
Get $|S|\sum_{j\in S} d(j,S)$. At least $\fc{n^2}{16}$.

%contains a lot of vertices: ...

Suppose there does not exist a point like this. 

Algorithm to produce $L,R$: Extension of Goemans-Williamson rounding. Put points depending on side of hyperplane. Take a random line, project, if $>\si$ put on one side, if $<-\si$ but on other side.

Define $O=\amax_{O\in V} B(O,4)$. Now pick a random Gaussian $r_i\sim N(0,1)$. Let 
\bal
L&=\set{i}{(v_i-v_o)\cdot \ga\ge \si}\\
R&=\set{j}{(v_j-v_o)\cdot \ge \le -\si}.
\end{align*}
Assume $\De=\rc{(\ln n)^{\fc 32}}$. Find a nonseparated pair, throw it out. Prune until $\De$-separated.
%centered at v_a
By definition, $2\si$-separated in sense $(v_i-v_j)\cdot \ga \ge 2\si$. $\si$ is a constant. $\si>\de$.
$\ga_i\sim N(0,1)$ not unit vector. But expectation is $\sqrt n$.
%$L_2$-norm?
In expectation 
\[
\E[\ve{(v_i-v_j)\cdot \ga}^2] =\ve{v_i-v_j}^2.
\]
easy to show $\ln n$ approximation by concentration. Without needing Bourgain's result.

Let $L',R'$ be the bad pairs. 
\begin{thm}
With constant probability $L$ and $R$ are both $\Om(n)$.
\end{thm}
%Define sets using 
%no point contains many things in ball
2 constant size disjoint which are both constant distance away from $o$. Project own to... large.

Condition on it. %change distribution on gaussian? 
%condition on constant prob event
\begin{thm}
With good probability, $L',R'$ are $\Om(n)$ large.
\end{thm}
Show by contradiction. Given $L,R$, suppose $L',R'$ are small, i.e., $L',R'\le \fc{\al}2n$. 
%\ge \al n
Show that for constant fraction, can come up with $v_i,v_j$ such that $(v_i-v_j)\cdot \ga$...
Why $\ln n$ approximation? Take $\de=\rc{\ln n}$. 
? cannot satisfy approximation. 
Suppose $v_i,v_j$, $\Pj((v_i-v_j)\cdot \ga \ge t) \le e^{-\fc{t^2}{\ve{v_i-v_j}^2}}$.
%replace by $\le 2\si$?
\[
\ve{v_i-v_j}^2 \le \fc C{\De}\le \fc C{\ln n}
\]
\[
e^{-\si^2(\ln n)}\le n^{-4}.
\]
Any 2 pairs less than $\ln n$ away cannot have projection $>\si$. Whp no pairs from it removed.

Proving $(\ln n)^{\fc 32}$: construct points that are 
Tke many pairs throw out. Each time throw out from fixed $L,R$, consider it a matching. Put matchings together, becomes a weighted graph. Find path $v_{i_1},\ldots, v_{i_k}$. $k$-length paths
\[
\Pj((v_{i_k}-v_{i_1})\ge \si k).
\]
Show $\ve{v_i-v_j}^2 \le k\De$. Discrepance between squares.
\[
e^{-\fc{\si^2k}{\De}}.
\]
As you build paths. 
With constant probability can find these paths. Concentration of measure, reduce projection, increase probability... extend paths...
At least $\De$ fraction of edges going out of each vertex. Directed graph.

%fraction for which ij  gets thrown out.

Vertex $i$, edge $2j$. 
%prob over rand choise of $\ga$...
fixing $\ga$, bipartite directed graph, matching. 
Find paths in this graph which have these large projections. 
%make edge in graph. sometimes only take you back.

\section{Cryptography in $NC^0$}
By Applebaum, Ishai, Kushilevitz, 2004.

Question: Are there OWF's in $NC^0$?

\begin{df}
A \vocab{one-way function} satisfies the following.
\begin{enumerate}
\item
easy to compute: there is a polynomial time algorithm to compute it. (Here, though, we want each bit to be computable in $NC^0$.)
\item
hard to invert: for any probabilistic polynomial time algorithm $A$, 
\[
\Pj_x[f(A(f(x)))=f(x)]<O\prc{n^k}.
\]
\end{enumerate}
\end{df}
$NC_c^0$ means $NC^0$ with locality $c$ (each output bit depends on $c$ input bits).

\subsection{Previous work}
Linial, Mansour, and Nisan in 1989 showed there is no PRF in $AC^0$. (PRF's imply one-way functions.)

Cryan and Mitterson, 2001 showed there is no PRG in $NC_2^0$, and no PRG in $NC_3^0$ chieving superlinear stretch.

On the positive side, Impagliazzo and Naor in 1996 showed there is a PRG in $AC^0$ based on the hardness of subset sum: is there a subset of a set that sums to a certain number? %average case hardness, and subsets of size k.

Goldreich in 2000 showed there is no OWF in $NC_2^0$.

%multiplication in $NC^1$.

Result: Any OWF in $NC^1$ or even $\bigopl L/\poly$ can be compiled into an OWF in $NC_4^0$.

Barrington's Theorem shows $NC^1\subeq L/\poly$. L/poly (log space with polynomial-sized advice) is exactly the class of PBP's. NC-circuit, can also be computed by a poly-size branching program.
%NL/poly multiple passes.
%description of circuit is poly.

\subsection{Mod-2 branching program}

\begin{df}
A \vocab{mod-2 branching program} is $(G=(V,E),\phi,s,t)$ 
where $G$ is an acyclic directed graph, $s,t\in V$, $\phi$ is a labeling assigning each edge $x_i,\ol{x_i}, 1$. On input $(w_1,\ldots, w_n)$, define $G_w=(V,E_w),e\in E_w$ iff $\phi(e)$ is satisfied by $w$. 
Let $f(w)$ be the number of paths from $s$ to $t$ mod 2. 
\end{df}

The main technique is randomized encodings.
\begin{df}
A \vocab{randomized encoding} of $f:B^n\to B^l$ is 
\[
\wh f:B^t\times B^m\to B^s
\]
that satisfies the following.
\begin{enumerate}
\item
$\de$-correctness: there exists a deterministic algorithm $C$ (the decoder) such that for all $x\in B^n$, 
\[
\Pj[C(\wh f(x,U_m))\ne f(x)]\le \de.
\]
\item $\ep$-privacy: there exists an algorithm $S$ (simulator) such that for all $x\in B^n$, 
\[
\ve{S(f(x))-\wh f(x,U_m)}\le \ep.
\]
(It does not have additional information about the input.)
\end{enumerate}
A perfect randomized encoding has 
\begin{enumerate}
\item
$\de=\ep=0$,
\item
is balanced: $S(U_L)=U_S$.
\item 
stretch-preserving: $L-n=S-n-m$.
\end{enumerate}
%important for OWP
\end{df}
\begin{lem}
Let $\wh f$ be a perfect randomized encoding for $f$. If $f$ is hard to invert, then $\wh f$ is hard to invert.
\end{lem}
\begin{proof}
Suppose $\wh B$ inverts $\wh f(x,r)$. We obtain $B$ that inverts $f(x)$ as follows.
\begin{enumerate}
\item
Get input $y=f(x)$.
\item  Let $\wh y=S(y)$. 
\item Run $\wh B$ on $\wh y$ and get $(x',r')$.
\item Output $x'$.
\end{enumerate}
Perfect correctness gives the following property: If $f(x)\ne f(x')$ then the support of $\wh f(x,U_m)$ and $\wh f(x',U_m)$ are disjoint.
%distribution b gives to b had
\end{proof}

Now we only have to construct, given $f\in \opl L/\poly$, a perfect RE $\wh f$ in $NC_4^0$.

The steps are as follows.
\begin{enumerate}
\item
Given a mod-2 BP, construct a degree-3 randomizing polynomials.

Letting the BP be $(G,\phi,s,t)$ of size $L$, let $A(x)$ be the $L\times L$ adjacency matrix of $G_x$. 
Now let $L(x)$ be the submatrix of $A(x)-I$ obtained by removing the first column and last row.
%-1's on lower diagonal.

Fact: $f(x)=\det(L(x))$. 
Proof: 
\begin{align*}
f(x)&=(I+A(x)+\cdots +A(x)^L)_{s,t}\\
&=(\cancelto{(I-A(x)^{L+1})}0(1-A(x))^{-1})_{s,t}\\
&=(I-A(x))^{-1}_{s,t} \\
&=(-1)\fc{\det(-L(x))}{\det(I-A(x))} = \det(L(x)).
\end{align*}
$L(x)$ is not very private; it tells it a lot about the input.

Fact: Let $M,M'$ be $(L-1)\times (L-1)$ matrices with $-1$ on the second diagonal (below the diagonal) with 0's below. Then 
\[
\det(M)=\det(M')
\]
iff there exist $r^{(1)},r^{(2)}$ such that $R_1(r^{(1)}) M R_2(r^{(2)}) = M'$.
%vector of \binom{L-1}{2} entries
Here $R_1$ packs the entries of $r^{(1)}$ above the diagonal and has 1's on the diagonal, and $R_2$ packs the entries of $r^{(2)}$ into the last column and has 1's on the diagonal.

Proof: $\Leftarrow$ is trivial.

$\implies$: there exist $r^{(1)}$ and $r^{(2)}$ with $R_1(r^{(1)})MR_2(r^{(2)})$ having $\det(M)$ in the upper-right corner, $-1$ on the diagonal, and $0$'s everywhere else. (?) $M'$  satisfies the same condition.

$R_1(\cdot),R_2(\cdot)$ form a multiplicative group.

\begin{lem}
Suppose a BP computes $f$. Let $\wh f(x,(r^{(1)}, r^{(2)}))$ be the $\binom L2$ entries of $R_1(r^{(1)})L(x)R_2(r^{(2)})=M$. Then $\wh f$ is a perfect randomized encoding of $f$. 
\end{lem} 
\begin{proof}
\begin{enumerate}
\item
Perfect correctness: The decoder outputs $\det(M)=\det(L(x))=f(x)$. 
\item
Simulator: $S(y) = R_1(r^{(1)})MR_2(r^{(2)})$ where $M$ has $y$ in the upper right-hand corner.
\end{enumerate}
$S(f(N))$ and $\wh (f,U_m)$ have same support.
The support size is $2^{\#\text{ of random bits}}$.
%used by simulator or encoding.
\item $S(U_1)=U_{\binom L2}$.
%L=1 in $f$.
Each output bit is a degree-3 polynomial.
%adj - I. $L(x)$ lin let ese
\end{proof}
\item Given a degree 3 polynomial, construct a function in $NC_4^0$. 
Write
\[
f(x)=T_1(x)+\cdots +T_k(x)
\]
where $T_i$ re degree 3 monomials. Then 
\[
\wh f(x_1(r_1,\ldots, r_k, r_1',\ldots, r_{k-1}'))= (T_1(x)-r_1,\ldots, T_k(x)-r_k, r-r_1',r_1'+r_2-r_2', \ldots, r_{k-2}' + r_{k_1}
- r_{k-1}', r_{k-1}'+r_k).
\]
\begin{proof}
\begin{enumerate}
\item
Decoder sum.
\item
$y=(r_1,\ldots, r_{2k}, y-\sum_{i=1}^{2k} r_i$.
\item Balanced length preserving.
\end{enumerate}
\end{proof}
\end{enumerate}•
Impractical: 3 digit times 3 digits requires 10000 modes. Blowup is $n^7$. Number of output bits too large.

Problem: $NC_3^0$?

Where use randomness? If some function is OWF, rhan.. lso.
%replace $R_1,R_2$ by the identity matrix, what breaks?
%too much info on the input.

\section{PRIMES is in P}

By Agarwal, K, and Saxena.

We start with this observation.
\begin{thm}
$n$ is prime iff (for any $a\perp n$, $a$ relatively prime to $n$)
\[
(X+a)^n\equiv X^n+a\pmod n.
\]
(Equality of polynomials: each coefficient matches.)
\end{thm}
\begin{proof}
If $n$ is prime then $\binom ni\equiv 0\pmod n$ for all $1\le i\le n-1$ and $a^n\equiv a\pmod n$ (Fermat's little theorem).

If $n$ is not prime consider some prime $p\mid n$. Suppose $p^r||n$. We claim that $n\nmid \binom{n}{p^r}$. We show this with Lucas's Theorem.
\end{proof}

This gives a primality test, but takes time $O(n)$.

The trick is the following. Instead check if
\[
(X+a)^n \equiv X^n+a \pmod{n,X^r-1}.
\]
We show that if there exists $r$ such that for $O(\ln^{O(1)} n)$ many $a$ the above holds, then $n$ is prime, and conversely.

The algorithm is as follows.
\begin{enumerate}
\item
Check if $n$ is a perfect power $a^b,b>1$. (There are $\lg n$ possible values for $b$.)
\item
Find the smallest $r$ such that $\ord_rn\ge (\ln n)^2$. Here $\ord_r n = \min \set{a>0}{n^a\equiv 1\pmod r}$.
\item
Check for all $1\le a\le \sqrt{\ph(r)}\ln n=:l$ that %mult group %totient is upper bound on $t$. Think of this as $\sqrt t \ln n$. $\ln n \le \srqt t$. Put $\ph(r)$ works as well. $t\ge (\ln n)^2$.
\[
(X+a)^n \equiv X^n+a \pmod{n,X^r-1}.
\] 
(Note you do repeated squaring, there are only $r$ coefficients to remember.)
%nothing smaller divides evenly into .
\end{enumerate}
If $n$ passes all these checks, $n$ is prime.

We want to show that $r=O(\ln^{O(1)} n)$.

If $n$ is composite; we show it can't pass all the tests.
\begin{thm}
$r\le (\ln n)^5$.
\end{thm}
\begin{proof}
Suppose by way of contradiction that for all $r\le (\ln n)^5$, $\ord_p n<(\ln n)^2$. 
If $\ord_p n = a$, $r\mid n^a-1$.
Hence 
\[\lcm (1,\ldots, (\ln n)^5)\mid \prod_{i=1}^{(\ln n)^2} (n^i-1)
\le n^{(\ln n)^4} = 2^{(\ln n)^5}.\]
%However, we can bound each of these guys
However
\[
\lcm (1,\ldots, m) \ge 2^{m-1}.
\]
\end{proof}
\begin{proof}
Assume $p\mid n$, $p<n$. WLOG $\ord_rp>1$. Assume the first equation is true. The second is also true since $p$ is prime.
\bal
(X+a)^n\equiv X^n+a \pmod{n,X^r-1}\\
(X+a)^p \equiv X^p+a \pmod{n,X^r-1}
\end{align*}
%Raising the first equation to the $\fc np$ power
The first equation is
\bal
((X+a)^p)^{\fc np} &\equiv (X^p)^{\fc np}+a
\iff 
(X^p+a)^{\fc np} & =(X^p)^{\fc np} +a.
\end{align*}
Choose $q$ such that $pq\equiv 1\pmod r$. Then
\bal
(X^{pq}+a)^{\fc np} &\equiv (X^{pq})^{\fc np}+a\\
(X+a)^{\fc np} &\equiv X^{\fc np}+a.
\end{align*}
We want to look at exponents for which things like this are true.
\begin{df}
Given a polynomial $f(X)$ and an integer $m$, $m$ is \vocab{introspective} for $f(X)$ if 
\[
f(X^m)\equiv f(X)^m\pmod{n,X^{r}-1}.
\]
\end{df}
We showed $n,p,\fc np$ are introspective for $X+a$ for all $a, 1\le a\le l$.
Introspective numbers are closed under multiplication.
\begin{thm}
If $m$ and $m'$ are introspective for $f$ then so is $mm'$.
\end{thm}
\begin{proof}
\bal
f(X)^{mm'}&\equiv f(X^m)^{m'}\pmod{n,X^r-1}\\
f(X^m)^{m'}&\equiv f(X^{mm'}) \pmod{n,X^{mr}-1}\\
f(X^m)^{m'}&\equiv f(X^{mm'}) \pmod{n,X^{r}-1}
\end{align*}
becuse $X^r-1\mid X^{mr}-1$.
\end{proof}
\begin{thm}
If $m$ is introspective for $f_1$ and $f_2$, then $m$ is introspective for $f_1f_2$.
\end{thm}
\begin{proof}
\[
f_1(X^m)f_2(X^m)\equiv f_1(X)^mf_2(X)^m \equiv (f_1(X)f_2(X))^m.
\]
\end{proof}
These generate a lot of introspection. This gives many polynomial relations, which is bad.

We define 2 groups based on introspection.
\begin{itemize}
\item
Define $H\subeq (\Z/r)^{\times}$ generated by $p,\fc np$ (both these are relatively prime to $r$).
\item
(Annoyingly, $\Z[X]/\an{n,X^{r}-1}$ is not a field. We convert it to a field in the following way.)
Let $h(X)$ be an irreducible factor of the $r$th cyclotomic polynomial modulo $p$. (It is not linear because the order is $>1$.)
Let $\fc{F_p[X]}{\an{h(X)}}$. (I.e., adjoin a $r$th root of unity to $\F_p$.)
Let $G\subeq F^{\times}$ be generated by $(X+1),\ldots, (X+l)$ in $F$.
%ord_rp=1. Already have $r$th roots mod $p$.
\end{itemize}•
We upper-bound and lower-bound $G$ to get a contradiction.

Let $t=|H|$. Since $\ord_r n \ge (\ln n)^2$, $t\ge (\ln n)^2$.
\begin{thm}
$|G|\ge 2^t$.
%lenstra
\end{thm}
\begin{proof}
Consider all products of $(X+i)$ of degree $\le t$. We claim all these products are distinct in $G$. 

By way of contradiction, suppose we have 2 products $f(X)=g(X)$ in $G$ ($F$). Choose $m\in H$. By construction of $H$, $m$ is introspective for $f=g$: $f(X)^m=g(X^m)$ so $f(X^m)=g(X^m)$. %$X$ is an $r$th root of unity here, $\ze$.
Let $Q(Y)=f(Y)-g(Y)$ in $\F_p[Y]$. ($f,g\in \F_p[X]$.)

This implies $X^m$ is a root of $Q(Y)$ for all $m\in H$. The $X^m$ are all distinct because everything in $H$ is relatively prime to $r$.

Thus $Q(Y)$ has at least $t$ roots. This is bad since the degree is $<t$. %$t=|H|\le \ph(r)\ln n$

We get a contradiction since $l\ge t$; there are at least $2^t$ different products.
\end{proof}
\begin{thm}
$|G|\le n^{\sqrt t}$.
%sudan
\end{thm}
\begin{proof}
Consider the set
\[
\wt I = \set{p^i\pf{n}{p}^j}{0\le i,j\le \sqrt t}.
\]
By Pigeonhole there exist $m_1,m_2\in \wt I$ such that $m_1\equiv m_2\pmod r$.
%because size of $H$ is $t$, and all products in $H$. 
Then $X^{m_1}=X^{m_2}$. 
Since $f(X)\in G$,
\[
f(X)^{m_1}=f(X^{m_1})=f(X^{m_2}) = f(X)^{m_2}.
\]
Let $Q(Y)=Y^{m_1}-Y^{m_2}$. Then $f(X)$ is a root of $Q$ for all $f\in G$. But $\deg Q\le \max (m_1,m_2)\le n^{\sqrt t}$. Hence $|G|\le n^{\sqrt t}$.

(This only works if $n$ is not a power of $p$.)
\end{proof}
These 2 theorems together imply that $2^t\le |G|\le n^{\sqrt t}$. However, $t> (\ln n)^2$, so 
\[
n^{\sqrt t}< 2^t\le |G|\le n^{\sqrt t}
\]
That's it!
\end{proof}
%cleaned it up.
%in annals.
In the paper they get $\fc{t+l}{t-1}$. Bur for $l\ge t$, it's $\ge 2^t$.

%RP\cap coRP. ZPP.


\section{Optimal inapproximability of MAXCUT from UGC}
%Unique Games Conjecture and hardness of approximation}
This is a paper from [KKMO04].

Goemans-Williamson achieves $\al\approx0.878$-approximation to MAXCUT.

If UGC is true than approximating MAXCUT better than $\al_{GW}+\ep$ is NP-hard for every $\ep>0$.

The way to show it is the following: Look at $\mathsf{Gap-MaxCut}_{(c,s)}$, $c>s$. If this is NP-hard for some parameters $c,s$, then a $\fc cs$ approximation is NP-hard.

They prove that for all $\ep>0,-1<\rh<0$, $\mathsf{Gap-MaxCut}_{(\fc{1-\rh}2-\ep,\fc{\cos^{-1}\rh}{\pi}+\ep)}$ is NP-hard assuming UGC.

Unconditionally, ($\fc{16}{17}+\ep$)-approximation of Gap-MaxCut is NP-hard.

Maximizing the bound gives the GW bound:
\[
\max_{-1<\rh<0} \fc{\pf{1-\rh}2}{\cos^{-1}\fc{\rh}{\pi}}=\al_{GW}\approx 0.878.
\]
The bad case for GW gives the same bad case here.

We construct a PCP-like theorem. For every $-1<\rh<0,\ep>0$, construct a 2-query PCP system with alphabet $\{-1,1\}$ and completeness $c=\fc{1-\rh}2-\ep$ and soundness $s=\fc{\cos^{-1}\rh}{\pi}+\ep$ in which the verifier.

The verifier looks at 2 bits and depending on the 2 bits, accepts or rejects. If $x$ is in the language, accept with probability $\ge c$ and if not, accept with probability $\le s$, using only $\ne$ checks.
%The only possible functions are equality and inequality.

What is MaxCut? It's some kind of equality constraint.

Given a PCP, estimate the proof that will make the verifier accept with the highest probability. The probability of acceptance is the max cut in the graph constructed as follows:  the vertices are the number of bits. The weight on $v_iv_j$ is $\Pj\pat{verifier queries $i,j$}$. The max cut is the probability of acceptance. %verifier pick distribution.

If there is  PCP system, you can convert it into a max cut system. Estimate $\max_{\text{proofs}} \Pj(V\text{ accepts }p)$.
%MAXCUT with $\om_{ij}\sim \Pj(i,j\text{ is queries})$.

%3-query is NP in general.
%3SAT to maxcut: if there's a gap there, there is a gap here.

%reduce a NP-problem to this?
%reduce UCG to this.

If $L$ is a NP-hard language, then we are done. We don't know existing NP-hard problems with this property, so we start with UGC instead.

Consider Gap-UniqueLabelCover$(\Si)_{(1-\de,\de)}$. Consider a bipartite graph $G=(V\cup W,E)$.  Each edge has a permutation $\pi_{vw}:\Si\to \Si$ that is left regular. We want $\si:V\cup W\to E$ such that for all $(v,w)\in E$, $\pi_{vw}(\si(w))=\si(v)$.  The value is the maximum fraction of edges satisfiable.

For example, say $\Si=\F_q$. The constraint is that $\si(v)+\si(w)=c_{vw}$. $\mathsf{Max2Lin}(\F_q)$.
%max stable cover
%label cover maps
Label cover is NP-hard.

%in NP.
%Either value $\ge 1-\de$ or $<\de$.

\begin{conj}[UGC (Khot)]
For all $\de>0$ there exists $m=|\Si|$ such that $GapUGC_{(1-\de,\de)}(\Si)$ is NP-hard.

\end{conj}

We give a 2-query PCP system for this problem.

We need the long code. It is $E:[m]\to \set{f}{\{-1,1\}^m\to \{-1,1\}}\cong B^{B^m}$, mapping $i\mapsto e_i=\Dict_i$ where $e_i(x)=x_i$.

Proofs only has $-1$ and 1. Can't give labels, so encode labels using encoding to turn into bits. Use long code to turn labels into bits.
%long code of ech label in the optimal labelling.

Find a random edge, check to see if the permutation is satisfied. However, only bits can be checked!

We give the dictatorship test.
Suppose we want to know whether the function is (close to) a long code or not.
Given $f:\{-1,1\}^m\to \{-1,1\}$ we want to check if $f$ is a dictator using 2 queries.

The simplest attempt is to take $x\sim \{-1,1\}^n$. Check if $f(x)\ne f(-x)$. This is bad because all odd functions satisfy this.

Attempt 2: Instead, we randomly flip some bits. $x\sim \{-1,1\}^n$. Take $y\sim_\rh x$ where 
\[
y_i=\begin{cases}
x_i,&\text{with probability } \fc{1+\rh}2\\
-x_i,&\text{with probability } \fc{1-\rh}2.
\end{cases}
\]
Check if $f(x)\ne f(y)$. 
For the dictator, we get
\begin{enumerate}
\item
$C=\fc{1-\rh}2$, the probability a single fixed bit is flipped.
\item
Soundness: 
\bal
s(f) & =\Pj(f\text{ is accepted})\\
&= \Pj_{x\sim_\rh y}(f(x)\ne f(y))\\
&=\rc2-\rc2\ub{\E_{x\sim_\rh y} [f(x)f(y)]}{\Stab_\rh(f)}.
%correlation with noisy version of itself.
\end{align*}
\end{enumerate}•
For example, consider $f=\Maj(x)$. 
\[
\Pj(\Maj\text{ is accepted})=\rc2-\rc2\Stab_\rh(\Maj_m)\to \fc{\cos^{-1}\rh}{\pi}
\]
as $m\to \iy$.
Proof: Use the central limit theorem to get $\Maj\pf{\sum x_i}{\sqrt m}$ approaches a Gaussian.

\begin{thm}[Majority is stablest, MOO05]
Let $-1<\rh<0,\ep>0$. There exists $\tau,C$ such that if $f$ passes the test with probability $>\fc{\cos^{-1}\rh}{\pi}+\ep$, then there exists $i\in [m]$ such that $\Inf_i^{\le c}(f)\ge \tau$.
\end{thm}
\begin{df}
Define $x^{\opl i}$ is $x$ with the $i$th coordinate flipped. Note the Fourier expansion is $f(x)=\sum_{S\subeq [m]}\wh f(S)\chi_S(x)$, $\wh f(S)=\E_x f(x)\chi_S(x)$, $\chi_S(x)=\prod_{i\in S}x_i$. Note $\sum_{S\subeq [m]}\wh f(S)^2 =\E [f(x)^2] = 1$.
\bal
\Inf_i(f)&=\Pj_{x\sim\{-1,1\}^n}[f(x)\ne f(x^{\opl i})]\\
&= \sum_{S\ni i} \wh f(S)^2.
\end{align*}
For example $f(x)=x_i$ has $\Inf_i(e_i)=1$. We also have $\Inf_i(f)=1$ for all $i$, which we want to exclude. Thus we look at 
\[
\Inf_i^{\le c}(f) =\sum_{S\ni s,|S|\le c}\wh f(S)^2.
\]
\end{df}
%equivalent to stable influences?
%(There must be an influential coordinate.)

Now we come to the 2-query PCP for GapULC$(\Si)_{(1-\de,\de)}$. Given a graph, %we ask the prover to give us the long code $f_v:\{-1,1\}^m\to \{-1,1\}$
the verifier asks the prover to encode the optimal labelling using the long code. How do we check it satisfies $1-\de$ fraction? This is inspired by the dictatorship test. We can only test whether one of the symbols is a dictator or not. I give you a sequence of functions, how can you test? The test combines!

Attempt 1: 
\begin{itemize}
\item
Pick a random $(v,w)\sim E$. Let $f_v,f_w$ be the supposed long codes of $\si(v)$ and $\si(w)$. 
\item
Let $\pi=\pi_{vw}$.
\item
Choose $x\sim_\rh y$ ($\rh$-correlated).
(We're trying to combine the 2 test.)
\item
Accept if $f_v(x)=f_w(y\circ \pi)$ where $(y\circ \pi)_i = y_{\pi(i)}$. Here $y\in \{-1,1\}^m$ and $\pi:[m]\to [m]$.
\end{itemize}•
This doesn't work because the prover can set $f_v=1$ for all $v\in V$ and $f_w=-1$ for all $w\in W$ and by bipartiteness the verifier always accepts.

$x_{\si(v)}=e_{\si(v)}(x)\ne e_{\si(w)}(y\circ \pi) = e_{\pi(\si(w))}(y) = y_{\si(v)}$.

We modify the test.
\begin{itemize}
\item
Pick $v\sim V$ and $w,w'\sim N(v)$. (We assume left-regularity.)
\item
Let $\pi=\pi_{vw},\pi'=\pi_{vw'}$.
\item Let $x\sim_\rh y$.
\item
Accept if $f_w(x\circ \pi)\ne f_{w'}(y\circ \pi')$. 
\end{itemize}

Completeness: An optimal labeling will satisfy $\ge 1-\de$ of the edges. The distribution of $(v,w),(v,w')$ are marginally uniform. 
\[
\Pj(\ge1\text{ of }(v,w),(v,w'))\le \Pj((v,w)\text{ not})+\Pj((v,w')\text{ not})\le 2\de.
\]
Write $f_w=e_{\si(w)}$, $f_{w'}=e_{\si(w')}$, 
\[
x_{\si(v)}=
x_{\pi_{vw}(\si(w))} = e_{\si(w)}(x\circ \pi) \ne e_{\si(w')} (y\circ \pi') = y_{\pi_{vw'}(\si(w'))} =y_{\si(v)} 
\]
For every $\de$, some $m$ for which NP-hard. Set $\de$ to be small.
Completeness is
\[
C\ge (1-2\de) \pf{1-\de}2 \ge \fc{1-\rh}2-\ep,
\]
$\de\le \eph$.

Soundness: Suppose $\Pj(V\text{ accepts})\ge C+\ep$. We want to show there exists a labeling which satisfies $>\de$ fraction of edges.
%we'll extract a labeling.

This is the main part of the proof.

For at least $\fc{\ep}2$ fraction of $v\in V$, the test conditioned on selecting $v$ accepts with probability $>\pf{\cos^{-1}\rh}{\pi}+\eph$.

Otherwise by averaging argument 
\[\Pj(\text{accept})\le \eph I + (1-\eph) (\fc{\cos^{-1}}{\rh} + \eph)<\cos^{-1}\fc{\rh}{\pi}+\ep.\]
Call such $v$'s good. Then 
\[
\Pj\pat{accept} = \E_{w,w'}\ba{
\E_{x\sim_\rh y} \ba{
\rc2-\rc2 f_w(x\circ \pi) f_{w'}(y\circ \pi')
}
}.
\]
Let 
\[
g_v(z)=\E_{w\sim N(v)}[f_w(z\circ \pi_{vw})].
\]
If this is a proper labeling, $e_{\si(w)}(z\circ \pi_{vw})=z_{\pi_{vw'}(\si(w))} = z_{\si(v)}$.
%extract labeling.
We'll extract the most influential coordinates.

The expectation goes inside. Each expectation you can replace with $g$.
\bal
\Pj\pat{accept} 
&= \E_{x\sim_\rh y} [\rc2-\rc2g_v(x)g_w(x)]\\
&=\rc2-\rc2\Stab_\rh (g_v) > \cos^{-1}\pf{\rh}{\pi} + \eph.
\end{align*}
Now we use Majority is Stablest. There exists $j\in [m]$ such that 
\[
\Inf_j^{\le C}(g_v)\ge \tau.
\]
Assign $\si^*(v)=j_v$. %the label should be dictator index.

%We need to keep in mind tht 
%Given $v$ can we use it fin
Assign a label for $w$ independent of $v$. The label of $v$ depend on its neighbors, the label for $w$ depends only on its long code.

We show that the label for $w$ should also have high influence for $f_w$. Expand 
\bal
f_w\circ \pi_{vw}&=\sum_{S\subeq [m]} \wh f_w(S)\chi_S(x\circ \pi_{vw})\\
&=\sum_T \wh{f_w}(\pi^{-1}(T))\chi_T\\
g_v&=\E_{w\sim N(v)}[f_w(x\circ \pi_{vw})]\\
&=\sum_T \E_v \wh f_w(\pi^{-1}(T))\chi_T.
\end{align*}
Influence high says that influence of neighbors $f_w$ for these neighbors high for some coordinates. By Jensen,
\bal
\tau & \le \Inf_{j_v}^{\le C}(g_v)\\
&=\sum_{|S|\le C,S\ni j_v} \wh g_v(S)^2\\
&=\sum_{|S|\le C, S\ni j_v} \pa{
\E_w\ba{\wh {f_w}(\pi^{-1}(S))}^2}\\
&\le \sum_{|S|\le C,S\ni j_v} \E_w\ba{\wh{f_w}(\pi^{-1}(S))^2}\\
&= \E_w \ba{\sum_{|S|\le C,S\ni j_v}\wh{f_w}(\pi^{-1}(S)^2)}\\
&=\E_w\ba{\Inf_{\pi^{-1}_{vw}(i_v)}^{\le C}(f_w)}.
\end{align*}
By another averaging argument, for at least $\fc{\tau}2$ fraction of $v$'s neighborhood, $\Inf_{\pi_{vw}(j_v)}^{\le C}(f_w)\ge \fc{\tau}2$.

Look at all coordinates with high influence and choose one with high influence randomly: Let 
\[
S_w = \set{k}{\Inf_k^{\le C}(f_w)\ge \fc{\tau}2}.
\]
Assigning $\si^*(w)\sim_R S_w$, $|S_w|\le \fc{C}{\pf{\tau}2}=\fc{2C}{\tau}$.
\begin{proof}
We have 
\[
\sum_{i=1}^m\Inf_i^{\le C}(f_w) = \sum_{|S|\le C}|S|\wh{f_w}(S)^2.
\]
($\Inf_i^{\le C}(f_w)=\sum_{S\ni i,|S|\le C}\wh{f_w}(S)^2$. The number of $S\ni i$ is $|S|$.)
Choose a random $(v,w)\sim E$. What is $\Pj(\si^*(v)=\pi_{vw}(\si^*(w)))$? It's
\bal
\Pj(\si^*(v)=\pi_{vw}(\si^*(w)))
&=\Pj\pat{$v$ good}
\Pj\pat{$w$ good for $v$|$v$ is good}
\Pj\pat{$\si^*(v) = \pi_vw(\si^*(w))$|...}\\
&\ge \eph\fc{\tau}2\rc{\pf{2C}{\tau}}\\
&=\fc{\ep\tau^2}{8C}=\de.
\end{align*}
%de independent of size of graph, depend on size of alphabet size.
%Last: $\Pj(\pi_{vw}^{-1}(j_v)=\si^*(w))$.
\end{proof}
$m$ depends on $\de$ depends on $\tau,C$ depends on $\ep,\rh$.
We reduced $\mathsf{Gap-ULC}_{(1-\de,\de)}(\ep)$ to $\mathsf{Gap-MaxCut}_{(\fc{1-\rh}2-\ep, \fc{\cos^{-1}\rh}{\pi}+\ep)}$.
%dependence (on alphabet size) can be bad.
%optimal SDP for every constraint sat problem 
%constant arity
%pick $m(\de)$ to make GapULC NP-hard. Now forget about $m$.

In MaxCSP, maximize the constraints satisfied. $C_i(x|_S)$, $|S|\le q$ constant arity. Maximize number of constraints satisfied. There is an SDP that achieves best approximation ratio. Generalizes GW.

Charikar: given ULC with $\ge 1-\de$ satisfiable, there is an algorithm to satisfy $\rc{|\ep|^{\fc{\de}2+O(S^2)}}$ fraction of constraints.

UGC captures limitations of SDP methods. Assuming UGC is like saying SDP is the best. 
%look ifferent.

There is a reason to disbelieve UGC if you think for some problem there can be something better than SDP.

Matching, determinants: require exponential size SDP.

Lower bounds: cannot compute parity using SDP?
%many algs are optimal
%if we want to improve we need so solve this problem
%algorithmic technique solve all these at once. if solve UGC then can solve all these CSP.
%win-win: conjecture true/false.
%american conjecture: win-win scenario.

%hastad 3bit pcp: max3sat. unique label, not unique label cover.
%proof idea same. test depends on problem. test at end reflect constraint.

%r o'donnell's course.

\section{Hastad's 3-bit PCP}
This was proved by Hastad in 1997. It's based on hardness of label cover.

\subsection{Hardness of label cover}
\begin{df}
An instance $I$ of label cover consists of
\begin{enumerate}
\item
labels $L,R$,
\item
$G=(U,V,E)$,
\item
for each edge $E$ in $G$, a function. $\Pi=\set{\pi_e:L\to R}{e\in E}$. (In unique label cover the functions have to be permutations, but not here.)
\end{enumerate}•
%Output: 

A labeling is a pair of functions $A:U\to L, B:V\to R$. An edge $(u,v)$ is satisfied iff $\pi_{(u,v)} (A(u))=B(v)$. 

Define the value $\val(I)$ to be the maximum possible fraction of satisfied edges.
\end{df}

\begin{thm}[Hardness of label cover]
For all $\de\in (0,1)$, there exist $L,R$ such that $|L|,|R|\le \exp\prc{\de}$ such that $\mathsf{Gap-LC}(L,R)_{(1,\de)}$ is NP-hard, where 
\begin{itemize}
\item
Yes instances are $I$ with $\val(I)=1$,
\item
No instances are $I$ with $\val(I)\le \ep$.
\end{itemize}
\end{thm}
The proof is by Ran Raz in 1995, by the Parallel Repetition Theorem.

\begin{thm}[3 bit PCP]
For all $\ep,\de\in (0,1)$, 
there exists a PCP for Gap-LC${}_{(1,\de)}$ over the boolean alphabet satisfying the following.
\begin{enumerate}
\item
The verifier queries 3 bits and checks $x_{i_1}\opl x_{i_2}\opl x_{i_3}=b$.
\item
Completeness is $1-\ep$.
(If it is a yes instance, there exists a proof that is accepted with probability $\ge 1-\ep$.)
\item
Soundness is $\rc2+\de$.
(If it is a no instance, all proofs are accepted with probability $\le \rc2+\de$.)
\end{enumerate}
\end{thm}

\begin{cor}[Hardness of MAX3LIN2]
It is NP-hard to distinguish: given a system of linear equations over $\F_2$ with 3 variables, are
\begin{itemize}
\item
$\ge 1-\ep$ satisfiable,
\item
$\le \rc2+\ep$ satisfiable.
\end{itemize}
%poly n
%for all $\ep,\de$ can find $\de'$.
\end{cor}
This is optimal because a random assignment gives a 2-approximation.

\begin{cor}[Hardness of MAX3SAT]
For all $\ep,\de\in (0,1)$, it is NP-hard to distinguish a $(1-\ep)$-satisfiable instance and $(\fc 78+\de)$-satisfiable instance.
\end{cor}
A random assignment will give a $\fc 78$ approximation.

If we can solve MAX3SAT with better than $\fc78$ approximation, we can get better than 2-approximation for MAX3LIN. This gives a reduction from MAX3SAT to MAX3LIN2. Encode $x_{i_1}\opl x_{i_2}\opl x_{i_3}=1$ with
\begin{gather*}
x_{i_1}\vee x_{i_2}\vee x_{i_3}\\
x_{i_1}\vee \ol{x_{i_2}}\vee \ol{x_{i_3}}\\
\ol{x_{i_1}} \vee \ol{x_{i_2}}\vee x_{i_3}\\
\ol{x_{i_1}} \vee x_{i_2}\vee \ol{x_{i_3}}\\
\end{gather*}

We use the long code,
\[
\text{long}_L(x_1,\ldots, x_L) = \chi_c.
\]
\subsection{Hastad's 3 bit PCP}
Hastad's 3-bit PCP is as follows.
The input is a LC instance $I$. The proof is
\bal
f_u:\{-1,1\}^L \to \{1,-1\} ,\forall u\in U\\
f_v:\{-1,1\}^R \to \{1,-1\} ,\forall v\in V
\end{align*}
The verifier does the following. 
\begin{enumerate}
\item
Pick $(u,v)\sim_R E$. 
\item
Pick $x\sim_R\{-1,1\}^R$, $y\sim_R\{1,-1\}^L$. 
\item
Pick $\mu\sim \{1,-1\}^L$ such that 
\[\mu_i=\begin{cases}
1,&\text{w. p. } 1-\ep\\
-1,&\text{w. p. }\ep.
\end{cases}
\]
\item
For all $i\in L$, take $z_i=X_{\pi_{(u,v)}(i)}y_i\mu_i$.
\item
Accept if $f_u(z)=f_v(x)f_u(y)$.
\end{enumerate}
%\ep removed by later proof?

\begin{proof}[Completeness.]
Let $A,B$ be labelings satisfying all edges. Let $f_u(x_1,\ldots, x_L)=X_{A(u)}$ and $f_v(x_1,\ldots, x_R)=X_{B(V)}$.
Then
\begin{align*}
\Pj\pat{accept} & = \Pj(f_u(z)=f_v(x)f_u(y))\\
&=\Pj[Z_{A(u)} = X_{B(V)} Y_{A(U)}\\
&=\Pj(X_{\pi_{(u,v)}(A(u))}Y_{A(u)} = X_{B(v)}Y_{A(u)})\\
&=\Pj(\mu_{A(u)}=1)=1-\ep.
\end{align*}
\end{proof}

Problem: if you set the bits all to 1, the verifier will always accept the proof.

\begin{proof}[Soundness]
The valid long code should satisfy the following condition: $f_u(-x)=-f_u(x)$ (we say it is folded). 

We don't ask the prover to give all the information; we just ask the prover to give half the information. For all $f_u$, and a pair $(x,-x)$, only one of $f_u(x),f_u(-x)$ is given in the proof.
%query and negate to get the other one.

Claim: if a function $f:\{-1,1\}^L \to \{-1,1\}$ is folded, then $\wh{f}(\phi)=0$. 

Proof: $\wh f(\phi)=\E_x[f(x)]=0$.

We have to show that if $\Pj\pat{accept} \ge \rc2+\de$ then there exists a labelling $(A,B)$ that satisfies $\ge \de'$ fraction of edge constraints. 

Notation: Given $S\subeq L$, $\pi:L\to R$, 
\bal
\pi(S)&= \set{j\in R}{\exists L\in S,\pi(L)=j}\\
\pi_2(S)&= \set{j\in R}{\text{odd number of $L$'s in $S$, $\pi(L)=j$}}
\end{align*}

Claim: For all $x\in \{-1,1\}^R$, $\chi_S(x\circ \pi)=\chi_{\pi_2(S)}(x)$. ($(x\circ \pi)_i=x_{\pi(i)}$)

Proof: $\chi_S(x\circ \pi)=\prod_{i\in S} x_{\pi(i)} = \prod_{j\in \pi(S)}x_j$. We have
\bal
\rc 2+\de \le \Pj\pat{accept} &= 
\E_{u,v,x,y,\mu} \ba{
\fc{1+f_u(x)f_v(x)f_u(y)}{2}
}\\
\implies \E[f_u(z)f_v(x)f_u(y)] & \ge  2 \de.
\end{align*}
This implies at for $\ge \de$ fraction of edges $(u,v)$, the value $\E f_u(z)f_v(x)f_u(y)\ge \de$.

To prove soundness, set $f_u=f, f_v=g$ for notational convenience.  Then
\bal
\E [ f(z) g(x)f(y)] & =\sum_{S,T,U} \wh f_S \wh g_T \wh f_U \E_{x,y,u} [\chi_S(x)\chi_T(x)\chi_U(y)]&z=(\chi \circ \pi) y \mu\\
&=\sum_{S,T,U} \wh f_S \wh g_T \wh f_U 
\E_{x}[\chi_S(x\circ \pi) \chi_T(x)]
\E_y [\chi_U(y)\chi_S(y)]
\E_\mu [\chi_S(\mu)]\\
&=\sum_S \wh f_S^2 \wh g_{\pi_2(S)} (1-2\ep)^{|S|}. %noise stability
\end{align*}
This is $\ge \de$ for $\de$ fraction of edges.

Decoding a labeling is done as follows. 
\begin{enumerate}
\item
For all $u$, pick $S$ with probability $\wh f_u^2(S)$.
\item 
Set $A(u) \sim_R S$.
\end{enumerate}•
For all $v$, 
\begin{enumerate}
\item
pick $T$ with probability $\wh f_v^2(T)$. 
\item Sample $B(v)\sim_R T$.
\end{enumerate}
Why does this labeling satisfy some fraction of edges?

Fix $(u,v)$.
\begin{align*}
\Pj[(u,v)\text{ is satisfied}] &= \Pj[\pi_{(u,v)}(A(u))=B(v)]\\
&\ge \sum_S\sum_{T\subeq \pi(S)} \wh f_u^2(S)\wh f_v^2(T) \rc{|S|}.
\end{align*}
%once fix $B(v)$ at least one element in $S$, $\pi$ of that element equals $v$.

What is the probability that a random edge is satisfied?
\bal
\Pj_{(u,v)\in E}[(u,v)\text{ is satisfied}]
&\ge %\sum_{S,T\subeq \pi(S)}
\sum_S \wh f_u^2(S)\wh f_v^2(\pi_v(S))\rc{|S|}\\
&=\sum_S (\wh f_u(S) \wh f_2(S)\rc{\sqrt|S|})^2 \sum_S\wh{f}_u^2(S)\\
&\stackrel{\text{CS}}\le \pa{
\sum_S \wh f_u^2(S) \wh f_v(\pi_2(S))\rc{\sqrt{|S|}}
}^2\\
&\ge 4\ep(\sum_S\wh f_u^2(S) \wh f_v(\pi_2(S))(1-2\ep)^{|S|})^2\\
&\ge \de \cdot 4\ep, \de^2=4\ep\de^2 =\de'
\end{align*}
%1/2 +\ep 
Use $\rc{\sqrt x}\ge \sqrt{4\ep} (1-2\ep)^x$.
(Take derivatives to see this.)
\end{proof}
Easier than PCP we saw last timel. Do 

%samples in poly time. 
%suppose cn solve this, show solve ...

\section{ZKP for NP}

What is a zero-knowledge proof? IP plus some notion of zero knowledge.

Recap: In IP
\begin{itemize}
\item
Prover (Merlin), computationally unbounded
\item
Verifier (Arthur), computationally bounded, probabilistic
\end{itemize}
Prover wants to convince verifier that $x\in L$. This procedure should be
\begin{itemize}
\item
Complete: if $x\in L$, V should accept (with perfect completeness---can always make one-sided (?))
\item
Sound: if $x\nin L$, V should reject with high probabilistic
\end{itemize}

What is zero knowledge?
Prover should convince $x\in L$ without giving any other info. The verifier can simulate the entire procedure, in the following sense. There exits an efficient algorithm $S$ such that 
\[
\Pj(S\text{ outputs }\pi) =\Pj((P,V)\text{ outputs }\pi).
\]
This is called Perfect Zero Knowledge.
Statistically Zero Knowledge is when the distributions are $\approx$: superpoly samples are required.
Computational Zero Knowledge means for all $x\in L$, no polytime algorithm can distinguish between output of $S$ and $(P,V)$.
%inputs to both.
This is the version used in crypto.

We give some canonical examples.
\begin{enumerate}
\item
Graph nonisomorphism (PZK):

Input: $G_1,G_2$. Prover wants to convince verifier $G_1\ncong G_2$.

Verifier chooses random permutation $H$ of either $G_1$ or $G_2$ and sends it to the prover.

Prover says whether $H\cong G_1$ or $H\cong G_2$.

If $G_1\cong G_2$, prover can't be right with probability $>\rc2$.

Here is the simulator: picks $G_1,G_2$, chooses random isomorphism, and outputs index chosen: $(\pi(G_b),b)$. 
\item
Graph isomorphism (PZK): Input $G_1,G_2$. 

Prover sends V a random permutation $H$ of $G$. 

Verifier says 1 or 2.

Prover sends permutation $\pi$ such that $\pi(H)=G_b$.

Simulator: You might think this is impossible, but the simulator doesn't need to solve graph isomorphism. With probability $\rc2$ output $(\pi(G_1),1,\pi^{-1})$. With probability $\rc2$ output $(\pi(G_2),2,\pi^{-1})$.
\end{enumerate}
We show 3-coloring has a zero-knowledge proof, conditioned on the existence of 1-way functions. (Actually, you need bit commitment. Encrypt a bit and send it to the verifier and then reveal later.)
%Reduce to the 3-coloring problem and then give a proof of that.

Let $G$ be the input graph with a 3-coloring $C$. 

$P$ randomly permutes colors in C and then commits. Each vertex is colored.

Verifier chooses a random edge. The verifier reveals the colors on that edge. Accept if the vertices have different colors.

$\rc{n^2}$ probability of failure.
%Reject if crypto failed, or the colors the same.
%encrypts with secret key, and later shares.

If $|G|=n$, then repeat $n^3$ times. Issues in parallel?

Question: Can you do in parallel? (Concurrent zero-knowledge.)

This is not perfect zero-knowledge because of bit-commitment. If it is perfectly binding, the verifier could figure all the colors from the commitment. for a computationally bounded verifier.

OWF$\implies$PRG$\implies$bit commitments. (OWP: $h$ hardcore predicate. $f(x),h(x\opl b)$. Given $f(x)$ hard to predict $h(x)$.)

If has a PZK proof, what happens? If any NP-complete problem has PZK proof, the poly hierarchy collapses to the 2nd level.

Simulator outputs random edge. Permutation ensures random colors. $e, (u,v)$, encrypts those. For the rest, assign random bit commitments. It doesn't have to start at the beginning, it just has to output a plausible transcript. It's not perfect; if you had enough time you can check the bit commitments are bogus. %Simulator has to be efficient and doesn't know coloring.
%break commitment scheme

%\chapter{Motivation}

%We'll see lots of analysis in action.
\section{Dirac delta function}

What is the derivative of the Dirac delta?

You may have seen the mysterious ``\textbf{Dirac delta function}" defined by 
\beq{eq:dist0-1}
\int_{-\iy}^{\iy} \de(x-x_0)f(x)\,dx=f(x_0), \qquad \de(x)=0,\,x\ne 0.
\eeq
This emerged from Fourier's classical treatise on heat. It was there implicitly in his work. Cauchy and Dirac noticed it. It is used in math, applied math, physics, engineering.
%does the job we're asking it

But there is no function in any sense of the word that does this job! It makes no mathematical sense!

%Let's look at~\eqref{eq:dist0-1} in an abstract sense. 
However, looking at~\eqref{eq:dist0-1} in an abstract sense, the ``process" which takes $\de(x-x_0)$ and the ``nice" function $f(x)$, and spits out $f(x_0)$ is well-defined.
%There's some $\de(x-x_0)$, given $f(x)$, some process happens and spits out $f(x_0)$. 

However, people don't just talk about the delta function, they also talk about its derivative! Trying to differentiate something that doesn't exist...? %We'll put on our applied maths hat on and try to define the derivative.

How can we define the derivative $\de'(x-x_0)$? A first attempt might be (assuming $f$ is nice)
\begin{align}
\nonumber
\int \de'(x-x_0)f(x)\,dx&=\lim_{h\to 0} \int 
\pf{\de(x-x_0+h)-\de(x-x_0)}{h}f(x)\,dx\\
\nonumber
&=\lim_{h\to 0} \fc{f(x_0-h)-f(x_0)}{h}\\
\nonumber
&=-f'(x_0)\\
\llabel{eq:dist0-2}
&=-\int \de(x-x_0)f'(x)\,dx.
\end{align}
The equality~\eqref{eq:dist0-2} suggests that we could have simply integrated by parts
\[
\int \de'(x-x_0)f(x)\,dx=-\int \de(x-x_0)f'(x)\,dx+\ub{\text{boundary term}}{0}.
\]
This function that doesn't exist seems to follow the usual rules of calculus! 
This suggests there is a way of interpreting all the integrals in a consistent way
We can make all this rigorous using the theory of distributions. %looks like the usual rules of integral calculus can be applied to $\de(x-x_0)$.

\section{Fourier transforms of polynomials}

The \textbf{Fourier transform} is defined by (abbreviate $\int:=\iiy$)
\[
\cal F:f\mapsto \wh f(x)=\int e^{-i\la x}f(x)\,dx.
\]
This makes sense if $f$ is absolutely integrable:
\[
\ab{
\int e^{-i\la x}f(x)
}\le \int |e^{-i\la x}f(x)|\,dx=\int |f(x)|\,dx<\iy.
\]
What if $f\nin L^1$, in particular, what if $|f|\not\to 0$ as $|x|\to \iy$? You may have seen
\beq{eq:dist0-3}
\de(\la)=\rc{2\pi}\int e^{-i\la x}\,dx.
\eeq
There is a way of interpreting this so that it makes sense. This seems to suggest that the Fourier transform of $\rc{2\pi}$ is equal to $\de(\la)$. What if $f(x)=x^n$? Can we take the Fourier Transform?
\bal
\int e^{-i\la x}x^n\,dx&=\int \pa{i\pd{}{\la}}^n e^{-i\la x}\,dx\\
&=\pa{i\pd{}{\la}}^n \int e^{-i \la x}\,dx\\
&=2\pi e^{-i \pi n/2}\de^{(n)}(\la).
\end{align*}
If we can make sense on the derivatives of $\de$, then we can define the Fourier transform of polynomials.

An alternative way of defining Fourier transform of $f(x)=x^n$ would be to use Parseval's Theorem, which states
\beq{eq:dist0-4}
\int f(x)\wh g(x)\,dx =\int \wh f(\la)g(\la)\,d\la
\eeq
for all ``nice" functions $f$ and $g$. We could define $\hat f(\la)$ where $f(x)=x^n$, as the function for which
\beq{eq:dist0-5}
\int x^n\hat g(x)\,dx=\int \hat f(\la)g(\la)\,d\la\text{ for all nice functions } g
\eeq
we could say that $\hat f(\la)$ is the Fourier transform of $f(x)=x^n$. Note $\hat g(x)$ decays quickly, so this makes sense. This can be done rigorously using the theory of distributions. %If we can find $\hat f(x)$

``Everything's easy when you know the answer." It's only perfectly natural when you've been shown its perfectly natural. To prove consistency is not quite so easy.

\section{Discontinuities to Differential Equations}
%we want to see things like that happen.

There are some important genuinely important physical scenarios in which we would like a solution to a PDE to have discontinuities. For example, in acoustics we want the pressure $p(x,t)$ to solve the wave equation
\beq{eq:dist0-6}
p_{xx}-p_{tt}=0,
\eeq
but for $p$ to jump either side of a shock wave.
%propagate out?
%pavilion g

Is there any meaningful way to say that the function 
\[
u(x,y)=\al(x-y)+\be(x+y)
\]
is a solution to the wave equation $u_{xx}-u_{yy}=0$ if $\al,\be \nin C^2$? Assume $\al,\be\in C^2$ and $u_{xx}-u_{yy}=0$ so that for any nice function $f(x,y)$ (say $f=0$ when $x^2+y^2$ is sufficiently large),
\bal
0&=\int\int f(x,y)\pa{\pd{{}^2u}{x^2}-\pd{{}^2u}{y^2}}\,dx\,dy\\
&=\int\int u(x,y) \pa{\pd{{}^2}{x^2}-\pd{{}^2f}{y^2}}\,dx\,dy
\text{integration by parts twice}.
\end{align*}
If we can find $u(x,y)$ such that
\beq{eq:dist0-7}
\int\int u(x,y) \pa{\pd{{}^2}{x^2}-\pd{{}^2f}{y^2}}\,dx\,dy=0
\text{ for all nice functions }f,
\eeq
we say that $u=u(x,y)$ is a \textbf{weak solution} to the PDE $u_{xx}-u_{yy}=0$. We can use distribution theory to study weak solutions to PDE's.
%3 reasons want formalize

\section{Summary}

In each motivating example we introduced a family of ``nice" functions that allowed us to extend classical definitions to a wider class of problems. This is the underlying idea in distribution theory. Given a vector space $V$ of ``nice" functions we define the distributions on $V$ to be the topological dual $V^*$, which consists of all the continuous linear forms $V\to \C$. 

For example, if $V=C(\R)$ then we can define Dirac delta by the linear form\footnote{You may be more familiar with the notation $\an{x,y}=x\cdot y$ for finite-dimensional vector spaces.}
\beq{eq:dist0-8}
\an{\de_{x_0},f}=f(x_0).
\eeq
%linear map.
In general, any $u\in V^*$ is linear, so $\an{u, \al f+\be g}=\al\an{u,f}+\be \an{u,g}$ for arbitrary constants $\al,\be$ and $f,g\in V$. We need functional analysis because we require continuity. (The algebraic dual is too big to be interesting. Hence we supplement $V$ with a topology, i.e. a notion of convergence $f_n\to V$ in $V$. This is the motion of $w^*$-convergence, $\an{u,f_n}\to \an{u,f}$.

\blu{Lecture 2}
\chapter{Distributions}

Recap:
\begin{itemize}
\item
Delta function doesn't make sense.
\item
Way to define distributions is to first define a ``nice" space of functions $V$ (having all the properties we want it to have) and define distributions as continuous linear maps from $V$ to $\C$. 
\end{itemize}
%V has all properties we want it to have.

We'll always work with continuous functions, so we can define continuity of functions $V\to \C$ very explicitly.

\section{Notation and preliminaries}

An element of $\R^n$ will be written $x,y,z,\ldots$ so that 
\[
x=(x_1,x_2,\ldots, x_n)
\]
and we will use $dx=dx_1\,dx_2\,\cdots \,dx_n$ to denote the volume element in $\R^n$. Capitals $X,Y,Z$ will always denote open subsets of $\R^n$ and $K$ will always denote a compact (closed and bounded) subset of $\R^n$. Integrals over all $\R^n$ or over $X\subeq \R^n$ will be denoted by $\int [\cdot]\,dx$ and $\int_X[\cdot]\,dx$, respectively.
We will use multi-index notation $\al,\be,\ga$ (Greek letters) will denote multi-indices $\al=(\al_1,\ldots, \al_n)\in \Z_+^n=\{0,1,2,3,\ldots\}^{\times n}$. Multi-index notation reads as follows.
\bal
\pl^{\al} &\equiv\pa{\pdd{x_1}}^{\al_1}\cdots \pa{\pdd{x_n}}^{\al_n},&x^{\al}&\equiv x_1^{\al_1}\cdots x_n^{\al_n}\\
\al!&\equiv \al_1!\cdots \al_n!&|\al|&\equiv \al_1+\cdots +\al_n.
\end{align*}
%functions for which it's switched on.
We will often write $\pl^{\al}_x$ to make it clear what we're differentiating with respect to. We will also use $D=-i\pl$ when we do Fourier analysis. Define the \textbf{support} of a function $f$ by 
\[
\Supp(f)=\ol{\set{x\in \R^n}{f(x)\ne 0}}.
\]
We will often want to take limits inside integrals. To do this we refer to the dominated convergence theorem: (See for instance~\url{https://dl.dropboxusercontent.com/u/27883775/math\%20notes/18.125.pdf}, Theorem 15.1.)
\begin{thm}[Dominated convergence theorem]\llabel{thm:dct}
Given a sequence of absolutely integrable functions $\{f_m\}_{m\ge 1}$ such that $f_m(x)\to f(x)$ for each $x$ and $|f_m(x)|\le g(x)$, where $g$ is absolutely integrable, then 
\[
\lim_{m\to \iy}\int_Xf_m(x)=\int_X\ba{\lim_{m\to \iy} f_m(x)}\,dx=\int_X f(x)\,dx.
\]
\end{thm}
If $f$ is absolutely integrable on $X$, i.e.,
\[
\int_X |f|\,dx<\iy,
\]
we say that $f\in L^1(X)$.

\section{Test functions and distributions}
We need to define our first vector space of test functions.
\begin{df}
The space $D(X)$ consists of all the smooth functions from $X$ to $\C$ that have compact support. We say that a sequence $\{\ph_m\}_{m\ge1}$ tends to zero in $D(X)$ if there exists some compact set $K\subeq X$ such that $\Supp(\ph_m)\subeq K$ and $\pl^{\al}\ph_m\to 0$ uniformly for each multi-index $\al$.

The space $D(X)$ is often written $C^{\iy}_c(X)$.
\end{df}
(For convergence, the function is not allowed to have its mass moving away to infinity.)
%integrate by parts, evaluated on the boundary, derivatives shift to the other side.

Since the functions in $D(X)$ vanish at the boundary of $X$, we have
\[
\int_X\ph \pl^{\al}\psi\,dx=(-1)^{\al}\int_X\psi\pl^{\al}\ph\,dx,
\qquad \ph,\psi\in D(X)
\]
by integration by parts $|\al|$ times. We have Taylor's Theorem to any order
\beq{eq:taylor}
\ph(x+h)=\sum_{|\al|\subeq N}\fc{h^{\al}}{\al!} \pl^{\al}\ph(x)+R_N(x,h)
\eeq
where the remainder is $o(|h|^N)$ uniformly in $x$. %treat it as  ncice polynomial plus remainder.

%we have space of tests. distributions will be defined as linear maps. We want them to be continuous. $u(\ph)\equiv \an{u,\ph}\in \C$. If $\ph_m\to 0$ in $D(X)$, $\an{u,\ph_m}\to 0$ in $\C$.
\begin{df}\llabel{df:distribution}
A linear map $u:D(X)\to \C$ is a \textbf{distribution} (on $X$) if
for each compact $K\subeq X$, there exist $C,N$ such that 
\beq{eq:dist1-1}
|\an{u,\ph}|\le C\sum_{|\al|\le N} \sup|\pl^{\al}\ph|
\eeq
%vs sequential continuity. 
for each $\ph\in D(X)$ with $\Supp(\ph)$ with $\Supp(\ph)\subeq K$. The space of all such maps is denoted $D'(X)$. If we can use the same $N$ for all $\ph\in D(X)$, we call the least such $N$ the order of $u$, denoted $\ord(u)$.
\end{df}
%in the back of mind have this example, go back to familiar example.
\begin{rem}
Thinking of $D(X)$ as a locally convex space (in fact, Fr\'echet space) with seminorms defined by $\sup|\pl^{\al}\ph|$, we have $D'(X)=D(X)^*$. See Example 2.2.4 in Functional Analysis\footnote{\url{https://dl.dropboxusercontent.com/u/27883775/math\%20notes/part_iii_functional.pdf}}.
The example there is actually a larger space ($\cal E(X)$ of the next chapter), but it contains our $D(X)$.
\end{rem}

\begin{ex}
\begin{enumerate}
\item
We check that the Dirac delta $\de_x$ is a distribution. $\de_x$ is defined by
\[
\an{\de_x,\ph}=\ph(x)\qquad\ba{\int \de(x-y)\ph(y)\,dy=\ph(x)}.
\]
We want to check if~\eqref{eq:dist1-1} holds:
\[
\ab{\an{\de_x,\ph}}=|\ph(x)|\le \sup|\ph|
\] 
so~\eqref{eq:dist1-1} holds with $C=1$, $N=0$,  no matter what $\ph$ we choose. So $\de_x$ is a distribution of order 0.
\item
Here is a more useful example. Consider the linear map $T$ on $D(X)$ defined by 
\[
\an{T,\ph}:=\sum_{|\al|\le M} \int_X f_{\al}\pl^{\al} \ph\,dx,
\]
$f_{\al}\in C(X)$. Now for $\ph\in D(X)$, with $\Supp(\ph)\subeq K$. By definition of $T$, 
\bal
|\an{T,\ph}|&=\ab{
\sum_{|\al|\le M}\int_K f_{\al} \pl^{\al}\ph\,dx
}\\
&\le \sum_{|\al|\le M} \int_K |f_{\al}||\pl^{\al} \ph|\,dx\\
&\le \sum_{|\al|\le M} \sup_{|\al|\le M}|\pl^{\al}\ph| \int_K |f_{\al}|\,dx\\
&\le \pa{\max_{\al}\int_K |f_{\al}|\,dx}\sum_{|\al|\le M} \sup |\pl^{\al} \ph|
\end{align*}
Note that the test functions have compact support, so it doesn't matter if $f_{\al}$ blows up at the boundary. So we have estimate~\eqref{eq:dist1-1} with 
\[
C=\max_{|\al|\le M}\int_K |f_{\al}|\,dx,\qquad N=M.
\]
Note that $C=C_K$. We could have done this only assuming that $\{f_{\al}\}$ were locally integrable on $X$ (integrable on every compact subset of $X$), written $f_{\al}\in L_{\text{loc}}^1(X)$.

Note here the constant $C$ depends on the support test function. $N$ can also depend on it.
%$N$ can also depend on the support of the test function.
\end{enumerate}
\end{ex}
\begin{lem}\llabel{lem:dist1-1}
A linear map $u:D(X)\to \C$ belongs to $D'(X)$ if $\an{u,\ph_m}\to 0$ for every sequence $\{\ph_m\}_{m\ge 1}$ in $D(X)$ that tends to 0.
\end{lem}
\blu{Lecture 3 (27 Jan)} 
\begin{rem}
That Definition~\ref{df:distribution} and Lemma~\ref{lem:dist1-1} are equivalent conditions for continuity is a special case of Lemma 2.2.5 in the functional analysis notes.
\end{rem}
\begin{proof}
\begin{itemize}
\item
$\implies$: If $u\in D(X)$ and $\ph_m\to 0$ in $D(X)$ then
\[
|\an{u,\ph_m}|\le \sum_{|\al|\le m} \sup|\pl^{\al} \ph_m|\to 0.
\]
\item
$\Leftarrow$: Assume not, so there exists a compact set $K\subeq X$ such that estimate~\eqref{eq:dist1-1} does not hold for any $C,N$. In particular, it doesn't hold for $C=N=m$. So there exist $\phi_m\in D(X)$ such that $|\an{u,\phi_m}|>m\sum_{|\al|\le m}\sup |\pl^{\al} \phi_m|$. WLOG, we can assume that $\an{u,\phi_m}=1$, by setting $\wt{\phi_m}=\fc{\phi_m}{\an{u,\phi_m}}$. This implies
\bal
\implies \sum_{|\al|\le m} \sup|\pl^{\al}\phi_m|&<\rc m\\
\implies \sup|\pl^{\al}\phi_m|&<\rc m,&|\al|\le m\\
\implies \phi_m\to 0& \text{ in }D(X).
\end{align*}
But this is a contradiction.
\end{itemize}
\end{proof}
\section{Limits in $D'(X)$}
Often we have  sequence of distributions $\{u_m\}_{m\ge 1}$. If there exist some $u\in D'(X)$ such that $\an{u_m,\ph}\to \an{u,\ph}$ for all $\ph\in D(X)$, then we say that $u_m\to u$ in $D'(X)$.

Limits in $D'(X)$ often look strange. 
\begin{ex}
For instance, define the distribution $u_m\in D'(\R)$ by the locally integrable function
\[
u_m(X):=\sin (mx).
\]
Then $u_m\to 0$ in $D'(\R)$. 

%compact support, ibp no boundary terms.
Proof: We have using \blu{integration by parts}
\bal
\an{u_m,\ph}&=\int \sin(mx)\ph(x)\,dx\\
&=\rc{m} \int \cos(mx)\ph'(x)\,dx
\end{align*}
Hence $u_m\to 0$ in $D'(\R)$. 
\end{ex}
%think about Fourier expansion of $\ph$
\begin{thm}[Closure under pointwise convergence]\llabel{thm:dist1-1}
If $u_m\in D'(X)$ is such that $\lim_{m\to \iy}\an{u_m,\ph}$ exists for every $\ph\in D(X)$, then the linear map
\[
\an{u,\ph}:=\lim_{m\to \iy}\an{u_m,\ph}
\]
is an element of $D'(X)$. 
\end{thm}
It's obvious that the LHS will satisfy the estimate or the definition. Use the principle of uniform boundedness (See the Banach-Steinhaus Theorem, 4.2.7 in FA notes).

\section{Basic operations}

\subsection{Differentiation and multiplication by smooth functions} 
If $u\in C^{\iy}(X)$, then $\pl^{\al}u$ defines an element of $D'(X)$ for every multi-index $\al$ by
\bal
\an{\pl^{\al}u,\ph}&=\int_{X}\pl^{\al}u\ph\,dx\\
&=(-1)^{|\al|} \int_X u\pl^{\al} \ph\,dx&\text{integration by parts}\\
&=\an{u,(-1)^{|\al|}\pl^{\al}\ph}.
\end{align*}
%add in f
%Makes sense for any distribution, so it's a good definition.
The RHS is well-defined for any $u\in D'(X)$. We arrive at the following.
\begin{df}\llabel{df:dist1-3}
For $f\in C^{\iy}(X)$, $u\in D'(X)$ and any multi-index $\al$ we define $\pl^{\al}(fu)$ by 
\[
\an{\pl^{\al}(fu),\ph}:=\an{u,(-1)^{|\al|} f\pl^{\al}\ph}.
\]
We call $\pl^{\al}u$ the \textbf{distributional derivatives} of $u$.
\end{df}
\begin{ex}
Take the Dirac delta $\de_x$. Then $\pl^{\al}\de_x$ is defined by
\[
\an{\pl^{\al}\de_x,\ph}:=\an{\de_x,(-1)^{|\al|} \pl^{\al}\ph} = (-1)^{|\al|} \pl^{\al}\ph(x).
\]

Consider the Heaviside function
\[
H(x)=\begin{cases}
1,&x>0\\
0,&x\le 0.
\end{cases}
\]
Then this defines an element of $D'(\R)$. We compute $H'$:
\[
\an{H',\ph}:=\an{H, -\ph'}=\int_0^{\iy} -\ph'(x)\,dx = -\ph|^{\iy}_0=\ph(0)=\an{\de_0,\ph}.
\]
Hence $H'=\de_0$. In general if $\an{u,\ph}=\an{v,\ph}$ for all $\ph\in D(X)$ then we say $u=v$ in $D'(X)$. 
%gradient infinite.
\end{ex}
Now we ask: how does the calculus for normal functions carries over to calculus for distributions?
\begin{lem}
If $u'=0$ in $D'(\R)$ then $u$ is a constant.
\end{lem}
\begin{proof}
Note that $u'=0$ means 
\[0=\an{u',\psi}=-\an{u,\psi'}.\]
%every function that is a total derive.
%{\it Note that }

{\it \blu{Idea: we'd like to say that given $\ph$, we can write $\ph=\ddd x\int_{-\iy}^x \ph(y)\,dy=\psi'$, and use the above to conclude $0=\an{u,\ph}$, so $u$ is constant. The problem is that when we integrate a test function, we don't necessarily get a test function. We need to adjust our function so that the integral is 0 for large $x$.}}

Fix $\te\in D(\R)$ with $\an{1,\te}=\int \te\,dx=1$. For arbitrary $\ph\in D(\R)$ write 
\bal
\ph&=(\ph-\an{1,\ph}\te)+\an{1,\ph}\te\\
&\equiv \ph_A+\ph_B
\end{align*}
Then 
\[
\an{1,\ph_A} =\an{1,\ph}-\an{1,\ph}\an{1,\te}=0.
\]
This is helpful because 
\[
\psi_A(x)=\int_{-\iy}^x \ph_A(y)\,dy\in D(\R).
\]
%
We have $\ph_A=\psi_A'$. So 
\bal
\an{u,\ph}&=\an{u,\ph_A}+\an{u,\ph_B}\\
&=\an{u,\psi_A'}+\an{1,\ph}\ub{\an{u,\te}}b=0+c\an{1,\ph} =\an{c,\ph}.
\end{align*}
This implies that $u$ is a constant.
\end{proof}
\subsection{Reflection and translation}
\begin{df}
For $\ph\in D(\R^n)$ then we can define its \textbf{translation} by $h\in \R^n$ by 
\[
(\tau_h\ph)(x):=\ph(x-h)
\]
and \textbf{reflection}
\[
\check \ph(x)=\ph(-x).
\]
\end{df}
(Motivation: $\an{u,\ph}=\int u\ph\,dx$ gives
$\an{\tau_h u,\ph}=\int u(x-h)\ph(x)\,dx=\int u(x)\ph(x+h)\,dx=\an{u,\tau_{-h}\ph}$.) By duality, the definitions of these operations on $u\in D'(\R^n)$,
\bal
\an{\tau_hu,\ph}&:= \an{u,\tau_{-h}\ph}\\
\an{\check u,\ph}&:=\an{u,\check{\ph}}.
\end{align*} 
\begin{lem}\llabel{lem:dist1-3}
For $u\in D'(\R^n)$ define
\[
v_h=\fc{\tau_{-h} u-u}{|h|}.
\]
Then $v_h\to n\cdot \pl u$, where $\lim_{h\to 0} \fc{h}{|h|}=n\in S^{n-1}$. 
\end{lem}
\begin{proof}
We have
\bal
\an{v_h,\ph}&=\an{\fc{\tau_{-h}u-u}{|h|},\ph}\\
&=\an{u,\fc{\tau_h\ph-\ph}{|h|}}.
\end{align*}
By Taylor's Theorem,
\bal
\tau_h \ph(x)-\ph(x)&=-h\pl \ph(x)+\ub{R(x,h)}{o(|h|)\text{ in }D(\R^n)}\\
\an{u,\fc{\tau_h\ph-\ph}{|h|}}&=\an{u,-\fc{h}{|h|}\pl \ph}+\ub{\an{u,\fc{(R(\cdot ,h))}{h}}}{\to 0\text{ as $|h|\to 0$}}\\
&=n\cdot \an{\pl u, \ph}.
\qedhere
\end{align*}
\end{proof}
This shows the distributional derivative coincides with the normal notion of derivative as difference quotient.

\blu{Lecture 4 (29 Jan)}

\subsection{Convolution between $D(\R^n)$ and $D'(\R^n)$}
If we combine the operations of reflection and translation, we get 
\[
(\tau_x\check{\ph})(y)=\check{\ph} (y-x)=\ph(x-y).
\]
If $u\in D(\R^n)$, we define the \textbf{convolution} of $u$ and $\ph\in D(\R^n)$ with
\[
(u*\ph)(x):=\int u(x-y)\ph(y)\,dy=\int \ph(x-y)u(y)\,dy=(\ph*u)(x).
\]
Using $\tau_h$ and $\check{\bullet}$, we can write 
\[
u*\ph(x)=\an{u,\tau_x \check{\ph}}.
\]
This is well defined for all $u\in D'(\R^n)$.
\begin{df}\llabel{df:dist1-5}
For $u\in D'(\R^n)$ and $\ph\in D(\R^n)$, define their convolution by 
\[u*\ph(x)=\an{u,\tau_x\check{\ph}}.\]
\end{df}
It is clear that $u*\ph(x)$ is just some function of $x\in \R^n$. It is actually smooth.
\begin{lem}\llabel{lem:dist1-4}
Define $\Phi_x(y)=\phi(x,y)$ where $\phi\in C^{\iy}(\R^n\times \R^n)$ and $\phi(\cdot, y)=0$ for $y$ outside some compact $K\subeq \R^n$. Then for $u\in D'(\R^n)$, 
\[
\pl_x^{\al} \an{u,\Phi_x}=\an{u, \pl^{\al}_x \Phi_x}.
\]
\end{lem}
We can take the derivatives inside the bracket.
\begin{proof}
By Taylor's theorem,  
%left with is linear map on $H$.
%use defn of $D'$
\[
\Phi_{x+h}(y)-\Phi_x(y)=\sum_i h_i \pd{\phi}{x_i} (x,y)+R_x(y,h)
\]
It is not difficult to show that $R_x(y,h)=o(|h|)$ in $D(\R^n)$ for each $x\in \R^n$. Hence
\[
\an{u,\Phi_{x+h}}-\an{u,\Phi_x}=\sum_i h_i \an{u,\pl{\Phi_x}{x_i}}+\an{u,R_x(\cdot, h)}.
\]
%sequence of test functions tend to 0, then continuity means $u$ acting on that sequence tends to 0.
Since $R_x(\cdot, h)=o(|h|)$ in $D(\R^n)$, dividing by $|h|$ and taking $h\to 0$ gives ($u=\fc{h}{|h|}$)
\[
n\cdot \an{u, \Phi_x}=n\cdot \an{u,\pl_x\Phi}.
\]
So the result follows.
\end{proof}
\begin{cor}\llabel{cor:dist1-1}
If $u\in D'(\R^n)$ and $\ph\in D(\R^n)$ then $u*\ph$ is smooth and $\pl^{\al}(u*\ph)=u*\pl^{\al}\ph$. 
\end{cor}
\begin{proof}
By Lemma~\ref{lem:dist1-4}, $\pl^{\al}(u*\ph)=\pl^{\al}_x\an{u,\tau_x \check{\ph}}=u*\pl^{\al}\ph$.
\end{proof}
%dist are normal functions, just take a derivative at the end.
\subsection{Density of $D(\R^n)$ in $D'(\R^n)$}
We have just seen the following.\\

\cpbox{
If $u\in D'(\R^n)$ and $\ph\in D(\R^n)$ then $u*\ph$ is smooth.}
\vskip0.15in
This is extremely useful. No matter how wild $u$ is, $u*\ph$ is nice. For this reason, $u*\ph$ is often called a  \textbf{regularisation} of the distribution $u$.
%wild creatures. give me any wild distribution, convolute with anything, get a smooth function. This is what we'll use to prove the density result.
We will use this fact to prove $D(\R^n)$ is dense in $D'(\R^n)$, i.e., for each $u\in D^1(\R^n)$ there exists a sequence of test functions $\{\ph_m\}_{m\ge 1}$ in $D(\R^n)$ such that $\ph_m\to u$ in $D'(\R^n)$, i.e., $\an{\ph_m,\te}\to \an{u,\te}$ for all $\te\in D(\R^n)$.

\blu{How to apply this: Suppose we have a problem about a distribution we don't know anything about. We know for some sequence of $\phi_m$, $\ph_m=u*\phi_m\to u$. We replace $u$ with $\ph_m$, do manipulations there, run the argument and take a limit.}


We need a technical lemma.
\begin{lem}\llabel{lem:dist1-5}
For $\ph,\psi\in D(\R^n)$ and $u\in D'(\R^n)$ we have
\[
(u*\ph)*\psi=u*(\ph*\psi).
\]
%insert dummy var
\end{lem}
\begin{proof}
The LHS is
\bal
(u*\ph)*\psi(x)&=\int (u*\ph)(x-y)\psi(y)\,dy\\
&=\int \an{u,\tau_{x-y} \wh \ph}\psi(y)\,dy\\
&=\int\an{u(z),\ph(x-y-z)\psi(y)}\,dy&\text{$z$ is ``dummy" variable}\\
&=\lim_{h\to 0} \sum_{m\in \Z^n}\an{u(z), \ph(x-z-hm)\psi(hm)h^n}&\text{Riemann sum}\\
&=\lim_{h\to 0}\an{u(z), \sum_{m\in\Z^n}\ph(x-z-hm)\ph(hm)h^n}
\end{align*}
(We want to take the $\int$ inside the $\an{\cdot}$, and we do this by turning it into a Riemann sum. Note the sum only has finitely many terms for each $m$, so this is legal.)
%sum eventually dead, only finitely many terms.
It is not hard to show that 
\[
\sum_{m\in \Z^n}\ph(x-hm)\psi(m)h^n\to \ph*\psi(x)\text{ in }D(\R^n)\text{ as }h\to 0.
\]
Continuing the above calculation,
\bal
&=\an{u(z),\ph*\psi(x-z)}\\
&=\an{u,\tau_x (\ph*\psi)\check{\,}}\\
&=u*(\ph*\psi)(x).
\end{align*}
%compact support, LHS has compact support, by fixed compact set, can differentiate
\end{proof}
\begin{thm}\llabel{thm:dist1-2}
$D(\R^n)$ is dense in $D'(\R^n)$. 
\end{thm}
We would like to say
\[
u(x)=\int\de(x-y)u(y)\,dy\stackrel?= \lim_{m\to \iy} \int \de_m(x-y)u(y)\,dy
\]
The $\de_m(x)$ are such that $\int \de_m(x)\,dx=1$ and get squashed closer and closer to $\de$. %(``a family of good kernels").
\begin{proof}
Fix $\psi\in D(\R^n)$ with $\int \psi\,dx=1$ and set
\[
\phi_m(x)=m^n \psi(mx)\qquad \pa{\int \phi_m\,dx=\int \psi\,dx=1}.
\]
%convolution of look-like de with original
Also introduce the bump function $\chi\in D(\R^n)$ with $\chi=1$ on $|x|<1$ and $\chi=0$ on $|x|>1$. Now set $\chi_m\pf xm$ and
\[
\ph_m=\pur{\chi_m(x)}(u*\phi_m)(x).
\]
%why can't $\ph$ of this guy?
(The purpose of $\pur{\chi_m(x)}$ is to make $\ph_m$ have compact support.)
Choose $\an{\ph_m,\te}$ for $\te\in D(\R^n)$ arbitrary, giving
\bal
\an{\ph_m,\te}&=\an{u*\phi_m, \chi_m\te}\\
&=(u*\phi_m)*(\chi_m\te)\check{\,}(0)\\
%&=u*\check{\ph}(0)\\
&=u*(\phi_m*(\chi_m \te)\check{\,})(0)\\
\phi_m*(\chi_m\te)\check{\,}(x)&=\int\phi_m (x-y)\chi_n(-y)\te(-y)\,dy\\
&=\int m^n \phi(m(x-y))\chi\pf{-y}{m}\te(-y)\,dy& y'=m(x-y)\implies y=x-\fc{y'}m\\
%Make the substitution 
&=\int\phi(y)\chi\pa{\fc y{m^2}-\fc xm}\te\pa{\fc ym -x}\,dy\\
&=\te(-x)+\ub{\int \phi(y)\chi\pa{\fc y{m^2}-\fc xm}\ba{
\te\pa{\fc ym-x}-\te(-x)
}\,dy}{=:R_m(-x)}\\
&=(\check{\te}+\check{R_m})(x).
\end{align*}
We can show $R_m\to 0$ in $D(\R^n)$. So \[\an{\ph_m,\te}=\an{u,\te}+\ub{\an{u,R_m}}{\to 0\text{ as }m\to \iy}.\] Hence $\an{\ph_m,\te}\to \an{u,\te},\te\in D(\R^n)$, giving $\ph_m\to u$ in $D'(\R^n)$ i.e. $D(\R^n)$ dense in $D'(\R^n)$. 
\end{proof}


This shows $NP\subeq ZK$. It's in fact true that PSPACE$\subeq ZK$. In fact because ZK$\subeq$IP, PSPACE$=$ZK$=$IP. Anything you can prove, you can prove in zero knowledge.

Take $L\in $IP. $L$ has an interactive protocol. Prover sends $y_1$. Veifier sends $x_1$, etc. Turn this into ZK by encrypting each of prover's messages with random key, $E(y_i, d_i)$. There are various problems. 
\begin{enumerate}
\item
First, how does the verifier know how to respond?
Every language in IP also has public coins IP protocol (Goldwasser, Sipser). Small-set certification with hash functions. 

The verifier only sends public coins! The prover can simulate the verifier. (The verifier is deterministic algorithm minus the randomness.) The prover can't see the coins ahead of time. 
%private but make more public.
Prover should not know $x_2$ when sending $y_2$.

Ex. 3-colorability is public-coin protocol? Prover should not know coin ahead of time.
%``Fresh public coins"
%Every ZKP can be made into a non-interactive ZKP; you just have to wait. 
%non-interactive 
%just share long random string
1-round. ???

\item
How does the verifier know when to accept? Can't just reveal the $d_i$ because this would give away the $y_i$ in the original protocol which is not ZK. 

Consider the following statement. ``If $d_1,\ldots, d_n$ are known to $V$, then $V$ will accept $x$." This is in NP. (guess $d_1,\ldots, d_n$.) %there exist $d_i$ such that the corresponding $y_i$ would make accept.
(Can also send $f(d_i)$, so can check. What if you plug in bad $d$ and something else?)

Prover can convince V of the statement in ZK.

$y_1\opl d_1$, but also including $f(d_1)$, $f$ a OWP; know when you guess $d_1$ right. 

%what if verifier is in NP.
%people care about bounded prover. 
%ZK 1 round vs. many

%share long public randomness. 
%ZKMA
\end{enumerate}
SZK$=$coSZK. PK$\subeq$coAM (Fortnow). NP language in , then NP$\subeq$coAM, collapse to 2nd level.

%What other things have perfect zero-knowledge proofs? 

factoring may have zero knowledge?

%\bibliographystyle{plain}
%\bibliography{refs}
\end{document}