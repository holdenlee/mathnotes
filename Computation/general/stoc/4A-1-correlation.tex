
\subsection{Forrelation}

What's the biggest dvantage a quantum computer ever guives you for anythign

$\wt O(n^2)$ and $2^{(\wt O(n^{\rc3})}$. These are only conjectural.

We like to work in the black box model, where we can prove exponential and larger quantum speedupt=s

$f:[N]\to [M]$. Querying means you can feed it a superposition and the black box gives a superposition of answers. 

Let $P$ be a promise problem about $f$, for example, is $f$ 1-to-1 or 2-to-1? Is $f$ periodic or far from periodic?

Two quantities we care about:
\begin{enumerate}
\item $Q(P)$, bounded-erro quantum query complexity of P
\item $R(P)$, bounded-error randomized query complexity of $P$. 
\end{enumerate}
Shor's result is
\begin{enumerate}
\item $Q(PeriodFinding)=O(1)$. %period <= \sqrt N
\item $R(PeriodFinding)=\wt{\Om(N^{\rc 4})}$.
\end{enumerate}

Buhrman et al.'s speedup question:
Is this the best possible? Could there be a property of $N$-bit strings that took only $O(1)$ queries to test quamtumly, but $\Om(N)$ classically?

For all of them the quantum randomized query complexity seems to hit  ceiling at $\sqrt N$. (Glued trees problem; a quantum algorithm finds the exit vertex in $Q=O(\poly\log N)$, $R=\Om(\sqrt N)$.) Permutation symmetry: only polynomial quantum speedups are possible. Until this morning, the best speedup was $N^{\rc2}$; now there's a $N^{\rc 4}$. Now there's a fourth root separation.
\begin{enumerate}
\item

We give the largest known quantum speedup, a problem with
\[
Q=1,\qquad R=\Om\pf{\sqrt N}{\ln N}
\]

for classical peopel:
A lower bound on number of randomized queries needed to detect small pairwise covariances in real Gaussian variables $x_1,\ldots, x_N$.

Optimality of speedup: for eveyr partial Boolean function $P$, if $Q(P)\le T$ then $R(P)=O(N^{1-\rc{2T}})$.
This answers Buhrman's question in the negative.

Classical:
Rand alg to approximate bounded low-degree polynoimals.
\end{enumerate}•
\begin{prb}
Given $f,g:B^n\to \{-1,1\}$, $N=2^n$, let (how correlated with Fourier transform)
\[
\Phi_{f,g} := \rc{2^{3n/2}} \sum_{x,y\in \{0,1\}^n} f(x)(-1)^{x\cdot y}g(y).
\]
Determine whether $\Phi_{f,g}\ge 0.6$ or $|\Phi_{f,g}|\le 0.01$.
\end{prb}
Introduced this problem as a candidate for a black-box problem in BQP but not in PH (needs a classical circuit lower bound). Showed that $R(Forrelation)=\Om(N^{\rc 4})$, $Q=1$.

The trivial quntum algorithm: start with in zero states, Hadamard, apply $f$, $H$, $g$, $H$. Can even reduce from 2 queries to 1. Use a control qubit; Hadamard the control qubit.

\begin{prb}[Gaussian distinguishing]
Given $N(0,1)$ Gaussian $x_{[1,M]}$, either all independenct or in fixed low-dimensional supspace $S\le R^M$, $\Cov\le \ep$ for all $i,j$.
\end{prb}

GD$\le$ F

$F(x)\sim N(0,1))$, $G=\wh F$.
$f()=\sign(F(x))$. Then $\Phi_{f,g}\approx \fc{2}{\pi}$. If iid...

Any classical algorithm for Gaussian distinguishing must query $\Om\pf{1/\ep}{\ln M}$. In the Forrelation case, $M=2N$ and $\ep=\rc{\sqrt N}$, so get $\Om\pf{\sqrt N}{\ln N}$.

query.

If orthogonal, queries returns independent $N(0,1)$ Gaussian.

Ue Gram-Schmidt and Azuma to argue that first $t$ query responses are close to independent Gaussians.



\subsubsection{Classical simulation of $k$-query Quantum algorithm}
$A$ be a quantum algoorithm makes $T$ queries, ten $p(X)=\Pj(A\text{ accepts }X)$ is a real poly in $x_i$'s of $\deg\le 2T$. 

Actually there is a $\deg=2T$ block-multilinear poly, which equals $P(X)$ whenever all $=X$, and bounded in $[-1,1]$  for $X_i\in B^n$.

\begin{thm}
TQuery only $O\pa{\pf{N}{\ep^2}^{1-\rc k}}$
\end{thm}
Identify influential variables, split...

$k$-fold forrelation. Conjecture that $k$-fold forrelation requires $\Om(N^{1-\rc k})$ randomized queries, optimal gap for all $k$>

$k$-fold forrelation is BQP complete for $k=\poly(n)$.

OPen
\begin{enumerate}
\item
Prove classical lower bound for $k$-fold forrelation.
\item
ny partial boolean $P$ such that $Q(P)=\poly\log N$, $R(P)\gg \sqrt N$.
\item
Extend to arbirary polynomials, (2: DKPO)
\item
Best quantum/classical query complexity separation for sampling problems.
%FS q, N/\ln N.
\end{enumerate}

%real var restre to subspace. 
%variables restricted to nonlinear manifold of degree $k-1$ no rotation variance, much harder to calculate posterior prob. Need new ideas.

%classical $2^{n/2}$ unless constant 

\subsection{Quantum information complexity}

How much quantum information need to exchange?

\begin{itemize}
\item
Definition of quantum information complexity. 
\item
Interpretation of amortized communication. 
\item
%compute parallel 
$QIC\le QCC$, additivity of QIC
\item
Direct sum for quantum communication
\item
Applications to communication lower bound.
\end{itemize}•

2 communication problems:
\begin{enumerate}
\item
compress messages with low info content.
\item
transmit messagee noislelessly over noisy channels.
\end{enumerate}
We focus on (1).

%Communiatio complexity 

Protocol transcript. %Alice and Bob can memorize

Coding for interactive protocols
\begin{enumerate}
\item
Can we compress protocols that do not convey much information?
\item
What is the amount of information conveyed by a protocol? Amount of info at end of protocol, optimal asymptotic rate.
\end{enumerate}
Define information complexity, cost. Additivity, $IC\le CC$, operational interpretation, direct sum on composite functions, convexity, continuity, etc.
\[
IC(f,\mu,\ep)=\inf_{\Pi}IC(\Pi,\mu).
\]

%bound for sym?
Applications of CIC: direct sum, direct product, exact communication complexity of DISJ${}_n$.

Quantum:
\begin{enumerate}
\item
quantum entropy $H(A)_P=-\Tr(\rh^A\lg \rh^A)=H(\la_i)$ for $\rh_A=\sum_i\la_i\kb{i}{i}$. 
\item
No pre-shared entanglement, classical messages; arbitrary pre-shared entanglement, classical messages, etc. Many problems.

In Yao model: no-cloning: cannot copy $m_i$, so no transcript. Can only evaluate information quantities on registers defined at asame moment in time. Not well defined.

Cleve-Buhrman model: $m_i$'s can be uncorrelated to inputs. (Teleport at each time step.)
%MI =0
%trivial
\item Solutions: keep as much info as possible, measure final correlations.

Reversible, no garbage, only additional info is function output.

Bt QIC trivial.
\item
Measure correlation at each step.

Problem: for $M$ messages and total communication $C$, can be $\Om(M\dot C)$. Want $QIC\le QCC$ independent of $M$, direct lower bound on communication.
\end{enumerate}

Approach: reinterpret classical info cost.
\begin{enumerate}
\item
Shannon task: simulate noiseless channel over noisy channel.
\item
Simulate noisy channel over noiseless channel.
\end{enumerate}

%new proof of $IC^M=ACC^M$?

%$QIC(\Pi,\mu)=

%Research directions
%\begin{enumerate}
%\item
%bounded round disjointness function.
%\end{enumerate}•