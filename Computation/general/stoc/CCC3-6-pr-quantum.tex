\subsection{Quantum parallel repetition}
Make games much harder against players with quantum power.

2-player 1-round games. Referee sends $x,y$ to 2 players, they send back $a,b$ (without communication); the referee decides whether to accept.

$\val(G)$ is the maximum probability Alice and Bol can win with deterministic strategies, over random questions $x,y$.

Here we are interested in the quantum value. $\val^*(G)$ is the maximum probability lice and Bob can win with entangled strategies, i.e., making measurements on preshared entangled state. (It doesn't allow them to communicate.)

Clearly $\val(G)\le \val^*(G)$. In the CHSH game, they receive uniform random bits; they win if $a+b=x\wedge y$. 
\bal
\val(CHSH)&=\fc 34\\
\val^*(CHSH)&=.85.
\end{align*}
How powerful is quantum entanglement?

%much harder game.
Parallel repetition is $G^n=(\mu^n,V^n)$. Referee chooses $n$ independent pairs $(x_i,y_i)$ from $\mu$. Alice and Bob win if they win all $n$ copies. How much harder is this?

%If Alice and Bob had a perfect strategy they also had one.

How does $\val(G^n)$ behave as a function of $\val(G),n$? Fortnow and Feige gave a counterexample: there exist games $G$ such that $\val(G^2)=\val(G)=\rc2$.

Parallel repetition theorem:
\[
\val(G^n)\le (1-\ep^c)^{\Om\pf ns}.
\]
Dependence on size of answer: $s=|a|+|b|$. The game value decays exponentially in $n$.

This parallel repetition theorem holds when Alice and Bob are classical.

How does $\val^*(G^n)$ depend on $\val^*(G),n$? Is there exponential decay, or decay at all? For general $G$, this is not known.

\begin{itemize}
\item
Feige-Killian repetition of a game $G$ goes to 0 with number of repetitions, but not exponential decay.
\item
XOR games satisfy perfect parallel repetition.
\item
For free games, $\val^*(G)\le (1-\ep^2)^{\Om\pf{n}2}$.
\item
For projection games, there exists $c\ge 1$, $\val^*(G^n)\le (1-\ep^c)^{\Om(n)}$.
\end{itemize}•
We show that for a free game (input distribution is product distribution),
\[
\val^*(G^n)\le (1-\ep^{\fc 32})^{\Om\pf ns}\le (1-\ep^2)^{\Om\pf ns}.
\]
We improved the exponent on $\ep$.
We don't know of a classical analogue of this theorem: parallel repetition of free games---we don't know we can make it $\fc32$. 
Rate of decay matters for hardness of approximation.
We use new tools from quantum communication complexity, so this may not be true for classical value!
Barak et al: $\val(G^n)\le (1-\ep^2)^{\Om \pf ns}$. Raz showed there are projection games where $\ep^2$ is optimal. It's plausible that the bound of Barak is tight! Then quantum and classical games behave differently under parallel repetition.

We use a faster quantum communication protocol to obtain a better bound than CS14. The communication protocol is a distributed version of the Grover search algorithm, improving $1-\ep^2$ to $1-\ep^{\fc 32}$.

High level proof idea: convert a ``too good" strategy for the repeated game to a strategy for the original game that does better than $1-\ep$, contradiction.

Intuition: use Bayes's rule:
\[
\Pj\pat{win all rounds}=\prod \Pj\pat{win round $i$|win rounds $<i$}.
\]If $\gg1-\ep$, then there are many rounds such that the factor is $>1-\ep$.

Embed inputs $(x,y)\sim \mu$ into some $i$ of the global strategy conditioned on some event $W$, $\Pj(\text{win }i|W)>1-\ep$. Simulate $S$ conditioned on $W$, the players won all games in a randomly chosen set $C$ of a certain size. If the subset is large enough ($|C|\ge \Om\pf{n\ga}{\ep}$,$\val^*(G^n)=2^{-\ga n}$).

When is simulation possible? The quantum mutual informations are small:
\[
I(X_i:E_BY|W=1),I(Y_i:E_AX|W=1)\le \de
\]%bound by give efficient protocol.
Then A and B can simulate $S$, condtioned on $W$ with $X_i=x,Y_i=y$, with $O(\de)$ error. So Alice and Bob can win $G$ whp, contradiction.

Play $G^n$ using strategy $S$. A sends answers in $C$ rounds to Bob, who computes if $W$ occured. (?) How much did Bob learn about Alice's inputs? $\le O(Q)$. Using chain rule to get average $\le O\pf{Q}{n}$. Here $Q$ is $|C|$ times Alice's answer length, $O(\fc{\ga n s}{\ep})$.

Alice and Bob are searching for a round $j$ in C. Any classical protocol must use at least $\Om(|C|)$ bits of communication. Use Grover search.
Grover search with low error: to search database of $C$ with error probability $\de$, suffices $\sqrt{|C|\ln \prc{\de}}$ queries. 

Lower communication implies smaller mutual information.

%arxiv:1501.00033
%Transmit quantum states as answers.
Open:
\begin{enumerate}
\item
Other uses of efficient communication protocols to prove better direct sum/direct product teorems.
\item Is the classical bound (Barak) for parallel repetition of free games tight?
\item Is $\fc 32$ tight?
\end{enumerate}
%Decay relative to quantum value.

%quantum value may be equal to 1.