\subsection{The social cost of cheap pseudonyms}

Eric Friedman, Paul Resnick, The social costs of cheap pseudonyms, 2001. (Started 1998.)

When you can easily have pseudonyms, it's hard to control behavior. Much of the beauty comes from the conciseness of the model, how little they assume. %Easy to res. 
When you have a small probability of malicious action, or innocent mistakes, cooperation is no longer equilibrium. Solution: Treat newcomers with skepticism.

%explore vs. exploit. new users
%advertisers under new identity.
%fb authentication
%tie back to 
%fb 5 years after
%google 
%1st year EC

%The social costs of cheap pseudonyms, 

We had to explain people what ebay was, why reputation on the Internet was important, etc.

Literature postview: the idea of reputation is now mainstream; since 1999, $10\%$ of EC papers study reputations.

%original problem, key results,

The cost of changing identity is key. This was a new aspect of the Internet. 
The technical approach is \vocab{dynamical games} (underutilized in the EC community). 
%$1L$ identifiers, negative progress

An economist, a computer scientist, and a philosopher walk into a bar...
\begin{itemize}
\item
Philosopher: what is the meaning of self on the Internet?
\item
Computer scientist: how can we secure identities using cryptography?
\item
Economist: How do we make money out of this?
\end{itemize}•

How can we create trust; how much loss do various models of trust imply? Why is trust important?
\begin{itemize}
\item
selling and trading websites like Ebay.

Reputation informs buyers, incentivizes sellers to honesty and quality service (``shadow of the future"). Note in these systems you don't usually interact with the same person again; how to make this work?

Self-selection: more honest people will be drawn to be sellers.
\item
online forums and discussion groups.
\item
cooperative groups like wikipedia.
\item networks of autonomous devices, like BitTorrent.
\end{itemize}


How/when should reputation systems work and how can we make them better?


Consider a reputation system with different kinds of interactions.
\begin{enumerate}
\item
anonymous (ID changes every action)
\item
pseudonyms (ID changes at will): you can start a new name if you want. If you rip somebody off, you can come in as a new person.
\item
identified (real name, ID never changes)
\end{enumerate}

We introduced 1L pseudonyms: you can change once in a lifetime. 

In every period, people get matched up and play a prisoner's dilemma. The problem is the folk theorem of economics: in repeated games anything can happen. There are many strange equilibria. 

If everyone starts at the same time, and one person comes back with a new name, that person can be identified. Instead we assume a bunch of people arrive/leave at specified times. 

What's public is the complete PD matrix and history of plays.

We considered only sequential equilibria, the expected per-period payoff.
Intuitively, $V=1-\fc{\pat{number of defections}}2$.

What can you do? Local punishment strategy: You have to do something bad to a newcomer. You defect against a new arrival, and cooperate against good people.

But you have to cooperate against new people, otherwise people will defect and come back as new people, and the system will break down.

Mathematically, use the public grim trigger strategy: if anyone non-compliant, shut down the whole strategy and everyone goes home. This isn't realistic.

We added the ideas of ``trembling": even if you choose to cooperate, with $\ep$ probability you actually defect (something gets lost in the mail for ebay...).

%On social norms in noisy environments, Friedman.
How much loss comes from free pseudonyms? We introduced Pay Your Dues: if you're around for a while and have a good reputation, you can defect against newcomers. (This is realistic.)

You can show this is an equilibrium.
There is inefficiency when you have to pay dues. 
%With pseudonyms the SF is $1-\pat{loss}$.

What are solutions? Pay for IDs. This discourages participation. Often you have to do some annoying tasks to do, ex. fill out a long form.

How would we implement this cryptographically? How could you implement pseudonyms so that people can get a pseudonym and never use it again? With this, you can do well. This never happened though. How could CA's cooperate, etc.?

Our conclusions:
\begin{itemize}
\item
Cheap pseudonyms imply distrust for newcomers.

We had to prove things in dynamic systems, modify the model with noise.
Without noise we get bizarre effects.
\item
Expensive pseudonyms help.
\item
Expensive to change psuedonyms are even better.
\end{itemize}

Some underserved research areas by EC:
\begin{itemize}
\item
application of dynamic games: most work is static or 2-period.
\item
noise (``trembles")
\item
self-selection (market for lemons)
\end{itemize}

Now reputation is still held centrally. %the gov doesn't want to be, people have security concerns. anonymous web browsing
%export rules on crypto.
%told not to do it, decide not to.

