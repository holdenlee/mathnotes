\subsection{Quantum query complexity}

\subsubsection{Bomb query complexity}

Modified quamtum query inspired by Elitzur-Vaidman bomb tester.

Given a bomb which is either a dud or a live bomb. Live: explodes on contact with photon. Dud: no interaction with photon. Can we tell them apart without blowing it up?

Put bomb in an interferometer: split the photon. If you see a photon on $D_2$ you know you have a live bomb. ($50\%$ explode, $25\%$ know it's a bomb.) %is this optimal.
We can improve to $\ep$ with a quantum zero effect.

We can rewrite the Elitzur-Vaidman bom in the circuit model: 
%\ket{0} -- I or X (* control)-- measure, explode if 1
Let $R(\te)$ be rotation matrix. Add a $R(\te)$ on top, and repeat $\fc{\pi}{2\te}$ times.

If live: first register is projected back to $\ket0$ on each measurement. Probability of explosion: $\Te(\te^2) \Te(1/\te)=\Te(\te)$.
%1/\te number of queries.

Bomb query: quanum query where each bit of $X$ is an EV-bomb.

Differences from quantum query:
\begin{itemize}
\item
extra control register $c$,
\item
record register must contain $0$ as input.
\item
must measure query after each input.
\end{itemize}•
$B_\ep(f)$ number of bomb queries needed to have probability of explosion $\le \ep$. 
\[
B_\ep=\Te(Q(f)^2/\ep).
\]
Upper bound: quantum zero effect. 
Simulate each quantum query using $\Te\prc{\te}$ bomb queries.

Lower bound: adversary method.
General weight adversary bound tightly characterized quantum query complexity.

\subsubsection{Algorithms}
$O(N)$ bomb query algorithm for OR. Stop at first live bomb. $O(N/\ep)$.
We get $Q(OP)=O(\sqrt N)$. This gives a nonconstructive proof of the existence of Grover's algorithm. Can we generalize this further?

If there is a classical randomized algorithm that computes $f$ using $\le T$ queries, and an algorithm that predicts the results of each query $\cal A$ makes, making $\le$ expected $G$ mistakes, then 
\[
B_\ep(f)=O(TG/\ep), \qquad Q(f)=O(\sqrt{TG}).
\]

For OR, $T=N$, $G=1$.

Cf. Kothari's algorithm for orace identification.

At each step, 
\begin{enumerate}
\item
use $G$ to predict remaining queries.
\item 
find the location of the first mistake using $O(\sqrt d)$ quantum queries
\item
This determines the actual query results up to the mistake we just found.
\end{enumerate}
Query complexity $O(\sqrt{TG})$. Log factors can be removed using span programs.

Application: Unweighted single-source shortest paths.
% Given the adjacency matrix of an unw
Guessing that $(v,w)$ is not an edge, we make $\le n-1$ mistakes, since each vertex is discovered once.
\[
Q(uSSSP)=O(\sqrt{TG})=O(n^{\fc 32}).
\]
%algorithm as black-box to another black-box
matches lower bound.

Problem: unweighted $k$-source shortest path. 
Classical: run BFS $k$ times. 
Quantum: $G=k(n-1)$ but $T=O(n^2)$ instead of $O(kn^2)$.

2109 on Scott A's blog.

$O(n^{1.75})$ quantum query algorithm for maximum bipartite matching.

Open:
\begin{itemize}
\item
relate $G$ to classical measures of query complexity (certificate, sensitivity).
\item 
time complexity of algorithms
\item
algorithm for adjacency list model
\item
other problems, e.g., matching for general graphs
\item $R(f)$ and $B(f)$
\end{itemize}


Conjecture: $R(f)=O(Q(f)^2)$ (attained by OR).

Projective query complexity, $P(f)$. Allow black-box access to a circuit that compotes $O_x$ but then forces a measurement.
Decohere result? $Q(f)\le P(f)\le R(f)$, $P(f)\le B(t)$. $P(OR)=\Om(N)$.
Conjecture: $P(f)=\Te(R(f))$ for all total functions.