\def\filepath{C:/Users/Owner/Dropbox/Math/templates}

\input{\filepath/packages_article.tex}
\input{\filepath/theorems_with_boxes.tex}
\input{\filepath/macros.tex}
\input{\filepath/formatting.tex}
%\input{\filepath/other.tex}

%\def\name{NAME}

%\input{\filepath/titlepage.tex}

\pagestyle{fancy}
%\addtolength{\headwidth}{\marginparsep} %these change header-rule width
%\addtolength{\headwidth}{\marginparwidth}
\lhead{Gems of theoretical computer science}
\chead{} 
\rhead{} 
\lfoot{} 
\cfoot{\thepage} 
\rfoot{} 
\renewcommand{\headrulewidth}{.3pt} 
%\renewcommand{\footrulewidth}{.3pt}
\setlength\voffset{0in}
\setlength\textheight{648pt}

\begin{document}
\tableofcontents
\section{List decoding of Reed-Solomon codes}
\subsection{Introduction}
An error-correcting code is $C\subeq \Si^n$, where $\Si,|\Si|=q$ is the alphabet and $n$ is the encoding length. 

Define the normalized Hamming distance
\[
d(x,y)=\rc n|\set{i}{x_i\ne y_i}|.
\]
We want 
\[
\de = \min_{x,y\in C,x\ne y} d(x,y)
\]
to be as large as possible (constant as $n\to \iy$). Imagine balls of radius $\de/2$ around each point.
We can send one of $|C|$ messages by mapping $[|C|]\to \Si^n$; they can withstand $\de/2$ errors. The rate of the code is $\fc{\lg|C|}{n}$ (how many bits of actual information are sent compared with encoding length); we want the rate and distance to be high, but there are fundamental limits to what can be achieved.

$\de/2$ is the unique decoding radius. Unique decoding is not possible beyond $\de/2$. %; they don't intersect. 

In many cases, even if there is not a unique codeword within a given radius ($\fc{\de}{2}+\ep$, say), there may be a only a small number of codewords.
\begin{df}
$C$ is $(\de,L)$-list decodable if for all $x\in \Si^n$, $|B(x,\de)\cap C|\le L$.
\end{df}
%random code, with high probability
If you choose your code randomly---that is, choose a codeword, removing a $\fc{\de}2$ ball around it, and repeat---with high probability it will have good list decodability, up to distance $(1-\ep)\de$. %; in fact, most of the time it can correct close to $\de$. %The code is good becuase most of the time it can correct $\de$.
List decoding is a better way to handle talk about how good the code is than stochastically, because it's worst-case.
%up to $(1-\ep)\de$
%People previously tried to model stochastically.
%This code is good becuase most of the time it can correct $\de$.
%List decoding is a better way to handle this than stochastically because it's worst-case.

The maximum $\de$ is the list-decoding radius. Here $L(n)$ is a function of $n$. The ideal setting is $L(n)=\poly(n)$.  Informally the list decoding radius is the maximum $\de$ such that $L(n)=\poly(n)$.

What does a random code give us; what is possible and not possible? What is the capacity of list decoding?
%minimum distance

\begin{thm}
Let $0<\de<1-\rc{q}$. (Above this error, the noisy word is essentially random, so we can say nothing.) Then there exists a $(\de,L)$-list decodable code with rate
\[
1-H_q(\de)-\rc{L}.
\]
Here $H_q$ is defined so that
\[
|B(0^n,\de)|\sim q^{H_q(\de)n}.
\]
\end{thm}
%\Si^n$.
\begin{proof}
Choose $M=q^{Rn}$ codewords at random. We want
\[
\Pj(C\text{ not }(\de,L)\text{-list decodable})<1.
\]
Do a union bound. The probability is at most, by the union bound,
\[
\binom{M}{L+1}q^n\pf{|B(x,\de)|}{q^n}^{L+1}<1.
\]
($q^n$ is the number of points; the probability of $L+1$ points in a ball centered at a point is $\le \binom{M}{L+1}\pf{|B(x,\de)|}{q^n}^{L+1}$.)
\end{proof}
This is almost tight. 
\begin{thm}
If a code is $(\de,L)$-list decodable with 
\[
R\ge 1-H_q(\de)+\ep
\]
then $L(n)\ge q^{\fc{\ep n}2}$.
\end{thm}
%1-H_q(\de)+\ep$
A random ball will contain too many points.
\begin{proof}
\bal
\E_{x\in_R \Si^n} [|B(x,\de)\cap C|]
&=
|C|\fc{|B(0^n,\de)|}{q^n}\\
&\ge 
2^{Rn}q^{n(H_q(\de)-1)}
%q^{n(1-H_q(\de)+\ep)}%%/\ep^n
\\
&\ge q^{\ep n/2}.
\end{align*}
\end{proof}
When $q\to \iy$, $H_q(\de)\to \de$. We can achieve rate $R\ge 1-\de-\ep$ with alphabet of size $2^{O\prc{\ep}}$  and list size $O\prc{\ep}$. 
%1/L
%all add up to at most $\rc \ep$. Plug in param.
%alphabet grows, can achieve any rate up to.

Graph: the achievable rates are below the line $\de+R=1$.

This is what is known existentially.

The current known explicit codes can achieve $R\ge 1-\de-\ep$ with alphabet of size $2^{\poly\prc{\ep}}$ and list size $n^{O\prc{\ep}}$.
%
The idea is a folded Reed-Solomon code concatenated with an asymptotically good list-decodable code.
%any code - at least this much list decodable.

\begin{thm}[Johnson bound]
Given $C\subeq \Si^n$ with min-distance $\de=1-\ep$, then $C$ is $(1-\sqrt{\ep},\poly(n))$ list-decodable.
%increase by $\sqrt{\ep}-\ep$.
\end{thm}
The Johnson bound says that if you slightly increase the radius, there cannot be an exponential number of codewords.

\begin{thm}[Singleton bound]
For any codewith distance $\de$, $R\le 1-\de$.
\end{thm}
\vocab{MDS (maximal distance separable) codes} are where $R+\de=1$, $\de=1-R$, R$=\ep$. MDS codes are $(1-\sqrt{R},\poly(N))$ list-decodable.
%if don't care about constant, simple. if want constant, need more careful calculation.

\subsection{Reed-Solomon Codes}
We introduce the Reed-Solomon code $RS_{k,n,\F}$, $k< n\le q$.

These are MDS codes. They achieve $R+\de=1$ so are $(1-\sqrt{R}, \poly(n))$-list decodable.

Let $\Si=\F_q=:\F$. Let
\[
C=\{\text{evaluations of degree $k$ polynomials on }\{\al_1,\ldots, \al_n\}\subeq \F\}
\]
where $\al_1,\ldots, \al_n$ are distinct and fixed.
Codewords correspond to degree $k$ polynomials in $\F[x]$.
2 distinct degree $k$ polynomials can only agree on $k$ points, so $\de=\fc{n-k}n=1-\fc kn$. 
%2 degree $k$ polynomials, distinct.
The rate is $\fc{\log_q(q^{k+1})}{n}=\fc{k+1}n$. 
The Reed-Solomon code is $(1-\sfc{k}n, \poly(n))$-list decodable. 

What is the algorithm? Given a set of points, we need to find all polynomials passing through enough of those points.
% Given a word $x$, give a list of the polynomials.
\begin{prb}
Given $(\al_1,y_1),\ldots, (\al_n,y_n)$, find all polynomials of degree $\le k$ such that $p(\al_i)=y_i$ for at least $\sqrt{nk}$ indices $i\in[n]$. 
\end{prb}
Madha Sudan (90's) showed that you can do this with $\sqrt{2nk}$. 
\begin{thm}
There is a polynomial-time algorithm (given in the proof) that given $n$ points as above, finds all degree $\le k$ polynomials agreeing on $\ge 2\sqrt{nk}$ points. 
\end{thm}
\begin{proof}
The algorithm is as follows.
Note this algorithm will work even if the $\al_i$ are not distinct.

Define the $(1,k)$-weighted degree of a polynomial $Q(x,y)$ as $\deg Q(X,Y^k)$.
The strategy is as follows.

We will find a low $(1,k)$-weighted degree polynomial $Q(x,y)$ of degree $D$ such that $Q(\al_i,y_i)=0$ for all $i$.

Look at $R(x)=Q(x,p(x))$ where $p\in L$. We have
\[
\deg R\le D.
\]
For all $i$ such that $p(\al_i)=y_i$, 
\[
R(\al_i)=Q(\al_i,p(\al_i))=Q(\al_i,y_i)=0.
\]
Suppose $R$ has at least $t$ roots and $\deg R\le D$.  If we arrange so that $t>D$, then $R(x)\equiv 0$ identically. Then $Q(x,p(x))\equiv 0$ implies $y-p(x)\mid Q(x,y)$.
%$Q(x,y+p(x))\equiv 0$, take $y=0$.
%t<q.
%cant have more roots than degree.
(There is a deterministic algorithm to factor bivariate polynomials.) Factor $R$ to find all factors in the form $y-p(x)$; then output those polynomials $p(x)$.

Now we just need to find the polynomial $R(x)$; this is polynomial interpolation. 

The number of coefficients in $Q(x,y)$ is $\rc{k}\binom{D+2}2$. We need to satisfy $n$ linear constraints; they are linear homogeneous equations in the coefficients. If $\fc{\binom{D+2}2}k>n$ then there is a nonzero solution: a nonzero $\ph(x,y)$ of $(1,k)$ weight degree $\le D$ with $\ph(\al_i)=y_i$ for all $i$.

If the number of roots is $t>D\approx \sqrt{2kn}$, then we can find all polynomials with agreement $t$.
%\item
%\end{enumerate}
\end{proof}
Guruswami and Sudan improved this to $\sqrt{kn}$.

\fixme{
Consider $k=1$, $\F=\R$, find all lines which pass through at least 3 points. If you can interpolate a degree 2 polynomial through all these points, get all lines as lines within the curve. The degree 2 curve will be 2 lines.

Let's look at a small example to see how to improve the bound.

Consider the following picture (see notebook). Here $n=10$ and $t=4$. Here we can choose $k=4$. However it can only have 4 linear factors. The maximum $D$ is 3. %D to small to contain all lines.
Peculiar: through each point there are 2 lines. An algebraic curve passing all points should vanish at the points with multiplicity 2. Fit a polynomial which vanishes with degree 2 at the points.}

First define multiplicity.
\begin{df}
$Q(x,y)$ vanishes with multiplicity $r$ at $(\al,\be)$ if $Q(x+\al,y+\be)$ doesn't have any monomial of degree $\le r$.
\end{df}
\begin{lem}
Let $Q(x,y)$ be with $(1,k)$ degree $\le D$, vanishing at $(\al_i,y_i)$ with multiplicity $r$ for all $i\in [n]$. Let $P$ be a degree $k$ polynomial with agreement $t>D/r$. Then $y-p(x)\mid Q(x,y)$.
%deg 5, should still contain line.
\end{lem}

\begin{proof}
Let $p\in L$, $P(\al_i)=y_i$. 
Define 
\[
Q^i(x,y) = Q(x+\al_i,y+y_i);
\]
it has no monomials of degree $<r$.
Then
\bal
R(x) &= Q(x,P(x))\\
&=Q^i (x-\al_i, P(x)-y_i)\\
&=Q^i (x-\al_i,\ub{P(x)-p(\al_i)}{x-\al_i\mid P(x)-P(\al_i)}).
\end{align*}
$Q^i$ has no monomials of degree $<r$. Thus $(x-\al_i)^r\mid R(x)$. The number of linear factors is $tr>D$. Thus $R(x)\equiv 0$ and $y-p(x)\mid Q(x,y)$. 
\end{proof}
The number of coefficients in $Q(x,y)$ of $(1,k)$-degree $D$ is $\rc k\binom{D+2}D$.
%no monom degree $r$. 
The number of homogeneous linear equations is $n\binom{r+1}2$. So a nonzero $Q$ exists if $\rc k\binom{D+2}2>n\binom{r+1}2$.

Choose $D=\sqrt{knr(r+1)}$. Let $t=\fc Dr = \sqrt{kn(1+\rc r)}$.

%find all polynomils. 
Make $r$ large enough, we approach the Johnson bound.

The number of polynomials in the list is at most $\fc Dk=\sfc{nr(r+1)}k$, the $y$-degree. 

Conclusion: 
\begin{enumerate}
\item
If $t>\sqrt{kn}$ the list size is $\le n^\ep$ and we can find all of them.
\item
The Reed Solomon code of rate $R$ can be list decoded up to $1-\sqrt{(1+\ep)R}$ errors with best size $\rc{\ep\sqrt R}$.
\end{enumerate}
%no more than $y$ degree

This method (the polynomial method) is very flexible. We give another application, the list recovering problem. Given $x\in C$, 
%replace each alphabet with small set
a noisy channel turns each coordinate into a set $s_i$ and spits out $s_1,\ldots, s_n$, $|s_i|\le \ell$. We have $|s_i|\le l $ such that for at least $1-\de$ fraction of $i$'s, $x_i\in s_i$. Find all $x$ such that $x_i\in s_i$ for at least $1-\de$ fraction of $i$'s. 
\[
\ab{
\pa{
\bigcup_{y\in S_1\times \cdots \times S_n} B(y,\de)
}\cap C
}\le L.
\]
Reed-Solomon codes are also good list-recoverable codes.

If $t>\sqrt{knl}$, where $t$ is the number of $i$ such that $x_i\in S_i$, then the Reed-Solomon code is $(1-\sqrt{kl},l,O(n^2l^2)=L)$ list recoverable.
%we want optimal paramters.

We only achieve $1-\sqrt R$. We want to attain $(1-R,L)$. Guruswami and Rudra came up with folded Reed-Solomon codes. 
Take a generator $\ga $ of $\F_q^{\times}$, $n=q-1$. $\al_1,\ldots, \al_n$ are $S=\{1,\ga,\ldots, \ga^{q-1}\}$. \fixme{Consider blocks. $\F^m$ by $m$. $P(1),\dots P(r),\ldots, P(r^{q-1})$.

It's an open problem to make $n^{O\prc{\ep}}$ independent of $n$. 
Use concatention to get length down. 

You can concatenate a list-recoverable code with a list-decodable codes to get a list-decodable code. $C_{\text{out}}\circ C_{\text{in}}$.}

%$C_{}$
%small list-decodable.
%folding: alphabet grow large

%S
%GS: RS (1-\srqt k)
%PVbetter than 1-R for small enough $R$
%FRS 1-R

%Extractors, expanders etc.

%to be list decodable don't need dep on n?
%codes achieve both poly n /poly n.

\section{$SL=L$}

%picture of Z.

Reingold, Vadhan, and Wigderson.

There are many stories in this problem.

One of the most fundamental questions in theoretical CS is the following.
\begin{clm}
Randomness is useless.
\end{clm}
This claim comes from very recent years. 20 years ago people actually believe it's useful (Papadimitriou gave $BPP\ne P$ as a homework exercise in his book).

There are 2 questions: is randomness useful in time and in space?
\begin{enumerate}
\item
$BPP\stackrel ? P$
\item
$RL\stackrel ? L$
\end{enumerate}
We focus on the second problem. The first problem is wide open: we don't know $BPP\stackrel?{\in}DTIME(2^{o(n)})$. We know more about the second question: Savitch proves
\[
BPP\subeq DSPACE(\ln^2n).
\]
We can do better: Saks and Zhou in 1999 show 
\[
RL\subeq DSPACE(\ln^{\fc 32}n).
\]
(They actual show this for BPP.)
Think of RL as (undirected) $s-t$ nonconnectivity problem. $coRL$ is interesting because it contains $s-t$ connectivity. (RL doesn't have a complete problem.) For $P\stackrel?=NP$, just look at a complete problem; here we don't have one.

Here we show $SL=L$. Reingold showed this in 2004. 

For simplicity, just think of SL as one problem, undirected $s-t$ connectivity.

Consider a graph. 


Jieming starts from a vertex and wants to find a way to Nanjing. Jieming has very little memory, and cannot remember the whole structure of the graph. He can't remember much more than the name of a city.

The standard way to solve this is to take a uniform random walk on the graph. If the graph has $|V|=n$, after $n^2\ln n$ steps, there is a high probability that he will have visited Nanjing.

But for a general graph, it's necessary to flip $n^2\ln n$ coins: Consider 2 complete graphs with a bridge: It takes order of $n$ time to hit the bridge vertex, and it has $O\prc n$ chance of crossing the bridge.

For which graphs can we do this randomized routing faster than the worst case? A natural family is the family of expaner graphs.

Expander graphs have rapid mixing under a random walk.
There are 2 definitions.
\begin{df}
A graph $G=(V,E)$ is a \vocab{$\la$-edge expander} if for all sets $S\subeq V$, $|S|\le \fc{|V|}2$, 
\[
\fc{E(S,\ol S)}{\Vol(S)}\ge \la.
\]
Here, $\Vol(S)=\sum_{v\in S}\deg(v)$. (We'll focus on the simple case when the graph is $d$-regular, i.e, for all $v\in V$< $\deg(v)=d$.) (Pictorially, a subset of $G=(V,E)$ is very spiky, like a sea urchin.)

A graph $G=(V,E)$ is a \vocab{$\la$-spectral expander}if $\la_2(L(G))\ge \la$. The Laplacian $L(G)$ is defined from the adjacency matrix $A_G$. $A_G$ is defined by $(A_G)_{ij}=1$ if there is a edge between $i$ and $j$. $M$ is the random walk matrix $\rc d A_G$, and 
\[
L=I-M.
\]
(Thinking of this matrix as an operator, $(Lx)_i=\rc d\sum_{(i,j)\in E}x_{ij}$, it flows from a vertex to adjacent vertices.)
\end{df}
The Laplacian originates in differential geometry, $L=\div \nb$.

These 2 definitions are quite close.
\begin{thm}
Let $h(G),\la(G)$ denote the edge and spectral expansions of $G$. Then
\[
\la(G)\le h(G)\le \sqrt{2\la(G)}.
\]
\end{thm}
The LHS is trivial, the RHS is Cheeger's inequality.

People in CS use a third definition that is more useful.
\begin{df}
A graph is a $\la$-expander if for all $x\perp u=\colthree{1}{\vdots}1,x\ne 0$, 
\[
\ab{\fc{x^TMx}{x^Tx}}\le \la.
\]
\end{df}
(For edge and spectral expansion we want $\la$ to be large. For random walk expansion we want $\la$ to be close to 0.) %For the other definitions we want $\la$ to be close to 1.)
$\la(L)$ range from 0 to 2, while $\la(M)$ range from 1 to $-1$.

Note that 
\[
\ab{\fc{x^T(I-L)x}{x^Tx}}=\ab{1-\fc{x^TLx}{x^Tx}}
\]
so spectral and random-walk expansion are not quite the same. For bipartite graphs, $\la(M)=1$: random walks are not mixing at all. The side you're on depends on the parity.

For the zig-zag product they consider the third definition.

\subsection{Zig-zag product}

Think of $G_1$ having large size and degree $(N_1,D_1)$, $G_2$ having small size and degree $(N_2,D_2)$, and suppose $D_1=N_2$. 
\[
G_3=G_1\zz G_2
\]
has large size $N_1N_2$ and small degree, $D_2^2$.

Let $\la_M(G_i)=\la_i$ denote the random walk expansion. Then  (a naive bound)
\[
\la_M(G_3)\le \la_1+\la_2+\la_2^2.
\]
It's easy to construct a large degree expander. It's easy to construct small-size expanders (try all possible graphs). It's easy to construct $G_1,G_2$, but an expander like $G_3$ is hard to construct. It takes as input 2 easy-to-construct expanders and outputs a harder to construct graph that's an expander.

Replace each vertex of the original graph with a copy of the second graph $G_2$. Its size is hence $N_1N_2$.

Each vertex in $G_3$ is labeled $(v,u)\in G_1\times G_2$. The $(i,j)$th neighbor is defined as follows. Take $u'$ the $i$th neighbor of $u$ in $G_2$, $u\xra i u'$. Take the $u'$th neighbor of $v$ in $G_1$; move to the cloud of $w$; $v\xra{u'} w$ in $G_2$. Now consider $\wt u$ such that $w\xra{\wt u}v$ in $G_1$; we move to $(w,\wt u)$. Now move $\wt u \xra j \wt u'$ in $G_2$. Define
\[
(v,u)\xra{(i,j)} (w,\wt u').
\]
%(Use $G_1$ inside the cloud, $G_2$ outside.)

What's the magic of the zig-zag product? We claim
\[
A_3:=A(G_3) = \wt{A_2}\wt{A_1}\wt{A_2}
\]
where $\wt{B_2}=I\ot A_2$ has $N_1$ blocks and each is a copy of $A_2$, and $\wt{A_1}$ is a permutation (actually matching) matrix where $(\wt{A_1})_{((v,u),(w,u'))}=1$ if 
\[
v\xra{u,G_1} w \xra{u',G_1}v.
\]

Consider when $(u,v)\xra{(i,j)} (w,\wt u')$:
\bal
u&\xra{i,G_2}u'\\
v&\xra{u',G_1}w\xra{\wt u,G_1} v\\
\wt u & \xra{j}\wt u'
\end{align*}
What is the action of $\wt{A_2}\wt{A_1}\wt{A_2}$ on $e_{(v,u)}$? It goes to $\sum_{uu'\in G_2}e_{(v,u')}=e_v\ot A_2e_u$. ``Stay in the block of $v$, move to all the neighbors of $u$."

In the second step, we move across blocks as given by 
$\wt{A_1}$.

%(If you sum over $u$, you get $A_1$.)
%want converge in log n steps
%replace tree - length log n.
%log n lg d.

We need to show
\[
%\la_m(C_{i_3}) = \max_{x\perp u} 
\forall x\in \R^{N_1\times N_2}, x\perp u \implies  \fc{x^TM_3x}{x^Tx}\le \la_1+\la_2+\la_2^2.
\]
Decompose $x$ as a vector that is uniform on each block, 
\[x=\ub{\al\ot u}{\al^{\parallel}} + \ub{x'}{\al^{\perp}}\] where $x'$ has blocks $\wt{\al_i}\perp u_i$. Then (noting $\wt{A_2}\al^{\parallel}=\al^{\parallel}$),
\bal
\an{x,\wt{A_2}\wt{A_1}\wt{A_2}x} 
&=\an{\wt{A_2}x, \wt{A_1}\wt{A_2}x}\\
&=\an{\wt{A_2}(\al^{\parallel}+\al^{\perp}), \wt{A_1}\wt{A_2}(\al^{\parallel}+\al^{\perp})}\\
&=\an{\al^{\parallel}+\wt{A_2}\al^{\perp}, \wt{A_1}(\al^{\parallel} + \wt{A_2} \al^{\perp})}\\
&=\an{\al^{\parallel}, \wt{A_1}\al^{\parallel}}
+\an{\al^{\parallel}, \wt{A_1}\wt{A_2}\al^{\perp}}
+\an{\wt{A_2}\al^{\perp}, \wt{A_1}\al^{\parallel}}
+\an{\wt{A_2}\al^{\perp}, \wt{A_1} \wt{A_2}\al^{\perp}}\\
&\le \la_1+2\la_2+\la_2^2
\end{align*}
%\wt{A_2}\al^{\parallel}=\al^{\parallel}
Because $\al_1+\cdots +\al_{N_1}=0$ we can use the expansion properties of $G_1$ to get the $\la_1$ bound for the first term.

The idea: 
\begin{enumerate}
\item
Suppose $G$ has parameters $(N,D)$ and expansion $\la$. (say $1-\rc{ND}$)
\item
$G^2$ has parameters $(N,D^2)$, expansion $\la^2$.
\item
Take $H$ with parameters $(D^2,\sqrt{D}), \la_2$. Then 
\[
G^2\zz H=G_{\text{new}}
\]
has parameters $(ND^2,D)$ expansion $\la(G_{\text{new}}) \le \la^2 + 2\la_2+\la_2^2 \le \la^2+\ep$.
\end{enumerate}
We go from $G$ with parameters $(N,D,\la)$ to $G_{\text{new}}$ with parameters $(ND, D,\la^2+\ep)$. Call this operator $Z$. Doing this operator $t$ times, $Z^tG=(ND^{2t}, D, \la^{2^t}+\ep')$. Then set $t=\lg \pf{ND}2$ to get $Z^tG$ with parameters $(N^2D, D, \rc e +\ep)$. (We're cheating a little, using the naive bound. We need a better bound to get $\ep$ small: $G_3$ has expansion $\rc2(1-\la_2^2)\la_1+\rc2 \sqrt{(1-\la_2^2)^2\la_1^2+t\la_2^2}$. 
%$\la_2=0$ this is 0
%if slightly bigger, constant $\la_1$ plus something. $(1-\la_2^2)\la_1+\la_2
%sum of norm squares equals 1. Just that problem. Just n
To go from our proof to the real proof, you just need to note $\al^{\parallel}\perp \al^{\perp}$, $\al^{\parallel}\perp \wt{A_2}\al^{\perp}$. Plug in this picture into the formula.

%($\ep$ should be $\rc{N}$).
%D^2 vertices and ... edges. 
%Start with 2 constant size expanders and blow up. The real constructor: start with arbitrary bound, but need better bound.

%\chapter{Motivation}

%We'll see lots of analysis in action.
\section{Dirac delta function}

What is the derivative of the Dirac delta?

You may have seen the mysterious ``\textbf{Dirac delta function}" defined by 
\beq{eq:dist0-1}
\int_{-\iy}^{\iy} \de(x-x_0)f(x)\,dx=f(x_0), \qquad \de(x)=0,\,x\ne 0.
\eeq
This emerged from Fourier's classical treatise on heat. It was there implicitly in his work. Cauchy and Dirac noticed it. It is used in math, applied math, physics, engineering.
%does the job we're asking it

But there is no function in any sense of the word that does this job! It makes no mathematical sense!

%Let's look at~\eqref{eq:dist0-1} in an abstract sense. 
However, looking at~\eqref{eq:dist0-1} in an abstract sense, the ``process" which takes $\de(x-x_0)$ and the ``nice" function $f(x)$, and spits out $f(x_0)$ is well-defined.
%There's some $\de(x-x_0)$, given $f(x)$, some process happens and spits out $f(x_0)$. 

However, people don't just talk about the delta function, they also talk about its derivative! Trying to differentiate something that doesn't exist...? %We'll put on our applied maths hat on and try to define the derivative.

How can we define the derivative $\de'(x-x_0)$? A first attempt might be (assuming $f$ is nice)
\begin{align}
\nonumber
\int \de'(x-x_0)f(x)\,dx&=\lim_{h\to 0} \int 
\pf{\de(x-x_0+h)-\de(x-x_0)}{h}f(x)\,dx\\
\nonumber
&=\lim_{h\to 0} \fc{f(x_0-h)-f(x_0)}{h}\\
\nonumber
&=-f'(x_0)\\
\llabel{eq:dist0-2}
&=-\int \de(x-x_0)f'(x)\,dx.
\end{align}
The equality~\eqref{eq:dist0-2} suggests that we could have simply integrated by parts
\[
\int \de'(x-x_0)f(x)\,dx=-\int \de(x-x_0)f'(x)\,dx+\ub{\text{boundary term}}{0}.
\]
This function that doesn't exist seems to follow the usual rules of calculus! 
This suggests there is a way of interpreting all the integrals in a consistent way
We can make all this rigorous using the theory of distributions. %looks like the usual rules of integral calculus can be applied to $\de(x-x_0)$.

\section{Fourier transforms of polynomials}

The \textbf{Fourier transform} is defined by (abbreviate $\int:=\iiy$)
\[
\cal F:f\mapsto \wh f(x)=\int e^{-i\la x}f(x)\,dx.
\]
This makes sense if $f$ is absolutely integrable:
\[
\ab{
\int e^{-i\la x}f(x)
}\le \int |e^{-i\la x}f(x)|\,dx=\int |f(x)|\,dx<\iy.
\]
What if $f\nin L^1$, in particular, what if $|f|\not\to 0$ as $|x|\to \iy$? You may have seen
\beq{eq:dist0-3}
\de(\la)=\rc{2\pi}\int e^{-i\la x}\,dx.
\eeq
There is a way of interpreting this so that it makes sense. This seems to suggest that the Fourier transform of $\rc{2\pi}$ is equal to $\de(\la)$. What if $f(x)=x^n$? Can we take the Fourier Transform?
\bal
\int e^{-i\la x}x^n\,dx&=\int \pa{i\pd{}{\la}}^n e^{-i\la x}\,dx\\
&=\pa{i\pd{}{\la}}^n \int e^{-i \la x}\,dx\\
&=2\pi e^{-i \pi n/2}\de^{(n)}(\la).
\end{align*}
If we can make sense on the derivatives of $\de$, then we can define the Fourier transform of polynomials.

An alternative way of defining Fourier transform of $f(x)=x^n$ would be to use Parseval's Theorem, which states
\beq{eq:dist0-4}
\int f(x)\wh g(x)\,dx =\int \wh f(\la)g(\la)\,d\la
\eeq
for all ``nice" functions $f$ and $g$. We could define $\hat f(\la)$ where $f(x)=x^n$, as the function for which
\beq{eq:dist0-5}
\int x^n\hat g(x)\,dx=\int \hat f(\la)g(\la)\,d\la\text{ for all nice functions } g
\eeq
we could say that $\hat f(\la)$ is the Fourier transform of $f(x)=x^n$. Note $\hat g(x)$ decays quickly, so this makes sense. This can be done rigorously using the theory of distributions. %If we can find $\hat f(x)$

``Everything's easy when you know the answer." It's only perfectly natural when you've been shown its perfectly natural. To prove consistency is not quite so easy.

\section{Discontinuities to Differential Equations}
%we want to see things like that happen.

There are some important genuinely important physical scenarios in which we would like a solution to a PDE to have discontinuities. For example, in acoustics we want the pressure $p(x,t)$ to solve the wave equation
\beq{eq:dist0-6}
p_{xx}-p_{tt}=0,
\eeq
but for $p$ to jump either side of a shock wave.
%propagate out?
%pavilion g

Is there any meaningful way to say that the function 
\[
u(x,y)=\al(x-y)+\be(x+y)
\]
is a solution to the wave equation $u_{xx}-u_{yy}=0$ if $\al,\be \nin C^2$? Assume $\al,\be\in C^2$ and $u_{xx}-u_{yy}=0$ so that for any nice function $f(x,y)$ (say $f=0$ when $x^2+y^2$ is sufficiently large),
\bal
0&=\int\int f(x,y)\pa{\pd{{}^2u}{x^2}-\pd{{}^2u}{y^2}}\,dx\,dy\\
&=\int\int u(x,y) \pa{\pd{{}^2}{x^2}-\pd{{}^2f}{y^2}}\,dx\,dy
\text{integration by parts twice}.
\end{align*}
If we can find $u(x,y)$ such that
\beq{eq:dist0-7}
\int\int u(x,y) \pa{\pd{{}^2}{x^2}-\pd{{}^2f}{y^2}}\,dx\,dy=0
\text{ for all nice functions }f,
\eeq
we say that $u=u(x,y)$ is a \textbf{weak solution} to the PDE $u_{xx}-u_{yy}=0$. We can use distribution theory to study weak solutions to PDE's.
%3 reasons want formalize

\section{Summary}

In each motivating example we introduced a family of ``nice" functions that allowed us to extend classical definitions to a wider class of problems. This is the underlying idea in distribution theory. Given a vector space $V$ of ``nice" functions we define the distributions on $V$ to be the topological dual $V^*$, which consists of all the continuous linear forms $V\to \C$. 

For example, if $V=C(\R)$ then we can define Dirac delta by the linear form\footnote{You may be more familiar with the notation $\an{x,y}=x\cdot y$ for finite-dimensional vector spaces.}
\beq{eq:dist0-8}
\an{\de_{x_0},f}=f(x_0).
\eeq
%linear map.
In general, any $u\in V^*$ is linear, so $\an{u, \al f+\be g}=\al\an{u,f}+\be \an{u,g}$ for arbitrary constants $\al,\be$ and $f,g\in V$. We need functional analysis because we require continuity. (The algebraic dual is too big to be interesting. Hence we supplement $V$ with a topology, i.e. a notion of convergence $f_n\to V$ in $V$. This is the motion of $w^*$-convergence, $\an{u,f_n}\to \an{u,f}$.

\blu{Lecture 2}
\chapter{Distributions}

Recap:
\begin{itemize}
\item
Delta function doesn't make sense.
\item
Way to define distributions is to first define a ``nice" space of functions $V$ (having all the properties we want it to have) and define distributions as continuous linear maps from $V$ to $\C$. 
\end{itemize}
%V has all properties we want it to have.

We'll always work with continuous functions, so we can define continuity of functions $V\to \C$ very explicitly.

\section{Notation and preliminaries}

An element of $\R^n$ will be written $x,y,z,\ldots$ so that 
\[
x=(x_1,x_2,\ldots, x_n)
\]
and we will use $dx=dx_1\,dx_2\,\cdots \,dx_n$ to denote the volume element in $\R^n$. Capitals $X,Y,Z$ will always denote open subsets of $\R^n$ and $K$ will always denote a compact (closed and bounded) subset of $\R^n$. Integrals over all $\R^n$ or over $X\subeq \R^n$ will be denoted by $\int [\cdot]\,dx$ and $\int_X[\cdot]\,dx$, respectively.
We will use multi-index notation $\al,\be,\ga$ (Greek letters) will denote multi-indices $\al=(\al_1,\ldots, \al_n)\in \Z_+^n=\{0,1,2,3,\ldots\}^{\times n}$. Multi-index notation reads as follows.
\bal
\pl^{\al} &\equiv\pa{\pdd{x_1}}^{\al_1}\cdots \pa{\pdd{x_n}}^{\al_n},&x^{\al}&\equiv x_1^{\al_1}\cdots x_n^{\al_n}\\
\al!&\equiv \al_1!\cdots \al_n!&|\al|&\equiv \al_1+\cdots +\al_n.
\end{align*}
%functions for which it's switched on.
We will often write $\pl^{\al}_x$ to make it clear what we're differentiating with respect to. We will also use $D=-i\pl$ when we do Fourier analysis. Define the \textbf{support} of a function $f$ by 
\[
\Supp(f)=\ol{\set{x\in \R^n}{f(x)\ne 0}}.
\]
We will often want to take limits inside integrals. To do this we refer to the dominated convergence theorem: (See for instance~\url{https://dl.dropboxusercontent.com/u/27883775/math\%20notes/18.125.pdf}, Theorem 15.1.)
\begin{thm}[Dominated convergence theorem]\llabel{thm:dct}
Given a sequence of absolutely integrable functions $\{f_m\}_{m\ge 1}$ such that $f_m(x)\to f(x)$ for each $x$ and $|f_m(x)|\le g(x)$, where $g$ is absolutely integrable, then 
\[
\lim_{m\to \iy}\int_Xf_m(x)=\int_X\ba{\lim_{m\to \iy} f_m(x)}\,dx=\int_X f(x)\,dx.
\]
\end{thm}
If $f$ is absolutely integrable on $X$, i.e.,
\[
\int_X |f|\,dx<\iy,
\]
we say that $f\in L^1(X)$.

\section{Test functions and distributions}
We need to define our first vector space of test functions.
\begin{df}
The space $D(X)$ consists of all the smooth functions from $X$ to $\C$ that have compact support. We say that a sequence $\{\ph_m\}_{m\ge1}$ tends to zero in $D(X)$ if there exists some compact set $K\subeq X$ such that $\Supp(\ph_m)\subeq K$ and $\pl^{\al}\ph_m\to 0$ uniformly for each multi-index $\al$.

The space $D(X)$ is often written $C^{\iy}_c(X)$.
\end{df}
(For convergence, the function is not allowed to have its mass moving away to infinity.)
%integrate by parts, evaluated on the boundary, derivatives shift to the other side.

Since the functions in $D(X)$ vanish at the boundary of $X$, we have
\[
\int_X\ph \pl^{\al}\psi\,dx=(-1)^{\al}\int_X\psi\pl^{\al}\ph\,dx,
\qquad \ph,\psi\in D(X)
\]
by integration by parts $|\al|$ times. We have Taylor's Theorem to any order
\beq{eq:taylor}
\ph(x+h)=\sum_{|\al|\subeq N}\fc{h^{\al}}{\al!} \pl^{\al}\ph(x)+R_N(x,h)
\eeq
where the remainder is $o(|h|^N)$ uniformly in $x$. %treat it as  ncice polynomial plus remainder.

%we have space of tests. distributions will be defined as linear maps. We want them to be continuous. $u(\ph)\equiv \an{u,\ph}\in \C$. If $\ph_m\to 0$ in $D(X)$, $\an{u,\ph_m}\to 0$ in $\C$.
\begin{df}\llabel{df:distribution}
A linear map $u:D(X)\to \C$ is a \textbf{distribution} (on $X$) if
for each compact $K\subeq X$, there exist $C,N$ such that 
\beq{eq:dist1-1}
|\an{u,\ph}|\le C\sum_{|\al|\le N} \sup|\pl^{\al}\ph|
\eeq
%vs sequential continuity. 
for each $\ph\in D(X)$ with $\Supp(\ph)$ with $\Supp(\ph)\subeq K$. The space of all such maps is denoted $D'(X)$. If we can use the same $N$ for all $\ph\in D(X)$, we call the least such $N$ the order of $u$, denoted $\ord(u)$.
\end{df}
%in the back of mind have this example, go back to familiar example.
\begin{rem}
Thinking of $D(X)$ as a locally convex space (in fact, Fr\'echet space) with seminorms defined by $\sup|\pl^{\al}\ph|$, we have $D'(X)=D(X)^*$. See Example 2.2.4 in Functional Analysis\footnote{\url{https://dl.dropboxusercontent.com/u/27883775/math\%20notes/part_iii_functional.pdf}}.
The example there is actually a larger space ($\cal E(X)$ of the next chapter), but it contains our $D(X)$.
\end{rem}

\begin{ex}
\begin{enumerate}
\item
We check that the Dirac delta $\de_x$ is a distribution. $\de_x$ is defined by
\[
\an{\de_x,\ph}=\ph(x)\qquad\ba{\int \de(x-y)\ph(y)\,dy=\ph(x)}.
\]
We want to check if~\eqref{eq:dist1-1} holds:
\[
\ab{\an{\de_x,\ph}}=|\ph(x)|\le \sup|\ph|
\] 
so~\eqref{eq:dist1-1} holds with $C=1$, $N=0$,  no matter what $\ph$ we choose. So $\de_x$ is a distribution of order 0.
\item
Here is a more useful example. Consider the linear map $T$ on $D(X)$ defined by 
\[
\an{T,\ph}:=\sum_{|\al|\le M} \int_X f_{\al}\pl^{\al} \ph\,dx,
\]
$f_{\al}\in C(X)$. Now for $\ph\in D(X)$, with $\Supp(\ph)\subeq K$. By definition of $T$, 
\bal
|\an{T,\ph}|&=\ab{
\sum_{|\al|\le M}\int_K f_{\al} \pl^{\al}\ph\,dx
}\\
&\le \sum_{|\al|\le M} \int_K |f_{\al}||\pl^{\al} \ph|\,dx\\
&\le \sum_{|\al|\le M} \sup_{|\al|\le M}|\pl^{\al}\ph| \int_K |f_{\al}|\,dx\\
&\le \pa{\max_{\al}\int_K |f_{\al}|\,dx}\sum_{|\al|\le M} \sup |\pl^{\al} \ph|
\end{align*}
Note that the test functions have compact support, so it doesn't matter if $f_{\al}$ blows up at the boundary. So we have estimate~\eqref{eq:dist1-1} with 
\[
C=\max_{|\al|\le M}\int_K |f_{\al}|\,dx,\qquad N=M.
\]
Note that $C=C_K$. We could have done this only assuming that $\{f_{\al}\}$ were locally integrable on $X$ (integrable on every compact subset of $X$), written $f_{\al}\in L_{\text{loc}}^1(X)$.

Note here the constant $C$ depends on the support test function. $N$ can also depend on it.
%$N$ can also depend on the support of the test function.
\end{enumerate}
\end{ex}
\begin{lem}\llabel{lem:dist1-1}
A linear map $u:D(X)\to \C$ belongs to $D'(X)$ if $\an{u,\ph_m}\to 0$ for every sequence $\{\ph_m\}_{m\ge 1}$ in $D(X)$ that tends to 0.
\end{lem}
\blu{Lecture 3 (27 Jan)} 
\begin{rem}
That Definition~\ref{df:distribution} and Lemma~\ref{lem:dist1-1} are equivalent conditions for continuity is a special case of Lemma 2.2.5 in the functional analysis notes.
\end{rem}
\begin{proof}
\begin{itemize}
\item
$\implies$: If $u\in D(X)$ and $\ph_m\to 0$ in $D(X)$ then
\[
|\an{u,\ph_m}|\le \sum_{|\al|\le m} \sup|\pl^{\al} \ph_m|\to 0.
\]
\item
$\Leftarrow$: Assume not, so there exists a compact set $K\subeq X$ such that estimate~\eqref{eq:dist1-1} does not hold for any $C,N$. In particular, it doesn't hold for $C=N=m$. So there exist $\phi_m\in D(X)$ such that $|\an{u,\phi_m}|>m\sum_{|\al|\le m}\sup |\pl^{\al} \phi_m|$. WLOG, we can assume that $\an{u,\phi_m}=1$, by setting $\wt{\phi_m}=\fc{\phi_m}{\an{u,\phi_m}}$. This implies
\bal
\implies \sum_{|\al|\le m} \sup|\pl^{\al}\phi_m|&<\rc m\\
\implies \sup|\pl^{\al}\phi_m|&<\rc m,&|\al|\le m\\
\implies \phi_m\to 0& \text{ in }D(X).
\end{align*}
But this is a contradiction.
\end{itemize}
\end{proof}
\section{Limits in $D'(X)$}
Often we have  sequence of distributions $\{u_m\}_{m\ge 1}$. If there exist some $u\in D'(X)$ such that $\an{u_m,\ph}\to \an{u,\ph}$ for all $\ph\in D(X)$, then we say that $u_m\to u$ in $D'(X)$.

Limits in $D'(X)$ often look strange. 
\begin{ex}
For instance, define the distribution $u_m\in D'(\R)$ by the locally integrable function
\[
u_m(X):=\sin (mx).
\]
Then $u_m\to 0$ in $D'(\R)$. 

%compact support, ibp no boundary terms.
Proof: We have using \blu{integration by parts}
\bal
\an{u_m,\ph}&=\int \sin(mx)\ph(x)\,dx\\
&=\rc{m} \int \cos(mx)\ph'(x)\,dx
\end{align*}
Hence $u_m\to 0$ in $D'(\R)$. 
\end{ex}
%think about Fourier expansion of $\ph$
\begin{thm}[Closure under pointwise convergence]\llabel{thm:dist1-1}
If $u_m\in D'(X)$ is such that $\lim_{m\to \iy}\an{u_m,\ph}$ exists for every $\ph\in D(X)$, then the linear map
\[
\an{u,\ph}:=\lim_{m\to \iy}\an{u_m,\ph}
\]
is an element of $D'(X)$. 
\end{thm}
It's obvious that the LHS will satisfy the estimate or the definition. Use the principle of uniform boundedness (See the Banach-Steinhaus Theorem, 4.2.7 in FA notes).

\section{Basic operations}

\subsection{Differentiation and multiplication by smooth functions} 
If $u\in C^{\iy}(X)$, then $\pl^{\al}u$ defines an element of $D'(X)$ for every multi-index $\al$ by
\bal
\an{\pl^{\al}u,\ph}&=\int_{X}\pl^{\al}u\ph\,dx\\
&=(-1)^{|\al|} \int_X u\pl^{\al} \ph\,dx&\text{integration by parts}\\
&=\an{u,(-1)^{|\al|}\pl^{\al}\ph}.
\end{align*}
%add in f
%Makes sense for any distribution, so it's a good definition.
The RHS is well-defined for any $u\in D'(X)$. We arrive at the following.
\begin{df}\llabel{df:dist1-3}
For $f\in C^{\iy}(X)$, $u\in D'(X)$ and any multi-index $\al$ we define $\pl^{\al}(fu)$ by 
\[
\an{\pl^{\al}(fu),\ph}:=\an{u,(-1)^{|\al|} f\pl^{\al}\ph}.
\]
We call $\pl^{\al}u$ the \textbf{distributional derivatives} of $u$.
\end{df}
\begin{ex}
Take the Dirac delta $\de_x$. Then $\pl^{\al}\de_x$ is defined by
\[
\an{\pl^{\al}\de_x,\ph}:=\an{\de_x,(-1)^{|\al|} \pl^{\al}\ph} = (-1)^{|\al|} \pl^{\al}\ph(x).
\]

Consider the Heaviside function
\[
H(x)=\begin{cases}
1,&x>0\\
0,&x\le 0.
\end{cases}
\]
Then this defines an element of $D'(\R)$. We compute $H'$:
\[
\an{H',\ph}:=\an{H, -\ph'}=\int_0^{\iy} -\ph'(x)\,dx = -\ph|^{\iy}_0=\ph(0)=\an{\de_0,\ph}.
\]
Hence $H'=\de_0$. In general if $\an{u,\ph}=\an{v,\ph}$ for all $\ph\in D(X)$ then we say $u=v$ in $D'(X)$. 
%gradient infinite.
\end{ex}
Now we ask: how does the calculus for normal functions carries over to calculus for distributions?
\begin{lem}
If $u'=0$ in $D'(\R)$ then $u$ is a constant.
\end{lem}
\begin{proof}
Note that $u'=0$ means 
\[0=\an{u',\psi}=-\an{u,\psi'}.\]
%every function that is a total derive.
%{\it Note that }

{\it \blu{Idea: we'd like to say that given $\ph$, we can write $\ph=\ddd x\int_{-\iy}^x \ph(y)\,dy=\psi'$, and use the above to conclude $0=\an{u,\ph}$, so $u$ is constant. The problem is that when we integrate a test function, we don't necessarily get a test function. We need to adjust our function so that the integral is 0 for large $x$.}}

Fix $\te\in D(\R)$ with $\an{1,\te}=\int \te\,dx=1$. For arbitrary $\ph\in D(\R)$ write 
\bal
\ph&=(\ph-\an{1,\ph}\te)+\an{1,\ph}\te\\
&\equiv \ph_A+\ph_B
\end{align*}
Then 
\[
\an{1,\ph_A} =\an{1,\ph}-\an{1,\ph}\an{1,\te}=0.
\]
This is helpful because 
\[
\psi_A(x)=\int_{-\iy}^x \ph_A(y)\,dy\in D(\R).
\]
%
We have $\ph_A=\psi_A'$. So 
\bal
\an{u,\ph}&=\an{u,\ph_A}+\an{u,\ph_B}\\
&=\an{u,\psi_A'}+\an{1,\ph}\ub{\an{u,\te}}b=0+c\an{1,\ph} =\an{c,\ph}.
\end{align*}
This implies that $u$ is a constant.
\end{proof}
\subsection{Reflection and translation}
\begin{df}
For $\ph\in D(\R^n)$ then we can define its \textbf{translation} by $h\in \R^n$ by 
\[
(\tau_h\ph)(x):=\ph(x-h)
\]
and \textbf{reflection}
\[
\check \ph(x)=\ph(-x).
\]
\end{df}
(Motivation: $\an{u,\ph}=\int u\ph\,dx$ gives
$\an{\tau_h u,\ph}=\int u(x-h)\ph(x)\,dx=\int u(x)\ph(x+h)\,dx=\an{u,\tau_{-h}\ph}$.) By duality, the definitions of these operations on $u\in D'(\R^n)$,
\bal
\an{\tau_hu,\ph}&:= \an{u,\tau_{-h}\ph}\\
\an{\check u,\ph}&:=\an{u,\check{\ph}}.
\end{align*} 
\begin{lem}\llabel{lem:dist1-3}
For $u\in D'(\R^n)$ define
\[
v_h=\fc{\tau_{-h} u-u}{|h|}.
\]
Then $v_h\to n\cdot \pl u$, where $\lim_{h\to 0} \fc{h}{|h|}=n\in S^{n-1}$. 
\end{lem}
\begin{proof}
We have
\bal
\an{v_h,\ph}&=\an{\fc{\tau_{-h}u-u}{|h|},\ph}\\
&=\an{u,\fc{\tau_h\ph-\ph}{|h|}}.
\end{align*}
By Taylor's Theorem,
\bal
\tau_h \ph(x)-\ph(x)&=-h\pl \ph(x)+\ub{R(x,h)}{o(|h|)\text{ in }D(\R^n)}\\
\an{u,\fc{\tau_h\ph-\ph}{|h|}}&=\an{u,-\fc{h}{|h|}\pl \ph}+\ub{\an{u,\fc{(R(\cdot ,h))}{h}}}{\to 0\text{ as $|h|\to 0$}}\\
&=n\cdot \an{\pl u, \ph}.
\qedhere
\end{align*}
\end{proof}
This shows the distributional derivative coincides with the normal notion of derivative as difference quotient.

\blu{Lecture 4 (29 Jan)}

\subsection{Convolution between $D(\R^n)$ and $D'(\R^n)$}
If we combine the operations of reflection and translation, we get 
\[
(\tau_x\check{\ph})(y)=\check{\ph} (y-x)=\ph(x-y).
\]
If $u\in D(\R^n)$, we define the \textbf{convolution} of $u$ and $\ph\in D(\R^n)$ with
\[
(u*\ph)(x):=\int u(x-y)\ph(y)\,dy=\int \ph(x-y)u(y)\,dy=(\ph*u)(x).
\]
Using $\tau_h$ and $\check{\bullet}$, we can write 
\[
u*\ph(x)=\an{u,\tau_x \check{\ph}}.
\]
This is well defined for all $u\in D'(\R^n)$.
\begin{df}\llabel{df:dist1-5}
For $u\in D'(\R^n)$ and $\ph\in D(\R^n)$, define their convolution by 
\[u*\ph(x)=\an{u,\tau_x\check{\ph}}.\]
\end{df}
It is clear that $u*\ph(x)$ is just some function of $x\in \R^n$. It is actually smooth.
\begin{lem}\llabel{lem:dist1-4}
Define $\Phi_x(y)=\phi(x,y)$ where $\phi\in C^{\iy}(\R^n\times \R^n)$ and $\phi(\cdot, y)=0$ for $y$ outside some compact $K\subeq \R^n$. Then for $u\in D'(\R^n)$, 
\[
\pl_x^{\al} \an{u,\Phi_x}=\an{u, \pl^{\al}_x \Phi_x}.
\]
\end{lem}
We can take the derivatives inside the bracket.
\begin{proof}
By Taylor's theorem,  
%left with is linear map on $H$.
%use defn of $D'$
\[
\Phi_{x+h}(y)-\Phi_x(y)=\sum_i h_i \pd{\phi}{x_i} (x,y)+R_x(y,h)
\]
It is not difficult to show that $R_x(y,h)=o(|h|)$ in $D(\R^n)$ for each $x\in \R^n$. Hence
\[
\an{u,\Phi_{x+h}}-\an{u,\Phi_x}=\sum_i h_i \an{u,\pl{\Phi_x}{x_i}}+\an{u,R_x(\cdot, h)}.
\]
%sequence of test functions tend to 0, then continuity means $u$ acting on that sequence tends to 0.
Since $R_x(\cdot, h)=o(|h|)$ in $D(\R^n)$, dividing by $|h|$ and taking $h\to 0$ gives ($u=\fc{h}{|h|}$)
\[
n\cdot \an{u, \Phi_x}=n\cdot \an{u,\pl_x\Phi}.
\]
So the result follows.
\end{proof}
\begin{cor}\llabel{cor:dist1-1}
If $u\in D'(\R^n)$ and $\ph\in D(\R^n)$ then $u*\ph$ is smooth and $\pl^{\al}(u*\ph)=u*\pl^{\al}\ph$. 
\end{cor}
\begin{proof}
By Lemma~\ref{lem:dist1-4}, $\pl^{\al}(u*\ph)=\pl^{\al}_x\an{u,\tau_x \check{\ph}}=u*\pl^{\al}\ph$.
\end{proof}
%dist are normal functions, just take a derivative at the end.
\subsection{Density of $D(\R^n)$ in $D'(\R^n)$}
We have just seen the following.\\

\cpbox{
If $u\in D'(\R^n)$ and $\ph\in D(\R^n)$ then $u*\ph$ is smooth.}
\vskip0.15in
This is extremely useful. No matter how wild $u$ is, $u*\ph$ is nice. For this reason, $u*\ph$ is often called a  \textbf{regularisation} of the distribution $u$.
%wild creatures. give me any wild distribution, convolute with anything, get a smooth function. This is what we'll use to prove the density result.
We will use this fact to prove $D(\R^n)$ is dense in $D'(\R^n)$, i.e., for each $u\in D^1(\R^n)$ there exists a sequence of test functions $\{\ph_m\}_{m\ge 1}$ in $D(\R^n)$ such that $\ph_m\to u$ in $D'(\R^n)$, i.e., $\an{\ph_m,\te}\to \an{u,\te}$ for all $\te\in D(\R^n)$.

\blu{How to apply this: Suppose we have a problem about a distribution we don't know anything about. We know for some sequence of $\phi_m$, $\ph_m=u*\phi_m\to u$. We replace $u$ with $\ph_m$, do manipulations there, run the argument and take a limit.}


We need a technical lemma.
\begin{lem}\llabel{lem:dist1-5}
For $\ph,\psi\in D(\R^n)$ and $u\in D'(\R^n)$ we have
\[
(u*\ph)*\psi=u*(\ph*\psi).
\]
%insert dummy var
\end{lem}
\begin{proof}
The LHS is
\bal
(u*\ph)*\psi(x)&=\int (u*\ph)(x-y)\psi(y)\,dy\\
&=\int \an{u,\tau_{x-y} \wh \ph}\psi(y)\,dy\\
&=\int\an{u(z),\ph(x-y-z)\psi(y)}\,dy&\text{$z$ is ``dummy" variable}\\
&=\lim_{h\to 0} \sum_{m\in \Z^n}\an{u(z), \ph(x-z-hm)\psi(hm)h^n}&\text{Riemann sum}\\
&=\lim_{h\to 0}\an{u(z), \sum_{m\in\Z^n}\ph(x-z-hm)\ph(hm)h^n}
\end{align*}
(We want to take the $\int$ inside the $\an{\cdot}$, and we do this by turning it into a Riemann sum. Note the sum only has finitely many terms for each $m$, so this is legal.)
%sum eventually dead, only finitely many terms.
It is not hard to show that 
\[
\sum_{m\in \Z^n}\ph(x-hm)\psi(m)h^n\to \ph*\psi(x)\text{ in }D(\R^n)\text{ as }h\to 0.
\]
Continuing the above calculation,
\bal
&=\an{u(z),\ph*\psi(x-z)}\\
&=\an{u,\tau_x (\ph*\psi)\check{\,}}\\
&=u*(\ph*\psi)(x).
\end{align*}
%compact support, LHS has compact support, by fixed compact set, can differentiate
\end{proof}
\begin{thm}\llabel{thm:dist1-2}
$D(\R^n)$ is dense in $D'(\R^n)$. 
\end{thm}
We would like to say
\[
u(x)=\int\de(x-y)u(y)\,dy\stackrel?= \lim_{m\to \iy} \int \de_m(x-y)u(y)\,dy
\]
The $\de_m(x)$ are such that $\int \de_m(x)\,dx=1$ and get squashed closer and closer to $\de$. %(``a family of good kernels").
\begin{proof}
Fix $\psi\in D(\R^n)$ with $\int \psi\,dx=1$ and set
\[
\phi_m(x)=m^n \psi(mx)\qquad \pa{\int \phi_m\,dx=\int \psi\,dx=1}.
\]
%convolution of look-like de with original
Also introduce the bump function $\chi\in D(\R^n)$ with $\chi=1$ on $|x|<1$ and $\chi=0$ on $|x|>1$. Now set $\chi_m\pf xm$ and
\[
\ph_m=\pur{\chi_m(x)}(u*\phi_m)(x).
\]
%why can't $\ph$ of this guy?
(The purpose of $\pur{\chi_m(x)}$ is to make $\ph_m$ have compact support.)
Choose $\an{\ph_m,\te}$ for $\te\in D(\R^n)$ arbitrary, giving
\bal
\an{\ph_m,\te}&=\an{u*\phi_m, \chi_m\te}\\
&=(u*\phi_m)*(\chi_m\te)\check{\,}(0)\\
%&=u*\check{\ph}(0)\\
&=u*(\phi_m*(\chi_m \te)\check{\,})(0)\\
\phi_m*(\chi_m\te)\check{\,}(x)&=\int\phi_m (x-y)\chi_n(-y)\te(-y)\,dy\\
&=\int m^n \phi(m(x-y))\chi\pf{-y}{m}\te(-y)\,dy& y'=m(x-y)\implies y=x-\fc{y'}m\\
%Make the substitution 
&=\int\phi(y)\chi\pa{\fc y{m^2}-\fc xm}\te\pa{\fc ym -x}\,dy\\
&=\te(-x)+\ub{\int \phi(y)\chi\pa{\fc y{m^2}-\fc xm}\ba{
\te\pa{\fc ym-x}-\te(-x)
}\,dy}{=:R_m(-x)}\\
&=(\check{\te}+\check{R_m})(x).
\end{align*}
We can show $R_m\to 0$ in $D(\R^n)$. So \[\an{\ph_m,\te}=\an{u,\te}+\ub{\an{u,R_m}}{\to 0\text{ as }m\to \iy}.\] Hence $\an{\ph_m,\te}\to \an{u,\te},\te\in D(\R^n)$, giving $\ph_m\to u$ in $D'(\R^n)$ i.e. $D(\R^n)$ dense in $D'(\R^n)$. 
\end{proof}

 
%\bibliographystyle{plain}
%\bibliography{refs}
\end{document}