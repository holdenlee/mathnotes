\def\filepath{C:/Users/Owner/Dropbox/Math/templates}

\input{\filepath/packages_article.tex}
\input{\filepath/theorems_with_boxes.tex}
\input{\filepath/macros.tex}
\input{\filepath/formatting.tex}

\pagestyle{fancy}
\lhead{Simons meeting}
\chead{} 
\rhead{Machine learning} 
\lfoot{} 
\cfoot{\thepage} 
\rfoot{} 
\renewcommand{\headrulewidth}{.3pt} 
\setlength\voffset{0in}
\setlength\textheight{648pt}

\addbibresource{bib.bib}

\begin{document}
\section{Yann LeCun}

``The interesting problems are nonconvex."

What's happened over the last 5 years is a revolution that has infected many areas in computer science: deep learning, neural nets rehashed and modernized, taking advantage of fast computers. There's been a gold rush of companies using this for speech recognition (the reason you can talk to your smartphone now), image processing, etc. The same thing happened with computer vision in 2013: everyone dropped what they were doing and did deep learning. %can't get accepted unless deep learning.
Industry picked up this much faster than academia: Microsoft, Google, facebook, IBM, etc.---because it works. %The way research communities are organized
We don't understand why it works; we only have intuition and hand-wavy arguments.
(Even simpler things than deep learning are nonconvex and we don't know why they work.)

People upload 2 billion photos to facebook every day (600 million directly to facebook). They go through 2 deep learning systems: face recognition and image understanding. This allows facebook to select content relevant to you.
%nearby people.
Ex. The Who's Nearby app instantly identifies your friends when you snap a photo. It's a convolutional neural network. %12000

%800 million 
Take image, run face detector, identify key points on the face, rectify,  normalize (rotate to be frontal, etc., so eyes, etc. are always in the same place), unfold the sides of the face, feed it to a convolutional neural net. It's been trained to classify thousands of people, and learn a metric. ``These people should be close together, these people should be far apart." It's really an embedding into Euclidean space. Start with 100$\times$100 pixels, reduce to 4000 dimensions. Use a hashing technique to compress to 256 bits for quick comparison.

%not earthmover, 
%Identify eyes and mouth, fit 3-D model of face, rotate so frontal, unfold sides. Eyes are always in the same place, etc. 

Naor: what about embedding before neural nets?

\emph{There is no well-defined initial metric to compare faces.}

Think of each person is a manifold. The ambient dimension is low, bounded by muscles and lighting sources (forget about hair). How do you separate the manifold?

Train the distance metric (``Siamese neural net"):
\[
\ve{G_v(X')-G_w(X')}^2.
\]
Train so pictures of the same person are close together, and of different people are far away.

What did people use before deep learning? PCA. Normalize the image and project it onto a low-dimensional space maximizing variance. It doesn't work that well. People call this ``eigenfaces." Then people used sparse representations, etc. Highly engineered things---hand crafted features like distance between eyes and nose. Deep learning means we don't have the hand-craft the features.

There is a technique called linear classification.

A linear classifier looks at $\sign(w^Tx+b)$. The categories should be linearly separable. Then people used a kernel $\Phi$ and calculated $w^T\Phi(x)$. But for say, facial recognition, the number of basis functions is ridiculously large---this method very sensitive to movements in a few pixels, etc. The main problem is invariance. The mapping $G_w(X)$ has to have complicated invariant properties. It has to be smooth: as long as the variability makes you move in a flat manifold you can separate it from others.
%Character recognition: Suppose we want to recognize 3. 
Variations should move on a flat manifold. If the last step is a dot product, then variations on a flat manifold don't matter.

The next step is to stack multiple layers of basis functions. Take basis functions from previous layer and combine them nonlinearly. 

%If we take inputs to particular node
Take a vector at a layer, multiply by a matrix, take each component and send through nonlinear function (ex. rectification). (We used to use sigmoid functions, but rectification $1_{\ge 0}x$ works better.)

Here's a handwavy argument for using layers. Suppose we're learning boolean functions, ex. most significant bit of sum of the two.

We can do this in two layers: compute the intermediate results. The bottom is $\wedge$, the top is $\vee$. But for most functions, the number of minterms in DNF is exponential. If you allow multiple layers, you allow algorithms with greater steps. Ex. for addition, you can do with a linear number of nodes in a linear number of layers. %Ex. for addition, you can do 
This took 30 years for the machine learning community to understand.

%AN: What do you mean by layers. 

It's rare that we use a full matrix. We give some structure, ex. for convolutional neural nets. The matrix is a ``filter bank."

An image is 3 2-dimensional arrays (RGB). Suppose we have only 1. 
\begin{enumerate}
\item
The first layer generates a set of planes the same size as the original ones. 
Each is a dot product of a region ($5\times 5$ neighborhood) with a weight vector. It's not smoothing/averaging, ex. looking at derivatives/differences.
\item
Do subsampling: each plane reduces resolution: each one takes the max over a region. This gives shift invariance: if we shift the original picture by 1 pixel, and subsample in $2\times2$ squares, then it's shifted by $\rc2$ pixels there. 
 The manifold of transforms is more linear; the curvature is smaller.
\item
Now repeat 5--7 times, so the resolution shrinks at every steps. At the end we get a vector each component having dimension 1.
\end{enumerate}
The planes are different features, ex. the first layer identifies oriented edges, crosses; the second corners, gratings; the third, circles, squares... The influence area of a particular unit increases in later layers---they give global features of the image.

%Feathery things
%The first 

Figure out all paths from an input to output going through all intermediate notes. It's either active if all the nodes are the linear (not flat) region of the rectifier, or inactive. The contribution is
\[
\wt y=\sum_p\de(p,w,x)x_p\prod_{i\in P}w_i.
\]
$x_p$ is the input taken from path $p$.
$\de(p)$ indicates whether the path is active. (Here we're just considering rectifiers. Max nodes also activates a switch and is similar.) 
%\chapter{Motivation}

%We'll see lots of analysis in action.
\section{Dirac delta function}

What is the derivative of the Dirac delta?

You may have seen the mysterious ``\textbf{Dirac delta function}" defined by 
\beq{eq:dist0-1}
\int_{-\iy}^{\iy} \de(x-x_0)f(x)\,dx=f(x_0), \qquad \de(x)=0,\,x\ne 0.
\eeq
This emerged from Fourier's classical treatise on heat. It was there implicitly in his work. Cauchy and Dirac noticed it. It is used in math, applied math, physics, engineering.
%does the job we're asking it

But there is no function in any sense of the word that does this job! It makes no mathematical sense!

%Let's look at~\eqref{eq:dist0-1} in an abstract sense. 
However, looking at~\eqref{eq:dist0-1} in an abstract sense, the ``process" which takes $\de(x-x_0)$ and the ``nice" function $f(x)$, and spits out $f(x_0)$ is well-defined.
%There's some $\de(x-x_0)$, given $f(x)$, some process happens and spits out $f(x_0)$. 

However, people don't just talk about the delta function, they also talk about its derivative! Trying to differentiate something that doesn't exist...? %We'll put on our applied maths hat on and try to define the derivative.

How can we define the derivative $\de'(x-x_0)$? A first attempt might be (assuming $f$ is nice)
\begin{align}
\nonumber
\int \de'(x-x_0)f(x)\,dx&=\lim_{h\to 0} \int 
\pf{\de(x-x_0+h)-\de(x-x_0)}{h}f(x)\,dx\\
\nonumber
&=\lim_{h\to 0} \fc{f(x_0-h)-f(x_0)}{h}\\
\nonumber
&=-f'(x_0)\\
\llabel{eq:dist0-2}
&=-\int \de(x-x_0)f'(x)\,dx.
\end{align}
The equality~\eqref{eq:dist0-2} suggests that we could have simply integrated by parts
\[
\int \de'(x-x_0)f(x)\,dx=-\int \de(x-x_0)f'(x)\,dx+\ub{\text{boundary term}}{0}.
\]
This function that doesn't exist seems to follow the usual rules of calculus! 
This suggests there is a way of interpreting all the integrals in a consistent way
We can make all this rigorous using the theory of distributions. %looks like the usual rules of integral calculus can be applied to $\de(x-x_0)$.

\section{Fourier transforms of polynomials}

The \textbf{Fourier transform} is defined by (abbreviate $\int:=\iiy$)
\[
\cal F:f\mapsto \wh f(x)=\int e^{-i\la x}f(x)\,dx.
\]
This makes sense if $f$ is absolutely integrable:
\[
\ab{
\int e^{-i\la x}f(x)
}\le \int |e^{-i\la x}f(x)|\,dx=\int |f(x)|\,dx<\iy.
\]
What if $f\nin L^1$, in particular, what if $|f|\not\to 0$ as $|x|\to \iy$? You may have seen
\beq{eq:dist0-3}
\de(\la)=\rc{2\pi}\int e^{-i\la x}\,dx.
\eeq
There is a way of interpreting this so that it makes sense. This seems to suggest that the Fourier transform of $\rc{2\pi}$ is equal to $\de(\la)$. What if $f(x)=x^n$? Can we take the Fourier Transform?
\bal
\int e^{-i\la x}x^n\,dx&=\int \pa{i\pd{}{\la}}^n e^{-i\la x}\,dx\\
&=\pa{i\pd{}{\la}}^n \int e^{-i \la x}\,dx\\
&=2\pi e^{-i \pi n/2}\de^{(n)}(\la).
\end{align*}
If we can make sense on the derivatives of $\de$, then we can define the Fourier transform of polynomials.

An alternative way of defining Fourier transform of $f(x)=x^n$ would be to use Parseval's Theorem, which states
\beq{eq:dist0-4}
\int f(x)\wh g(x)\,dx =\int \wh f(\la)g(\la)\,d\la
\eeq
for all ``nice" functions $f$ and $g$. We could define $\hat f(\la)$ where $f(x)=x^n$, as the function for which
\beq{eq:dist0-5}
\int x^n\hat g(x)\,dx=\int \hat f(\la)g(\la)\,d\la\text{ for all nice functions } g
\eeq
we could say that $\hat f(\la)$ is the Fourier transform of $f(x)=x^n$. Note $\hat g(x)$ decays quickly, so this makes sense. This can be done rigorously using the theory of distributions. %If we can find $\hat f(x)$

``Everything's easy when you know the answer." It's only perfectly natural when you've been shown its perfectly natural. To prove consistency is not quite so easy.

\section{Discontinuities to Differential Equations}
%we want to see things like that happen.

There are some important genuinely important physical scenarios in which we would like a solution to a PDE to have discontinuities. For example, in acoustics we want the pressure $p(x,t)$ to solve the wave equation
\beq{eq:dist0-6}
p_{xx}-p_{tt}=0,
\eeq
but for $p$ to jump either side of a shock wave.
%propagate out?
%pavilion g

Is there any meaningful way to say that the function 
\[
u(x,y)=\al(x-y)+\be(x+y)
\]
is a solution to the wave equation $u_{xx}-u_{yy}=0$ if $\al,\be \nin C^2$? Assume $\al,\be\in C^2$ and $u_{xx}-u_{yy}=0$ so that for any nice function $f(x,y)$ (say $f=0$ when $x^2+y^2$ is sufficiently large),
\bal
0&=\int\int f(x,y)\pa{\pd{{}^2u}{x^2}-\pd{{}^2u}{y^2}}\,dx\,dy\\
&=\int\int u(x,y) \pa{\pd{{}^2}{x^2}-\pd{{}^2f}{y^2}}\,dx\,dy
\text{integration by parts twice}.
\end{align*}
If we can find $u(x,y)$ such that
\beq{eq:dist0-7}
\int\int u(x,y) \pa{\pd{{}^2}{x^2}-\pd{{}^2f}{y^2}}\,dx\,dy=0
\text{ for all nice functions }f,
\eeq
we say that $u=u(x,y)$ is a \textbf{weak solution} to the PDE $u_{xx}-u_{yy}=0$. We can use distribution theory to study weak solutions to PDE's.
%3 reasons want formalize

\section{Summary}

In each motivating example we introduced a family of ``nice" functions that allowed us to extend classical definitions to a wider class of problems. This is the underlying idea in distribution theory. Given a vector space $V$ of ``nice" functions we define the distributions on $V$ to be the topological dual $V^*$, which consists of all the continuous linear forms $V\to \C$. 

For example, if $V=C(\R)$ then we can define Dirac delta by the linear form\footnote{You may be more familiar with the notation $\an{x,y}=x\cdot y$ for finite-dimensional vector spaces.}
\beq{eq:dist0-8}
\an{\de_{x_0},f}=f(x_0).
\eeq
%linear map.
In general, any $u\in V^*$ is linear, so $\an{u, \al f+\be g}=\al\an{u,f}+\be \an{u,g}$ for arbitrary constants $\al,\be$ and $f,g\in V$. We need functional analysis because we require continuity. (The algebraic dual is too big to be interesting. Hence we supplement $V$ with a topology, i.e. a notion of convergence $f_n\to V$ in $V$. This is the motion of $w^*$-convergence, $\an{u,f_n}\to \an{u,f}$.

\blu{Lecture 2}
\chapter{Distributions}

Recap:
\begin{itemize}
\item
Delta function doesn't make sense.
\item
Way to define distributions is to first define a ``nice" space of functions $V$ (having all the properties we want it to have) and define distributions as continuous linear maps from $V$ to $\C$. 
\end{itemize}
%V has all properties we want it to have.

We'll always work with continuous functions, so we can define continuity of functions $V\to \C$ very explicitly.

\section{Notation and preliminaries}

An element of $\R^n$ will be written $x,y,z,\ldots$ so that 
\[
x=(x_1,x_2,\ldots, x_n)
\]
and we will use $dx=dx_1\,dx_2\,\cdots \,dx_n$ to denote the volume element in $\R^n$. Capitals $X,Y,Z$ will always denote open subsets of $\R^n$ and $K$ will always denote a compact (closed and bounded) subset of $\R^n$. Integrals over all $\R^n$ or over $X\subeq \R^n$ will be denoted by $\int [\cdot]\,dx$ and $\int_X[\cdot]\,dx$, respectively.
We will use multi-index notation $\al,\be,\ga$ (Greek letters) will denote multi-indices $\al=(\al_1,\ldots, \al_n)\in \Z_+^n=\{0,1,2,3,\ldots\}^{\times n}$. Multi-index notation reads as follows.
\bal
\pl^{\al} &\equiv\pa{\pdd{x_1}}^{\al_1}\cdots \pa{\pdd{x_n}}^{\al_n},&x^{\al}&\equiv x_1^{\al_1}\cdots x_n^{\al_n}\\
\al!&\equiv \al_1!\cdots \al_n!&|\al|&\equiv \al_1+\cdots +\al_n.
\end{align*}
%functions for which it's switched on.
We will often write $\pl^{\al}_x$ to make it clear what we're differentiating with respect to. We will also use $D=-i\pl$ when we do Fourier analysis. Define the \textbf{support} of a function $f$ by 
\[
\Supp(f)=\ol{\set{x\in \R^n}{f(x)\ne 0}}.
\]
We will often want to take limits inside integrals. To do this we refer to the dominated convergence theorem: (See for instance~\url{https://dl.dropboxusercontent.com/u/27883775/math\%20notes/18.125.pdf}, Theorem 15.1.)
\begin{thm}[Dominated convergence theorem]\llabel{thm:dct}
Given a sequence of absolutely integrable functions $\{f_m\}_{m\ge 1}$ such that $f_m(x)\to f(x)$ for each $x$ and $|f_m(x)|\le g(x)$, where $g$ is absolutely integrable, then 
\[
\lim_{m\to \iy}\int_Xf_m(x)=\int_X\ba{\lim_{m\to \iy} f_m(x)}\,dx=\int_X f(x)\,dx.
\]
\end{thm}
If $f$ is absolutely integrable on $X$, i.e.,
\[
\int_X |f|\,dx<\iy,
\]
we say that $f\in L^1(X)$.

\section{Test functions and distributions}
We need to define our first vector space of test functions.
\begin{df}
The space $D(X)$ consists of all the smooth functions from $X$ to $\C$ that have compact support. We say that a sequence $\{\ph_m\}_{m\ge1}$ tends to zero in $D(X)$ if there exists some compact set $K\subeq X$ such that $\Supp(\ph_m)\subeq K$ and $\pl^{\al}\ph_m\to 0$ uniformly for each multi-index $\al$.

The space $D(X)$ is often written $C^{\iy}_c(X)$.
\end{df}
(For convergence, the function is not allowed to have its mass moving away to infinity.)
%integrate by parts, evaluated on the boundary, derivatives shift to the other side.

Since the functions in $D(X)$ vanish at the boundary of $X$, we have
\[
\int_X\ph \pl^{\al}\psi\,dx=(-1)^{\al}\int_X\psi\pl^{\al}\ph\,dx,
\qquad \ph,\psi\in D(X)
\]
by integration by parts $|\al|$ times. We have Taylor's Theorem to any order
\beq{eq:taylor}
\ph(x+h)=\sum_{|\al|\subeq N}\fc{h^{\al}}{\al!} \pl^{\al}\ph(x)+R_N(x,h)
\eeq
where the remainder is $o(|h|^N)$ uniformly in $x$. %treat it as  ncice polynomial plus remainder.

%we have space of tests. distributions will be defined as linear maps. We want them to be continuous. $u(\ph)\equiv \an{u,\ph}\in \C$. If $\ph_m\to 0$ in $D(X)$, $\an{u,\ph_m}\to 0$ in $\C$.
\begin{df}\llabel{df:distribution}
A linear map $u:D(X)\to \C$ is a \textbf{distribution} (on $X$) if
for each compact $K\subeq X$, there exist $C,N$ such that 
\beq{eq:dist1-1}
|\an{u,\ph}|\le C\sum_{|\al|\le N} \sup|\pl^{\al}\ph|
\eeq
%vs sequential continuity. 
for each $\ph\in D(X)$ with $\Supp(\ph)$ with $\Supp(\ph)\subeq K$. The space of all such maps is denoted $D'(X)$. If we can use the same $N$ for all $\ph\in D(X)$, we call the least such $N$ the order of $u$, denoted $\ord(u)$.
\end{df}
%in the back of mind have this example, go back to familiar example.
\begin{rem}
Thinking of $D(X)$ as a locally convex space (in fact, Fr\'echet space) with seminorms defined by $\sup|\pl^{\al}\ph|$, we have $D'(X)=D(X)^*$. See Example 2.2.4 in Functional Analysis\footnote{\url{https://dl.dropboxusercontent.com/u/27883775/math\%20notes/part_iii_functional.pdf}}.
The example there is actually a larger space ($\cal E(X)$ of the next chapter), but it contains our $D(X)$.
\end{rem}

\begin{ex}
\begin{enumerate}
\item
We check that the Dirac delta $\de_x$ is a distribution. $\de_x$ is defined by
\[
\an{\de_x,\ph}=\ph(x)\qquad\ba{\int \de(x-y)\ph(y)\,dy=\ph(x)}.
\]
We want to check if~\eqref{eq:dist1-1} holds:
\[
\ab{\an{\de_x,\ph}}=|\ph(x)|\le \sup|\ph|
\] 
so~\eqref{eq:dist1-1} holds with $C=1$, $N=0$,  no matter what $\ph$ we choose. So $\de_x$ is a distribution of order 0.
\item
Here is a more useful example. Consider the linear map $T$ on $D(X)$ defined by 
\[
\an{T,\ph}:=\sum_{|\al|\le M} \int_X f_{\al}\pl^{\al} \ph\,dx,
\]
$f_{\al}\in C(X)$. Now for $\ph\in D(X)$, with $\Supp(\ph)\subeq K$. By definition of $T$, 
\bal
|\an{T,\ph}|&=\ab{
\sum_{|\al|\le M}\int_K f_{\al} \pl^{\al}\ph\,dx
}\\
&\le \sum_{|\al|\le M} \int_K |f_{\al}||\pl^{\al} \ph|\,dx\\
&\le \sum_{|\al|\le M} \sup_{|\al|\le M}|\pl^{\al}\ph| \int_K |f_{\al}|\,dx\\
&\le \pa{\max_{\al}\int_K |f_{\al}|\,dx}\sum_{|\al|\le M} \sup |\pl^{\al} \ph|
\end{align*}
Note that the test functions have compact support, so it doesn't matter if $f_{\al}$ blows up at the boundary. So we have estimate~\eqref{eq:dist1-1} with 
\[
C=\max_{|\al|\le M}\int_K |f_{\al}|\,dx,\qquad N=M.
\]
Note that $C=C_K$. We could have done this only assuming that $\{f_{\al}\}$ were locally integrable on $X$ (integrable on every compact subset of $X$), written $f_{\al}\in L_{\text{loc}}^1(X)$.

Note here the constant $C$ depends on the support test function. $N$ can also depend on it.
%$N$ can also depend on the support of the test function.
\end{enumerate}
\end{ex}
\begin{lem}\llabel{lem:dist1-1}
A linear map $u:D(X)\to \C$ belongs to $D'(X)$ if $\an{u,\ph_m}\to 0$ for every sequence $\{\ph_m\}_{m\ge 1}$ in $D(X)$ that tends to 0.
\end{lem}
\blu{Lecture 3 (27 Jan)} 
\begin{rem}
That Definition~\ref{df:distribution} and Lemma~\ref{lem:dist1-1} are equivalent conditions for continuity is a special case of Lemma 2.2.5 in the functional analysis notes.
\end{rem}
\begin{proof}
\begin{itemize}
\item
$\implies$: If $u\in D(X)$ and $\ph_m\to 0$ in $D(X)$ then
\[
|\an{u,\ph_m}|\le \sum_{|\al|\le m} \sup|\pl^{\al} \ph_m|\to 0.
\]
\item
$\Leftarrow$: Assume not, so there exists a compact set $K\subeq X$ such that estimate~\eqref{eq:dist1-1} does not hold for any $C,N$. In particular, it doesn't hold for $C=N=m$. So there exist $\phi_m\in D(X)$ such that $|\an{u,\phi_m}|>m\sum_{|\al|\le m}\sup |\pl^{\al} \phi_m|$. WLOG, we can assume that $\an{u,\phi_m}=1$, by setting $\wt{\phi_m}=\fc{\phi_m}{\an{u,\phi_m}}$. This implies
\bal
\implies \sum_{|\al|\le m} \sup|\pl^{\al}\phi_m|&<\rc m\\
\implies \sup|\pl^{\al}\phi_m|&<\rc m,&|\al|\le m\\
\implies \phi_m\to 0& \text{ in }D(X).
\end{align*}
But this is a contradiction.
\end{itemize}
\end{proof}
\section{Limits in $D'(X)$}
Often we have  sequence of distributions $\{u_m\}_{m\ge 1}$. If there exist some $u\in D'(X)$ such that $\an{u_m,\ph}\to \an{u,\ph}$ for all $\ph\in D(X)$, then we say that $u_m\to u$ in $D'(X)$.

Limits in $D'(X)$ often look strange. 
\begin{ex}
For instance, define the distribution $u_m\in D'(\R)$ by the locally integrable function
\[
u_m(X):=\sin (mx).
\]
Then $u_m\to 0$ in $D'(\R)$. 

%compact support, ibp no boundary terms.
Proof: We have using \blu{integration by parts}
\bal
\an{u_m,\ph}&=\int \sin(mx)\ph(x)\,dx\\
&=\rc{m} \int \cos(mx)\ph'(x)\,dx
\end{align*}
Hence $u_m\to 0$ in $D'(\R)$. 
\end{ex}
%think about Fourier expansion of $\ph$
\begin{thm}[Closure under pointwise convergence]\llabel{thm:dist1-1}
If $u_m\in D'(X)$ is such that $\lim_{m\to \iy}\an{u_m,\ph}$ exists for every $\ph\in D(X)$, then the linear map
\[
\an{u,\ph}:=\lim_{m\to \iy}\an{u_m,\ph}
\]
is an element of $D'(X)$. 
\end{thm}
It's obvious that the LHS will satisfy the estimate or the definition. Use the principle of uniform boundedness (See the Banach-Steinhaus Theorem, 4.2.7 in FA notes).

\section{Basic operations}

\subsection{Differentiation and multiplication by smooth functions} 
If $u\in C^{\iy}(X)$, then $\pl^{\al}u$ defines an element of $D'(X)$ for every multi-index $\al$ by
\bal
\an{\pl^{\al}u,\ph}&=\int_{X}\pl^{\al}u\ph\,dx\\
&=(-1)^{|\al|} \int_X u\pl^{\al} \ph\,dx&\text{integration by parts}\\
&=\an{u,(-1)^{|\al|}\pl^{\al}\ph}.
\end{align*}
%add in f
%Makes sense for any distribution, so it's a good definition.
The RHS is well-defined for any $u\in D'(X)$. We arrive at the following.
\begin{df}\llabel{df:dist1-3}
For $f\in C^{\iy}(X)$, $u\in D'(X)$ and any multi-index $\al$ we define $\pl^{\al}(fu)$ by 
\[
\an{\pl^{\al}(fu),\ph}:=\an{u,(-1)^{|\al|} f\pl^{\al}\ph}.
\]
We call $\pl^{\al}u$ the \textbf{distributional derivatives} of $u$.
\end{df}
\begin{ex}
Take the Dirac delta $\de_x$. Then $\pl^{\al}\de_x$ is defined by
\[
\an{\pl^{\al}\de_x,\ph}:=\an{\de_x,(-1)^{|\al|} \pl^{\al}\ph} = (-1)^{|\al|} \pl^{\al}\ph(x).
\]

Consider the Heaviside function
\[
H(x)=\begin{cases}
1,&x>0\\
0,&x\le 0.
\end{cases}
\]
Then this defines an element of $D'(\R)$. We compute $H'$:
\[
\an{H',\ph}:=\an{H, -\ph'}=\int_0^{\iy} -\ph'(x)\,dx = -\ph|^{\iy}_0=\ph(0)=\an{\de_0,\ph}.
\]
Hence $H'=\de_0$. In general if $\an{u,\ph}=\an{v,\ph}$ for all $\ph\in D(X)$ then we say $u=v$ in $D'(X)$. 
%gradient infinite.
\end{ex}
Now we ask: how does the calculus for normal functions carries over to calculus for distributions?
\begin{lem}
If $u'=0$ in $D'(\R)$ then $u$ is a constant.
\end{lem}
\begin{proof}
Note that $u'=0$ means 
\[0=\an{u',\psi}=-\an{u,\psi'}.\]
%every function that is a total derive.
%{\it Note that }

{\it \blu{Idea: we'd like to say that given $\ph$, we can write $\ph=\ddd x\int_{-\iy}^x \ph(y)\,dy=\psi'$, and use the above to conclude $0=\an{u,\ph}$, so $u$ is constant. The problem is that when we integrate a test function, we don't necessarily get a test function. We need to adjust our function so that the integral is 0 for large $x$.}}

Fix $\te\in D(\R)$ with $\an{1,\te}=\int \te\,dx=1$. For arbitrary $\ph\in D(\R)$ write 
\bal
\ph&=(\ph-\an{1,\ph}\te)+\an{1,\ph}\te\\
&\equiv \ph_A+\ph_B
\end{align*}
Then 
\[
\an{1,\ph_A} =\an{1,\ph}-\an{1,\ph}\an{1,\te}=0.
\]
This is helpful because 
\[
\psi_A(x)=\int_{-\iy}^x \ph_A(y)\,dy\in D(\R).
\]
%
We have $\ph_A=\psi_A'$. So 
\bal
\an{u,\ph}&=\an{u,\ph_A}+\an{u,\ph_B}\\
&=\an{u,\psi_A'}+\an{1,\ph}\ub{\an{u,\te}}b=0+c\an{1,\ph} =\an{c,\ph}.
\end{align*}
This implies that $u$ is a constant.
\end{proof}
\subsection{Reflection and translation}
\begin{df}
For $\ph\in D(\R^n)$ then we can define its \textbf{translation} by $h\in \R^n$ by 
\[
(\tau_h\ph)(x):=\ph(x-h)
\]
and \textbf{reflection}
\[
\check \ph(x)=\ph(-x).
\]
\end{df}
(Motivation: $\an{u,\ph}=\int u\ph\,dx$ gives
$\an{\tau_h u,\ph}=\int u(x-h)\ph(x)\,dx=\int u(x)\ph(x+h)\,dx=\an{u,\tau_{-h}\ph}$.) By duality, the definitions of these operations on $u\in D'(\R^n)$,
\bal
\an{\tau_hu,\ph}&:= \an{u,\tau_{-h}\ph}\\
\an{\check u,\ph}&:=\an{u,\check{\ph}}.
\end{align*} 
\begin{lem}\llabel{lem:dist1-3}
For $u\in D'(\R^n)$ define
\[
v_h=\fc{\tau_{-h} u-u}{|h|}.
\]
Then $v_h\to n\cdot \pl u$, where $\lim_{h\to 0} \fc{h}{|h|}=n\in S^{n-1}$. 
\end{lem}
\begin{proof}
We have
\bal
\an{v_h,\ph}&=\an{\fc{\tau_{-h}u-u}{|h|},\ph}\\
&=\an{u,\fc{\tau_h\ph-\ph}{|h|}}.
\end{align*}
By Taylor's Theorem,
\bal
\tau_h \ph(x)-\ph(x)&=-h\pl \ph(x)+\ub{R(x,h)}{o(|h|)\text{ in }D(\R^n)}\\
\an{u,\fc{\tau_h\ph-\ph}{|h|}}&=\an{u,-\fc{h}{|h|}\pl \ph}+\ub{\an{u,\fc{(R(\cdot ,h))}{h}}}{\to 0\text{ as $|h|\to 0$}}\\
&=n\cdot \an{\pl u, \ph}.
\qedhere
\end{align*}
\end{proof}
This shows the distributional derivative coincides with the normal notion of derivative as difference quotient.

\blu{Lecture 4 (29 Jan)}

\subsection{Convolution between $D(\R^n)$ and $D'(\R^n)$}
If we combine the operations of reflection and translation, we get 
\[
(\tau_x\check{\ph})(y)=\check{\ph} (y-x)=\ph(x-y).
\]
If $u\in D(\R^n)$, we define the \textbf{convolution} of $u$ and $\ph\in D(\R^n)$ with
\[
(u*\ph)(x):=\int u(x-y)\ph(y)\,dy=\int \ph(x-y)u(y)\,dy=(\ph*u)(x).
\]
Using $\tau_h$ and $\check{\bullet}$, we can write 
\[
u*\ph(x)=\an{u,\tau_x \check{\ph}}.
\]
This is well defined for all $u\in D'(\R^n)$.
\begin{df}\llabel{df:dist1-5}
For $u\in D'(\R^n)$ and $\ph\in D(\R^n)$, define their convolution by 
\[u*\ph(x)=\an{u,\tau_x\check{\ph}}.\]
\end{df}
It is clear that $u*\ph(x)$ is just some function of $x\in \R^n$. It is actually smooth.
\begin{lem}\llabel{lem:dist1-4}
Define $\Phi_x(y)=\phi(x,y)$ where $\phi\in C^{\iy}(\R^n\times \R^n)$ and $\phi(\cdot, y)=0$ for $y$ outside some compact $K\subeq \R^n$. Then for $u\in D'(\R^n)$, 
\[
\pl_x^{\al} \an{u,\Phi_x}=\an{u, \pl^{\al}_x \Phi_x}.
\]
\end{lem}
We can take the derivatives inside the bracket.
\begin{proof}
By Taylor's theorem,  
%left with is linear map on $H$.
%use defn of $D'$
\[
\Phi_{x+h}(y)-\Phi_x(y)=\sum_i h_i \pd{\phi}{x_i} (x,y)+R_x(y,h)
\]
It is not difficult to show that $R_x(y,h)=o(|h|)$ in $D(\R^n)$ for each $x\in \R^n$. Hence
\[
\an{u,\Phi_{x+h}}-\an{u,\Phi_x}=\sum_i h_i \an{u,\pl{\Phi_x}{x_i}}+\an{u,R_x(\cdot, h)}.
\]
%sequence of test functions tend to 0, then continuity means $u$ acting on that sequence tends to 0.
Since $R_x(\cdot, h)=o(|h|)$ in $D(\R^n)$, dividing by $|h|$ and taking $h\to 0$ gives ($u=\fc{h}{|h|}$)
\[
n\cdot \an{u, \Phi_x}=n\cdot \an{u,\pl_x\Phi}.
\]
So the result follows.
\end{proof}
\begin{cor}\llabel{cor:dist1-1}
If $u\in D'(\R^n)$ and $\ph\in D(\R^n)$ then $u*\ph$ is smooth and $\pl^{\al}(u*\ph)=u*\pl^{\al}\ph$. 
\end{cor}
\begin{proof}
By Lemma~\ref{lem:dist1-4}, $\pl^{\al}(u*\ph)=\pl^{\al}_x\an{u,\tau_x \check{\ph}}=u*\pl^{\al}\ph$.
\end{proof}
%dist are normal functions, just take a derivative at the end.
\subsection{Density of $D(\R^n)$ in $D'(\R^n)$}
We have just seen the following.\\

\cpbox{
If $u\in D'(\R^n)$ and $\ph\in D(\R^n)$ then $u*\ph$ is smooth.}
\vskip0.15in
This is extremely useful. No matter how wild $u$ is, $u*\ph$ is nice. For this reason, $u*\ph$ is often called a  \textbf{regularisation} of the distribution $u$.
%wild creatures. give me any wild distribution, convolute with anything, get a smooth function. This is what we'll use to prove the density result.
We will use this fact to prove $D(\R^n)$ is dense in $D'(\R^n)$, i.e., for each $u\in D^1(\R^n)$ there exists a sequence of test functions $\{\ph_m\}_{m\ge 1}$ in $D(\R^n)$ such that $\ph_m\to u$ in $D'(\R^n)$, i.e., $\an{\ph_m,\te}\to \an{u,\te}$ for all $\te\in D(\R^n)$.

\blu{How to apply this: Suppose we have a problem about a distribution we don't know anything about. We know for some sequence of $\phi_m$, $\ph_m=u*\phi_m\to u$. We replace $u$ with $\ph_m$, do manipulations there, run the argument and take a limit.}


We need a technical lemma.
\begin{lem}\llabel{lem:dist1-5}
For $\ph,\psi\in D(\R^n)$ and $u\in D'(\R^n)$ we have
\[
(u*\ph)*\psi=u*(\ph*\psi).
\]
%insert dummy var
\end{lem}
\begin{proof}
The LHS is
\bal
(u*\ph)*\psi(x)&=\int (u*\ph)(x-y)\psi(y)\,dy\\
&=\int \an{u,\tau_{x-y} \wh \ph}\psi(y)\,dy\\
&=\int\an{u(z),\ph(x-y-z)\psi(y)}\,dy&\text{$z$ is ``dummy" variable}\\
&=\lim_{h\to 0} \sum_{m\in \Z^n}\an{u(z), \ph(x-z-hm)\psi(hm)h^n}&\text{Riemann sum}\\
&=\lim_{h\to 0}\an{u(z), \sum_{m\in\Z^n}\ph(x-z-hm)\ph(hm)h^n}
\end{align*}
(We want to take the $\int$ inside the $\an{\cdot}$, and we do this by turning it into a Riemann sum. Note the sum only has finitely many terms for each $m$, so this is legal.)
%sum eventually dead, only finitely many terms.
It is not hard to show that 
\[
\sum_{m\in \Z^n}\ph(x-hm)\psi(m)h^n\to \ph*\psi(x)\text{ in }D(\R^n)\text{ as }h\to 0.
\]
Continuing the above calculation,
\bal
&=\an{u(z),\ph*\psi(x-z)}\\
&=\an{u,\tau_x (\ph*\psi)\check{\,}}\\
&=u*(\ph*\psi)(x).
\end{align*}
%compact support, LHS has compact support, by fixed compact set, can differentiate
\end{proof}
\begin{thm}\llabel{thm:dist1-2}
$D(\R^n)$ is dense in $D'(\R^n)$. 
\end{thm}
We would like to say
\[
u(x)=\int\de(x-y)u(y)\,dy\stackrel?= \lim_{m\to \iy} \int \de_m(x-y)u(y)\,dy
\]
The $\de_m(x)$ are such that $\int \de_m(x)\,dx=1$ and get squashed closer and closer to $\de$. %(``a family of good kernels").
\begin{proof}
Fix $\psi\in D(\R^n)$ with $\int \psi\,dx=1$ and set
\[
\phi_m(x)=m^n \psi(mx)\qquad \pa{\int \phi_m\,dx=\int \psi\,dx=1}.
\]
%convolution of look-like de with original
Also introduce the bump function $\chi\in D(\R^n)$ with $\chi=1$ on $|x|<1$ and $\chi=0$ on $|x|>1$. Now set $\chi_m\pf xm$ and
\[
\ph_m=\pur{\chi_m(x)}(u*\phi_m)(x).
\]
%why can't $\ph$ of this guy?
(The purpose of $\pur{\chi_m(x)}$ is to make $\ph_m$ have compact support.)
Choose $\an{\ph_m,\te}$ for $\te\in D(\R^n)$ arbitrary, giving
\bal
\an{\ph_m,\te}&=\an{u*\phi_m, \chi_m\te}\\
&=(u*\phi_m)*(\chi_m\te)\check{\,}(0)\\
%&=u*\check{\ph}(0)\\
&=u*(\phi_m*(\chi_m \te)\check{\,})(0)\\
\phi_m*(\chi_m\te)\check{\,}(x)&=\int\phi_m (x-y)\chi_n(-y)\te(-y)\,dy\\
&=\int m^n \phi(m(x-y))\chi\pf{-y}{m}\te(-y)\,dy& y'=m(x-y)\implies y=x-\fc{y'}m\\
%Make the substitution 
&=\int\phi(y)\chi\pa{\fc y{m^2}-\fc xm}\te\pa{\fc ym -x}\,dy\\
&=\te(-x)+\ub{\int \phi(y)\chi\pa{\fc y{m^2}-\fc xm}\ba{
\te\pa{\fc ym-x}-\te(-x)
}\,dy}{=:R_m(-x)}\\
&=(\check{\te}+\check{R_m})(x).
\end{align*}
We can show $R_m\to 0$ in $D(\R^n)$. So \[\an{\ph_m,\te}=\an{u,\te}+\ub{\an{u,R_m}}{\to 0\text{ as }m\to \iy}.\] Hence $\an{\ph_m,\te}\to \an{u,\te},\te\in D(\R^n)$, giving $\ph_m\to u$ in $D'(\R^n)$ i.e. $D(\R^n)$ dense in $D'(\R^n)$. 
\end{proof}

We want to minimize the loss function. A good one is the hinge loss function. Suppose we want to do binary classification. A training sample is a pair $(X^k,Y^k)$. We want to minimize the loss function
\[
L(w)=\rc{Q}\sum_{k=1}^Q L(Y^k,\wt Y(X^k ,W)).
\]
%We're trying to predict something.
Often people take
\[
L=[m-Y^k\wt Y (X^k,W)]^+
\]
where $[\cdot]^+$ is rectification.

The loss function is \[L(w)=\rc{Q}\sum_k \sum_p\de(p, X^k,Y^k) X_p^k \prod_i w_i\].
(I'm forgetting the structure of the network, except that each path has $L$ components where $L$ is the number of layers.) The same $w_i$ appears in many paths. You're trying to learn a sequence of matrices: the collection of $w_i$ from each layer to the next.
%I only care that each path has $k$ terms.
%Piecewise linear.
It's a piecewise polynomial function whose pieces are very small.

Let's turn ourselves into physicists and do something ridiculous:
\begin{enumerate}
\item
$\de(p,X^k,Y^k)$ is a random variable, sometimes 1 vs. 0 independent of the product. (It's actually deterministic.)
\item
Every path starts at a different input.
\end{enumerate}
This is an ``ideal gas model" of neural nets (particles in gas never touch each---but this gives good results, $PV=n RT$.)
But it's not that different. 
(cf. Belief propagation assuming an infinite tree.) 
%not decay of correlation.

How do you optimize a function like this? We use stochastic gradient descent. Compute the gradient of the loss (one sample/one term in the sum). We do this efficiently by applying the chain rule. This idea popped up in 1985; it's called propagation; it started the 2nd wave of neural nets in the late 80's. Compute partial derivative with respect to each weight coordinate and update:
\[
W\mapsfrom W-\eta \nb_w L(X^k,Y^k,w).
\]
There are 2 schools: use stochastic gradient descent with 1 sample, vs. compute the entire sum for all samples. Use stochastic gradient descent because in training sets, many samples are redundant. You're wasting computation computing an accurate gradient. 

The number of $w$'s is typically in the range $[10^8,10^9]$. The amount of computation to compute 1 output (the size of the network) is in $[10^9,10^{11}]$. (Because of shared parameters there are fewer weights than nodes.) The number of examples is in $[10^6,10^9]$. We have armies of people sitting around labelling images. We can also use tagged images. We want to predict hashtags 

We also flip, distort the image for better training.

You can train a reasonably sized network in a week on a good laptop; in facebook/google we do it in a few hours with parallel processing.

Suppose it costs 0 to find a global minimum. Look at space of $x$'s; there is some hidden partition.
Two questions: 
\begin{enumerate}
\item
Suppose can get global minimum, how many layers do you need? 2, but the middle layer is exponentially large. (If the answer is 2 what is the question.) Given a function defined by a lot of $(X,Y)$ pairs.
Under reasonable smoothness assumptions, every function can be arbitrary approximated by sigmoids
\[
f=\sum_j w_j \phi_j(\sum_k u_{ik} x_k).
\] (cf. Kolmogorov superposition theorem) People in pattern recognition, theory, have been hung up: that's the way to do things, why would you do anything else? SVM's are not efficient.
%For an arbitrary
Sample classification.

\fixme{$10^8$ eqn for $10^9$ param
want something that's stable, works on new examples.
Generalization ability.
Retrain with a few more examples, the top layer, and is good for new species.
Is it out of the question that things done layer by layer?
%Train each layer unsupervised. 
If you are facebook or google, you're better off spending your money getting more samples.
}

Layer by layer supervised don't really work. It's not clear how to do that because the loss is only defined at the output.

It's common to add a regularizer to the loss function. It's like a prior. %make this precise
\[
L=\cdots + \la \ph(w).
\]
Think of SGD as getting a noisy estimate of the gradient. In practice we minibatch over samples of 100 because GPU's are faster when you do this.

Convergence is noisy and random. It creates a preference for solutions that are robust: relatively flat. %If the parameter ,
Bouncing around increases the value of the objective. Otherwise it's smaller and will not affect the average of the loss. Stochastic methods prefer the flatter methods. The regularizer says I want the solution to be flat.
\end{enumerate}

A lot is known about polynomials with Gaussian random coefficients (random matrix theory). Look at the critical points of these functions, as a function of the index of the critical point. (If there's a huge amount of saddle points with roughly equal number of dimensions going up and going down.)

We can plot a histogram of the saddle points with equal number of points going up and down. What about those with most dimensions going up? For 0 dimension going down, the loss function is clustered. 
Histogram of how many for each value of loss. Pretend that loss function is polynomial.
%the difference between those don't matter
\begin{enumerate}
\item
Regardless of optimization algorithm, you will get trapped where there are a lot of local minima. You almost always get the same value of the loss. (Never trapped in high loss.)
\item
Once you minimize loss function...
What you care about is average of loss function over %all samples, 
samples you've never seen before. If you did too well on the data you have, you probably overfit. The solution with smallest loss is probably not what you want.
\end{enumerate}
The game is to minimize the function on samples you've never seen before!

Dropout: in the top layers, you do as if half the nodes do not exist. It's a random subset that's different at each step. You can use only one half of the resources at each step.
It improves the performance.
EH: Bound on $L^1$ norm of $w$? No.
$L^1-L^2$ doesn't help. Size of weights don't matter because halfway rectification. You have the same function scaled by 2. You get some kind of generalization bound.

There are 2 challenges: what's the representational power of these systems: why is it good to stack things up. From analysis, Mark Tiger: if you make hypotheses that the signals are noise filtered, and you want to recover the filter, then stacking the layers helps.
%tyber?
%5--20

Also recurrent networks: the matrix wraps on itself. It's a dynamical system where the state at time $t+1$ is $x_{t+1}=[f(x_t)]^+$. This goes with \fixme{time}.

\fixme{
Bring 2 vectors close if they represent the same object. ``Is it Alex?" Some number in cosine space. Take the closest. Too large: threshold. It's not part of the training. Threshold built into argument.
Define argument arbitrarily.
}

If you want to build an image search technique---give me all images that match a sentence---train an embedding. Tomasz Nikolov, word2vec. Images to vectors. Cost: all images that match query closer than images than don't match.

Google uses this technique for image search, and uses whether people click on the image to further train.

If you make them small, Because number of parameters is more than samples, each solution is degenerate; lots of weights are ignored; there are lots of directions where loss function doesn't change of is not sensitivity. This makes the function easy to optimization, find good local minimum. If you try to train a smaller network than necessary, there will be bad local minimum. If you choose the right size, it becomes hard. In regression if number of samples equals the number of dimensions you can get infinitely bad condition number.

Proper vs. improper learning. Improper easier: to learn an object of size 1000 learn it with 10000 parameters.

Add another dimensions, get around local minimum without getting trapped. The likelihood to get trapped becomes 0.

Why does this generalize? Size your network according to number of samples?
SVM: overparametrize your function like crazy but regularize the hell out of it. 

%The size of the intermediate vector
%vector per training sample
%memory: you can get any output you want by setting the weight.
%This gives flexible options which you can regularize by choosing the shape of basis functions, or making the weight vector short so things don't change too quickly.
Oversize the neural nets like mad, and then regularize them. The fact that you can't actually get to the minimum is also a regularizer. There are many effects we don't understand.

AN: We want generative model for the data, so that if you apply your optimization method, there is an efficient procedure so that if you apply nearest neighbor, you get what you want?
Embedding a vector space is a side effect. I'm interested in any loss function. I'm just interested in producing a vector which  categorizes cats, dogs, etc.
That's too specific. This model is ridiculous because the hypothesis is wrong. What is it as a function of the number of layers? Bounded width. Deep learning: forget about 2 layers, work with many layers.

How many layers do we need for a particular task or class of tasks?

How many layers do we need as a function of the class of the function you want to find?

Let's say I have a collection of images.  I want my function invariant to shift. Is there a general statements about how many elements I need to do a good job? Assuming we can find a local minimum. Here's this class of function: with 2 layer need exponential, with $\ln n$ layer need linear or quadratic...

We can't separate 2 from 3 in circuit complexity.

Why do we always get a solution regardless of where we start from?

The geometry of the objective function. %where
How spread in the space? Can you get from minimum by just going down, or have to search around space?

Can we come up with better algorithms? Badly behaved, conditioned things. Conditioning, spectrum of Hessian at every point well-behaved?

Conjecture: if you have a particular problem, ex. 1000000 images from ImageNet, if you train neural nets of various sizes, for any solution you find, the number of dimensions is only a function of the problem, not of the architecture: augment dimensions that are flat.

Are there situation where deep learning fails, but something else works? There are situations where you have a good understanding of the data and can get a solution directly.

Optimization with multiple layers, overfitting prevention, all mixed found. Everything found the hard way by experiments. Any theoretical framework would be good.
%How hard to do linear regression if system doesn't know linear regression.
Put sigmoid: logistic regression. Hinge: SVM. Whatever ML can do can be done with deep learning? No, find better methods (deeper learning...)

Big nuts to crack: unsupervised learning. I only have $Y$'s, no $X$'s. Most of our learning is unsupervised. We learn the structure of the world by observing it. We learn object permanence Thsi takes several months. 

We can't do this except with text. You can predict the distribution of words that follow. 
%poetry prediction

If you want to apply this to learning physics in the world, you need video. There is a lot of fuzziness. You can't predict what exactly the world will look at. How can you represent a distribution/ensemble, etc. of video frames? There's a lot of literature that doesn't work.

Represent a single frame with hidden variable theory: things that you don't observe, like person's intentions. Depending on wind, unpredictable, if you can infer hidden variabels you can make a crisp predictions. 

\section{Language models}

For example, word2vec. These models give word embeddings that capture the meaning for the word. Given a corpus (sequence of words), map words $w$ to vectors $v_w\in \R^d$ where $d=300$ say. 

This has been studied fo many decades. %relating search query to the meaning: understand what the word meant.

A recurrent neural net is given text. Given words $w_1,\ldots, w_5$, predict the next word. The first layer maps to those vectors. There's a tradition of neural nets predicting the next word from the last 5. 
%crazy nonconvex function.
In what way does the embedding make sense?
\begin{enumerate}
\item
$\an{v_w,v_{w'}}$ correlates with humanly judged similarity.
%internal rep of word.
%Start with a word as a vector where 1 component is on.
\item
It solves analogies. Given a testbed of 1000 analogies. To solve man:woman::king:?. Find $w$ such that $v_{\text{man}}-v_{\text{woman}}=v_{\text{king}}-v_w$. This works $70-80\%$ of the time. 
\end{enumerate} 
This is plausible neurally. Mitchell, fMRI experiments: capture brain activation in 50000 voxels. Present a person with words. Using word embeddings, given activation patterns for 50 words and then a new word, you can predict whether a voxel will be on with $>50\%$ accuracy.

One version of word embedding:
\[
\Pj(w_1,\ldots, w_5) \propto \exp\pa{\an{v_{w_6}, \rc5 \sum_{i=1}^5 w_i}}.
\]
Human experiments: if you get a sentence scrambled you can probably unscramble it. Bag of words: scramble the local order.
%smallvec

This is a simple example of an energy-based model. 
What are the models actually representing? 
\begin{itemize}
\item
Understand these models (how they capture meaning).
\item
Understand the training algorithm.
\end{itemize}

People tried to explain these in various ways. The things to realize is that these are discriminative models. Here we want to predict the sixth word from the five words---these are called \vocab{discriminative models}.

If we want to prove any type of theorem we have to hypothesize some distribution on the input, a \vocab{generative model}.

Empirically it's been discovered that the embeddings have the property that 
\[
p_{ww'} = \Pj(w,w'\text{ co-occur in window of size 5}).
\]
Levi-Goldberg find these models fit $\an{v_w,v_{w'}} \approx \ln \fc{p_{ww'}}{p_wp_{w'}}=PMI(w,w')$. This suggests the matrix of PMIs is approximately PSD with low rank. SVD is not the best way; you need to put a weighting on it. 
The correct way is to put a weighting on the terms.

SVD is finding a rank $d$ approximation, finding $B$ of rank $d$ minimizing $\sum |A_{ij}-B_{ij}|^2$. We need to put a weighting $w_{ij}$. (Weighted SVD is NP-hard for $d\ge 3$, perhaps $d\ge 2$. But in practice this works.)

Initially linguists just took $p_{ww'}$. Then they found all types of reweightings. They found the PMI worked well. %linguists

Model: there is a discourse vectors $c_t\in \R^d$. Each direction has some semantic meaning. Nature does a random walk for $c_t$ on the space in $[-1,1]^d$, and
\[
\Pj(w|c_t)\propto \exp(v_w\cdot c_t)
\]
(Mnih-Hinton 2007). We introduced the random walk; we integrated over the random walk to get a closed form.

The new things is that in order to compute our closed-form expression, we assume the $v_w$'s are spatially isotropic: the global distribution is $sN(0,I)$ where $s$ is a scalar. The $v_w$'s are not unit vectors---they have information. Assuming $N(0,I)$ distribution allows us to calculate the messy integrals.
\[
\ln p_{ww'}=\fc{\ve{v_w+v_{w'}}^2}{2d}-2\ln Z+\ub{\ep_{n,d}}{\to 0}.
\]
We have $\ln p_w = \fc{|v_w|^2}{2d}-\ln Z+\ep'_{n,d}$. %predicted some correlations

We can prove the analogy solving more rigorously. Why is queen the right answer? For most words $\chi$,
\[
\fc{\Pj(\chi|\text{king})}{\Pj(\chi|\text{queen})} \approx \fc{\Pj(\chi|\text{man})}{\Pj(\chi|\text{woman})},
\]
ex. eat, walk. But for $x=$dress, John, Elizabeth this is not true. So we minimize 
\[
\min_\om \sum_{\chi} \ab{
\ln \fc{\Pj(\chi|\text{king})}{\Pj(x|w)}- \ln \fc{\Pj(\chi|\text{man})}{\Pj(\chi|\text{woman})}
}.
\]

Polysemy problem: Words like ``tie" have many meanings. These embeddings represent words by a single vectors. What does it do for polysemous words. 
We found a survey by linguists a few years ago. ``Obviously, you cannot just use word occurence counts for polysemy."

We set up sparse coding on $\{v_w\}$ with basis size $\approx 2000$ with sparsity $5$, such that every word is a linear combination of 5 words. The five nonzero coefficients capture the meanings.

%We had some intuition; it works. 
We created artificial polysemous words: take random words $w,w'$ and create $w_{\text{new}}$ the same token representing these words. This sequence of letters represents 3 different words. 
We find that $v_{w_{\text{new}}}=v_w+0.2v_{w'}$, say. 

What does the basis correspond to? Atoms of discourse. Here are 2000 directions corresponding to things people talk about (social media, cell biology, Indian cities, etc.). 

Look at tie: get atoms corresponding to clothing, rope, games.

What about multiple languages?

3 takeaways:
\begin{enumerate}
\item
Current efforts to understand word2vec. 
\item
All kinds of linear algebra that are NP-hard so you will never learn in linear algebra calss.
\item
Fitting a models are like fitting, factoring matrices. The more general problem is for tensors (Rong Ge).
\end{enumerate}

\section{Simple, efficient, and neural algorithms for sparse coding}

We'll think about algorithms for sparse coding by connecting them to approximate gradient descent. How iterative algorithms on nonconvex algorithms make progress.

Olshausen and Field introduced sparse coding to try to understand a scientific mystery: the response properties of neurons---they were trying to solve an optimization problem.

They took many natural images, broke them into patches, and applied sparse coding: find a basis for them. They look like Gabor filters: localized, bandpassed, and oriented. 

Contrast to a theorist's swiss army knife of singular value decomposition. You'd find a noisy, different basis that is difficult to interpret. 

Applying sparse coding gives much more meaningful bases. 

Are there efficient, neural algorithms for sparse coding with provable guarantees? The algorithms should not just be polynomial time, but made of simple building blocks so they might be implemented neurally.

\subsection{Olshausen-Field Update Rule}

Many types of data are sparse in an appropriately chosen basis: images, signals, etc. Think of them as $n$ vectors; we're given $p$ of them. We have a $n\times m$ dictionary $A$,
\[
Ax^{(i)} \approx b^{(i)}.
\]
We want the representation to be simple. Sparse recovery: if you were given $A$ and $b$ can you find a sparse $x$ such that $Ax\approx b$. 
But now the meat of the issue is to learn $A$ from examples! Given only the $b$'s, you're promised that they come from $A$ and $k$-sparse $x$'s. You could add a penalty function that become denser and denser; you can use analytic sparseness, etc.

For example, 1-sparse reduces to $k$-means: look for $m$ means.

Here's the usual nonconvex version:
\[
\min_{A,x^{(i)}}\sum_{i=1}^p \ve{b^{(i)}-Ax^{(i)}} + \sum_{i=1}^p L(x^{(i)})
\]
where the last term is a nonlinear penalty function to encourage sparsity. (We can do a hard constraint like $\iy$ for $k+1$ nonzeros, or something softer.)

We'll give iterative rules and talk about why they work.

OF's approach is as follows. 
\begin{itemize}
\item
Architecture: There is a 3-layer network. Given an image patch, keep track of the residual, how much of the image you haven't represented yet. The dictionary is stored in the strength of connections between wires. Start with image, add to residuals, multiply by $A$. There is feedback to penalize dense $x$. 

The network performs gradient descent on $\ve{b-Ax}^2 + L(x)$. Alternate between 
\begin{enumerate}
\item
$r\mapsfrom b-Ax$
\item
$x\mapsfrom x+\eta(A^T r -\nb L(x))$.
\end{enumerate}
\end{itemize}
$A$ is updated through Hebbian rules (strengthen/weaken connections). There are no provable guarantees but this works well.
%Iterative approach to solve something like Candes-Tao
%how do you learn from x's.
It works well empirically: given natural images, it makes Gabor filters that are interesting. You're happy by the result because it has low reconstruction error with sparse entries.

The generative model we assume will have sparse basis. Do we find that basis? 

The iterative rule finds a basis. In the model there is a global minimum. 
%how know find global optimal solutions.

Other approaches and applications.
\begin{itemize}
\item
signal processing/statistics: MOD, kSVD
\item
Machine learning
\end{itemize}

Generative model: 
\begin{enumerate}
\item
given unknown dictionary $A$,
\item
generate $x$ with support size $k$ u.a.r.; choose non-zero values independently, observe $b=Ax$. 
\end{enumerate}•
We can also add in noise, $Ax+\be$. Many extensions continue to work, but we'll focus on the simple model.

Spielman, Wang, and Wright (13): works for full column rank up to sparsity $\sqrt n$ ($m\le n$). If makes more sense for $A$ to have many more columns to rows---it's much more flexible. Often you mix and match dictionaries---sparse in some combination of basis elements.

Arora, Ge, and Moitra (14): works for overcomplete, $\mu$-incoherent $A$ up to sparsity $n^{\rc2-\ep}/\mu$. 

Agarwal et l. (14): overcomplete, $\mu$-incoherent $A$ up to $n^{\rc 4}/\mu$ via alternating minimization. Use $L^1$ optimization as subroutine.

Barak, Kelner, Steurer (14): works for overcomplete $A$ up to $n^{1-\ep}$, running time exponential in accuracy.

Arora, Ge, Ma, Moitra 14: There is a variant of OF-update rule that converges to the true dictionary.

New update rule:
\begin{enumerate}
\item
$\wh x^{(i)} = threshold(\wh A^Tb^{(i)})$: how correlated your vector is with basis elements so far, threshold to zero out nonzero entries. Take that as representaiton
\item
$\wh A\mapsfrom \wh A + \eta \sum_{i=1}^q (b^{(i)}-\wh A\wh x^{(i)})\sign (\wh x^{(i)})^T$. Update with rank-1 outerproduct: residual error.
\end{enumerate}
Nice:
\begin{itemize}
\item
The samples arrive online. In contrast, previous provble algorithms might need to compute a new estimate from scratch, when new samples arrive. 
\item
The computation is local: a neuron looks at neighbors and do thresholds.
\item
The update rules is explicitly Hebbian: neurons that fire together wire together.
\end{itemize}
The weight $\wh{A_{i,j}}$ is the product of the activations at the residual layer and the decoding layer. 

What is neurally plausible? Skip.

Approximate gradient descent: we give a general framework for designing and analyzing iterative algorithms for sparse coding. 
Usual approach is to think of minimize non-convex function
\[
\min_{\wh A, \text{coln-sparse} \wh X} E(\wh A, \wh X) = \ve{B-\wh A\wh X}_2^2
\]
Why should a iterative problem make progress on nonconvex function? What if think of it as minimizing unknown convex function.

Think of $X$ as unknown: 
\[
\min_{\wh A} E(\wh A, X)=\ve{B-\wh AX}_F^2. 
\]
Now the function is strongly convex, and has global optimum that can be reached by gradient descent. 

Separately convex in each variable. 

New goal: prove that with high probability the step (2) approaches the gradient of this function.

Just have some nontrivial inner product to not get stuck. This is where we use distributional properties.

%If you have a function which is convex in each variable separately, under some distribution.

Conditions for convergence (convex optimization): consider the following general setup:
\begin{itemize}
\item
optimal solution $z^*$
\item
update $z^{s+1}=z^s-\eta g^s$.
\end{itemize}
This works even if it's correlated with the gradient:
\begin{df}
$g^s$  is $(\al,\be,\ep_s)$-correlated with $z^*$ if for all $s$, 
\[
\an{g^s,z^s-z^*}\ge \al \ve{z^s-z^*}^2 + \be\ve{g^s}^2 -\ep_s.
\]
\end{df}
\begin{thm}
If $g^s$ is $(\al,\be,\ep_s)$-correlated with $z$, then
\[
\ve{z^s-z^*}^2 \le (1-2\al \eta)^2\ve{z^0-z*}^2 + \cdots.
\]
\end{thm}
Take usual proof almost verbatim; can make progress as long as correlated. 
Fresh randomness golf you out of where you are.

\begin{enumerate}
\item
$\wh x^{(i)}=$threadhold$(\wh A^Tb^{(i)})$. 

Formualte decoding lemma. 
\item
Update $\wh A$.

Calculate expectation of column-wise update rule. Expectation is $A_j-\wh{A_j}$. The rest is systemic bias $\xi \E_R[\wh A_R\wh A_R^T]A_j$.
Then auxiliary lemma. 

Various conditioning trips.
\end{enumerate}•
%numerical bound for . won't be very different from top $k$. Uses the generative model.

An initialization procedure: we give an initialization algorithm that outputs $\wh A$ that is column-wise $\de$-close to $A$ for $\de\le \rc{\poly\log(n)}, \ve{\wh A-A}\le 2$. 
\begin{enumerate}
\item
Choose samples $b,b'$. There's a reasonable chance they intersect in 1 element (constant/polylog chance).
\item
Calculate $M_{b,b'}$. Filter the guys which share that column. 
\item
Hope there is one large eigenvalue and others noticably smaller.
\end{enumerate}•
This happens when $\Supp(x)\cap \Supp(x')$ is a singleton.

Further results: adjusting an iterative algorithm can have subtle effects on its behavior.  Think accelerated gradient method vs. high ball method (?).

We can use our framework to systematically design and analyze new update rules. ``Crosstalk" between columns. Once know where error comes in, can remove systemic bias, by carefully pojecting out along direction being updated.
\begin{enumerate}
\item
$\wh x_j^{(i)} = threshold(\wh C_j^Tb^{(i)})$ where $\wh C_j$ has all columns except $\wh A_j$ projected to the orthogonal complement of $\wh A_j$.
\item
Update $\wh A_j$.
\end{enumerate}
Think of the convex programming problem you wish you had.
%golf out of local optimum without convex.

%This is what's going on.
Aren't you hiding in original initialization? All these things need careful initialization.

It's easy to analyze when you're very close to the minimum. Here is the medium range $\rc{\poly\log(n)}$. 

Summary
\begin{enumerate}
\item
Online, local, Hebbian algorithms for sparse coding that find globally optimal solution (whp).
\item
Introduced framework for analyzing iterative algorithms by thinking of them as trying to minimize unknown convex function
\item
The key is working with a generative model.
\item Is computational intractability really a barrier to a rigorous theory of neural computation?
\end{enumerate}•
%between every pair they could be on, chance both on is close to if independent.
Conditional pairwise independent.

Put other alternate minimization into this framework, and engineer more alternate minimization.

%\subsection{A New Update Rule}

\section{Tensor decomposition}

Two parts:
\begin{enumerate}
\item
Many unsupervised learning problems can be rephrased as tensor decomposition problems. 
\item
Stochastic gradient descent for tensor decomposition.
%find k components
\end{enumerate}

One problem is the mixture of Gaussians problem. Observe samples from Gaussian distributions. Assume they have the same covariance matrix and are spherical. For each sample, you don't know which Gaussian it came from. Try to find the center of the Gaussian distributions. The $k$ components are the $k$ centers. 

Many of the problems discussed today, like sparse coding, falls in this category. Find vectors that can sparsely represent all the given vectors.
%Think of each layer as vectors you want to find.

All these problems are necessarily nonconvex because the solution we are looking for are components $u_1,u_2,\ldots, u_k\in \R^d$. Suppose the objective function is $L(u_1,u_2,\ldots, u_k)$. %, we want to say if we optimize, then 
The components do not come in specific order: if we swap then it's still the same solution. If we take a convex combination then it's no longer an optimal solution. Because of the symmetry there are many equivalent global optimum solutions.
%ex. consider $u_i$ increasing. Don't know of set that works for all problems. 

%smaller subset of $\R^d$.

\begin{prb}[Independent component analysis]
There are $n$ unknown independent sources (people talking) and $n$ signals observed (microphones). What are the original sources?

Suppose $x$ has independent components $\E x =0$, $\E xx'=1$. (Ex. Consider $x\sim_R \{\pm 1\}^n$. The general notion is that there is a measure of non-Gaussianity (if all components are Gaussian it looks the same from every direction))

There is an unkown linear transform $A\in \R^{m\times n}$. 

Observe $y=Ax$.

Goal: find (approximate) $A$.
\end{prb}

One way to solve this problem is to use tensor decomposition. 

Tensors are higher-dimensional arrays. We consider 4-dimensional tensors $T\in \R^{n^4}$. A tensor is rank 1 if it can be written as a product of tensors, $T=x\ot x\ot x\ot x$ (we consider symmetric tensors). Then $T_{i,j,k,l} = x_ix_jx_kx_l$. A low rank decomposition is $T=\sum_{i=1}^4 x_i x_i^{\ot 4}$.

Define
\[
T(u) = \sum_{i,j,k,l} T_{ijkl} u_iu_ju_ku_l.
\] 
This is a degree 4 polynomial. For this particular tensor, $T(u)=\sum_{i=1}^r \la_i\an{x_i,u}^4$. 

This is similar to spectral decomposition: breaking up into a sum of $r$ different rank 1 components. The benefit from tensor decomposition is that it is unique.

%suppose we have a tensor with low-rank decomposition.
%want to find its decomposition

How do we construct tensors? Use the 4th order cumulant $\ka_4(X)$. We need the following properties:
\begin{enumerate}
\item
If $X,Y$ are independent, $\ka_4(X+Y) = \ka_4(X)+\ka_4(Y)$. 
\item (looks like degree 4 homogeneous polynomial)
For $C\in \R$, $\ka_4(CX)= C^4 \ka_4(X)$.
($\ka_4(X) = \E[X^4] - 3 \E[X^2]^2 $ when $\E[X]=0$.) For Gaussian this is 0. If it's nonzero it's far from Gaussian.
\end{enumerate}
Using this cumulant you can construct a tensor with low-rank form. 

\begin{clm}
If $T(u)=\ka_4(u^Ty)$ then $T=\sum_{i=1}^n\ka_4(x_i)A_i^{\ot 4}$.
\end{clm}
\begin{proof}{}
\bal
\ka_4(u^Ty) &= \ka_4(u^TAx)\\
&=\ka_4 \pa{\sum_{i=1}^n \an{u,A}x_i}\\
&=\sum_{i=1}^n \ka_4(x_i) \an{u,A}^4.
\end{align*}
\end{proof}
%We 
%Why not represent the characteristic function and... moment generating, Laplace transform?
People have tried different functions. Some work better in practice. In theory this is easiest to analyze because it's a degree 4 polynomial. 

AN: If look at characteristic functions, product of functions of 1 variable. Moment generating: anything takes sum into product. 
\[
\E e^{iu^Ty}
\]
This breaks down as a product. Functions of 1 variable? (Santosh? ICA paper.)
%Problem becomes simpler? 
It's important to represent function in finite way. When something breaks down into product, it's a different way of representing it simply. %There's really no $u$ at the end.
%different notion of being simple.

In general tensor decomposition is NP-hard. Our problem we can solve in polynomial time; there are many algorithms. WLOG assume $A$ is orthonormal ($A^TA=I$). We have $\E[yy^T]=\E[Axx^TA^T]=AA^T$. We compute a whitening matrix $W$ such that $W^TAA^TW=I$. (Use matrix factorization.)

We can change the problem to $Z=W^Ty = W^TAx$. %Linear transformation is orthronormal.
WLOG can always assume linear transformation orthonormal.

The algorithm works like this.
\begin{enumerate}
\item
Pick random vectors $a\in \R^n$. 
Compute $T(a,a)= \sum_{i=1}^n \ka_4(x_i)\an{A_i,a}^2 A_iA_i^T = ADA^T$. 
%Exactly same as SVD. 
The entries of $D$ are likely to be distinct. Just do SVD of $T(a,a)$ to find $A$.
\end{enumerate}•

Do perturbation analysis: even if don't know exactly, we can approximate $A$. 

In practice the tensor is not exactly row rank. This algorithm is not so robust. Then I show how we can use basic optimization techniques to solve this kind of tensor decomposition problem. 

Tensor decomposition can solve mixture of Gaussian (with arbitrary covariance), hidden Markov models, topic models, stochastic block models, models for trees (phylogeny reconstruction). 
%Explicit formula to compute $T(a,a)$. 

%Interesting class of functions we can use sgd/tensor decomp.

Let $f(u)=-\rc 2 T(u) =\sum_{i=1}^n \an{u,A_i}^4 = \ve{A^T u}_4^4$. We want
\[
\max_{\ve{u}_2=1} \ve{A^Tu}_4^4.
\]
Rotating to basis $A$, the $L^2$ norm does not change. 
\[
\max_{\ve{u}_2=1} \ve{u}_4^4.
\]
This is maximized when $u$ is one of the basis vectors. If we can find a local max, then we can find one component of the tensor. 
This is still nonconvex; there are many saddle points. 

The above is just a motivating example. It doesn't have any bad local maximum. They are global maximum and correspond to meaningful components.
%To go from one to many, 
If you start from bad points the gradient will not move. It's not clear why you can use gradient descent. People use second-order optimization techniques: Find a positive eigenvalue in your Hessian.

Guaranteed to find one of local solutions. We come up with the definition of strict saddle functions. 
\begin{df}
Critical point of $f$ is $x$ such that $\nb f(x)=0$. If $x$ is neither a local max nor min, then we say $x$ is a saddle point. %just look at Hessian, can't even tell if saddle point.

Assume $f\in C^{\iy}$. $f(x)$ is a strict saddle if for any point $x$ one of the following holds. 
\begin{enumerate}
\item
$x$ is close to a local minimum. (Assume $f$ is strongly convex there so the Hessian is strictly convex.)
\item
$\ve{\nb f(x)}_2$ is large
\item
$\la_{\min{}}(\nb^2 f(x))<-\ga$. It has a negative direction, and is upperbounded by $-\ga$.
\end{enumerate}• 
Then you can use a second-order algorithm.
\end{df}
(On a manifold redefine the gradient to be the projection onto the tangent space. Hessian with Lagrange multiplier.)

In 2 and 3 you are guaranteed to make fixed progress, so that you can get close; then you are in case 1. 

We show not only you can optimize it, you can optimize this using stochastic gradient descent. 
Local minimizations are permutations of global minimizers, tensor decomposition.

What do I mean by a stochastic gradient algorithm? 
\[
x^{(t+1)} \mapsfrom x^{(t)} - \eta(\nb f(x^{(t)}) +\ep_d)
\]
where the $\ep_d$ is a random variable that capturres the error because we are not using all samples. We have 
\[\forall x, \E[\ep_t|x^{(t)}] = 0,\] 
i.e., we have an unbiased estimator for the gradient.
\begin{thm}[GHuangJinYang, COLT]
If $f(x)$ is a strict saddle then stochastic gradient converges to a local minimum in polynomial number of steps.
\end{thm}
This is different from Ankur's talk: no matter where you start, as long as your function has this property, your function will always converge. 

Ankur broke the symmetry by the initialization algorithm (unique permutation of components closest to solution). Here we don't have that, but because we are doing SGD, can break symmetry so will converge to one of multiple equivalent solutions.

The intuition is:
\begin{enumerate}
\item
Trying to optimize $f(x)$, a strongly convex function.
\item
As long as your step size is not too large, we will in expectation make progress after 1 step. 
\item
\end{enumerate}•
What if we start close to saddle point? Consider $x^2-y^2$. Then $\nb f= (2x,-2y)$. If $\ep_d$ is a standard Gaussian, or any other whose covariance matrix is something we have a sense of (ex. $I$), then we can recursively compute $\E[(x^{(t+1)})^2] = (1-2\eta)^2 \E[(x^m)^2] + \eta^2$. We can solve this recursion. In a small number of steps the expectation converges to $\te(\eta)$. 
%x coord at time t.
On the other hand, for $y$ you get $(1+2\eta)$ and this goes to infinity. It becomes large after $O\prc{\eta}$ steps. In $\rc{\eta}$ steps, the expectation of $y^2$ is larger in the $x$ direction.
The same intuition holds in more complicated settings: you might have more directions that have positive eigenvalues, but they will be bounded in $\E\bullet^2$, those with negative eigenvalues will $\to -\iy$. This is cheating because for this function the Hessian is constant. The Hessian is the same no matter how far you go. In the real proof what's hard is to show you don't go very far so we can use smoothness of the Hessian to make sure that even if it's not constant the argument goes through.

Finally, the construction for tensor decomposition is as follows. Let $T=\sum_{i=1}^n A_i^{\ot 4}$. We happen to have a construction ($u_i\in \R^n$, $\ve{u_i}=1$)
\bal
f(u_1,\ldots, u_n) &= \sum_{i\ne j, 1\le i\le n} T(u_i,u_i,u_j,u_j)\\
&=\sum_{k=1}^n \an{A_k, u_i}^2 \an{A_k,u_j}^2
\end{align*}
Compute just from $T,u_i,u_j$.
Lemma: Local min of $f$ will have $u_i=\om A_{\pi(i)}$. 

Write out the $u_i$ in the basis of $A_i$'s. 

Zero when disjoint support. Otherwise positive contribution: nothing will cancel it. One term is 0 only when $u_i,u_j$ have disjoint support when represented in basis of $A$. $n$ vectors pairwise disjoint support. None empty.%Only way for support to form permutation is if no


\printbibliography
\end{document}